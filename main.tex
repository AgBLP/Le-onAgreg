%%%%%%%%%%%%%%%%%%%%%%%%%%%%%%%%%%%%%%%%%%%%%%%%%%%%
\documentclass[14pt]{extbook}


%%%% french character
\usepackage[french]{babel}
\usepackage[T1]{fontenc}
\usepackage[utf8]{inputenc}


%%%% useful package
\usepackage[left=1.8cm, right=1.8cm, top=2cm, bottom=2.5cm]{geometry}
\usepackage{subcaption} % for figure caption
\usepackage{graphicx} % image
\usepackage{tabularx} % table
\usepackage[table]{xcolor} % color in table
\usepackage{amsmath} % math
\usepackage{amssymb} % bold math
\usepackage{wasysym} % integral
\usepackage[many]{tcolorbox} % colored box
\usepackage{fancyhdr} % headers
\usepackage{enumitem} % for bullet in itemize
\usepackage{hyperref} % for link
\usepackage[version=4]{mhchem}
\usepackage{braket}
\usepackage{awesomebox}
\usepackage{chemist}
\usepackage[squaren, Gray, cdot]{SIunits}

%%%% settings
\setlength{\parskip}{0cm}
\setlength{\parindent}{0.8cm}
\setlength{\headheight}{25.86253pt}
\renewcommand{\baselinestretch}{1}
\renewcommand{\thesection}{\Roman{section}}
\addto\captionsfrench{\renewcommand\chaptername{Leçon}}

\newenvironment{changemargin}[2]{%
  \begin{list}{}{%
    \setlength{\topsep}{0pt}%
    \setlength{\leftmargin}{#1}%
    \setlength{\rightmargin}{#2}%
    \setlength{\listparindent}{\parindent}%
    \setlength{\itemindent}{\parindent}%
    \setlength{\parsep}{\parskip}%
  }%
  \item[]}{\end{list}}

%%%% small macros
%\newcommand{\unit}[1]{\; \mathrm {#1}}
\newcommand{\ex}{\mathrm {e}}
%\newcommand{\deriv}{\mathrm {d}}
\newcommand{\timeD}[1]{\frac{\deriv #1}{\deriv t}}
\newcommand{\fracD}[2]{\frac{\deriv #1}{\deriv #2}}
\newcommand{\partialD}[2]{\frac{\partial #1}{\partial #2}}
\newcommand{\timeCode}[1]{\textit{#1}\bigskip}
\newcommand{\Int}{\displaystyle\int}
\newcommand{\grad}{\Vec{\mathrm{grad}}}
\newcommand{\Frac}[2]{\displaystyle\frac{#1}{#2}}
\newcommand{\vecb}[1]{\mathbf {#1}}
\newcommand{\sinc}{\mathrm{sinc}}
% fluide
\newcommand{\Rey}{$R_e$ }
\newcommand{\convec}{(\Vec{v}\cdot\grad)\Vec{v}}
\newcommand{\visq}{\eta\Delta\Vec{v}}
\newcommand{\ud}{\mathrm{d}}


%%%% color
\definecolor{correction}{rgb}{0.58, 0.00, 0.82}
\definecolor{experience}{rgb}{0.00, 0.48, 0.44}
\definecolor{highlight} {rgb}{0.78, 0.12, 0.12}
\definecolor{section}   {rgb}{0.00, 0.45, 0.00}
\definecolor{slide}     {rgb}{0.89, 0.00, 0.30}

%%%% color macros
\newcommand{\correction}[1]{\textcolor{correction}{#1}}
\newcommand{\experience}[1]{\textcolor{experience}{#1}}
\newcommand{\partC}[1]{\textcolor{section}{\textbf{#1}}}
\newcommand{\slide}[1]{\textcolor{slide}{$\rightarrow$ #1}}

%%%% bigger macros
%%%% header
\pagestyle{fancy}
\lhead { % left header
  \textbf{\footnotesize \'Ecole normale supérieure}
  \newline
  \footnotesize Préparation à l'agrégation de physique-chimie option physique
}
\chead{ % central header
}
\rhead{ % right header
  \hfill \textbf{\footnotesize Compte-rendu leçon physique}
  \newline \hfill
  \footnotesize 2022-2023
}
\renewcommand{\headrulewidth}{0.4pt}


%%%% blocks for report
\newenvironment{reportBlock}[1]{
  \begin{tcolorbox}
  [
    breakable, enhanced jigsaw, % to break box over page
    arc = 0mm, % straight line
    title = \textbf{#1}, % title
    coltitle = black, % title font color
    colbacktitle= black!10!white, % light-gray title
    colback= white, % white background
    colframe= black % dark frame
  ]
} 
{
  \end{tcolorbox}
}


%%%% blocks for experiment
\newenvironment{experimentBlock}{
  \begin{tcolorbox}
  [
    breakable, enhanced jigsaw, % to break box over page
    arc = 0mm, % straight line
    colback= white, % white background
    colframe= black % dark frame
  ]
}
{
  \end{tcolorbox}
}


%%%% header block
\newenvironment{headerBlock}{
  \begin{tcolorbox} 
  [
    arc= 0mm, % straight line
    colback= black!10!white, % light-gray background
    colframe= white % no frame
  ]
}
{
  \end{tcolorbox}
}


%%%% Section
\newcommand{\sectionC}[1]{%
  \refstepcounter{section}
  % number and text
  \textcolor{section} {\textbf {\Roman{section}-- #1}}
  
  % subsection counter update
  \setcounter{subsection}{0}
  \addcontentsline{toc}{section}{\protect\numberline{} #1}
}%


%%%% Subsection
\newcommand{\subsectionC}[1]{%
  \refstepcounter{subsection}
  % number and text
  \textcolor{section} {\arabic{subsection}--  #1}
  
  % subsection counter update
  \addcontentsline{toc}{subsection}{\protect\numberline{} #1}
}%


%%%% Highlight frame
\newenvironment{highlightFrame}[1]{
  \begin{tcolorbox}
  [
    breakable, enhanced jigsaw, % to break box over page
    arc = 0mm, % straight line
    colback= white, % white background
    colframe= highlight % orange frame
  ]
} 
{
  \end{tcolorbox}
}

%%%% document
\begin{document}
\tableofcontents
\newpage
\part{Leçons de physique}


%\vspace*{\stretch{0.5}}
%\begin{center}
%    \vspace{5cm}
%    \Huge{\textbf{Leçons de Physique}}
%\end{center}
%\vspace*{\stretch{1}}

\input{LP_avancement.tex}
\newpage
%%%%%%%%%%%%%%%%%%%%%%%%%%%%%%%%%%%%%%%%%%%%%%%%%%%%
%%%% En-tête leçon
\begin{headerBlock}
  \chapter{Gravitation}
    \label{LP_Gravitation}
\end{headerBlock}

%%%%%%%%%%%%%%%%%%%%%%%%%%%%%%%%%%%%%%%%%%%%%%%%%%%%
%%%% Références
\begin{center}
\begin{tabularx}{\textwidth}{| X | X | c | c |}
  \hline
  \rowcolor{gray!20}\multicolumn{4}{c}{Bibliographie de la leçon : } \\
  \hline 
  Titre & Auteurs & Editeur (année) & ISBN \\
  \hline
La physique par la pratique    & B. Portelli et J. Barthes   &  H\&K (2005) &    \\
  \hline 
     Mécanique & E. Hecht  & de boeck (2006)  &   \\
  \hline 
    Mécanique, Fondements et applications & J.-Ph. Pérez   & Dunod (2014)   &      \\
  \hline 
  Physique tout-en-un 2ème année PC-PSI & M.-N. Sanz & Dunod (2004) & \\
  \hline
  Mécanique I & BFR & Dunod (1995) & \\
  \hline
  
\end{tabularx}
\end{center}

\begin{reportBlock}{Commentaires des années précédentes :}
    \begin{itemize}
        \item \textbf{2017 :} Les applications ne doivent pas nécessairement se limiter à la gravitation terrestre,
        \item \textbf{2016 :} Les analogies entre l’électromagnétisme et la gravitation classique présentent des
limites qu’il est pertinent de souligner
    \end{itemize}
\end{reportBlock}

%%%%%%%%%%%%%%%%%%%%%%%%%%%%%%%%%%%%%%%%%%%%%%%%%%%%

%%%%%%%%%%%%%%%%%%%%%%%%%%%%%%%%%%%%%%%%%%%%%%%%%%%%
%%%% Plan
\begin{reportBlock}{Plan détaillé}
  \textbf{Niveau choisi pour la leçon :} CPGE 2ème année
  \newline
  \textbf{Prérequis : }Mécanique newtonienne ; Equations de Maxwell en électrostatique ; Loi de l'hydrostatique ; Référentiels non galiléens ; Lois de Kepler
  \newline
  
  \textbf{Déroulé détaillé de la leçon: }   \newline
  
  \section*{Introduction}

Introduction historique avec Aristote et Galilée et la chute des corps. Expérience de pensée de Galilée : tous les objets tombent à la même vitesse dans le vide. Enfin, Newton qui décrit les lois de la mécanique classique, en particulier de gravitation universelle.


\section{Interaction gravitationnelle}

\subsection{Force de gravitation}

Gravitation = interaction entre deux masses. Schéma avec deux masses. $\overrightarrow{F}_G = - \frac{m_1 m_2}{r_{12}^3} \overrightarrow{r_{12}}$. \\
\begin{itemize}
\item Force attractive dirigée suivant $\mathbf{r_{12}}$.
\item $G = 6.67\times10^{-11} kg^{-1} m^3 s^{-2}$
\end{itemize}

Rappel force électrostatique $\overrightarrow{F}_e = \frac{q_1 q_2}{4\pi \epsilon_0 r_{12}^3} \overrightarrow{r_{12}}$.


\subsection{Analogie avec le champ électrostatique (sur slide)}

Analogie entre :

\begin{itemize}
\item Forces $F_e$ vs $F_G$
\item charge vs masse
\item $\frac{1}{4 \pi \epsilon}$ vs $-G$
\item Champs électrique vs champ gravitationnel
\item Equations de Maxwell-Gauss et Maxwell-Faraday en statique vs leurs analogue en mécanique
\item Potentiel électrique vs potentiel gravitationnel
\end{itemize}

\subsection{Champ gravitationnel d'une boule homogène}

Schéma boule, coordonnées sphériques. Détermination des invariances et des plans de symétrie. Application du théorème de Gauss dans le cas $r\leq R$ pour le calcul du champ gravitationnel terrestre. Tracé du champ gravitationnel vs $r$. Discussion de la limite de l'analogie entre les interactions électrostatique et gravitationnelle.

\section{Dynamique terrestre (10'35)}

\subsection{Caractère non galiléen terrestre}

Référentiel géocentrique supposé galiléen.
$\Omega_T = \frac{2 \pi}{T}=7.27\times10^{-5} rad/s$.  \\

Calcul de l'accélération dans le référentiel terrestre (non galiléen). PFD dans le référentiel terrestre :
\begin{equation}
    \mathbf a = \mathbf{\mathcal{G}}_T(M) - \mathbf \Omega_T \land \mathbf \Omega_T \land \mathbf{HM} + \sum_p \left(\mathbf{\mathcal{G}}_p(M) - \mathbf{\mathcal{G}}_p(T) \right) - 2  \mathbf \Omega_T \land \mathbf v_{/R_t}(M)
\end{equation}
\begin{itemize}
    \item Les deux premiers termes : $\mathbf g(M)$ 
    \item Dernier terme : négligeable
    \item Terme de somme : terme de marée
\end{itemize}
\begin{equation}
    \mathbf{a} = \mathbf{g} + \sum_{p} \left( \mathbf{\mathcal{G}}_p(M) - \mathbf{\mathcal{G}}_p(T) \right)
\end{equation}
avec $p$ une planète, le soleil ou un satellite.\\

Ordres de grandeurs des accélérations de Coriolis et d'entraînement:
\begin{align}
v & \simeq  1m/s \\
\mid a_c \mid & = 2 \Omega_T v \sim 10^{-4} m s^{-2}   \\
\mid a_e \mid & = \Omega_T^2 R_T \sin^2(\lambda) \sim 3\times10^{-2} m s^{-2}
\end{align}

\subsection{Mesure du champ de pesanteur à la latitude 48.8 N (19'30)}

Schéma pendule simple. PFD $\ddot{\theta} + \frac{g}{L} \sin \theta$. Approximation des petits angles $\ddot{\theta} + \frac{g}{L} \theta$. Période $T = 2\pi\sqrt{\frac{L}{g}}$. Mesure expérimentale de la longueur de la tige, le rayon et la hauteur du cylindre. Tracé de l'angle en fonction du temps avec Latis Pro et fit sur Qtiplot.\\

Discussion champ de pesanteur non uniforme sur la Terre dû à la non sphéricité de la Terre.

\section{Effets de marées (33'20)}
Le Gié (Mécanique I) ou Faroux-Renault p278-286

Schéma système Terre+Lune. Champ de marée $\mathbf{\mathcal{C}}(M) = \mathbf{\mathcal{G}}_L(M) - \mathbf{\mathcal{G}}_L(T) =  G m_L \left( \frac{\mathbf{MA}}{MA^3} - \frac{\mathbf{TA}}{TA^3} \right) = G \frac{m_A r}{2d^3} \left[(3cos^2(\theta) - 1) \mathbf{e_r} - \frac{3}{2} sin(2\theta) \mathbf{e_\theta} \right] = - \nabla V_A(M)$.


Discussion $\theta = 0$ et $\theta = \pi$ et bourrelets océaniques.

(41'30)

\end{reportBlock}


%%%%%%%%%%%%%%%%%%%%%%%%%%%%%%%%%%%%%%%%%%%%%%%%%%%%
%%%% Questions
\begin{reportBlock}{Questions posées par l’enseignant (avec réponses)}
  \textbf{C : Vous nous avez parlé du potentiel de marée. Je n'ai pas compris le lien entre le champ $\mathbf{\mathcal{C}}$ et la rotation de la Terre.}  \textcolor{purple}{Du fait que la Terre tourne sur elle-même, il faut prendre en compte la force d'inertie d'entraînement qui donne le terme $\sum_{p}\mathbf{G}_{p}(T)$ dans l'expression du champ de marée sur Terre. % En raison de la rotation de la Terre autour de l'axe passant pas ses pôles, chaque point à la surface de la Terre passe par deux marées basses et deux marées hautes par jour. C'est un modèle approximatif car il existe des endroits sur Terre où les marées ne sont pas au nombre de 4 par jours. Mais c'est plus compliqué, il faut prendre en compte l'orientation avec le soleil, la profondeur de l'eau, etc ...Pour un même point M sur Terre, ce point M est aligné avec la lune et le soleil tous les 29jours12h44min (c'est ce qu'on appelle la lunaison). La Terre elle fait une révolution toutes les 24h. Si on imagine que la lune est au zénith à une certaine heure T d'un point M sur Terre, le jour d'après la lune sera légèrement décalée d'un angle $\theta=360\frac{T_{rot}}{T_{rev}}\sim13^{\circ}$
  }.\newline
  
  \textbf{C : Quel est le lien entre les marées et l'eau ?}  \textcolor{purple}{C'était pour expliquer le fait qu'on ait des marées hautes et basses sur Terre.}\newline
  
  \textbf{C : Mais ça s'applique sur tout, à la fois l'eau, la terre, les hommes.}  \textcolor{purple}{La croûte terrestre aussi peut subir une force de marée mais moins perceptible à l'\oe uil humain. }\newline
  
  \textbf{C : Vous modélisez la terre comme un fluide ?}  \textcolor{purple}{ça dépend sur quelle durée d'observation on se place. \'{A} l'échelle des temps géologiques, la Terre est effectivement un fluide}\newline
  
  \textbf{C : Est-ce qu'il y a aussi un effet de marée de la Terre sur la Lune ?}  \textcolor{purple}{Oui. Et c'est grâce à cet effet que la rotation s'est synchronisée et qu'on voit toujours la même face.}\newline
  
  \textbf{C : Connaissez-vous un autre exemple notable des champs de marées sur un satellite autour d'une autre planète ?}  \textcolor{purple}{Io, un satellite de Jupiter, présente une activité volcanique importante du fait des compressions et dilatations importantes du manteau générées par les forces de marées. }\newline
  
  \textbf{C : Je n'ai pas compris l'orgine du second lobe de l'autre côté de la Lune ? Comment tu l'expliques}  \textcolor{purple}{Le terme de marée est un terme différentiel. La force de marée créée par la lune en un point M de la terre a pour origine la différence entre le champ de gravitation créé par la lune au point M de la terre et le champ de gravitation créé par la lune au centre de la terre. Dans l'expression du champ de marée, on voit qu'il est maximal en $\theta=0$ et $\theta=\pi$. Dans le premier cas, le champ est dirigé suivant $\mathbf{e_r}=\mathbf{e_x}$ et dans le second cas $\mathbf{e_r}=-\mathbf{e_x}$.}\newline
  
  \textbf{C : Vous avez introduit deux visions différentes entre Aristote et Galilée. Une façon de déterminer qui a raison et qui a tort ?}  \textcolor{purple}{Galilée s'est posé la question de savoir qu'est-ce qui se passerait si on faisait une expérience de chute libre dans le vide. Il a fait des expériences dans différents fluides pour voir l'effet de la viscosité pour mais il avait pas le moyen de le faire à l'époque.}\newline
  
  \textbf{C : On a les moyens de le faire aujourd'hui ? Avez-vous une vidéo d'expérience spectaculaire faite ?}  \textcolor{purple}{https://www.sciencesetavenir.fr/fondamental/video-qu-est-ce-qui-tombe-le-plus-vite-une-plume-ou-une-boule-de-billard\_23185. Franchement incroyable.}\newline
  
  \textbf{C : Unité de $G$ ?}  \textcolor{purple}{Analyse dimensionnelle donne $kg^{-1}.m^3.s^{-1}$}\newline
    
  \textbf{C : Rappelez l'expression du champ électrique en fonction des potentiels.}  \textcolor{purple}{$\mathbf E = - \nabla V - \frac{\partial \mathbf A}{\partial t}$}\newline
    
  \textbf{C : $A$ a-t-il une analogie en mécanique ?}  \textcolor{purple}{Si on a un flux de masse, on peut définir l'équivalent du champ $\mathbf B$.}\newline
    
  \textbf{C : Peut-on mesurer son effet ? A-t-on un ordre de grandeur ? Quelle serait la correction dans la force ?}  \textcolor{purple}{Voir Effet Lense-Thirring. C'est une correction extrêmement faible qui nécessite des masses importantes et une vitesse des corps relativiste pour qu'elle soit observable. À titre d'exemple, le pendule de Foucault devrait osciller environ 16000 ans avant de précesser de 1 degré (cf. Wikipédia).}\newline
    
  \textbf{C : Une expérience de pensée qui permettrait de mettre ça en évident ?}  \textcolor{purple}{Un tube dans lequel je fais passer des billes à vitesse constante dans le vide, il faudrait regarder l'effet que ça fait sur une bille suspendue sur un ressort.}\newline
    
  \textbf{C : Si on place une personne à l'intérieur de la terre, qu'est-ce qu'elle ressent ?}  \textcolor{purple}{Elle serait en apesanteur au centre de la terre.}\newline
    
  \textbf{C : Et au centre d'une étoile à neutron ?}  \textcolor{purple}{Au centre, en apesanteur. Par contre les forces exercées en chaque point du corps, sont très différentes et on est écartelé.}\newline
  
  \textbf{C : Dans les limites de l'analogie, vous avez dit que l'interaction électrostatique était plus grand que l'interaction gravitationnelle. Pouvez-vous quantifier ça ?}  \textcolor{purple}{Si on considère deux électrons : $\frac{F_G}{F_e} = \frac{G 4 \pi \epsilon_0 m_e^2}{e^2} \sim 10^{-44}$} \newline

  \textbf{C : Expérience : quel théorème utilisé pour l'équation ?}  \textcolor{purple}{Principe fondamental de la dynamique dans le référentiel terrestre}\newline

  \textbf{C : Et les forces d'inertie ?}  \textcolor{purple}{On veut juste $g$, c'est un $g$ effectif qui inclus l'accélération d'entraînement.}\newline

  \textbf{C : Vous avez mesuré ce $g$ effectif ? Donc vous avez pris en compte les forces d'inertie d'entraînement.}  \textcolor{purple}{Oui.}\newline

  \textbf{C : Et de coriolis ?}  \textcolor{purple}{Non car négligeable : $F_c = 2 m \Omega_T v_{pendule}$, $F_e \sim m \Omega_T^2 T_T$ d'où $\frac{F_e}{F_c} \sim \frac{\omega_T}{2 v_{pendule}} \sim 10^6 $}\newline

  \textbf{C : Système ?}  \textcolor{purple}{masse et tige. J'ai négligé la masse de la tige, a peu près ok car facteur 10 entre les deux.}\newline

  \textbf{C : La mesure que vous avez faîtes est indépendante de la masse ?}  \textcolor{purple}{Pas tout à fait vrai. Si je voulais faire une mesure propre, il aurait fallu que je détermine le moment d'inertie du système total.}\newline

  \textbf{C : Slide du champ de pensanteur terrestre. Réexpliquez ce qu'est $g_0$}  \textcolor{purple}{Le rayon terrestre n'est pas homogène partout sur la terre donc variation de $g_0$ en fonction de la latitude.}\newline
 
  \textbf{C : et $g$ ?}  \textcolor{purple}{Correction due à l'accélération d'entraînement.}\newline 
  
  \textbf{C : Comment vous définiriez le poids pour une classe de 1ere année sup ?}  \textcolor{purple}{Force de réaction c'est  l'opposé du poids. Une bonne mesure du poids est l'utilisation d'une balance qui, lorsqu'il y a équilibre, donne la force de réaction du support de norme égale au poids.}\newline 

  
\end{reportBlock}


%%%%%%%%%%%%%%%%%%%%%%%%%%%%%%%%%%%%%%%%%%%%%%%%%%%%
%%%% Commentaires
\begin{reportBlock}{Commentaires lors de la correction de la leçon}

Les choix faits sont intéressants. En revanche, il y a un manque de continuité dans la leçon et dans les transitions. Le calcul du champ gravitationnel était bien maîtrisé. Vu ton intro, je m'attendais à ce que tu partes de ton expérience puis que tu discutes la dépendance en la latitude et que tu dises qu'il manque des choses puis partir sur ça.

\end{reportBlock}



%%%%%%%%%%%%%%%%%%%%%%%%%%%%%%%%%%%%%%%%%%%%%%%%%%%%
%%%% Correction
\begin{reportBlock}{Partie réservée au correcteur}
  \textbf{Avis général sur la leçon (plan, contenu, etc.) Le plan est raisonnable. Pour une leçon docteure, je m'attendrais plus à partir des expériences et aller vers la modélisation.\\:}
  
  
  \textbf{Notions fondamentales à aborder, secondaires, délicates :} À aborder en priorité: la gravitation, la pesanteur, notion de référentiel, champ gravitationnel, Kepler \\
  
  Notions possibles: l'égalité masse grave masse inerte; l'altitude des satellites selon leur vitesse angulaire et en particulier pour la Terre la différence d'altitude entre satellite géostationnaire et l'ISS par exemple; de l'astrophysique (galaxie, amas, Jupiter, trou noir, exoplanète, etc.) \\
  
  Notion importante mais subtile: les effets de marées. Il faut savoir différencier ce qui vient de la rotation propre de la Terre et de sa révolution autour du Soleil. On peut parler de limite de Roche, de la façon dont se créent les anneaux d'une planète (Saturne bien sûr, mais aussi Jupiter, trou noir, etc.). Les marées océaniques bien sûr, mais c'est un problème très complexe avec pleins d'effets subtils. À aborder avec précaution. \\
  
  \textbf{Ne pas oublier de contextualiser la leçon}, et ici il y avait un boulevard: pleins de prix Nobel récents (2020 trous noirs, 2019 exoplanètes, 2017 ondes gravitationnelles), parler d'autres satellites (Io, Europe), parler des satellites du GPS, de Starlink, du James Webb telescope lancé récemment sur un point de Lagrange, etc.\\
  
  
  
  \textbf{Expériences possibles (en particulier pour l'agrégation docteur) :} Pendule, balle de ping pong, gyroscope, viscosimètre de Stokes, ondes de surface (régime gravitaire)
  
  
  \textbf{Bibliographie conseillée : }Le Bocquet, Faroux, Renault, \emph{Toute la mécanique}, est très complet sur la partie effet de référentiel non galiléen sur la Terre, tout en restant au niveau prépa. Je recommande.
\end{reportBlock}


\newpage
%%%%%%%%%%%%%%%%%%%%%%%%%%%%%%%%%%%%%%%%%%%%%%%%%%%%
%%%% En-tête leçon
\begin{headerBlock}
  \chapter{Lois de conservation en mécanique}
    \label{LP_LoisConservation}
\end{headerBlock}

%%%%%%%%%%%%%%%%%%%%%%%%%%%%%%%%%%%%%%%%%%%%%%%%%%%%
%%%% Références
\begin{center}
\begin{tabularx}{\textwidth}{| X | X | c | c |}
  \hline
  \rowcolor{gray!20}\multicolumn{4}{c}{Bibliographie de la leçon : } \\
  \hline 
  Titre & Auteurs & Editeur (année) & ISBN \\
  \hline
  Mécanique 1 & H. Gié et J.-P. Sarmant & Tec\&Doc (1984) & \\
  \hline
  Mécanique 2ème année & H. Gié et J.-P. Sarmant & Tec\&Doc (1996) & \\
  \hline
\end{tabularx}
\end{center}

\begin{reportBlock}{Commentaires des années précédentes :}
    \begin{itemize}
        \item \textbf{2017 :} Des exemples concrets d’utilisation des lois de conservation sont attendus,
        \item \textbf{2016 :} Lors de l’entretien avec le jury, la discussion peut aborder d’autres domaines que celui de la mécanique classique,
        \item \textbf{2015 :} Cette leçon peut être traitée à des niveaux très divers. L’intérêt fondamental des lois de conservation et leur origine doivent être connus et la leçon ne doit pas se limiter à une succession d’applications au cours desquelles les lois de conservation se résument à une propriété anecdotique du problème considéré.
    \end{itemize}
\end{reportBlock}

%%%%%%%%%%%%%%%%%%%%%%%%%%%%%%%%%%%%%%%%%%%%%%%%%%%%
\begin{reportBlock}{Plan détaillé}
  \textbf{Niveau choisi pour la leçon :} CPGE 2ème année
  \newline
  \textbf{Prérequis : }
  \begin{itemize}
      \item Lois de la mécanique classique,
      \item Définition énergie mécanique, moment cinétique, moment d'une force, quantité de mouvement
      \item Système fermé
  \end{itemize}


\section*{Introduction}
Les principes de conservation sont nombreux en physique : conservation de la charge en électromagnétisme, conservation de la masse ou de l'énergie pour un système thermodynamique fermé (1er principe). Nous allons ici nous restreindre aux principes de conservations en mécanique newtonienne.\\

En mécanique, nous connaissons trois lois d'évolution :
\begin{enumerate}
    \item le \underline{\textbf{théorème de la résultante cinétique :}} $\frac{d\mathbf{p}}{dt}=\sum_{i}\mathbf{F_i}=\mathbf{R}$,
    \item le \underline{\textbf{théorème du moment cinétique :}} $\frac{d\mathbf{L_O}}{dt}=\sum_{i}\mathbf{M_{O,i}}$,
    \item le \underline{\textbf{théorème de la puissance mécanique :}} $\frac{dE_m}{dt}=\sum_{i}P_i$.
\end{enumerate}

\textcolor{red}{Transition :}Les termes $\sum_{i}\mathbf{F_i}$, $\sum_{i}\mathbf{M_{O,i}}$ et $\sum_{i}P_i$ jouent le rôle de \textit{sources} pour les grandeurs $\mathbf{p}$. On va s'intéresser aux relations de conservation qui découlent de l'annulation de ces sources 

\section{Conservation de la quantité de mouvement}

\subsection{Système isolé}
\textcolor{green}{Définition :} Système fermé qui ne subit aucune action extérieure. \\
En pratique cela n'arrive jamais car il y a toujours des forces gravitationnelles mais fondamental d'un point de vue conceptuel. L'Univers est par définition un système isolé donc tout sous-système A de l'Univers qui reçoit de la quantité de mouvement $\Delta \mathbf{P_A}$ d'un sous-système B avec lequel il intéragit, ce dernier a dû lui céder une quantité de mouvement $\Delta \mathbf{P_B}$ telle que :
\begin{equation}
    \Delta(\mathbf{P_A}+\mathbf{P_B}= 0 \Leftrightarrow \Delta\mathbf{P_A} = -\Delta\mathbf{P_B}
\end{equation}
Exemple de la diffusion de Rutherford où il y a un transfert de la quantité de mouvement.
\subsection{Système pseudo-isolé}
\textcolor{green}{Définition :} un système est dit pseudo-isolé s'il est soumis à des actions extérieures qui se compensent c'est-à-dire dont la résultante $\mathbf{R}$ est nulle.
\subsection{Collision élastique}
\textcolor{blue}{Expérience : }Conservation de $\mathbf{p}$ de mobiles autoporteurs.

\section{Conservation du moment cinétique}

\subsection{TMC}
\subsection{Système pseudo-isolé}
\subsection{Cas d'une force centrale}

\section{Conservation de l'énergie}

\subsection{TEC}
\subsection{Forces conservatives}
\subsection{Cas du pendule simple}

\section*{Conclusion}
Ouverture sur les théorèmes de N\oe ther (Emmi N\oe ther, mathématicienne allemande (1882-1935):
\begin{enumerate}
    \item La conservation de la quantité de mouvement est une conséquence de \textbf{l'invariance des lois de la physique par translation spatiale}, c'est-à-dire de l'homogénéité de l'espace,
    \item  La conservation du moment cinétique est une conséquence de \textbf{l'invariance des lois physiques par rotation}, c'est-à-dire de l'isotropie de l'espace,
    \item La conservation de l'énergie est une conséquence de \textbf{l'invariance des lois physiques par translation temporelle}, c'est-à-dire de l'homogénéité du temps.
\end{enumerate}

\end{reportBlock}
\newpage
%%%%%%%%%%%%%%%%%%%%%%%%%%%%%%%%%%%%%%%%%%%%%%%%%%%%
%%%% En-tête leçon
\begin{headerBlock}
  \chapter{Notions de viscosité d'un fluide. Ecoulement visqueux.}
  \label{LP_Viscosite} 
\end{headerBlock}




%%%%%%%%%%%%%%%%%%%%%%%%%%%%%%%%%%%%%%%%%%%%%%%%%%%%
%%%% Références
\begin{center}
\begin{tabularx}{\textwidth}{| X | X | c | c |}
  \hline
  \rowcolor{gray!20}\multicolumn{4}{c}{Bibliographie de la leçon : } \\
  \hline 
  Titre & Auteurs & Editeur (année) & ISBN \\
  \hline
  Hydrodynamique physique & Guyon, Hulin, Petit & EDP Sciences & \\
  \hline 
  Poly de cours & Marc Rabaud & &    \\
  \hline 
  Physique Spé PC/PC* & S. Olivier & Tec\&Doc (2000) &    \\
  \hline 
\end{tabularx}
\end{center}

%%%%%%%%%%%%%%%%%%%%%%%%%%%%%%%%%%%%%%%%%%%%%%%%%%%%

%%%%%%%%%%%%%%%%%%%%%%%%%%%%%%%%%%%%%%%%%%%%%%%%%%%%
%%%% Plan
\begin{reportBlock}{Plan détaillé}

  \textbf{Niveau choisi pour la leçon :} CPGE 2ème année
  \newline
  \textbf{Prérequis} : \begin{itemize}
  \item cinématiques des fluides
      \item loi de l'hydrostatique, poussée d'Archimède
      \item diffusion
  \end{itemize}

  \begin{reportBlock}{Commentaires des années précédentes :}
    \begin{itemize}
        \item \textbf{2017 :} Il peut être judicieux de présenter le fonctionnement d’un viscosimètre dans cette leçon,
        \item \textbf{2016 :} Le jury invite les candidats à réfléchir d’avantage à l’origine des actions de contact mises en jeu entre un fluide et un solide,
        \item \textbf{2014, 2013, 2012, 2011 :} L’exemple de l’écoulement de Poiseuille cylindrique n’est pas celui dont les conclusions sont les plus riches. Les candidats doivent avoir réfléchi aux différents mécanismes de dissipation qui peuvent avoir lieu dans un fluide. L’essentiel de l’exposé doit porter sur les fluides newtoniens : le cas des fluides non newtoniens, s’il peut être brièvement mentionné ou résenté, ne doit pas prendre trop de temps et faire perdre de vue le message principal.
    \end{itemize}
\end{reportBlock}

  \textbf{Déroulé détaillé de la leçon: }  
  
  \section*{Introduction}
  On a décrit dans un cours précédent l'équation du mouvement pour un fluide parfait. Hors, il existe beaucoup de situation ou le modèle du fluide parfait ne permet pas d'expliquer le comportement de l'écoulement du fluide. \textcolor{blue}{Manip qualitative : cylindre pour l'expérience de Taylor-Couette.} Brancher une alim stabilisée +12V-12V au moteur. Serrer la vis en bas du cylindre pour libérer le mouvement du cylindre extérieur. On voit la diffusion de la quantité de mouvement qui est impossible à expliquer dans le cadre du modèle du fluide parfait. Il y a ce qu'on appelle une force de viscosité qui va diffuser la quantité de mouvement du fluide. C'est ce qu'on va voir dans cette leçon.

  \section{Notion de viscosité}
cf Chap 2 Guyon Hulin Petit.
\subsection{Viscosité de cisaillement}
Voir Tec\&Doc p418. On considère un écoulement de la forme $\mathbf{v}=v_x(z)$ dans un fluide. On considère un élément de surface $\mathbf{dS}=dxdy\mathbf{\hat{u_z}}$ d'ordonnée z séparant le fluide situé au-dessus de z et en dessous de z. L'action de contact, appelée \textcolor{green}{force de viscosité}, exercée par le fluide situé au-dessus de z sur le fluide situé en-dessous de z est tangentielle à la vitesse de l'écoulement :
\begin{equation}
    d\mathbf{F_t} = \eta\partialD{v_x}{z}dS\mathbf{\hat{u_x}}
\end{equation}
où on introduit le coefficient de viscosité cinématique $\eta$ de dimension Pa.s$^{-1}$ comme on peut le voir avec la formule. \textcolor{green}{Donner quelques exemples de $\eta$ sur slide (miel et eau)}. 
\textbf{Remarques :}
\begin{itemize}
    \item Plus $\eta$ est grand plus la force de viscosité de cisaillement est élevée (c'est pour ça qu'on dit que le miel est plus visqueux que l'eau),
    \item force non nulle si champ de vitesse inhomogène
    \item le sens de la force est tel qu'il tend à homogénéiser le champ des vitesses : la couche de fluide du dessus qui va plus vite met en mouvement la couche de dessous : \textbf{il y a une diffusion de la quantité de mouvement} des couches du fluide les plus rapides vers les couches les moins rapides.
\end{itemize} 

  \subsection{Equivalent volumique de la force de viscosité}
Considérons un pavé élementaire de volume $d\tau=dxdydz$ en prenant le même champ de vitesse que précédemment. Un bilan des forces (sans la force de pression normale aux surfaces) conduit à :
\begin{itemize}
    \item $d\mathbf{F}_{z+dz}=\eta\partialD{v_x}{z}(z+dz)dxdy\mathbf{u_x}$ car le fluide sur la couche de dessus va plus vite,
    \item  $d\mathbf{F}_{z}=-\eta\partialD{v_x}{z}(z)dxdy
    \mathbf{u_x}$ car le fluide sur la couche de dessous va moins vite.
\end{itemize}
Un développement de Taylor à l'ordre 1 permet d'écrire la résultante comme :
\begin{align}
    d\mathbf{F} &= \eta\partialD{^2v_x}{d_z^2}d\tau\mathbf{u_x} \\
    \mathbf{f}_{visc}&=\frac{d\mathbf{F}}{d\tau} = \eta\mathbf{\Delta v}
\end{align}
\textbf{La dernière formulation est valable pour les écoulements incompressibles !} C'était une hypothèse dont on se passait pour les fluides parfaits.\\
\textbf{Remrque :} On a l'analogie avec les phénomènes de transport diffusif de la chaleur $\lambda\Delta T$ et de la quantité de matière $D\Delta n$. 


  %\subsection{Tenseur des déformations}
  %Voir Guyon Hulin Petit Chap 3 p125.
  
  \textcolor{red}{Transition :} Maintenant qu'on a vu la force volumique de viscosité pour un fluide visqueux, on peut l'injecter dans les équations du mouvement du champ de vitesse eulérien du fluide.

  \section{Dynamique d'un fluide visqueux}

  \subsection{Equation de Navier-Stokes et nombre de Reynolds}
  Le principe fondamental de la dynamique dans un référentiel galiléen appliquée à une particule fluide suivant le champ de vitesse eulérien $\mathbf{v}$ du fluide s'écrit désormais :
  \begin{equation}
      \left(\partialD{\mathbf{v}}{t} + (\mathbf{v}\cdot\grad)\mathbf{v}\right) = -\grad P + \rho \mathbf{g} + \nu\Delta\mathbf{v} + \mathbf{f}_{vol}
  \end{equation}
  où on a introduit ici la viscosité cinématique $\nu=\frac{\eta}{\rho}$. C'est l'équation de Navier-Stokes.\\
  Donner les coefficients de viscosité cinématique de fluides (eau-air-silicone-poix(?)) homogène à un coefficient de diffusion. voir Tec\&Doc p423. Parler de l'expérience de la goutte de poix (cf Wikipédia) ?\\
  
  \textbf{Remarques :}comme l'équation d'Euler, cette équation est non-linéaire donc très difficile à résoudre numériquement ou analytiquement (problème du millénaire en maths). Il faut donc comparer les termes entre eux pour pouvoir simplifier possiblement les équations.\\
  
  C'est possible par exemple (les autres sont dans le cours de Marc Rabaud) en introduisant le nombre de Reynolds qui compare l'accélération convective au terme de viscosité :
  \begin{equation}
      Re = \frac{\lVert (\mathbf{v}\cdot\grad)\mathbf{v}\rVert}{\lVert\nu\Delta\mathbf{v}\rVert}
  \end{equation}
  Dans un problème possédant une échelle spatiale unique L, une vitesse d'écoulement typique U, le nombre de Reynolds devient :
  \begin{equation}
      Re = \frac{UL}{\nu}
  \end{equation}
  On décrit les limites $Re>>1$ et $Re<<1$. On va voir qu'il est possible de mesurer la viscosité dans le cas ou le nombre de Reynolds est très petit devant 1 et qu'on a un écoulement stationnaire.
  
  \subsection{Viscosimètre à chute de bille : force de Stokes}
  Lorsquon fait tomber une bille de rayon r dans un fluide, on constate phénoménologiquement (on eut la retrouver par une analyse dimensionnelle) qu'il existe une force dites \textcolor{green}{force de traînée} s'opposant à la vitesse dont l'expression pour les vitesses faibles est donnée par (voir HPrépa p153 exercice 6) :
  \begin{equation}
      \mathbf{F} = -6\pi\eta r\mathbf{v}
  \end{equation}
  
\textcolor{blue}{Manip quantitative : mesure de la viscosité du silicone} :
\begin{itemize}
    \item écran lumineux,
    \item cylindre contenant une huile de silicone $\nu=500$~cst = $5.00\times10^{-4}$~m$^2$/s,
    \item une caméra ultra-rapide,
    \item un ordi avec le logiciel Tracker,
    \item des billes de métal de différents diamètres
\end{itemize}
Incertitude : faire varier un peu la longueur et fitter x=f(t) pour avoir différentes vitesses + incertitude sur le pointage : prendre quelques pixels (moitié de la sphère).\\
Bilan des forces : poids, poussée d'Achimède, force de traînée :
\begin{align*}
    \mathbf{\Pi} + \mathbf{F_t} + \mathbf{P} &= 0 \\
    6\pi\eta R v_{\infty} +\frac{4}{3}\pi R^3\left(\rho_{silicone}-\rho_{acier}\right)g &= 0 \\
    v_{\infty} &= \frac{2}{9}\frac{\left(\rho_{acier}-\rho_{silicone}\right)g}{\eta}R^2
\end{align*}
On trace $v_{\infty}=f(R^2)$, on trace également le Reynolds. On montre que la loi de Stokes n'est valable que pour les petites sphères et qu'effectivement on tend vers la valeur de $\nu=500$~m$^2$/s\\

Introduire le coeffcient de traînée $C=\frac{F_t}{\frac{1}{2}\rho Sv^2}$ (dépend de la géométrie) et montrer la courbe C=f(Re) p325 Dunod 2019.\\

Réversibilité cinématique : 16min01\url{https://www.youtube.com/watch?v=51-6QCJTAjU&list=PL0EC6527BE871ABA3&index=8}

\subsection{Notion de couche limite}
  \subsection{Transport diffusif de la quantité de mouvement}


\subsection{Aspect énergétique}
Fluide visqueux = dissipation d'énergie. Démo p265 Guyon Hulin Petit.
\subsection{Conditions aux limites cinématiques et dynamiques}
Tableau de comparaison fluides parfaits fluide réels.



\end{reportBlock}
\newpage
%%%%%%%%%%%%%%%%%%%%%%%%%%%%%%%%%%%%%%%%%%%%%%%%%%%%
%%%% En-tête leçon
\begin{headerBlock}
  \chapter{Modèle de l'écoulement parfait d'un fluide}
    \label{LP_EcoulementParfait}
\end{headerBlock}

%%%%%%%%%%%%%%%%%%%%%%%%%%%%%%%%%%%%%%%%%%%%%%%%%%%%
%%%% Références
\begin{center}
\begin{tabularx}{\textwidth}{| X | X | c | c |}
  \hline
  \rowcolor{gray!20}\multicolumn{4}{c}{Bibliographie de la leçon : } \\
  \hline 
  Titre & Auteurs & Editeur (année) & ISBN \\
  \hline
  Hydrodynamique & Guyon, Hulin, Petit &  & \\
  \hline
  Tout-en-un PC/PC* & M.-N. Sanz & Dunod (2019) & \\
  \hline
  Wikipédia & Il y a pas mal de choses cool & & \\
  \hline
  Mécanique 2ème année & H. Gié et J.-P. Sarmant & Tec\&Doc (1996) & \\
  \hline
  Physique-Chimie PSI/PSI* & Pascal Olive & Ellipses (2022) & \\
  \hline
\end{tabularx}
\end{center}

\begin{reportBlock}{Commentaires des années précédentes :}
    \begin{itemize}
        \item \textbf{2017 :} La multiplication des expériences illustrant le théorème de Bernoulli n’est pas souhaitable, surtout si celles-ci ne sont pas correctement explicitées,
        \item \textbf{2016 :} Les limites de ce modèle sont souvent méconnues,
        \item \textbf{2015 :} Le jury invite les candidats à réfléchir davantage à l’interprétation de la portance et de l’effet Magnus. Les exemples cités doivent être correctement traités, une présentation superficielle de ceux-ci n’étant pas satisfaisante,
        \item \textbf{2014 2013, 2012, 2011 :} La notion de viscosité peut être supposée acquise.
    \end{itemize}
\end{reportBlock}

%%%%%%%%%%%%%%%%%%%%%%%%%%%%%%%%%%%%%%%%%%%%%%%%%%%%
\begin{reportBlock}{Plan détaillé}
  \textbf{Niveau choisi pour la leçon :} CPGE 2ème année
  \newline
  \textbf{Prérequis : }
  \begin{itemize}
      \item Viscosité, équation de Naviers-Stokes, equation de l'hydrostatique
      \item calcul de l'accélération convective dans un écoulement parallèle
      \item Champ eulérien, lignes de champ, lignes de courant
  \end{itemize}


\section*{Introducion}
Les équations de la mécanique des fluides décrivent l'évolution du champ eulérien de vitesse dans le référentiel d'un observateur fixe. Dans certains cas comme un écoulement d'eau sous un pont, ou un écoulement d'air sur une aile d'avion (\textcolor{green}{Montrer sur slide étude IFREMER sur l'effet d'un écoulement turbulent sur une hydrolienne}), il se créé des phénomènes de vorticité/turbulences lié aux effets de viscosité des fluides. Parler de ressaut hydrolique ? Dans le cadre de cette leçon, on va se placer loins de ces zones complexes c'est-à-dire en amont ou loin en aval de l'écoulement.

\section{Modèle de l'écoulement parfait}

\subsection{Equation d'Euler}
\textcolor{green}{Slide : équation de Navier-Stokes}.\\

\textcolor{red}{Modèle de l'écoulement parfait : $\eta=0$ dans Navier-Stokes.}\\
\textbf{Remarques :}
\begin{enumerate}
    \item Si fluide au repos, on retrouve l'équation de l'hydrostatique,
    \item Equation toujours non linéaire donc extrèmement difficile à résoudre (problème du Millénaire avec l'équation de Navier-Stokes),
    \item Dans les cas pratiques que nous verrons plus loin, on peut considérer l'écoulement parfait si $\eta\mathbf{\Delta v}<<$ autres termes de l'équation. Par exemple, si $Re = `\frac{\lvert \mathbf{v}\cdot\grad \rvert}{\lvert \nu\mathbf{\Delta v}\rvert}>>1$
\end{enumerate}


\subsection{Conséquence : effet de la courbure des lignes de courant}
Voir Dunod. Prendre une vitesse orthoradiale, montrer que $\partialD{P}{r}>0$. Faire le dessin d'une balle soumise à un flux d'air. L'air va adhérer à la surface par les intéractions moléculaires \textcolor{blue}{Manip qualitative : Montrer l'effet Coanda avec la balle de ping-pong. On peut aussi faire avec une cuillère et un filet d'eau, la cuillère est attirée vers le filet d'eau qui impose un gradient de pression de la surface de la cuillère vers l'extérieur.}

\textcolor{red}{Transition : Puisqu'on a annulé les effets de frottements visqueux qui sont dissipatifs, on va voir qu'on a conservation de l'énergie;}

\subsection{Théorème de Bernouilli}
Hypothèses :
\begin{itemize}
    \item Ecoulement parfait et stationnaire
    \item Ecoulement incompressible : $\rho=cste$,
    \item que forces de pression et de pesanteur (dirigée suivant $-\mathbf{\hat{u_z}}$)
\end{itemize}
Pour des hypothèses plus générales, voir H. Gié p190.\\

\textbf{Le long d'une ligne de courant (Daniel Bernouilli (1738):}
\begin{equation}
    \rho\frac{v^2}{2}+P+gz = cste
\end{equation}
Cela constitue une équation de conservation de l'énergie d'un fluide. Savoir qu'on peut le démontrer pour un écoulement compressible, ou potentiel $\grad\wedge\mathbf{v}=0$.\\

\textcolor{red}{Transition : on va voir quelques applications de ce théorème.}

\section{Applications}

\subsection{Les sondes Pitot}
Bien fait dans le Pascal Olive.\\
Du nom d'Henri Pitot (1730) qui voulait mesurer le débit de la Seine.\\

Faire le dessin. Attention, mettre point d'arrêt A à l'intérieur du tube. On a un écoulement parfait hors de la couche limite et il n'y a pas de discontinuité de la pression dans la couche. Cf Guyon-Hulin-Petit p494.\\

\underline{Application Bernouilli le long de la ligne de courant passant par A$_{\infty}$A :}
\begin{equation}
    P_A = P_0 + \frac{1}{2}\mu v_0^2
\end{equation}
\underline{Application Bernouilli le long de la ligne de courant passant par B$_{\infty}$B' :}
\begin{equation}
    P_{0} = P_{B'} = P_B
\end{equation}
comme les lignes de courant sont parallèles à l'axe BB' (cf équation d'Euler.\\

On en déduit la conversion de la différence de pression à l'intérieur du tube et la vitesse de l'écoulement :
\begin{equation}
    v_0 = \sqrt{\frac{2(P_A-P_B}{\mu}}
\end{equation}
\textbf{Remarques :}
\begin{enumerate}
    \item On a négligé les variations d'altitude des points A et A$_{\infty}$, B$_{\infty}$ et B', B' et B par rapport aux variation de vitesse ou de pression : $g\Delta z\sim 0.01-0.1m^2.s^{-2}$ vs $0.5*v_0^2\sim 50m^2.s^{-2}$
    \item On a supposé que le fluide ne rentrait pas dans l'ouverture au niveau de B ce qui n'est pas justifiable dans un modèle d'écoulement parfait. Du fait de la viscosité, il y a une couche limite (de très faible épaisseur $\delta(x)\propto\frac{1}{Re_x}$) qui donne une vitesse nulle au point B. Mais il y a continuité de la pression à la normale à la paroi ce qui permet d'utiliser Bernouilli loin de la couche limite.
\end{enumerate}

\textcolor{red}{Transition : on va vérifier expérimentalement le théorème de Bernouilli et voir qu'on peut en déduire la masse volumique de l'air.}

\subsection{Mesure de la masse volumique de l'air}

\textcolor{blue}{Manip quantitative :} utilise une soufflerie, un anémomètre à fil chaud et un tube de Pitot relié à un manomètre différenciel lui même relié à un voltmètre METRIX. Il faut une alimentation continue +12V pour le manomètre et bien mesurer la tension à pression atmosphérique ($\sim 2.266V$).\\

Attention au sens de l'anémomètre à fil chaud, la flèche doit être dans le sens de l'écoulement.\\

On représente $P_A-P_B = f(v^2/2)$ et on obtient une droite de coefficient $\mu_{air}(T)=1.43(14)$~kg.m$^{-3}$. A comparer à celle attendue à cette température (cf Wikipédia). Commenter les incertitudes, le Z-score, l'anémomètre (précision : $0.03m/s + 5\% valeur moyenne$ ce qui est important !).

\subsection{Effet Venturi}
Une autre application du théorème de Bernouilli est l'effet Venturi. Faire le schéma.\\

Equation d'incompressibilité :
`\begin{equation}
    v_A = v_B\frac{s}{S} = v_C
\end{equation}
Théorème de Bernouilli le long de ABC :
\begin{equation}
    P_A - P_B = \frac{1}{2}\left(v_B^2-v_A^2\right) = \frac{1}{2}v_A^2\left(\frac{S^2}{s^2}-1\right)
\end{equation}
qui relie différence de pression et géométrie du dispositif par l'intermédiaire du débit. On peut donc mesurer le débit !
Montrer qu'il y a des pertes de charges car le niveau en C n'est pas le même qu'en A, il faut prendre en compte les effets de viscosité pour être vraiment précis mais au premier abord ça marche bien.


\section*{Conclusion}
Conclure sur l'hélium liquide. \textcolor{green}{Photo sur slide ou vidéo youtube à partir de 1min :} \url{https://www.google.com/search?client=firefox-b-d&q=helium+superfluide#fpstate=ive&vld=cid:fcf88f2b,vid:2Z6UJbwxBZI}.\\

Ouvrir sur le paradoxe de d'Alembert : si pas de viscosité, pas de résistance au mouvement d'une plaque ou d'une aile d'avion : une aile d'avion ne peut pas voler ? Hé non car viscosité induit un décollement des lignes de champs qui implique une force de trainée.

\end{reportBlock}
\newpage
%%%%%%%%%%%%%%%%%%%%%%%%%%%%%%%%%%%%%%%%%%%%%%%%%%%%
%%%% En-tête leçon
\begin{headerBlock}
  \chapter{Phénomènes interfaciaux impliquant des fluides}
    \label{LP_PhenomenesInterfaciaux}
\end{headerBlock}




%%%%%%%%%%%%%%%%%%%%%%%%%%%%%%%%%%%%%%%%%%%%%%%%%%%%
%%%% Références
\begin{center}
\begin{tabularx}{\textwidth}{| X | X | c | c |}
  \hline
  \rowcolor{gray!20}\multicolumn{4}{c}{Bibliographie de la leçon : } \\
  \hline 
  Titre & Auteurs & Editeur (année) & ISBN \\
  \hline
 Gouttes, bulles, perles et ondes & David Quéré, Françoise Brochard-Wyart et Pierre-Gilles de Gennes & \'Editions Belin (2002) &    \\
  \hline 
 Hydrodynamique physique & \'Etienne Guyon, Jean-Pierre Hulin et Luc Petit  &  CNRS ÉDITIONS (3ème édition) &    \\
  \hline 
  Capillarité (cours) & P. Lidon  & \url{https://cel.hal.science/cel-01332274} &  \\
  \hline
Notes de cours sur les fluides (2019-2020) & Marc Rabaud  &  Site agreg Montrouge  &    \\
  \hline 
 Why is surface tension a force parallel to the interface? & Marchand, A., Weijs, J. H., Snoeijer, J. H., \& Andreotti, B   & American Journal of Physics (2011)  &    \\
  \hline
\end{tabularx}
\end{center}

%%%%%%%%%%%%%%%%%%%%%%%%%%%%%%%%%%%%%%%%%%%%%%%%%%%%

%%%%%%%%%%%%%%%%%%%%%%%%%%%%%%%%%%%%%%%%%%%%%%%%%%%%
%%%% Plan
\begin{reportBlock}{Plan détaillé}

  \textbf{Niveau choisi pour la leçon :} License 3
  \newline
  \textbf{Prérequis} : \begin{itemize}
      \item Forces, travaux de force, énergie
      \item Principe de travaux virtuels
      \item Equation de l'hydrostatique
  \end{itemize}

  
  \textbf{Déroulé détaillé de la leçon: }   \newline
  
  \section*{Introduction (3min)}
L'étude des phénomènes interfaciaux permet de répondre à un certain nombre de questions comme : pourquoi est-ce que insectes marchent sur l'eau, pourquoi les gouttes et les bulles ont la formes qu'elles ont, ou encore pourquoi en TP de chimie, lorsqu'on met un liquide dans un tube, on voit la formation d'un ménisque ?
  \section{Tension de surface}
  \subsection{Définition}
  \textcolor{purple}{Expérience qualitative films de savon sur des surfaces (cubes) : il faut que le système minimise son énergie de surface.}\\
  Si on augmente la surface d'une interface \textit{dA}, le coût en énergie associé vaut : 
  \begin{equation}
      \delta W = \gamma dA
  \end{equation}
  avec $\gamma$ le coefficient de tension de surface (J.m$^{-2}$). C'est l'énergie qu'il faut fournir pour augmenter la surface d'une interface d'une unité. Ex : pour l'interface eau-air, $\gamma = 72.8$~mJ/m$^2$ à $20^{\circ}$C. 
  
  \subsection{Origine microscopique}
  Une molécule en volume subit des interactions de cohésion de la part de ses voisines qui la stabilisent. Une molécule à l'interface n'a plus de voisines au dessous, cette configuration augmente son énergie. Le fluide ajuste donc sa forme pour exposer le minimum de surface afin de minimiser son énergie.

  La forme minimale pour une goutte ou une bulle en l'absence d'interaction est une sphère (vidéo bulle d'eau dans l'espace : \url{https://www.youtube.com/watch?v=bKk_7NIKY3Y}). 
  
  \subsection{Force capillaire (7min)}
  Le coefficient de tension de surface peut également être vu comme une force par unité de longueur. \\
  Vidéo force capillaire \url{https://www.youtube.com/watch?v=g4c_tj1CccE}.
  Permet de répondre à pourquoi le gerris peut marcher sur l'eau: les pattes hydrophobes du gerris déforment la surface de l'eau qui va chercher à retrouver sa forme en appliquant une force capillaire sur les pattes. Comme l'insecte est suffisamment léger, la force arrive à compenser le poids. 
  
 \subsection{Pression et tension de surface}
 Question : La pression est-t-elle plus grande dans les petites bulles ou dans les grandes bulles ? \\
 \textcolor{purple}{Expérience avec générateur de bulles : la petite bulle est "mangée" par la grosse.}
 \begin{center}
     \includegraphics[scale=0.1]{LP_TensionSurface/Manip_Laplace.jpg}
     
 \end{center}
 CCL : la pression est donc plus grande dans les petites bulles : on peut le démontrer mathématiquement via la \textbf{Loi de Laplace.}
 Pour cela, on applique le principe des travaux virtuels. On imagine que l'on déforme une bulle de $dR$ : \\
 $\delta W = -P_1dV_1-P_2dV_2+\gamma dA =0$ à l'équilibre.\\
 $\ud V_1 = - \ud V_2 = 4 \pi R^2 \ud R$\\
 $dA = d(4\pi R^2) = 8\pi RdR$.\\
 Finalement : 
 \textcolor{red}{Loi de Laplace : }
 \begin{equation}
     (P_1-P_2)=\frac{2\gamma}{R}
 \end{equation}
 \begin{itemize}
     \item Plus $R$ est petit, plus la pression à l'intérieur est grande. 
     \item Lorsque $R \rightarrow \infty$ (surface plane), on a continuité de la pression.
     \item Cette loi nous permet de mesurer $\gamma$ expérimentalement.
 \end{itemize}
 \subsection{Mesure de \text{$\gamma$} (17min)}
 \textcolor{purple}{Expérience :}  Pour une bulle de savon, deux interfaces donc $\Delta P =\frac{4\gamma}{R}$. On génère une bulle, on mesure son rayon et la pression à l'intérieur grâce à un manomètre différentiel de mesurer la différence de pression entre l'intérieur de la bulle et l'extérieur à travers une mesure de tension (piézo). J'ai pris plusieurs points en préparation, je vais en prendre un devant vous. Une regression linéaire $\delta P = \frac{A}{R}$ permet d'obtenir $\gamma$. \\
 \textbf{19min30}\\

La pente de $\Delta P = 4 \gamma\times 1/R$ donne $\gamma=28.8 \pm 2.2$~mN/m$^2$ qui est plus faible que celle de l'eau dû à la présence d'un tensioactif (savon = tensioactif). Par ailleurs, on retrouve le bon ordre de grandeur pour une eau savonneuse (internet donne $25$~mN/m$^2$). \\


  \section{Contact à 3 phases : mouillage (22min)}
  \textcolor{green}{Mouillage :} étude de l'étalement d'une liquide sur un substrat (solide ou liquide). Utile dans l'industrie (peinture, encre, traitement des pneus, crème, maquillage, etc.).\\
  $\theta =$ angle de contact, permet de savoir si un liquide mouille plus ou moins bien un substrat. Résulte d'une compétition entre les tensions de surface intervenant dans les trois interfaces (L/G, L/S, S/G). \\
  \textbf{26min}\\
  Le système est \{\textbf{dl}$\in$ ligne triple\}. 
  $\textbf{dF}=0$ à l'équilibre. La projection sur l'axe du solide donne la \textcolor{red}{Loi de Young-Dupré} :
  \begin{equation}
      \gamma_{LG}\cos(\theta) = \gamma_{SG}-\gamma_{SL}
  \end{equation}
  
  \section{Influence de la gravité}
  On voit que plus la taille de la goutte est importante, plus la goutte est applatie : il y a un effet de gravité qui n'est plus négligeable à partir d'une certaine taille : laquelle ?
  \subsection{Nombre de Bond (30min)}
  Compétition entre gravité et tension de surface : $B_0=\frac{\rho R^2g}{\gamma}$. La longueur capillaire $l_c$ est telle que: $B_0=\frac{\rho l_c^2g}{\gamma}=1$ (frontière entre les deux régime), soit $l_c = \sqrt{\frac{\gamma}{\rho g}}$.
  Pour l'eau, $l_c=3$mm.\\
  Deux régimes : 
  \begin{itemize}
      \item $R>>l_c$, la gravité domine : goutte plate.
      \item $R<<l_c$, la tension de surface domine, la bulle est sphérique.
  \end{itemize}
  Photo ménisque : résulte de cette compétition entre tension de surface responsable de sa formation et gravité qui s'y oppose. Menisque concave pour un fluide mouillant (ex: eau dans tube en verre) et convexe pour un fluide peu mouillant (ex: mercure dans tube en verre)
  \begin{center}
      \includegraphics[scale=0.1]{LP_TensionSurface/Menisque.jpg}
  \end{center}

  \subsection{Ascension dans un tube (34min)}
  Vidéo de l'ascension d'un liquide dans différents tubes capillaires. Plus le tube est fin, plus le liquide monte haute.
  Schéma liquide dans un tube de rayon $r$, rayon de courbure du ménisque $R$, angle de contact $\theta$. On applique la loi de Laplace : $\Delta P = \frac{2\gamma}{R}=\frac{2\gamma\cos(\theta)}{r}$. En appliquant l'équation de l'hydrostatique, on obtient \textcolor{red}{la Loi de Jurin} :
  \begin{equation}
      h = \frac{2\gamma\cos(\theta)}{\rho g r}
  \end{equation}
  En effet, plus $r$ est petit, plus $h$ est grand. Cette loi permet de mesurer $\gamma$.
  
  \section*{Conclusion (40min)}
  Expérience de conclusion : Pince à nourrice dans eau flotte, puis coule avec tensio-actif (savon).



\end{reportBlock}


%%%%%%%%%%%%%%%%%%%%%%%%%%%%%%%%%%%%%%%%%%%%%%%%%%%%
%%%% Questions
\begin{reportBlock}{Questions posées par l’enseignant (avec réponses)}
  \textbf{C : C'est quoi un tensioactif ?} \textcolor{purple}{\'{E}lément qui diminue la tension de surface. Il est composé d'une tête hydrophile et d'une queue hydrophobe. La tête va se mettre à l'interface du côté de l'eau tandis que la queue sera du côté de l'air, faisant le lien entre l'eau et l'air. Cette configuration permet de diminuer l'énergie de l'interface, et donc diminue la tension de surface. } \newline

  \textbf{C : La dynamique de l'ascension dans la loi de Jurin semble diférrente en fonction du diamètre du tube ? Pouvez-vous discuter des éléments à prendre en compte pour essayer de comprendre pourquoi ça monte plus vite ou plus lentement ?} \textcolor{purple}{Plus le rayon est grand, plus la vitesse est grande. Peut se comprendre en ordre de grandeur via $\nabla P = \eta \Delta v$ soir $\frac{P}{h} \sim \eta \frac{v}{r^2}$}
  %En comparant à Poiseuille cylindrique, il y a une dépression à l'interface air/liquide qui fait augmenter brutalement la vitesse. Lorsque le liquide monte, le gradient de pression diminue ($\frac{\delta P}{\delta z}=\frac{2\gamma}{rh}$, r rayon du tube et h hauteur de l'ascension) et en prenant en compte la gravité et les forces de viscosité ($\frac{\delta^2v_r}{\delta z^2}$), ça va ralentir l'ascension.

  \textbf{C : Où placeriez-vous cette leçon dans le cours global de mécanique des fluides ?} \textcolor{purple}{Après l'équation de Navier-Stokes. C'est presque une domaine un peu à part.}

  \textbf{C : Peut-on imaginer qu'on n'ait pas de saut de pression à travers une surface courbe ?} \textcolor{purple}{En utilisant l'équation de Laplace pour une surface quelconque, $\Delta P =\gamma(1/R+1/R')$, R et R' sont les deux rayons de courbures de la surface en un point donné et sont algébriques (compté positivement si le centre de la coubure est à l'intérieur de la surface, négativement sinon. Par exemple, pour un point selle, on a $R' = -R$ impliquant une continuité de la pression.}

  \textbf{C : Dans l'équation de Young-Dupré, vous avez dessiné 3 forces mais il ne me semblait pas qu'on était en équilibre (au moins à la normale à la surface)} \textcolor{purple}{La force de réaction du support compense la composante normale de la force de tension de surface.}

  
  \textbf{C : D'autres façon pour déterminer $\gamma$ ?} \textcolor{purple}{Tensiomètre à plaque de Wilhelmy: on mesure la force exercée sur la plaque par le liquide grâce à une microbalance, et on peut en déduire $\gamma$.}

  
  \textbf{C : Dessin d'une goutte sur un plan incliné ? De quoi dépendrait l'angle d'inclinaison ?} \textcolor{purple}{Vitesse, viscosité, et coefficient de tension de surface.}

  
  \textbf{C : Pouvez-vous expliquer pourquoi on peut faire des chateaux de sables avec du sable mouillé ?} \textcolor{purple}{La présence d'eau va créer un pont capillaire entre deux grains qui va augmenter la force de cohésion entre les grains. Cette cohésion se fait sur une surface beaucoup plus grande pour le grain mouillée grâce au pont capillaire alors que pour un grain sec, cette cohésion se fait au niveau d'une rugosité (très petite surface) et donc insuffisance pour maintenir un château de sable.}



  \textbf{C : Pourquoi les serviettes sont rêches quand elle sèche ? Quel est le principe d"un adoucissant ?} \textcolor{purple}{Lorsqu'une serviette est mouillée, l'eau va chercher à minimiser son interface avec l'air en collant et écrasant les fibres de la serviette (c'est aussi pour ça que nos cheveux sont collés en sortant de la douche). En séchant, la serviette durcit, lui donnant une texture rêche. L'adoucissant agit chimiquement sur les fibres pour rendre le linge plus doux au toucher.}


  \textbf{C : $\gamma(T)$ ? Que se passe-t-il s'il y a des endroits du fluide qui sont plus chauds que d'autres ?} \textcolor{purple}{Création d'un gradient de tension de surface qui va causer un déplacement du liquide des zones de faibles $\gamma$ vers les zones de $\gamma$ élevée (cf. effet Marangoni).}

  
  \textbf{C : On parle parfois de liquide mouillant et de liquide non-mouillant. Valeurs de $\theta$ ?} \textcolor{purple}{Mouillage total : $\theta=0$, non-mouillage total : $\theta=\pi$.}

  
  \textbf{C : Commenter les incertitudes sur l'expérience ?} \textcolor{purple}{Incertitudes dominantes sur la mesure du rayon avec le pied à coulisse que j'ai pris de $0.5$mm, ce qui est une grande incertitude.}


  \textbf{C : Principe du manomètre ?} \textcolor{purple}{Un piézoélectrique subit une contrainte mécanique sous l'effet de la différence de pression, générant une tension que l'on mesure. Le constructeur nous donne la relation de conversion entre tension et pression.}

  \end{reportBlock}
  
%%%%%%%%%%%%%%%%%%%%%%%%%%%%%%%%%%%%%%%%%%%%%%%%%%%%
%%%% Commentaires
\begin{reportBlock}{Commentaires lors de la correction de la leçon}
Bonne leçon, beaucoup apprécié toutes les expériences montrées. Vous maîtrisez pas mal de choses. Attention à ce que le dispositif ne gène pas la vue sur le tableau.\\


\end{reportBlock}



%%%%%%%%%%%%%%%%%%%%%%%%%%%%%%%%%%%%%%%%%%%%%%%%%%%%
%%%% Correction
\begin{reportBlock}{Partie réservée au correcteur}
  \textbf{Avis général sur la leçon (plan, contenu, etc.) :}
  
  
  \textbf{Notions fondamentales à aborder, secondaires, délicates :}
  
  
  \textbf{Expériences possibles (en particulier pour l'agrégation docteur) :}
  
  
  \textbf{Bibliographie conseillée :}
\end{reportBlock}


\begin{reportBlock}{Partie réservée au correcteur}
  \textbf{Avis général sur la leçon (plan, contenu, etc.) :}
  
  
  \textbf{Notions fondamentales à aborder, secondaires, délicates :}
  
  
  \textbf{Expériences possibles (en particulier pour l'agrégation docteur) :}
  
  
  \textbf{Bibliographie conseillée :}
\end{reportBlock}
\newpage
%%%%%%%%%%%%%%%%%%%%%%%%%%%%%%%%%%%%%%%%%%%%%%%%%%%%
%%%% En-tête leçon
\begin{headerBlock}
    \chapter{Premier principe de la thermodynamique}
    \label{LP_PremierPrincipe}
\end{headerBlock}




%%%%%%%%%%%%%%%%%%%%%%%%%%%%%%%%%%%%%%%%%%%%%%%%%%%%
%%%% Références
\begin{center}
\begin{tabularx}{\textwidth}{| X | X | c | c |}
  \hline
  \rowcolor{gray!20}\multicolumn{4}{c}{Bibliographie de la leçon : } \\
  \hline 
  Titre & Auteurs & Editeur (année) & ISBN \\
  \hline
 Physique Tout en 1 MPSI PTSI & Bernard Salamito et al.  & Dunod &    \\
  \hline 
 Thermodynamique 1ère année MPSI-PCSI-PTSI & Jean-Marie Brébec  & H Prépa (Hachette Supérieur)  &    \\
  \hline 
 &   &   &    \\
  \hline 
 &   &   &    \\
  \hline
\end{tabularx}
\end{center}

%%%%%%%%%%%%%%%%%%%%%%%%%%%%%%%%%%%%%%%%%%%%%%%%%%%%

%%%%%%%%%%%%%%%%%%%%%%%%%%%%%%%%%%%%%%%%%%%%%%%%%%%%
%%%% Plan
\begin{reportBlock}{Plan détaillé}
  \textbf{Niveau choisi pour la leçon :} 1ère année de CPGE
  \newline
  \textbf{Prérequis :} Système thermodynamique fermé ; Transformation thermodynamique quasi-statique ; Fonctions d'état ; Equilibre thermodynamique ; Energies cinétique et potentielles, Travail d'une force 
  \newline
  
  \textbf{Déroulé détaillé de la leçon: }   \newline
  
  \section*{Introduction (3min)}
 La thermodynamique classique est une branche de la physique qui s'intéresse aux propriétés macroscopiques d'un système, et qui étudie les transformations de la matière et les échanges d'énergie sous différentes formes entre ce système et son environnement sans chercher à comprendre ce qui se passe au niveau microscopique. C'est une théorie axiomatique basée sur principalement sur deux principes.\\
  Animation : https://phet.colorado.edu/sims/html/energy-forms-and-changes/latest/energy-forms-and-changes\_en.html. Pour un système isolé, l'énergie ne se créé pas et ne disparaît pas, mais elle se transforme d'une forme à une autre : Ex animation : conversion énergie mécanique-électrique, électrique-thermique. \\
  Si le système est isolé, il y a conservation de l'énergie.  \\
  Leçon placée en 1re année CPGE. Review des prérequis.
  
  \section{Premier principe de la thermodynamique}
  Système ($\Sigma$) fermé.
  
  \subsection{Enoncé} Il existe une fonction d'état extensive \textbf{U} appelée énergie interne, telle que : 
  \begin{equation}
      \Delta E = \Delta E_c + \Delta E_p + \Delta U = W + Q
  \end{equation}
  \textcolor{red}{Interprétation :}
  \begin{itemize}
  \item U : $\sum_{i} E_{c,micro} + E_{p,micro}$
  \item $E_c$ : énergie cinétique macroscopique
  \item $E_p$ : énergie potentielle macroscopique (ex : $E_p=mgz$)
  \item $W$ : travail des forces macroscopiques extérieures non conservatives
  \item $Q$ : transfert thermique (chaleur) dû à l'agitation thermique aléatoire des particules
  \end{itemize}
  $Q$ et $W$ deux modes de transferts d'énergie. Ils sont algébriques (comptés positivement si reçus par le système. \\
  
  \textcolor{red}{Conséquences : } Si système isolé : $\Delta E = 0$ (conservation de l'énergie). \\
  
  \textcolor{red}{Version infinitésimale : } $dE_c + dE_p + dU = \delta W + \delta Q$.
  
  \subsection{Travaux des forces de pression (11min30)}
  Schéma fluide contenu dans une paroi. Pression extérieure $P_e$ uniforme et constante. Système = {fluide + paroi}. Variation du volume total de $dV$ entre entre $t$ et $t+dt$. En $M$ déplacement de $\mathbf{dM}$. \\
  Force apppliquée au système en un point $M$ : $\mathbf{dF}=-P_e\mathbf{dS}$. Travail asssocié $\delta^2 W = \mathbf{dS}\cdot\mathbf{dM} = -P_e \mathbf{dS}\cdot\mathbf{dM} = - P_e d^2 V$. Travail totale $\delta W$ s'obtient en sommant les travaux : $\delta^2 W$ $\delta W = -P_e dV$\\
  \underline{Si transformation quasi-statique et équilibre mécanique avec l'extérieur $P=P_e$} $\delta W = - p \ud V$ et $W = - \int_A^B p \ud V$ entre deux états $A$ et $B$.
  
 \subsection{Exemples de bilan d'énergie}
 \subsubsection{Trasformation isochore}
 $dV=0$ d'où $W=0$ donc $\Delta U = Q$.
 
 \subsubsection{Transformation monobare et enthalpie}
 $P_e=cste$ donc $W=-P_e(V_f-V_i)$ donc $\Delta U = Q + W$ donc $Q = \Delta U - W = [U_f +P_f V_f]-[U_i+P_iV_i] = H_f - H_i$.\\
 On définit l'enthalpie $H=U+PV$.\\
 
 \textcolor{red}{Premier Principe (transormation monobare):} $\Delta H = Q + W_{autre}$.
 
  \section{Applications du premier principe (20min)}
  \subsection{Définitions préliminaires}
  \textcolor{red}{Capacité thermique à volume constant : } $C_V = (\frac{\partial{U}}{\partial{T}})_V \rightarrow c_V = \frac{C_V}{m}$. \\
  \textcolor{red}{Capacité thermique à pression constante : } $C_P = (\frac{\partial{H}}{\partial{T}})_V \rightarrow c_P = \frac{C_P}{m}$. \\
  
  \subsection{Calorimétrie (24min)}
  \textcolor{red}{Expérience} Vase Dewar avec agitateur permettant d'homogénéiser le contenu. On peut remonter à la capacité calorifique.  La pression extérieure est fixée : transformation monobare. On suppose la transformation adiabatique : $\Delta H = Q = 0$.\\
  Pour le système \{eau+calorimètre+fer\} : $\Delta H = \Delta H_{eau} + \Delta H_{calo} + \Delta H_{fer} = c_{eau}\times m_{eau}\times \Delta T_{eau} + c_{eau}\times \mu \times \Delta T_{cal} + c_{fer}\times m_{fer}\times \Delta T_{fer}$.\\
  $\mu =29(3)$ déterminé en préparation. Mesure des masses à la balance : $m_{eau}$ et $m_{fer}$ ; et mesure des températures au thermocouple : $T_{eau+cal}$ et $T_{fer}$ à l'état initial et  $T_f$ à l'état final.\\
  On en déduit : $c_{fer}=\frac{c_{eau}(m_{eau}+\mu)(T_{eau+cal}-T_f)}{m_{fer}(T_f-T_i)}=1016\pm 184$~J.kg$^{-1}$.K$^{-1}$ à comparer à $c_{fer}=449$~J.kg$^{-1}$.K$^{-1}$.\\
  
  \subsection{Détente de Joule - Gay Lussac (38min)}
  Deux enceintes séparées par un robinet. Une enceinte est remplie par un gaz, l'autre par un fluide. Ces enceintes sont calorifugées et avec des parois rigides. On ouvre le robinet. \\
  Système = {gaz+vide+enceintes}. On a $\Delta U=0$. Si $U(T)$ (première loi de Joule) : $\Delta U = C_{V}\Delta T = 0$ donc transformation isotherme. \\
  Cette expérience permet de vérifier si un gaz vérifie la première loi de Joule en mesurant la variation de température.
  
  \section*{Conclusion (40min)}
  Dans cette leçon, on a parlé du premier principe qui est un principe de conservation de conservation. Ce principe est complété par le second principe, qui lui, est plutôt un principe d'évolution et qui porte sur le caractère réversible ou irréversible d'une transformation. Pour finir, ces principes et la thermodynamique classique en général a été formalisée plus tard par la mécanique statistique qui permet d'expliquer les résultats de la thermodynamiques en faisant le lien entre l'échelle microscopique et macroscopique.



\end{reportBlock}


%%%%%%%%%%%%%%%%%%%%%%%%%%%%%%%%%%%%%%%%%%%%%%%%%%%%
%%%% Questions
\begin{reportBlock}{Questions posées par l’enseignant (avec réponses)}
  \textbf{C : Sur la vidéo, comment on fait conversion énergie mécanique et électrique ?}  \textcolor{purple}{Avec un alternateur : un aimant est entrainé par l'énergie mécanique qui créé un courant variable dans une bobine par induction. Exemple : une dynamo de bicyclette. Si pas de champs magnétiques préexistant, induction électromagnétique.} \newline
  
  \textbf{C : Dans l'énoncé du premier principe, quelles sont les particularités de A et B ?}  \textcolor{purple}{Ils sont à l'équilibre thermodynamique pour qu'on puisse leur défiir une énergie .} \newline
  
  \textbf{C : W est le travail des forces non conservatives ? Si je prends un piston qui est bloqué par un poids, il y a le poids dans W ? Si j'ajoute une force, ça agit sur E$_p$ par sur U (par exemple la force de Lorentz qui agit sur chaque particule ?}  \textcolor{purple}{Il est bien présent dans la partie gauche de l'équation. Si on le met à doite ça agit sur $E_p$ et on met un signe \og - \fg. C'est un peu indifférent de le mettre à droite où à gauche mais il ne faut pas le compter deux fois et mettre le bon signe. } \newline
  
  \textbf{C : De façon non ambigüe, peut-on mesurer une quantité de chaleur ou savor ce que c'est ?}  \textcolor{purple}{La chaleur ça sera toute la variation d'énergie sauf le travail des forces macroscopiques. On peut mesurer la variation d'énergie interne dans certaines conditions.} \newline
  
  \textbf{C :Si transfo quasi-statique, on peut faire $-P_edV$. Peut-on juste dire que la transformation est réversible ?}  \textcolor{purple}{Oui ça fonctionne mais dans la vie il n'existe pas de transformation réversible.} \newline
  
  \textbf{C : Définition de réversible ?}  \textcolor{purple}{On peut changer le sens et changeant infiniment peu les contraintes extérieures. On peut prendre l'exemple d'un piston où il y a des frottements pour la différence entre réversible et quasi-statique.} \newline
  
  \textbf{C : Dans la détente Joule Gay Lussac, est-ce que c'est important que l'enceinte soit vide ou remplie d'un gaz à pression plus faible par exemple ?}  \textcolor{purple}{Il faut juste faire attention à la définition du système. } \newline
  
  \textbf{C :Si on veut une variation de chaleur $\delta Q$ pour un fluide, quels sont les coefficients importants ?}  \textcolor{purple}{Il y a 6 variables calorimétriques importantes. Suivant le système, il faut prendre la dépendance en volume etc... On peut montrer qu'il n'y en a que deux d'indépendantes.} \newline
  
  \textbf{C :Différentes façon d'exprimer $\delta Q$ : $\delta Q = c_vdT + ldV = c_pdT + hdP = \lambda dP + \mu dV$. On peut montrer la relation entre pente adiabatique et pente isotherme. Quelle est la pente la plus importante entre une adiabatique ou une isotherme dans le diagramme (P,V) ? Calculer la pente pour une adiabatique et une isotherme ? }  \textcolor{purple}{Isotherme : $dT=0$ donc $\delta Q= ldV = hdP$. Adiabatique : $dV = -\frac{c_v}{l}dT$ et $dP = -\frac{c_p}{h}dT= \frac{c_pl}{hc_v}dV$.} \newline
  
  \textbf{C :Comment exprimer de manière générale $dP$ en fonction de $dT$ et $dV$ à partir d'une équation d'état ?}  \textcolor{purple}{$dT = (\frac{\partial{T}}{\partial{P}})_VdP + (\frac{\partial{d}T}{\partial{d}V})_VdV$. On en déduit } \newline
  
  \textbf{C :Dans le diagramme (P,V), on considère des transformations infinitésimales entre A et D (entre A et B : isochore, B et C : isobare, C et D : isotherme et entre D et A : isobare). Calculer la variation infinitésimale $\delta Q$ sur le cycle en commençant d'abord par les chemins A-B-C et A-D-C.}  \textcolor{purple}{Sur A-B-C : $\delta Q = \lambda dP + \mu dV$. Sur A-D-C, $\delta Q = c_pdT + hdP$. Donc sur le cycle : $\delta Q = -\delta W = \lambda dP + \mu dV - c_pdT - hdP$. } \newline
  
  
\end{reportBlock}


%%%%%%%%%%%%%%%%%%%%%%%%%%%%%%%%%%%%%%%%%%%%%%%%%%%%
%%%% Commentaires
\begin{reportBlock}{Commentaires lors de la correction de la leçon}
Très bonne leçon. Basile : je n'aurai pas du tout évoquer le terme chaleur.\\
Pour l'expérience, on aurait pu faire l'inverse en faisant chauffer la barreau de fer et mettre de l'eau à température ambiante. Il y aurait peut-être moins de pertes.\\
Tu as parlé moins vite et c'était mieux par rapport à la dernière fois.\\
Pas de forces à longue portée pour que U soit extensive.


\end{reportBlock}



%%%%%%%%%%%%%%%%%%%%%%%%%%%%%%%%%%%%%%%%%%%%%%%%%%%%
%%%% Correction
\begin{reportBlock}{Partie réservée au correcteur}
  \textbf{Avis général sur la leçon (plan, contenu, etc.) :}
  
  
  \textbf{Notions fondamentales à aborder, secondaires, délicates :}
  
  
  \textbf{Expériences possibles (en particulier pour l'agrégation docteur) :}
  
  
  \textbf{Bibliographie conseillée :}
\end{reportBlock}


\begin{reportBlock}{Partie réservée au correcteur}
  \textbf{Avis général sur la leçon (plan, contenu, etc.) :}
  
  
  \textbf{Notions fondamentales à aborder, secondaires, délicates :}
  
  
  \textbf{Expériences possibles (en particulier pour l'agrégation docteur) :}
  
  
  \textbf{Bibliographie conseillée :}
\end{reportBlock}
\newpage
%%%%%%%%%%%%%%%%%%%%%%%%%%%%%%%%%%%%%%%%%%%%%%%%%%%%
%%%% En-tête leçon
\begin{headerBlock}
  \chapter{Transition de phase}
  \label{LP_TransitionPhase} 
\end{headerBlock}




%%%%%%%%%%%%%%%%%%%%%%%%%%%%%%%%%%%%%%%%%%%%%%%%%%%%
%%%% Références
\begin{center}
\begin{tabularx}{\textwidth}{| X | X | c | c |}
  \hline
  \rowcolor{gray!20}\multicolumn{4}{c}{Bibliographie de la leçon : } \\
  \hline 
  Titre & Auteurs & Editeur (année) & ISBN \\
  \hline
  Thermodynamique & BFR & Dunod & \\
  \hline 
  Thermodynamique & Diu &  &    \\
  \hline 
   &  & &    \\
  \hline 
\end{tabularx}
\end{center}

%%%%%%%%%%%%%%%%%%%%%%%%%%%%%%%%%%%%%%%%%%%%%%%%%%%%

%%%%%%%%%%%%%%%%%%%%%%%%%%%%%%%%%%%%%%%%%%%%%%%%%%%%
%%%% Plan
\begin{reportBlock}{Plan détaillé}

  \textbf{Niveau choisi pour la leçon :} Licence 3
  \newline
  \textbf{Prérequis} : \begin{itemize}
      \item 
  \end{itemize}

  \textbf{Déroulé détaillé de la leçon: }  
  
  \section*{Introduction}
Faire une accroche phénoménologique. Comment décrire l'eau qui bout ? Pourquoi la température reste constante ? Pourquoi c'est la même chose pour l'eau qui gèle ? Est-ce que c'est la même chose pour toutes les transitions de phase ? Définir une phase.

  \section{Règle des phases}

  \subsection{Diagramme (P,T) de l'eau}
  
  \subsection{Description du phénomène}
  
  
  \section{Etude du diagramme (P,V) de l'eau/azote}
  
  
  \subsection{Enthalpie de changement d'état} 


  \section{La transition ferro-para}
  Coexistence entre deux phases. 
\subsection{Transition paramagnétisme-ferromagnétisme}


\end{reportBlock}
\newpage
%%%%%%%%%%%%%%%%%%%%%%%%%%%%%%%%%%%%%%%%%%%%%%%%%%%%
%%%% En-tête leçon
\begin{headerBlock}
  \chapter{Phénomènes de transport}
    \label{LP_Transport}
\end{headerBlock}

%%%%%%%%%%%%%%%%%%%%%%%%%%%%%%%%%%%%%%%%%%%%%%%%%%%%
%%%% Références
\begin{center}
\begin{tabularx}{\textwidth}{| X | X | c | c |}
  \hline
  \rowcolor{gray!20}\multicolumn{4}{c}{Bibliographie de la leçon : } \\
  \hline 
  Titre & Auteurs & Editeur (année) & ISBN \\
  \hline
  Thermodynamique & BFR & Dunod (1989) & \\
\end{tabularx}
\end{center}

%%%%%%%%%%%%%%%%%%%%%%%%%%%%%%%%%%%%%%%%%%%%%%%%%%%%
\begin{reportBlock}{Plan détaillé}
  \textbf{Niveau choisi pour la leçon :} 
  \newline
  \textbf{Prérequis : }
  \newline


\section{Généralités sur les phénomènes de transport}

\textcolor{blue}{Expérience qualitative :} Barreau  chauffé + caméra thermique
\subsection{Type de transport}
\subsection{Système hors équilibre}
\subsection{Equilibre thermodynamique local}

\section{Diffusion de particules}

\subsection{Loi de Fick (1855)}
\subsection{Conservation du nombre de particules et équation de diffusion}
\subsection{Exemple de solution dans un cas non-stationnaire}

\section{Correspondance avec d'autres phénomènes de diffusion}

\subsection{Analogie}


\section*{Conclusion}
Ouverture sur le théorème de N\oe ther.

\end{reportBlock}
\newpage
%%%%%%%%%%%%%%%%%%%%%%%%%%%%%%%%%%%%%%%%%%%%%%%%%%%%
%%%% En-tête leçon
\begin{headerBlock}
  \chapter{Conversion de puissance électromécanique}
    \label{LP_ConversionPuissance}
\end{headerBlock}

%%%%%%%%%%%%%%%%%%%%%%%%%%%%%%%%%%%%%%%%%%%%%%%%%%%%
%%%% Références
\begin{center}
\begin{tabularx}{\textwidth}{| X | X | c | c |}
  \hline
  \rowcolor{gray!20}\multicolumn{4}{c}{Bibliographie de la leçon : } \\
  \hline 
  Titre & Auteurs & Editeur (année) & ISBN \\
  \hline
  TP Moteur &  & Site agreg Montrouge &   \\
  \hline 
  Cours Jérémy Neveu & J. Neveu & Site agreg Montrouge & \\
  \hline
\end{tabularx}
\end{center}

%%%%%%%%%%%%%%%%%%%%%%%%%%%%%%%%%%%%%%%%%%%%%%%%%%%%
\begin{reportBlock}{Plan détaillé}
  \textbf{Niveau choisi pour la leçon :} 
  \newline
  \textbf{Prérequis : }
  \newline

\section*{Introduction}
Parler de l'intérêt dans la vie de tous les jours :
\begin{itemize}
    \item Conversion mécanique - électrique : ex : éolienne, production d'électricité, courants de Foucault
    \item Conversion électrique - mécanique : moteur, haut-parleur, freinage par courants de Foucault
\end{itemize}

\section{Principe de la conversion}
\subsection{Rails de Laplace}
\subsection{Bilan de puissance}

\section{Moteurs synchrones}
Cf cours de Jérémy
\subsection{Etude du MCC}
Avantage du moteur synchrone : réversible (si mode générateur : alternateur)
Attention : utiliser des multimètres métrix, poids de 100g à 5kg (x2 à chaque fois). Pied à coulisse pour $r$, chrono pour mesure $\Omega$ à la main, typiquement $v$ stable pour $\Delta x=10$~cm (cf tracker).\\
\textcolor{blue}{Manip 1 : }Déterminer $K_{em}$ la résistance du moteur et la comparer à la valeur donnée dans la notice. Je trouve $K_{em}=1.30(2)$~V.rad.s$^{-1}$.\\
\textcolor{blue}{Manip 2 :} Tracer $\eta$ en fonction de la puissance utile et comparer à charge nominale = 2kg et $\eta_{nom}=50\%$. Le courant lu est assez critique, ne pas hésiter à refaire plusieurs fois.

\subsection{Bilan de puissance}

\end{reportBlock}
\newpage
%%%%%%%%%%%%%%%%%%%%%%%%%%%%%%%%%%%%%%%%%%%%%%%%%%%%
%%%% En-tête leçon
\begin{headerBlock}
 \chapter{Induction électromagnétique}
 \label{LP_Induction}
\end{headerBlock}

%%%%%%%%%%%%%%%%%%%%%%%%%%%%%%%%%%%%%%%%%%%%%%%%%%%%
%%%% Références
\begin{center}
\begin{tabularx}{\textwidth}{| X | X | c | c |}
  \hline
  \rowcolor{gray!20}\multicolumn{4}{c}{Bibliographie de la leçon : } \\
  \hline 
  Titre & Auteurs & Editeur (année) & ISBN \\
  \hline
   Electromagnatisme 3 : magnétostatique, induction, équations de Maxwell et compléments électroniques & M. Bertin, J. P. Faroux, J. Renault & Dunod Université (1986) & 2-04-016916-4 \\
  \hline 
   Physique Spé. MP*, MP et PT*, PT & Hubert Gié, Jean-Pierre Sarmat, Stéphane Olivier, Christophe More & Editions Tec \& Doc (2000) & 2-7430-0398-7 \\
  \hline 
   Physique Spé. PSI*, PSI & Stéphane Olivier, Christophe More, Hubert Gié & Editions Tec \& Doc (2000) & 2-7430-0399-5 \\
  \hline 
\end{tabularx}
\end{center}

%%%%%%%%%%%%%%%%%%%%%%%%%%%%%%%%%%%%%%%%%%%%%%%%%%%%

%%%%%%%%%%%%%%%%%%%%%%%%%%%%%%%%%%%%%%%%%%%%%%%%%%%%
%%%% Plan
\begin{reportBlock}{Plan détaillé}
  \textbf{Niveau choisi pour la leçon :} L3
  \newline
  \textbf{Prérequis : } Equations de Maxwell ; Forces de Lorentz, de Laplace ; ARQS magnétique ; Potentiels scalaire et vecteur ; Electrocinétique
  \newline
  
  \textbf{Déroulé détaillé de la leçon: }\newline
  \section*{Introduction} 
  \begin{itemize}
      \item \textbf{Cadre général :} ARQS magnétique
      \item \textbf{Introduction historique :} Oersted (1820): courants éléctriques induisent $\mathbf B$. Faraday (1831): Variations de $\mathbf B$ qui induisent des courants électriques. 
  \end{itemize}
  \vspace{1cm}
  \section{Approche expérimentale} 3min20
  \begin{itemize}
      \item \textbf{Expérience qualitative 1:} Approche un aimant et éloigne un aimant droit d'une bobine fixe branchée à un oscilloscope : apparition d'une tension. Même observation avec déplacement de la bobine dans aimant fixe. Amplitude de l'intensité proportionnelle à la vitesse de variation de $\mathbf B$.
      \item \textbf{Définition générale de l'induction (slide)} apparition d'une f.e.m et, s'ils peuvent s'écouler, de courants, dans un conducteur mobile placé d'un champ magnétique variable.
      \item \textbf{Deux cas particuliers:} \\
      Induction de Neumann (circuit fixe, champ variable) ; \\ Induction de Lorentz (circuit mobile, champ stationnaire).
        \end{itemize}
\begin{enumerate}
      \item \textbf{Loi de Faraday :} 4min20 $e = - \frac{\ud \phi}{\ud t}$ ; \\
      Validité : circuits filiformes ; \\
      Définition du flux : $\phi = \iint \mathbf B \cdot \ud \mathbf S$ \\
      Unités de $e$ et $\phi$, convention d'orientation de la surface par rapport au circuit (règle de la main droite) ; \\ Convention générateur de la f.e.m.
      \item \textbf{Loi de Lenz :} discussion du signe $-$ dans la loi de Faraday, expérience qualitative 2 : chute d'un aimant dans un tube conducteur.
\end{enumerate}

\vspace{1cm}
\section{Théorie de l'induction} 6min10
\begin{enumerate}
    \item \textbf{Définition formelle de la fem}: $e = \frac{1}{q} \oint \mathbf F(\mathbf r, t) \cdot \ud \mathbf l$. \\
    Ici : $\mathbf F$ force de Lorentz $\rightarrow e = \oint \mathbf E \cdot \ud \mathbf l + \oint (\mathbf B \land \mathbf v) \cdot \ud \mathbf l$.
    \item \textbf{Induction de Neumann}: $\mathbf v \slash\slash
 \ud \mathbf l \rightarrow e = \oint \mathbf E \cdot \ud \mathbf l$. \\
 $\mathbf E = - \nabla V - \frac{\partial \mathbf A}{\partial t} \rightarrow e = \oint \mathbf{E_m} \cdot \ud \mathbf l $ où $E_m = - \frac{\partial \mathbf A}{\partial t}$ est le \textbf{champ électromoteur de Neumann}. \\
 Equation de Maxwell-Faraday : $\nabla \land \mathbf E = - \frac{\partial \mathbf B}{\partial t}$ donne la loi de Faraday.
    
    \item \textbf{Induction de Lorentz}: Schéma. \\
    Non relativiste : $\mathbf v = \mathbf{v_r} + \mathbf{v_e}$, $\mathbf{v_r} \slash\slash \ud \mathbf l$ \\
    $\mathbf E = - \nabla V$. \\
    $e = \oint (\mathbf v_e \land \mathbf B) \cdot \ud \mathbf l$. Le terme $\mathbf v_e \land \mathbf B$ se subtitue au champ électromoteur de Neumann. \\
    A l'oral (sans faire le calcul) : on peut montrer qu'en utilisant l'équation de Maxwell-Flux, on retrouve la loi de Faraday.
\end{enumerate}
\textbf{Conclusion orale}: Cas général : somme des deux cas (Neumman et Lorentz). Même phénomène : on le voit par changement de référentiel (exemple avec la première expérience qualitative).

\vspace{1cm}
\section{Aspects pratiques} 15min30
\begin{enumerate}
    \item \textbf{Auto-induction :} Dessin spire avec ligne de champ. \\
    Flux propre : $\phi_p = L i$, $L$ inductance propre (H). \\
    f.e.m : $e = - L \frac{\ud i}{\ud t}$. \\
    Schéma équivalent en éléctrocinétique : convention générateur avec générateur, convention récepteur avec bobine. \\
    - \textbf{Expérience quantitative 1 : Mesure de L}: Circuit RL, mesure du temps caractéristique sur oscilloscope.
    26min
    \item \textbf{Inductance mutuelle}: Dessin spire 1 avec ligne de champ et spire 2 dans champ magnétique créé par spire 1. \\
    - Flux créé par spire 1 à travers spire 2: $\phi_{21} = M_{21} i_1$ ; \\
    - Flux créé par spire 2 à travers spire 1: $\phi_{12} = M_{12} i_2$ ; \\
    - $M_{12} = \oint \oint \frac{\mu_0 \ud \mathbf{l_1} \cdot \ud \mathbf{l_2}}{4 \pi r_{12}} = M_{21}$. \\
     - \textbf{Expérience quantitative 2 : Mesure de M}: Modèle simple de transformateur (schéma sur slide). Secondaire en circuit ouvert $(i_2 = 0)$. Loi des mailles (en complexe) donne: $ M = L_1 \left| \frac{U_2}{e_g} \right|$. Expérience avec deux bobines placées dans un entrefer en fer doux. \\
     \textbf{\textcolor{red}{Attention: cette mesure expérimentale est fausse ! (ne pas reproduire). Mettre un entrefer modifie M et L qui peuvent dépendre des courants.}}
     \newline 37min20
\end{enumerate}

\vspace{1cm}
\section*{Conclusion} 
Applications diverses (on a vu bobines et transformateurs). \\
Autres applications (slides) : Plaques à induction, Feinage par induction.\newline
39min12
\end{reportBlock}


%%%%%%%%%%%%%%%%%%%%%%%%%%%%%%%%%%%%%%%%%%%%%%%%%%%%
%%%% Questions
\begin{reportBlock}{Questions posées par l’enseignant (avec réponses)}
  \textbf{C :} Revenir sur la toute première manip (aimant+bobine). Peut-on remonter directement au flux $\phi$ ? \textcolor{purple}{Oui en faisant l'intégrale de la tension sur un aller à l'oscilloscope. Par ailleurs, on peut diviser le nombre de spire par deux et voir que la tension est également divisée par deux.} Peut-on retrouver que c'est une tension négative en premier et positive en deuxième ? \textcolor{purple}{Il faut connaître dans quel sens est orienté le bobinage de la bobine. A partir de là, le sens des lignes de champ de l'aimant étant connu, on peut connaître le signe du flux et de sa variation temporelle, et ainsi connaître le signe de la tension aux bornes de la bobine.}\newline
  
  \textbf{C :} Symbole du Weber ? \textcolor{purple}{Wb, ce sont des T.m$^{-2}$.}\newline
  
  \textbf{C :} Dans la loi de Faraday, vous avez dit que $e$ doit être orienté suivant le sens du courant. Pourtant il n'y a pas de courant à la base car c'est la fem qui induit le courant (s'il peut circuler) ? \textcolor{purple}{On choisit une convention d'orientation du circuit. Celle-ci nous impose l'orientation de la surface du flux sortant. La fém est alors \textbf{dans le sens de l'orientation du circuit}. Enfin, si le courant peut circuler, pour $e>0$, le champ de force tend à faire circuler les porteurs de charge positive dans le sens positif du circuit, c'est-à-dire à produire une intensité $i>0$ (on rappelle que le sens de circulation du courant est par convention inverse au sens de circulation des électrons).}\newline
  
  \textbf{C :} Peut-on prendre la surface du circuit filiforme pour calculer le flux de B ? \textcolor{purple}{Oui car le flux est indépendant du choix de la surface (il faut par contre que celle-ci s'appuie sur le contour défini par le circuit).}\newline
  
  \textbf{C :} Pourquoi la manip avec le tube est une illustration de la loi de Lenz ? \textcolor{purple}{Le mouvement de chute est freiné par la force de Laplace créée par les courants induits dans le conducteur : la force s'oppose donc au mouvement qui lui a donné naissance.} Peut-on retrouver que la force doit être opposée au mouvement avec la loi de Faraday ? \textcolor{purple}{Oui, on peut voir le conducteur cylindrique comme une superposition de spires infiniment fines et faire le raisonnement avec la loi de Faraday sur chacune des spires et sommer le tout.}  \newline
  
  \textbf{C :} Dans quel cas un fil peut-être considéré comme filiforme (pour la validité de la loi de Faraday) ? \textcolor{purple}{Si la largeur du fil est petite devant la taille caractéristique de variation du champ magnétique (alors le champ magnétique est quasi-constant dans le volume).} \newline
  
  \textbf{C :} Quelle est le cadre de l'ARQS magnétique ? \textcolor{purple}{L'ARQS revient à négliger le temps de propagation des quantités électromagnétiques. Si $T$ est le temps typique de variation des densités de courants $\mathbf j$ et de charges $\rho$ et $L$ la distance typique entre les sources et le champ électromagnétique observé alors l'ARQS est valable si $T \gg \frac{L}{c}$. Si de plus $\mid \mathbf j \mid \gg \mid \rho c \mid$, on est dans l'ARQS \textbf{magnétique}. Odg : pour un circuit $L \sim 1$m, on doit avoir $f = \frac{c}{T} \ll 10^{14}$ Hz.} \newline
  
  \textbf{C :} Y-a-t'il une raison pour laquelle on associe la fem à une tension autre que l'homogénéité de son unité ? \textcolor{purple}{Par exemple, si je n'ai que un champ électrique, $\frac{\mathbf F}{q} = \mathbf E$ et $e = \oint \mathbf E \cdot \ud \mathbf l$ : c'est la tension. On a bien une force qui met en mouvement les électrons, ce qui ressemble à ce que fait une tension électrique, c'est comme un générateur. \\
  Par ailleurs, on peut retrouver à partir de la loi $W = \oint \mathbf{F}(\mathbf{r},t) \cdot \ud \mathbf l$ la loi d'Ohm généralisée en électrocinétique.} \newline
  
  \textbf{C :} Vous avez obtenu la fem dans le cadre de Neumann, quid dans le cas de Lorenz ? \textcolor{purple}{On prend un circuit filiforme orienté dont chaque élement $\ud \mathbf l$ subit un déplacement $\ud \mathbf r$ (on suppose que l'orientation du circuit ne change pas au cours du temps). On a $e = \oint \mathbf{v}_e\land\mathbf{B} \cdot \ud \mathbf l$ avec $\mathbf{v}_e = \deriv \mathbf{r}/\deriv t$ la vitesse d'entrainement des électrons par déplacement due au déplacement du circuit au cours de $\deriv t$. Par permutation circulaire du produit vectoriel avec le produit scalaire, on a : $\mathbf{v}_e\land\mathbf{B} \cdot \ud \mathbf l = - \frac{1}{\ud t} (\mathbf{B} \cdot \ud^2 S_c) = - \frac{\ud^2 \phi_c}{\ud t}$ où on a introduit $\deriv^2 S_c \equiv \deriv \mathbf{r} \cdot \deriv \mathbf l$ l'aire balayée pendant $\ud t$ par le déplacement du conducteur et $\ud \phi_c \equiv \mathbf B \cdot \deriv^2 S_c$ le flux "coupé" (flux du champ magnétique à travers $\deriv^2 S_c$, "coupé" car lors de son déplacement, l'élement de conducteur "coupe" les lignes de champs magnétiques). Enfin, en utilisant l'équation de Maxwell-Flux, on peut montrer que pour un champ magnétique stationnaire, $\oint \ud^2 \phi_c = \ud \phi$, nous permettant de retrouver la loi de Faraday.} \newline
  
  \textbf{C :} \`{A} t-on toujours une fem dans le cas où le circuit se déplace ? \textcolor{purple}{Si le circuit se déplace sans déformation, la variation de flux magnétique est nulle ($\phi(t+\ud t) = \phi(t)$) et donc $e=0$.}\newline
  
  \textbf{C :} Pourquoi la manip n'a pas marché pour la mesure de l'inductance ? \textcolor{purple}{Mesure mal faite, on a trouvé la bonne valeur quand elle a été refaite pendant la séance de question.}  \newline
  
  \textbf{C :} Est-ce que la bobine peut créer une inductance mutuelle sur le circuit électrique et donc modifier l'inductance du circuit ? \textcolor{purple}{C'est possible en théorie, elle est d'autant plus importante que le circuit est grand car l'inductance mutuelle est proportionnelle à la surface d'intégration.} Est-ce que c'est valable pour un solénoïde infini ? \textcolor{purple}{Non car le champ est nul à l'extérieur de la bobine.}\newline
  
  \textbf{C :} Retour sur la manip de l'inductance mutuelle ? Quelle est l'influence du fer doux ? \textcolor{purple}{Il canalise les lignes de champs dans les bobines, cela permet d'avoir un champ plus fort.} Pourquoi la tension n'est pas sinusoïdale à la sortie de la bobine alors que celle à l'entrée l'est ? \textcolor{purple}{Les inductances propres et mutuelle peuvent dépendre des courants induits. Ces courants dépendent de la réponse du fer doux à une excitation magnétique qui n'est pas linéaire avec le champ. Cela entraîne donc des non-linéarités du courants et donc de la fem en sortie.}\newline
  
  \textbf{C :} Quel instrument permet de mesurer beaucoup mieux la tension crête à crête ? \textcolor{purple}{Un voltmètre en configuration AC.}\newline
  
  \textbf{C :} Contrairement à $L$, $M$ est algébrique. Qu-est-ce que ça veut dire ? \textcolor{purple}{$M$ dépend des conventions d'orientation des circuits. $L$ est toujours positive car changer l'orientation du circuit change l'orientation du courant d'une part, et celui du vecteur surface donc du flux d'autre part, de sorte que $L$ reste toujours positive.}\newline
  
  \textbf{C :} Pourquoi il y a-t'il un déphasage dans le signal de sortie de la manip de mesure d'inductance ? \textcolor{purple}{Le circuit est un filtre passe-bas d'ordre 1, la fréquence de travail induit un déphasage dans la réponse du système qui vaut $0$ pour $\omega=0$ et $-\pi/2$ pour $\omega\rightarrow\infty$}.
  
  
\end{reportBlock}


%%%%%%%%%%%%%%%%%%%%%%%%%%%%%%%%%%%%%%%%%%%%%%%%%%%%
%%%% Commentaires
\begin{reportBlock}{Commentaires lors de la correction de la leçon}
\end{reportBlock}


%%%%%%%%%%%%%%%%%%%%%%%%%%%%%%%%%%%%%%%%%%%%%%%%%%%%
%%%% Correction
\begin{reportBlock}{Partie réservée au correcteur}
  \textbf{Avis général sur la leçon (plan, contenu, etc.) :}
  
  
  \textbf{Notions fondamentales à aborder, secondaires, délicates :}
  
  
  \textbf{Expériences possibles (en particulier pour l'agrégation docteur) :}
  
  
  \textbf{Bibliographie conseillée :}
\end{reportBlock}


\begin{reportBlock}{Partie réservée au correcteur}
  \textbf{Avis général sur la leçon (plan, contenu, etc.) :}
  
  
  \textbf{Notions fondamentales à aborder, secondaires, délicates :}
  
  
  \textbf{Expériences possibles (en particulier pour l'agrégation docteur) :}
  
  
  \textbf{Bibliographie conseillée :}
\end{reportBlock}
\newpage
%%%%%%%%%%%%%%%%%%%%%%%%%%%%%%%%%%%%%%%%%%%%%%%%%%%%
%%%% En-tête leçon
\begin{headerBlock}
  \chapter{Rétroactions et oscillations.}
  \label{LP_RetroactionOscillation} 
\end{headerBlock}




%%%%%%%%%%%%%%%%%%%%%%%%%%%%%%%%%%%%%%%%%%%%%%%%%%%%
%%%% Références
\begin{center}
\begin{tabularx}{\textwidth}{| X | X | c | c |}
  \hline
  \rowcolor{gray!20}\multicolumn{4}{c}{Bibliographie de la leçon : } \\
  \hline 
  Titre & Auteurs & Editeur (année) & ISBN \\
  \hline
  Electronique & Pérez & Dunod & \\
  \hline 
  Tout-en-un PSI/PSI* & Cardini & Dunod &    \\
  \hline 
  Physique Spé. PSI/PSI* & Olivier, More, Gié & Tec\&Doc & \\
  \hline 
  Cours & Jérémy Neveu & & \\
  \hline
\end{tabularx}
\end{center}

%%%%%%%%%%%%%%%%%%%%%%%%%%%%%%%%%%%%%%%%%%%%%%%%%%%%

%%%%%%%%%%%%%%%%%%%%%%%%%%%%%%%%%%%%%%%%%%%%%%%%%%%%
%%%% Plan
\begin{reportBlock}{Plan détaillé}

  \textbf{Niveau choisi pour la leçon :} Licence 3
  \newline
  \textbf{Prérequis} : \begin{itemize}
      \item 
  \end{itemize}

  \textbf{Déroulé détaillé de la leçon: }  
  
  \section*{Introduction}

  \section{Rétroactions}
  \subsection{Enthalpie de changement d'état} 


  \section{Oscillations}
\subsection{Oscillateur à Pont de Wien}
Prendre $R_1\sim 1k\Omega$ pour l'amplification et $R=1k\Omega$ dans le filtre. Mettre $t-t_0$ pour le fit et enregistrer les oscillations par clé USB directement sur l'oscillo. Fitter par 
\begin{equation}
    A\exp\left(-a\omega_0(t+t_0)\right)\times\sin\left(\sqrt{1-\alpha^2}\omega_0(t+t_0)\right)
\end{equation}


\end{reportBlock}
\newpage
%%%%%%%%%%%%%%%%%%%%%%%%%%%%%%%%%%%%%%%%%%%%%%%%%%%%
%%%% En-tête leçon
\begin{headerBlock}
  \chapter{Traitement d'un signal. Etude spectrale.}
    \label{LP_TraitementSignal}
\end{headerBlock}

%%%%%%%%%%%%%%%%%%%%%%%%%%%%%%%%%%%%%%%%%%%%%%%%%%%%
%%%% Références
\begin{center}
\begin{tabularx}{\textwidth}{| X | X | c | c |}
  \hline
  \rowcolor{gray!20}\multicolumn{4}{c}{Bibliographie de la leçon : } \\
  \hline 
  Titre & Auteurs & Editeur (année) & ISBN \\
  \hline
  Electronique & Pérez & Dunod &   \\
  \hline 
  Traitement des signaux et acquisition de données & Francis Cottet & Dunod & 2-10-006312-X \\
  \hline 
  Tout-en-un PSI &  & Tec\&Doc & \\
  \hline
  Dictionnaire de physique & R. Taillet, L. Villain, P. Febvre & de Boeck & \\
  \hline
  Cours Jérémy Neveu & J. Neveu & & \\
  \hline
\end{tabularx}
\end{center}

%%%%%%%%%%%%%%%%%%%%%%%%%%%%%%%%%%%%%%%%%%%%%%%%%%%%
\begin{reportBlock}{Plan détaillé}
  \textbf{Niveau choisi pour la leçon :} 
  \newline
  \textbf{Prérequis : }
  \newline

\section*{Introduction}
\textcolor{red}{Accroche : }L'enjeu des communications est de pouvoir envoyer un signal (\textit{i.e.} une information) depuis un émetteur jusqu'à un récepteur afin que celui-ci puisse être d'une part reçu et d'autre part compris.
\begin{center}
    \includegraphics[scale=0.8]{LP_TraitementSignal/Codage_LuckyLuke.jpg}
\end{center}
\textcolor{green}{signal :} (ref Taillet p674) Variation temporelle ou spatiale d'une quantité physique mesurable (tension, force, lumière, ...) portant une information.\\
\textcolor{green}{Traitement du signal :} Transformation d'un signal reçu par un récepteur pour en retirer l'information transmise initialement par un émetteur. Ex : si un observateur cherche à analyser la lumière émise par une étoile (\textcolor{green}{signal}) à travers l'atmosphère, il doit se séparer de celle émise par l'atmosphère (\textcolor{green}{bruit}) par différents moyen (filtrage par exemple).

\section{Représentation des signaux}

\subsection{Types de signaux}
Signaux analogiques vs numériques.\\
Spectre d'énergie d'un signal cf Cottet.\\
\subsection{Décomposition Fourier}
Tout signal peut se décomposer comme une série de signaux sinusoïdaux.\\
Propriétés TF.
\subsection{Echantillonnage}
Mettre Shannon, repliement de spectre (effet de fenêtrage)

\section{Traitement d'un signal}

\subsection{Filtrage, fonction de transfert}
A voir si généralité ou si étude d'un filtre en particulier. Peut-être reprendre leçon de Vincent.
\subsection{Application à la détection synchrone}
\textcolor{blue}{Manip : mesure d'une impédance par détection synchrone}

\subsection{Modulation/démodulation}

\subsection{FFT}

\end{reportBlock}
\newpage
%%%%%%%%%%%%%%%%%%%%%%%%%%%%%%%%%%%%%%%%%%%%%%%%%%%%
%%%% En-tête leçon
\begin{headerBlock}
  \chapter{Ondes progressives, ondes stationnaires.}
  \label{LP_OndesProgressives} 
\end{headerBlock}




%%%%%%%%%%%%%%%%%%%%%%%%%%%%%%%%%%%%%%%%%%%%%%%%%%%%
%%%% Références
\begin{center}
\begin{tabularx}{\textwidth}{| X | X | c | c |}
  \hline
  \rowcolor{gray!20}\multicolumn{4}{c}{Bibliographie de la leçon : } \\
  \hline 
  Titre & Auteurs & Editeur (année) & ISBN \\
  \hline
  Tout-en-un PC/PC* & M.-N. Sanz & Dunod (2022) & \\
  \hline 
   \url{http://www.etienne-thibierge.fr/agreg/ondes_poly_2015.pdf} & Ethienne Thibierge & 2015 &    \\
   \hline
   \url{http://images.math.cnrs.fr/Spectre.html\#nh7} &  San Vu Ngoc & CNRS & \\
  \hline 
  Ondes H-prépa & J-M Brébec & Hachette (2004) & \\
  \hline
  \url{https://dropsu.sorbonne-universite.fr/s/nyD9Ppz3kH6BHZE} & Clément Sayrin & & \\
\end{tabularx}
\end{center}

\begin{reportBlock}{Commentaires des années précédentes :}
    \begin{itemize}
        \item \textbf{2015 :} Les candidats doivent être attentifs à bien équilibrer leur exposé entre ces deux familles d’ondes qui, d’ailleurs, ne s’excluent pas entre elles,
        \item \textbf{2014 :} À l’occasion de cette leçon, le jury tient à rappeler une évidence : avec un tel titre, la leçon doit être équilibrée et ne peut en aucun cas se limiter pour l’essentiel aux ondes progressives.
    \end{itemize}
\end{reportBlock}

%%%%%%%%%%%%%%%%%%%%%%%%%%%%%%%%%%%%%%%%%%%%%%%%%%%%

%%%%%%%%%%%%%%%%%%%%%%%%%%%%%%%%%%%%%%%%%%%%%%%%%%%%
%%%% Plan
\begin{reportBlock}{Plan détaillé}

  \textbf{Niveau choisi pour la leçon :} CPGE 2ème année
  \newline
  \textbf{Prérequis} : \begin{itemize}
      \item Forces, énergie
      \item 
      \item 
  \end{itemize}
  
  \section*{Introduction}
\textcolor{green}{Slide} Quel est le point commun entre tous ces phénomènes : Olà dans un stade, effet d'une goutte qui tombe dans une flaque, un séisme, une corde soumise à des vibrations ? L'objectif de cette leçon est de donner les similitudes entre ces phénomènes et de pouvoir en retirer des caractéristiques générales.

  \section{Approches du phénomène de propagation}

  \subsection{Elongation d'une corde sans raideur}
  Voir Hprépa p30.
  \textbf{Hypothèses :} On regarde des petites perturbations de la hauteur de la corde notée $y(x,t)$. 
  Obtenir l'équation de d'Alembert unidimentionnelle.\\
  
  Donner les équations couplant $y(x,t)$ et $T(x,t)$ (p32) en disant que le phénomène de propagation est contenue dans ces équations liant vitesse transverse $\fracD{y}{t}$ et la projection de la tension suivant $\mathbf{\hat{u_y}}$.\\

  \textcolor{red}{Transition : Ce couplage entre deux grandeurs rappelle celui de la tension et du courant. On peut effectivement faire l'analogie de la propagation de U et i dans un câble coaxial avec la tension et la déformation de la corde.}

  \subsection{Analogie électrocinétique : le câble coaxial} 
  Voir HPrépa p58. \textcolor{blue}{Prendre câble coaxial, montrer l'âme et la gain et le diélectrique}. Faire le schéma d'un câble coaxial avec âme et gaine et le schéma électriue équivalent.\\
  \textbf{Hypothèses :} On ne se place plus dans l'ARQS pour le câble tout entier car on veut voir des phénomènes de propagation mais on peut le découper en tronçons dans lesquels la propagation est négligeable. \\
  
  Obtenir les équations couplées (loi des mailles et loi des n\oe uds) et obtenir l'équation de d'Alembert unidimentionnelle.\\
  \textcolor{blue}{Manip quantitative : mesure de la vitesse de propagation de i ou u dans le câble coaxial avec un câble de 100m de long. Envoyer un pulse d'impulsion 100ns, montrer le décalage sur la voie Y et mesurer ce décalage avec les curseurs. On obtient $v=\frac{2c}{3}$. On peut en déduire $n=\frac{v}{c}=\sqrt{\epsilon_r}$}.

  Définir la notion d'onde (poly Thibierge p10) : \og Une onde correspond à la propagation d’une perturbation à travers un milieu. Cette perturbation est générée par une source, qui apporte de l’énergie au milieu. L'existence de deux grandeurs couplées qui se créent l'une l'autre est à la base du phénomène de propagation.\\

  Dans le cas de la corde, une personne à mis en mouvement la corde. Dans le cas de la ligne, un expérimentateur a décidé de mettre un GBF. Dans le cas de la olà, un groupe de personne déclanche ce phénomène. Dans le cas de la goutte d'eau, un robinet qui fuit a amené de l'eau au milieu.\\

  \textcolor{green}{Slide : tableau analogie}
  
  \textcolor{red}{On a décrit le phénomène de propagation comme l'existence de deux grandeurs couplées qui se propagent dans un milieu. On va maintenant voir la forme des solutions de l'équation de d'Alembert.}
  
  

  \section{Ondes progressives}
  \subsection{Solution de l'équation de d'Alembert unidimentionnelle}
  A voir si on a le temps de la démontrer.
  
  \subsection{Interprétation}
  Voir Tec\&Doc p128 ou Dunod p806-807 et Thibierge Exos C1. Faire les dessins avec correspondance temporelle et spatiale.

  \subsection{OPPH}
  Idée : tout signal périodique peut se décomposer en série de Fourier dont chaque terme est solution de l'équation de d'Alembert. Cela est possible par la linéarité de l'équation.\\
  Obtenir la relation de dispersion. Définir vitesse de phase et vitesse de groupe. 
  
  \section{Onde stationnaire}
  On peut imaginer une onde progressive et une onde regressive qui se propagent et qui se rencontrent. 
  

  \subsection{Corde de Melde}

  \section{Conclusion}
  Conclure sur absorption, dispersion, impédance. Prendre exemple Dunod p818 en introduisant une force de frottement.


\end{reportBlock}
\newpage
%%%%%%%%%%%%%%%%%%%%%%%%%%%%%%%%%%%%%%%%%%%%%%%%%%%%
%%%% En-tête leçon
\begin{headerBlock}
  \chapter{Ondes acoustiques.}
  \label{LP_OndeAcoustique} 
\end{headerBlock}




%%%%%%%%%%%%%%%%%%%%%%%%%%%%%%%%%%%%%%%%%%%%%%%%%%%%
%%%% Références
\begin{center}
\begin{tabularx}{\textwidth}{| X | X | c | c |}
  \hline
  \rowcolor{gray!20}\multicolumn{4}{c}{Bibliographie de la leçon : } \\
  \hline 
  Titre & Auteurs & Editeur (année) & ISBN \\
  \hline
  Tout-en-un PC/PC* & M.-N. Sanz & Dunod (2022) & \\
  \hline 
   \url{http://www.etienne-thibierge.fr/agreg/ondes_poly_2015.pdf} & Ethienne Thibierge & 2015 &  \\
  \hline 
  Ondes H-prépa & J-M Brébec & Hachette (2004) & \\
  \hline
  \url{https://dropsu.sorbonne-universite.fr/s/nyD9Ppz3kH6BHZE} & Clément Sayrin & & \\
  \hline
  Ondes acoustiques & A. Chaigne & Ecole Polytechnique (2011) & \\
  \hline
\end{tabularx}
\end{center}

%%%%%%%%%%%%%%%%%%%%%%%%%%%%%%%%%%%%%%%%%%%%%%%%%%%%

%%%%%%%%%%%%%%%%%%%%%%%%%%%%%%%%%%%%%%%%%%%%%%%%%%%%
%%%% Plan
\begin{reportBlock}{Plan détaillé}

  \textbf{Niveau choisi pour la leçon :} 
  \newline
  \textbf{Prérequis} : \begin{itemize}
      \item 
      \item 
      \item 
  \end{itemize}
  
  \section*{Introduction}

  \section{}

  \subsection{}   

  \section{}

  \subsection{}
  
  \section{}
  


\end{reportBlock}
\newpage
%%%%%%%%%%%%%%%%%%%%%%%%%%%%%%%%%%%%%%%%%%%%%%%%%%%%
%%%% En-tête leçon
\begin{headerBlock}
  \chapter{Propagation guidée des ondes.}
  \label{LP_PropagationGuidee} 
\end{headerBlock}




%%%%%%%%%%%%%%%%%%%%%%%%%%%%%%%%%%%%%%%%%%%%%%%%%%%%
%%%% Références
\begin{center}
\begin{tabularx}{\textwidth}{| X | X | c | c |}
  \hline
  \rowcolor{gray!20}\multicolumn{4}{c}{Bibliographie de la leçon : } \\
  \hline 
  Titre & Auteurs & Editeur (année) & ISBN \\
  \hline
   Electromagnétisme & Pérez & Dunod & \\
  \hline 
    & & &    \\
  \hline 
   &  & &    \\
  \hline 
\end{tabularx}
\end{center}

%%%%%%%%%%%%%%%%%%%%%%%%%%%%%%%%%%%%%%%%%%%%%%%%%%%%

%%%%%%%%%%%%%%%%%%%%%%%%%%%%%%%%%%%%%%%%%%%%%%%%%%%%
%%%% Plan
\begin{reportBlock}{Plan détaillé}

  \textbf{Niveau choisi pour la leçon :}
  \newline
  \textbf{Prérequis} : \begin{itemize}
      \item 
  \end{itemize}

  \textbf{Déroulé détaillé de la leçon: }  
  
  \section*{Introduction}

  \section{Propagation}
  \subsection{Champs couplés}
  \subsection{Impédance caractéristique}
  Montrer ce qui se passe si $Z=+\infty$, $Z=0$ et $Z=Z_c$.
  
  \section{Ondes TEM dans un câble coaxial}

  \subsection{Equation des télégraphistes}

  \subsection{Rapport des ondes stationnaires (ROS)}

  \section{Ondes centimétrique en propagation guidée}

  \textcolor{blue}{Expérience quantitative :} Mesure de la relation de dispersion dans un guide et mesure de la fréquence de propagation dans un guide d'onde (TP Onde 2).
  

\end{reportBlock}
\newpage
%%%%%%%%%%%%%%%%%%%%%%%%%%%%%%%%%%%%%%%%%%%%%%%%%%%%
%%%% En-tête leçon
\begin{headerBlock}
  \chapter{Microscopies optiques}
  \label{LP_Microscopie} 
\end{headerBlock}




%%%%%%%%%%%%%%%%%%%%%%%%%%%%%%%%%%%%%%%%%%%%%%%%%%%%
%%%% Références
\begin{center}
\begin{tabularx}{\textwidth}{| X | X | c | c |}
  \hline
  \rowcolor{gray!20}\multicolumn{4}{c}{Bibliographie de la leçon : } \\
  \hline 
  Titre & Auteurs & Editeur (année) & ISBN \\
  \hline
  Les instruments d'optique & Luc Detwiller & Ellipses (1997) & \\
  \hline 
  \url{http://ressources.agreg.phys.ens.fr/media/ressources/RessourceFichiers/11-Maxime_Dahan_-_Microscopie_pour_la_biologie.pdf} & M. Dahan & Vraiment bien & \\
  \hline
  \url{https://www.nikonsmallworld.com/} & Nikon & &    \\
  \hline 
  Les nouvelles microscopies & L. Aigouy & Belin &   \\
  \hline 
  Optique & Sylvain Houard & de Boeck & \\
  \hline
  Optique & Eugène Hecht & Pearson & \\
  \hline
  OPtique Physique & R. Taillet & de Boeck (2006) & \\
  \hline
\end{tabularx}
\end{center}

%%%%%%%%%%%%%%%%%%%%%%%%%%%%%%%%%%%%%%%%%%%%%%%%%%%%
\begin{reportBlock}{Commentaires des années précédentes :}
    \begin{itemize}
        \item \textbf{2017 :} L’intérêt des notions introduites doit être souligné,
        \item \textbf{2016 :} Une technique récente de microscopie optique à haute résolution doit être présentée,
        \item \textbf{2010 :} La propagation guidée ne concerne pas les seules ondes électromagnétiques ou optiques. Il faut insister sur les conditions aux limites introduites par le dispositif de guidage.
    \end{itemize}
\end{reportBlock}
%%%%%%%%%%%%%%%%%%%%%%%%%%%%%%%%%%%%%%%%%%%%%%%%%%%%
%%%% Plan
\begin{reportBlock}{Plan détaillé}

  \textbf{Niveau choisi pour la leçon :} 
  \newline
  \textbf{Prérequis} : \begin{itemize}
      \item Optique géométrique : lentilles, construction images par une lentille
      \item Diffraction par un cercle
      \item 
  \end{itemize}

  \textbf{Déroulé détaillé de la leçon: }  
  
  \section*{Introduction}
  L'homme a voulu voler, voir loin dans l'Univers mais aussi voir l'infiniment petit.\\
  On appelle micrsocopie l'ensemble des techniques permettant de rendre discernables et visuels des objets indiscernables à l'\oe il nu (environ 1 minute d'arc = $0.017\degree$ pour une vision 10/10 soit environ 100km de la surface de la Lune). 

  \section{Le principe du microscope}

  \subsection{Présentation du dispositif}
  Faire le schéma du microscope.\\
  \textcolor{green}{L'objectif} : lentille convergente de courte focale qui fait une image intermédiaire $A_1B_1$ de l'objet $AB$ à aggrandir. \textcolor{green}{L'oculaire} : lentille convergente faisant une image à l'infini de $A_1B_1$ pour que l'\oe il qui s'y accole n'accomode pas. On doit donc avoir $\bar{O_2A_1}=-f_2'$.

  \subsection{Grossissement commercial}
  \textcolor{green}{Grossissement oculaire }: $G_{c,oc}$, rapport entre l'angle sous lequel est vu l'objet ($A_1B_1$) à travers l'oculaire et l'angle sous lequel est vu le même objet à travers $G_{c,oc}=\frac{\alpha'}{\alpha_1}=\frac{A_1B_1/f_2'}{A_1B_1/PP}=\frac{PP}{f_2'}$.\\
  \textcolor{green}{Grandissement objectif} : $`\gamma_{obj}=\frac{\bar{A_1B_1}}{AB}=-\frac{\Delta}{f_1'}$ par le théorème de Thalès+formule de conjugaison de Descartes.\\
  \textcolor{green}{Grandissement commercial} : $G_{c}=\frac{\alpha'}{\alpha}=\frac{A_1B_1/f_2'}{AB/PP}=\lvert \gamma_{obj} \rvert G_{c,oc} = \frac{PP\Delta}{f_1'f_2'}$\\  
  Le grossissement commercial d'un microscope est donné par :
  \begin{equation}
     G_{com} = \lvert \gamma_{ob}\rvert G_{oc} = \frac{\alpha'}{\alpha}
  \end{equation}
 \textcolor{blue}{Expérience quantitative :} Faire l'image d'une mire micrométrique par un microscope optique. Pour cela : \begin{itemize}
      \item utiliser une lampe quartz-iode/LED,
      \item 1 condenseur de 8 ou 12 cm pour focaliser l lumière sur la mire
      \item un microscope avec une mire micrométrique (pas 0.1mm),
      \item une lentille de focale 1m ou 150cm,
      \item fixer l'écran à la distance focale de la lentille,
      \item ajuster le microscope pour avoir une image nette sur l'écran
  \end{itemize} 
 On mesure $\alpha'= \frac{Taille-objet-sur-l'ecran}{distance-objet-ecran}$ ainsi que les incertitudes associées. 
 \importantbox{On doit faire des traits bien droits pour la mesure de la distance à l'écran. Pour cela, prendre une feuille et la coller à l'écran, reproduire les traits sur la feuille. Faire la mesure à la règle proprement sur la feuille en traçant des angles droits.}
 En préparation, j'ai trouvé $G_{com}=40.5(3)$ à comparer à 40. Bon ordre de grandeur, la valeur peut-être différente de 40, le microscope ne coûte pas cher et pas d'incertitudes sur la valeur du constructeur.\\
  
  %Parler de : 
  %\begin{itemize}
   %   \item conditions d'éclairement (éclairage Köhler
    %  \item résolution optique (ouverture numérique, diffraction, aberrations) 
     % \item contraste
      %\item microscopie plein champ/ point par point
  %\end{itemize}

  \textcolor{green}{Slide :}Montrer le nouveau microscope par rapport à celui de 1930 sur slide.
  
  \subsection{Eclairage de Köhler}
  Voir Wastiaux p130. En parler rapidement sur slide, montrer qu'on fait l'image du filament de la source à l'infini pour ne pas être gêné par celui-ci. Parler du diaphragme de champ et d'ouverture.
  
  \textcolor{red}{Transition :} Est-ce que si je prends des focales f' ou infinement petites, je peux observer des choses infinement petites ? On va voir que non car il y a des limitations.


  \section{Limitations (trouver mieux pour le nom)}

  \subsection{Résolution latérale : critère de Rayleigh}
  p136 Aigouy et Taillet p.227. Présenter l'ouverture numérique. Plus on augmente en $\Delta k$ plus on est résolu en $\Delta x$. Faire des applications numériques voir Detwiller p110. Parler des objectifs à immersion en faisant un dessin.
  
  \subsection{Profondeur de champ}
  Faire le dessin avec trois points sources. Il faut diaphragmer pour améliorer la profondeur de champ.

  \subsection{Aberrations (partie qui peut sauter)}
  Sur slide ? Parler des lentilles achromatiques, apochromatique.

  \textcolor{red}{Transition :} Comment rendre visible des choses que ne le sont pas ?
  
\section{Microscopie à contraste de phase}
Cette technique s'intéresse en particulier à des échantillons transparents dont les épaisseurs sont faibles \textcolor{green}{Slide photos avec ou sans contraste de phase + photos microscopies Nikon}. Elle a valu le prix Nobel à Frederik Zernike en 1953.
\subsection{Principe}
Voir Hecht p635. On envoie de la lumière sur un objet dit de phase qui va modifier localement la phase de la lumière incidente :
\begin{equation}
    E_i = E_0e^{i\phi} = E_0 + E_0\times(e^{i\phi}-1) \sim 
\end{equation}
On fait passer la lumière à travers une lentille qui va donner la figure de diffraction dans fait l'image de cette 


\section*{Ouverture}
Microscopie électronique par effet tunnel

\end{reportBlock}
\newpage
%%%%%%%%%%%%%%%%%%%%%%%%%%%%%%%%%%%%%%%%%%%%%%%%%%%%
%%%% En-tête leçon
\begin{headerBlock}
  \chapter{Interférences à deux ondes en optique}
    \label{LP_InterferencesDeuxOndes}
\end{headerBlock}

%%%%%%%%%%%%%%%%%%%%%%%%%%%%%%%%%%%%%%%%%%%%%%%%%%%%
%%%% Références
\begin{center}
\begin{tabularx}{\textwidth}{| X | X | c | c |}
  \hline
  \rowcolor{gray!20}\multicolumn{4}{c}{Bibliographie de la leçon : } \\
  \hline 
  Titre & Auteurs & Editeur (année) & ISBN \\
  \hline
Physique Spé MP-MP* & Olivier, Gié, Sarmant & Tec \& Doc & \\
  \hline 
  Sextant &  & Hermann &  \\
  \hline 
   Tout-en-un, MP & M.-N. Sanz. & Dunod &  \\
   \hline
   Optique & Eugene HECHT & Pearson (2005) & \\
   \hline
   Optique & S. Houard & de Boeck & \\
   \hline
\end{tabularx}
\end{center}

%%%%%%%%%%%%%%%%%%%%%%%%%%%%%%%%%%%%%%%%%%%%%%%%%%%%
\begin{reportBlock}{Plan détaillé}
  \textbf{Niveau choisi pour la leçon :} CPGE
  \newline
  \textbf{Prérequis : }Modèle scalaire d'une onde, chemin optique, différence de marche, intensité lumineuse, formules trigonométriques
  \newline
  
  \textbf{Déroulé détaillé de la leçon: } \newline
\textcolor{green}{Manip introductive :} si on superpose deux lasers, il ne se passe rien. Si on les fait passer à travers un dispositif qui élargit le faisceau + une fente source + une bifente : on voit une figure d'interférence.
  \section{Interférences à deux ondes}
  \textcolor{red}{Définition :} phénomène ondulatoire qui résulte d'une interaction entre deux ondes (lumineuses) qui produit une intensité totale qui diffère de la somme des intensités individuelles.
  \subsection{Superposition de deux ondes}
  On considère deux sources ponctuelles $S_1$ et $S_2$ et des amplitudes vibratoires $a_i(M,t)=A_i\cos\left(\omega_it-\phi_{S_i} - \frac{2\pi[S_iM]}{\lambda_0i}\right)$. L'amplitude totale est : $a(M,t)=a_1(M,t)+a_2(M,t)$. L'intensité est : $I(M,t) = <a^2(M,t)>$.\\

  En développant, on obtient :
  \begin{equation}
      I = I_1 + I_2 + I_{1,2}
  \end{equation}
  avec $I_{1,2} = 2A_1A_2\cos\left(\omega_1t-\phi_1(M)\right)\cos\left(\omega_2t-\phi_2(M)\right)$

  \subsection{Conditions d'interférence, notion de cohérence}
  \begin{itemize}
      \item $I_{1,2}\neq$, dans ce cas on dit que les ondes sont cohérentes,
      \item si $\omega_1\neq\omega_2$, $I_{1,2}=0$
  \end{itemize}

  \underline{\textcolor{red}{Condition 1 :}} deux ondes de pulsations différentes sont incohérentes.\\

  Présentation du modèle du train d'onde : paquet d'onde séparés par un temps $\tau$. Comme $\phi_{S1}$ et $\phi_{S2}$ varient aléatoirement, on obtient $I_{1,2}$ non nulles sur le détecteur si :\\
  \underline{\textcolor{red}{Condition 2 :}} Il faut que les deux ondes soient issus du même train d'onde. \\

  On obtient alors la formule de Fresnel :
  \begin{equation}
      I = I_1 + I_2 + 2\sqrt{I_1I_2}\cos{\Delta\phi(M)}  \end{equation}
      où $\Delta\phi(M) = \frac{2\pi([S_2M]-[S_1M])}{\lambda_0}$

\section{Exemple d'interféromètre : les trous d'Young}
Cf photo.

\begin{equation}
    I = 2I_0\left[1+\cos\left(\frac{2\pi ax}{D}\right)\right]
\end{equation}
Succesion de franges brillantes et de franges sombres. La distance entre deux franges brillantes est appelée \textcolor{red}{interfrange} notée $i$ qui vaut ici : $i=\frac{\lambda_0D}{a}$.\\
\textcolor{green}{Manipulation 2 (quantitative) :} Mesure de l'interfrange de la figure d'interférences pour en déduire $a$. On mesure $a=0.16\pm0.04$mm à comparer avec la valeur $a_{fabricant}=0.2$mm.

\section{Notion de cohérence spatiale}
Effet de la largeur de la source en reprenant le problème avec deux sources séparées par une distance $b$. On obtient à l'aide des formules obtenues dans la partie précédente : 
\begin{equation}
    I_{tot} = 4I_0\left[1+\cos\left(\frac{\pi ab}{\lambda D}\right)\cos\left(\frac{2\pi a x}{\lambda D}+\frac{2\pi a b}{\lambda D}\right)\right]
\end{equation}
\section{Conclusion}
Ouverture sur les dispositifs à division du front d'onde et division d'amplitude.
\end{reportBlock}


\begin{reportBlock}{Questions posées par l’enseignant (avec réponses)}
   \textbf{C : D'autres phénomènes d'interférences autres que lumineuses ?}  \textcolor{purple}{Oui, exemple de la cuve à onde.} Qu'est-ce qui fait la spécificité des interférences des ondes lumineuses ? \textcolor{purple}{On peut faire des mesures super précises.}\\
   \textbf{C : Conditions de cohérence pour l'eau ?}  \textcolor{purple}{On somme directement les amplitudes, il n'y a pas de notion de cohérence pour une onde mécanique.}\\
   \textbf{C : Dépendence de la durée d'intégration ? Odg temps de réponse d'un détecteur ?}  \textcolor{purple}{Période de la lumière  : $10^{-15}$s, \oe uil : $10^{-2}$s,  photorésistance $10^{-2}$s, photodiode (standard): $10^{-6}$s, thermopile : $1$s}\\
   \textbf{C : lien entre intensité $I$ et éclairement $\epsilon$ ?}  \textcolor{purple}{On a $\epsilon = KI = K<s^2(M,t)>$, où $<...>$ représente la valeur moyenne temporelle, K est une constante qui dépend du détecteur et $s(M,t)$ représente une composante du champ électrique de la lumière par rapport à un axe perpendiculaire à sa direction de propagation. L'éclairement est la puissance surfacique moyenne de l'onde lumineuse (autrement dit la valeur moyenne temporelle du vecteur de Poynting).}\\
   \textbf{C : Pourquoi il faut un vide entre deux trains d'ondes ?}  \textcolor{purple}{Lié à la désexcitation de l'atome, la durée de vie d'un niveau d'énergie.} Un train d'onde c'est un photon du coup ? \textcolor{purple}{C'est l'aspect ondulatoire du photon.}\\
   \textbf{C : C'est quoi la cause de l'incohérence spatiale ?} \textcolor{purple}{Emission de trains d'onde de phase à l'origine aléatoire suivant l'atome émetteur.}\\

   \textbf{C : Différences/avantages interférométrie à division d'amplitude/ division du front d'onde ?} \textcolor{purple}{Division du front d'onde : on fait interférer de la lumière provenant de deux sources différentes. Les interférences ne sont pas localisées mais il y a un problème de brouillage du fait de la cohérence spatiale des sources. Division d'amplitude : on fait interférer de la lumière provenant d'un même faisceau incident dont on a séparé en deux (au moins) l'amplitude. Il n'y a pas de problème lié à la cohérence spatiale de la source mais le prix à payer est la localisation des interférences (à l'infini pour une lame d'air, à distance finie pour un coin d'air). L'avantage est de pouvoir utiliser des sources de lumière très étendues, on gagne en luminosité.}\\

   \textbf{C : Stratégies à mettre en \oe uvre pour éviter $20\%$ d'erreur sur les mesures ?} \textcolor{purple}{Caméra CCD, mettre une lentille pour agrandir l'image} Ca change quoi avec une lentille ? \textcolor{purple}{On remplace $D$ par $f'$ dans la formule de $I_{tot}$.} C'est mieux du coup ? \textcolor{purple}{On peut mesurer $f'$ de façon assez précise} Quoi d'autre ? \textcolor{purple}{Pied à coulisse, banc optique, ...}\\
   
\end{reportBlock}

\newpage
%%%%%%%%%%%%%%%%%%%%%%%%%%%%%%%%%%%%%%%%%%%%%%%%%%%%
%%%% En-tête leçon
\begin{headerBlock}
  \chapter{Interférométrie à division d'amplitude}    \label{LP_DivisionAmplitude}
\end{headerBlock}



%%%%%%%%%%%%%%%%%%%%%%%%%%%%%%%%%%%%%%%%%%%%%%%%%%%%
%%%% Références
\begin{center}
\begin{tabularx}{\textwidth}{| X | X | c | c |}
  \hline
  \rowcolor{gray!20}\multicolumn{4}{c}{Bibliographie de la leçon : } \\
  \hline 
  Titre & Auteurs & Editeur (année) & ISBN \\
  \hline
   Optique physique et électronique  & Daniel Mauras & Presse Universitaire de France (2004) &  \\
  \hline 
   Optique Physique & Richard Taillet & de boeck (2015) & 953-087864-7\\
  \hline 
   Optique & Sylvain Houard & de boeck (2014) & \\
  \hline 
  Optique et phyique ondulatoire & Bertin Faroux Renault & Dunod (1986) & \\
  \hline
\end{tabularx}
\end{center}

\begin{reportBlock}{Commentaires des années précédentes :}
    \begin{itemize}
        \item \textbf{2017 :} Le candidat doit réfléchir aux conséquences du mode d’éclairage de l’interféromètre (source étendue, faisceau parallèle ou non...). Il est judicieux de ne pas se limiter à l’exemple de l’interféromètre de Michelson,
        \item \textbf{2016 :} La distinction entre divisions du front d’onde et d’amplitude doit être précise. Le jury rappelle que l’utilisation d’une lame semi-réfléchissante ne conduit pas nécessairement à une division d’amplitude,
        \item \textbf{2015 :} Les notions de cohérence doivent être présentées,
        \item \textbf{2014 :} Un interféromètre comportant une lame séparatrice n’est pas obligatoirement utilisé en diviseur d’amplitude. La notion de cohérence et ses limites doivent être discutées
    \end{itemize}
\end{reportBlock}


%%%%%%%%%%%%%%%%%%%%%%%%%%%%%%%%%%%%%%%%%%%%%%%%%%%%
%%%% Plan
\begin{reportBlock}{Plan détaillé}
  \textbf{Niveau choisi pour la leçon :} Licence 3
  \newline
  \textbf{Prérequis :} Optique Géométrique, Interférences à deux ondes, Interférences à division du front d'onde, cohérence temporelle/cohérence spatiale
  \newline
  
  \textbf{Déroulé détaillé de la leçon: }\newline
  \section*{Introduction : interférences à division du front d'onde}
  Interférences sont beaucoup utilisées en physique et même dans la vie de tous les jours : exemple du casque anti-bruit qui fait interférer les ondes sonores de l'extérieur avec celles du son dans le casque audio.\\
  Dans la suite, on s'intéresse aux interférences à deux ondes lumineuses.\\
  Retour sur les fentes d'Young (\textcolor{green}{slide 1}). Avec un éclairement des fentes par une source ponctuelle, on observe des franges d'interférences non localisées sur un écran.\\
  \textbf{Problème :} (\textcolor{green}{slide 2}) quand on augmente la taille de la source, on observe un brouillage de la figure d'interférences et donc une perte de contraste : problème de cohérence spatiale de la source.\\
  On va voir que l'interférométrie à division d'amplitude permet de s'affranchir de se problème.
  Annonce du plan : 
  \begin{itemize}
      \item Division d'amplitude
      \item Un exemple d'interféromètre à division d'amplitude
      \item Une application pratique de la division d'amplitude
  \end{itemize}
  
  \section{Division d'amplitude}
  \subsection{Théorème de localisation}
  Considérons un système interféromètrique éclairé par une source monochromatique étendue tel que représenté sur la Figure \ref{fig:localisation}. On souhaite qu'au point M, les intensités lumineuses provenant de S et S' qui s'ajoutent ne produisent pas de brouillage. Autrement dit, nous avons la condition de non-brouillage au point M suivante :\newline
  \textcolor{red}{Condition de non-brouillage :} $\Delta(M)=\delta(S',M)-\delta(S,M)=0$ \newline
  On montre (dans D. Mauras p. 159) que :
  \begin{equation}
      -n(\mathbf{u}_2-\mathbf{u}_1)\cdot\mathbf{SS'}=0
  \end{equation}
  Cette équation nous permet de distinguer deux cas :
  \begin{itemize}
      \item $\mathbf{u}_2\ne\mathbf{u}_1$ : c'est une contrainte sur la source, les rayons qui interfèrent en M ne sont pas issus du même rayon incident. C'est le cas des interféromètres à \textbf{division du front d'onde}.
      \item $\mathbf{u}_2=\mathbf{u}_1$ : c'est une contrainte sur l'interféromètre. Les rayons qui interfèrent sont issus du même rayon incident. Le système optique doit contenir une lame séparatrice qui divise en deux un rayon incident puis fait interférer les deux rayons ainsi créés: c'est \textbf{la division d'amplitude}.
  \end{itemize}
  \textcolor{red}{Théorème de localisation :} (\textcolor{green}{Slide 3}) Seuls les dispositifs à division d'amplitude peuvent donner des interférences contrastées produites par des sources arbitrairement larges. Ces interférences sont alors \textbf{localisées} au voisinage des points d'intersection des couples de rayons lumineux issus du même rayon incident.
  
  \subsection{Lame d'air}
  Un exemple simple d'interféromètre à division d'amplitude est celui de la lame d'air. On peut montrer que la différence de marche entre deux rayons lumineux sortant de la lame d'air est donné par :
  \begin{equation}
      \delta(M) = 2ne\cos{i}
  \end{equation}
  avec $e$ l'épaisseur de la lame d'air et $i$ l'angle d'incidence du rayon lumineux provenant de la source S.
  On peut faire quelques remarques : 
  \begin{itemize}
      \item Les interférences sont localisées à l'infini, il faut donc une lentille convergente et un écran placé dans le plan focal image de cette dernière pour observer la figure d'interférences,
      \item Pour une même épaisseur $e$, l'intensité lumineuse ne dépend que de $i$ : on observe des franges circulaires dites \textbf{d'égales inclinaison} qu'on comprend bien en analysant les hyperboloïdes d'interférences (\textcolor{green}{slide 4}),
      \item Pour une source étendue, les franges sont plus brillantes.
  \end{itemize}
  On va voir à présent un système interférométrique réalisant effectivement la lame d'air : il s'agit de l'interféromètre de Michelson.
  
  \section{Interféromètre de Michelson}
  Rappel historique : Expérience de Michelson et Morlay au 19$^{e}$ qui démontre que la vitesse de la lumière est la même dans toutes les directions. Prix Nobel de Physique décerné à Albert Michelson en 1907.
  \subsection{Présentation du montage en configuration lame d'air}
  Le schéma de principe est présenté (\textcolor{green}{slide 5}). Schéma équivalent au tableau tel que représenté sur la Figure \ref{fig:Michelson}.
  Le Michelson n'est certes pas sensible à la cohérence spatiale de la source mais est sensible à sa cohérence temporelle : $\delta(M)\sim 2e<L_c=c\tau_c=c/\Delta\nu=\Delta\lambda/\lambda^2 $. En odg : 
  \begin{itemize}
      \item laser He-Ne à $\Delta\nu=10MHz$, $L_c\sim 30$m
      \item lumière blanche, $L_c\sim 1\mu$m
  \end{itemize}
  Ainsi pour régler le Michelson, on commence par utiliser un laser qui est une source spatialement et temporellement cohérente. Si on souhaite changer de source temporellement moins cohérente que le laser, on se met au contact optique ($e=0$) en utlisant a propriété suivante :on peut montrer que le rayon des anneaux est proportionnel à $1/\sqrt{e}$, ainsi si $e$ diminue, les anneaux s'aggrandissent et rentrent vers leur centre.\newline
  $\rightarrow$ \textcolor{blue}{observation de la figure d'interférence d'un Michelson éclairé par lampe à vapeur de sodium. On voit les anneaux avec une lentille de $f'=20$cm (faible luminosité de la source et salle assez éclairée, utiliser un condenseur à la sortie de la source pour maximiser l'éclairement sur les miroirs) et on peut repérer le contact optique. On observe également un brouillage des franges pour certaines positions du miroir (M1) car doublet du sodium.}\newline
  On va mesurer l'écart en longueur d'onde de ce doublet.
  
  \subsection{Mesure interféromètrique du doublet du sodium}
  On rappelle la formule de l'éclairement pour deux sources lumineuses de longueur d'onde différente (\textcolor{green}{slide 6}). L'écart $\Delta e$ entre deux anticoïncidences est donné par :
  \begin{equation}
      \Delta e = \frac{\bar{{\lambda}}}{2\Delta\lambda^2}
  \end{equation}
  avec $\Delta\lambda =|\lambda_2-\lambda_1|$ et $\bar{{\lambda}}=\frac{\lambda_1+\lambda_2}{2}$.\newline
  $\rightarrow$ \textcolor{blue}{S'éloigner du contact optique de ma nière à pouvoir visualiser 5 ou 6 anticoïncidences. Mesurer $\Delta e$, déterminer $\Delta\lambda$ avec son incertitude et comparer à la valeur tabulée $\Delta\lambda=0,597$nm (Encyclopedia Brittanica).}\\
  Grâce au Michelson, on est capable de faire des mesures interféromètriques très précises (inférieures au nm !).
  
  \section*{Conclusion}
  Présentation de la tomographie en cohérence optique (\textcolor{green}{Slide 7}). Voir R. Taillet.\\
  Ouverture sur le Fabry-Pérot avec la visualisation des anneaux du doublet du sodium obtenues par Fabry-Pérot (contraste des systèmes d'anneaux de $\lambda_1$ et $\lambda_2$) (\textcolor{green}{Slide 8})
\end{reportBlock}

\clearpage
  \begin{figure}[!htbp]
  \centering
  \includegraphics[scale=0.6]{LP_DivisionAmplitude/localisation.jpg}
  \caption{\label{fig:localisation}Système interférométrique (SO) éclairé par une source étendue. Les voies 1 et 2 constituent les voies de l'interféromètre. S et S' sont deux points sources.}
  \end{figure}
  
  \begin{figure}[!htbp]
  \centering
  \includegraphics[scale=0.4]{LP_DivisionAmplitude/Michelson.jpg}
  \caption{\label{fig:Michelson}Schéma équivalent du Michelson. S$_1$ et S$_2$ sont les images de S à travers respectivement la voie 1 (Miroir M1) et la voie 2 (miroir M2) de l'interféromètre. Le miroir (M1') est l'image de (M1) à travers la séparatrice.}
  \end{figure}
  \clearpage
  
  

%%%%%%%%%%%%%%%%%%%%%%%%%%%%%%%%%%%%%%%%%%%%%%%%%%%%
%%%% Questions
\begin{reportBlock}{Questions posées par l’enseignant (avec réponses)}
  \begin{enumerate}
      \item Types d'interférence pour le casque anti-bruit ? \\ \textcolor{purple}{Il y a un micro qui permet d'enregistrer le bruit ambiant. Le signal est ensuite analysé puis un haut-parleur génère un signal de bruit avec une phase exactement opposée à celui qui vient d'être enregistré pour que les deux signaux interfèrent destructivement. Il s'agit donc d'une division du front d'onde.}
      
      \item Utilisation des intérférences à division d'amplitude ? \\
       \textcolor{purple}{Regarder la structure spatiale d'un l'échantillon, voir des défauts de planéité des miroirs, mesurer la différence d'indice de réfraction d'un gaz, mesurer une variation de température}. 
       
    \item Comment mesure-t-on l'indice de réfraction d'un gaz ? \\
       \textcolor{purple}{En configuration lame d'air, on peut regarder comment changent les franges rectlignes lorsque l'indice de refraction $n$ du milieu varie. Connaissant la taille du miroir, on peut remonter à $n$.}
       
      \item Quelle configuration pour la planéité des miroirs ? \\
       \textcolor{purple}{Configuration coin d'air.}
       
      \item En configuration coin d'air, où sont localisées les intérférences ? \\
       \textcolor{purple}{On verra des franges au niveau des miroirs.}
       
     \item Pourquoi au niveau des miroirs ? \\
       \textcolor{purple}{Il faut considérer deux points sources $S_1$ et $S_2$ proches et tracer les rayons issus de ces points et passant par un point $P$. Si $P$ est proches du coin d'air, la différence de marche $\delta_1$ entre les rayons issus de $S_1$ et celle entre les rayons issus de $S_2$ sont quasiment les mêmes (et égales à $2e(P)$): il n'y a donc pas brouillage. En revanche, si $P$ s'éloigne du coin d'air, $\delta_1$ devient très différent de $\delta_2$ et on a brouillage (cf. corrigé du TD d'optique sur les intéreférences). \\
       Par ailleurs, une construction géométrique permet de montrer que le lieu d'intersection des rayons réfléchis correspondant \textbf{à un même rayon incident} (condition de localisation vu en $1.1$, à savoir $\mathbf{u_2} = \mathbf{u_1}$) est approximativement le plan faisant l'angle $i$ (où $i$ est l'angle d'incidence) avec (M1'). En pratique, cet angle est si petit (quelques minutes d'arc) que ce plan d'intersection est presque confondu avec (M1') (Dunod Physique tout-en-un 2ème année, PC-PC* 2004).}
    
    \item Lorsque la source est ponctuelle, les interférences sont-elles localisées ? \\
       \textcolor{purple}{Non. Elles le sont lorsque la source est étendue.}
       
    \item Si on a des trous d'Young et une source déplacée de $b$ sur l'axe parallèle à l'axe des trous, comment seront les franges ? \\
       \textcolor{purple}{Il y aura une différence de chemin optique additionnelle avant les trous. Les nouvelles franges obtenues se décalent de $\frac{-bD}{d}$, avec $d$ la distance entre la source et les trous d'Young et $D$ la distance entre les trous et l'écran.}
    
    \item Peut-il y avoir brouillage si on considère deux sources ponctuelles incohérentes ?\\
    \textcolor{purple}{Oui si les deux systèmes de franges créés par les sources sont en anticoïncidence (les franges brillantes d'un système de franges se superposent aux franges sombres de l'autre système de franges).}
       
    \item Pour la lame d'air, pourquoi appelle-t-on les anneaux "anneaux d'égale inclinaison" ? \\
       \textcolor{purple}{Un ordre d'interférence donné correspond à une inclinaison, c'est-à-dire à un même angle d'incidence des rayons lumineux.}
       
    \item Pour la lame d'air, est-ce qu'on a le même signe de refléxion des deux côtés ? \\
       \textcolor{purple}{Non, $r_1 < 0$, $r_2 > 0$. Il y a un déphasage de $\pi$ en plus. Dans le Michelson, on oublie car c'est plus compliqué que ça, il y a des traitements en plus. Ce qui compte c'est que le contact optique est défini non pas par la longueur géométrique mais par la longueur optique.}
    
    \item A quoi sert la compensatrice ? On pourrait s'en passer pour le laser en réglant la longueur entre les deux bras de sorte à compenser la différence de marche correspondant à l'épaisseur de la séparatrice. \\
       \textcolor{purple}{Si la source est polychromatique, on ne peut pas trouver un pas du miroir qui compense la différence de marche pour toutes les longueurs d'onde car l'indice optique de la séparatrice dépend de la longueur d'onde, d'où la nécessité d'utiliser une compensatrice. Si une seule longueur d'onde, ce n'est pas nécessaire (mais ça n'existe pas dans la vraie vie).}
    
    \item Pourquoi tu utilises un verre anticalorique ? \\
    \textcolor{purple}{Pour ne pas diffuser de la chaleur provenant de la source sur l'interféromètre, cela modifierait $n$ qui dépend de la température (le Michelson est sensible à des variations d'indice de réfraction de l'ordre de $10^{-4}$.)}
       
    \item Le laser He-Ne : 10 MHz. Lorsqu'on a fait le calcul on a trouvé 400 MHz. Pourquoi cette différence ? \\
       \textcolor{purple}{En réalité, le spectre comporte plusieurs raies. L'enveloppe est à 400 MHz mais la largeur de chaque raie est beaucoup moins: 10 MHz semble raisonnable.}

       
    \item Que mesures-tu dans la tomographie ? \\
       \textcolor{purple}{Les interférences associées à une certaine épaisseur.}
      
  \end{enumerate}
\end{reportBlock}


%%%%%%%%%%%%%%%%%%%%%%%%%%%%%%%%%%%%%%%%%%%%%%%%%%%%
%%%% Commentaires
\begin{reportBlock}{Commentaires lors de la correction de la leçon}
\end{reportBlock}


%%%%%%%%%%%%%%%%%%%%%%%%%%%%%%%%%%%%%%%%%%%%%%%%%%%%
%%%% Correction

\begin{reportBlock}{Partie réservée au correcteur}
  \textbf{Avis général sur la leçon (plan, contenu, etc.) :}
  
  
  \textbf{Notions fondamentales à aborder, secondaires, délicates :}
  
  
  \textbf{Expériences possibles (en particulier pour l'agrégation docteur) :}
  
  
  \textbf{Bibliographie conseillée :}
\end{reportBlock}

\newpage
%%%%%%%%%%%%%%%%%%%%%%%%%%%%%%%%%%%%%%%%%%%%%%%%%%%%
%%%% En-tête leçon
\begin{headerBlock}
  \chapter{Diffraction de Fraunhofer.}
  \label{LP_DiffractionFraunhofer} 
\end{headerBlock}




%%%%%%%%%%%%%%%%%%%%%%%%%%%%%%%%%%%%%%%%%%%%%%%%%%%%
%%%% Références
\begin{center}
\begin{tabularx}{\textwidth}{| X | X | c | c |}
  \hline
  \rowcolor{gray!20}\multicolumn{4}{c}{Bibliographie de la leçon : } \\
  \hline 
  Titre & Auteurs & Editeur (année) & ISBN \\
  \hline
  Tout-en-un PC/PC* & M.-N. Sanz & Dunod (2022) & \\
  \hline
  Optique - Fondements et applications & J.P. Pérez & Dunod (2011) & \\
  \hline
  \url{https://www.lkb.upmc.fr/cqed/teaching/teachingsayrin/} & C. Sayrin & & \\
  \hline
  Optique Physique & R. Taillet & de boeck (2015) & \\
  \hline
  Optique Physique & D. Mauras & PUF (2001) & \\
  \hline
\end{tabularx}
\end{center}

%%%%%%%%%%%%%%%%%%%%%%%%%%%%%%%%%%%%%%%%%%%%%%%%%%%%
\begin{reportBlock}{Commentaires des années précédentes :}
    \begin{itemize}
        \item \textbf{2017 :} Les conditions de Fraunhofer et leurs conséquences doivent être présentées, ainsi que le lien entre les dimensions caractéristiques d’un objet diffractant et celles de sa figure de diffraction,
        \item \textbf{2014-2011 :} Les conditions de l’approximation de Fraunhofer doivent être clairement énoncées. Pour autant, elles ne constituent pas le coeur de la leçon.
    \end{itemize}
\end{reportBlock}
%%%%%%%%%%%%%%%%%%%%%%%%%%%%%%%%%%%%%%%%%%%%%%%%%%%%
%%%% Plan
\begin{reportBlock}{Plan détaillé}

  \textbf{Niveau choisi pour la leçon :} Licence 3
  \newline
  \textbf{Prérequis} : 
  \begin{itemize}
      \item optique géométrique,
      \item modèle scalaire des ondes lumineuses,
      \item transformées de Fourier,
      \item microscopie optique ?
  \end{itemize}
  \importantbox{Le message essentiel de la leçon est que la figure de diffraction de Fraunhofer se trouve dans le plan image de la source qui éclaire l’objet diffractant}
  \section*{Introduction}
  
  \section{Conditions de Fraunhofer de la diffraction}

  \subsection{Principe d'Huygens-Fresnel} 
  Voir Pérez p263. Enoncé, faire le dessins. Introduire le facteur d'obiquité et dire qu'il est constant si on considère des rayons proches de l'axe optique.\\
  La démo de Huygens-Fresnel est sur Wikipédia \url{https://fr.wikipedia.org/wiki/Th\%C3\%A9orie_de_Kirchhoff}.\\
  Donner l'amplitude scalaire de l'onde $s(M)$ (prendre les notations de C. Sayrin).

  \subsection{Amplitude d'une onde à travers un diaphragme plan}
  Cf TD Clément Sayrin Diffraction (1).\\

  \importantbox{On peut garder $\frac{e^{ikr}}{r}\sim\frac{e^{ikr}}{D}$ mais pour évaluer la phase $\mathbf{k}\cdot \mathbf{PM}$, il faut tenir compte des variations de PM=r à l’échelle de $\lambda$ et l’on garde donc les ordres plus élevés du développement.}

  \subsection{Conditions et approximations de Fraunhofer}
  On néglige le terme quadratique en $r^2$ dans la phase de l'onde. Enoncer les conditions pour avoir le terme de Fraunhofer nul voir Sextant p139 :
  \begin{itemize}
      \item onde plane ($d=0$) à l'infini ($D=0$), il suffit de placer une source lumineuse dans le plan focal objet d'une lentille convergente et observer la figure dans le plan focal image d'une autre lentille convergente,
      \item en pratique, condition pas très restrictive car : onde plane (source à l'infini) mais $kr^2<<2D\Leftrightarrow D>>\frac{r^2}{4\lambda}$. Application numérique : pour un laser de longueur d'onde $\lambda=648nm$, on trouve $D>>500m$ et si $r=1cm$, $D>>5m$ si $r=1mm$, $D>>5\mu m$ si $r=1\mu m$. Il est très facile d'être dans les conditions de Fraunhofer pour les petits objets diffractants ! Pour les objets plus grand, il faut faire l'image de la source lumineuse par une lentille,
      \item cas particulier de la source sur l'écran (d=-D) (à omettre, garder pour les questions ?)
  \end{itemize}
  On voit quand même que l'amplitude de l'onde résultante est la transformée de Fourier de la fonction $t(x,y)$ de l'objet diffractant. Il y a une relation de réciprocité entre les distances angulaires de la figure de diffraction et les distances dans le plan d'ouverture de l'objet diffractant.

  \textcolor{red}{Transition :} on va mettre en application ce qu'on vient de détailler en observant la figure de diffraction d'objets simples.
  \section{Deux exemples de systèmes diffractants}
  
  \subsection{Diffraction par une fente}
  Faire le calcul + la manip. Il faut juste prendre $t(x,y)=rect_x(-a/2,a/2)$.\\
  \textcolor{blue}{Manip qualitative : mesure d'une largeur de fente}. Voir Poly TP Montrouge Diffraction. On a besoin 
  \begin{itemize}
      \item barette CCD + logiciel MIGHTEX,
      \item laser + boys + lentille f=50mm,
      \item ordinateur avec Spectra Suite + câble connexion caméra,
      \item fentes sources de différentes taille + fente de taille variable,
  \end{itemize}
  Montrer visuellement que plus la largeur de la fente diminue, plus l'espacement entre les maxima est grand. Parler de la réciprocité entre espace réel et espace de Fourier.\\

  \textcolor{red}{Transition : avec l'étude de la figure de diffraction de la fente, on peut en déduire celle d'un trou.}
  

  \subsection{Diffraction par un trou d'Young}
  Le faire rapidement sur slide ou en estimant à l'aide de la formule précédente : un trou de rayon $a$ est contenu entre deux carrés de côté $\sqrt{2}a$ donc :
  \begin{equation}
      \frac{\lambda}{a} < \theta < \frac{\sqrt{2}\lambda}{a}\sim 1.4\frac{\lambda}{a}
  \end{equation}
  En prenant la moyenne : $\theta \sim (1.2\pm0.2)\frac{\lambda}{a}$. Le calcul réel se fait dans le Taillet p154.\\
  Conséquence voir Perez p276-277 : 
  \begin{itemize}
      \item parler du critère de Rayleigh donnant la résolution latérale maximale d'un microscope optique standard,
      \item \textcolor{red}{Attention, nécessite le théorème de Babinet}. \textcolor{green}{slide} : tâches de diffraction circulaire autour de la Lune attribuées aux goutellettes d'eau/glace dans l'atmosphère. Calcul du diamètre "moyen" des gouttes d'eau : $\theta=\frac{1.22\lambda}{D}$, si on prend $\theta\sim6\theta_{L}=0.5\degree$, on trouve en prenant $\lambda=600nm$ : $D=\frac{1.22\times 600e^{-9}}{\times6\times0.5\pi/180}=14\mu m$
  \end{itemize}
  Apodisation voir Mauras p258 et Clément Sayrin Diffraction (2) : pour observer les exoplanètes de faible intensité lumineuse proche des étoiles qui ont une très grande intensité lumineuse, on peut choisir de mettre un masque dont la fonction de transmission de la lentille décroît de l'unité en son milieu à zéro à son bord : réduit l'intensité de des lobes secondaires de diffraction d'un objet très lumineux.
  
  \subsection{Ensemble de structure diffractante (peut sauter si besoin, ou en ouverture)}
  Si on prend pleins de trous d'Young et qu'on connait leur répartition spatiale, 

  \section{Applications de la diffraction}
  Au choix : expérience d'Abbe avec le filtrage spatiale, microscopie à contraste de phase,
  \section{}

  \section*{Conclusion}
  Ouverture sur la diffraction des rayons X. Conditions de Laue.

\end{reportBlock}
\newpage 
%%%%%%%%%%%%%%%%%%%%%%%%%%%%%%%%%%%%%%%%%%%%%%%%%%%%
%%%% En-tête leçon
\begin{headerBlock}
  \chapter{Diffraction sur des structures périodiques}
    \label{LP_DiffractionPeriodique}
\end{headerBlock}

%%%%%%%%%%%%%%%%%%%%%%%%%%%%%%%%%%%%%%%%%%%%%%%%%%%%
%%%% Références
\begin{center}
\begin{tabularx}{\textwidth}{| X | X | c | c |}
  \hline
  \rowcolor{gray!20}\multicolumn{4}{c}{Bibliographie de la leçon : } \\
  \hline 
  Titre & Auteurs & Editeur (année) & ISBN \\
  \hline
   Physique du solide & Ashcroft et Mermin & EDP Sciences &   \\
  \hline 
   Tout-en-un MP & M.-N. Sanz & Dunod (2009) &  \\
  \hline 
  Optique & J.-P. Pérez & Dunod & \\
  \hline 
  Optique Physique et électronique & D. Mauras & PUF (2011) & \\
  \hline
  \url{http://ressources.univ-lemans.fr/AccesLibre/UM/Pedago/physique/02/optiondu/reseauphase.html} & Simulation réseau & Université du Mans & \\
\end{tabularx}
\end{center}

\begin{reportBlock}{Commentaires des années précédentes :}
    \begin{itemize}
        \item \textbf{2017 :} Il faut traiter de diffraction par des structures périodiques et pas seulement d’interférences à N ondes,
        \item \textbf{2015 :} Il est important de bien mettre en évidence les différentes longueurs caractéristiques en jeu,
        \item \textbf{2014-2012 :} Cette leçon donne souvent l’occasion de présenter les travaux de Bragg ; malheureusement, les ordres de grandeur dans différents domaines ne sont pas toujours maîtrisés,
        \item \textbf{2010-2009 :} La notion de facteur de forme peut être introduite sur un exemple simple. L’influence du nombre d’éléments diffractants doit être discutée.
    \end{itemize}
\end{reportBlock}

%%%%%%%%%%%%%%%%%%%%%%%%%%%%%%%%%%%%%%%%%%%%%%%%%%%%
\begin{reportBlock}{Plan détaillé}
  \textbf{Niveau choisi pour la leçon :} Licence
  \newline
  \textbf{Prérequis : }
  \begin{itemize}
      \item Principe de retour inverse de la lumière, théorème de Malus
      \item Diffraction de Fraunhofer, diffraction par une fente rectangulaire
      \item notion de cohérence spatiale et temporelle
  \end{itemize} 

  
  \textbf{Déroulé détaillé de la leçon: }   \newline
La diffraction, en particulier dans les conditions de  Fraunhofer, permet de faire le lien entre les caractéristiques de l’objet diffractant et la figure de diffraction résultante. On va dans cette leçon s'intéresser aux propriétés d'objet diffractant possédant des structures périodiques et montrer en particulier deux choses : 1) si on connait les caractéristiques de l'objet diffractant, on peut connaitre les caractéristiques de la source, 2) on peut utiliser la figure de diffraction pour remonter à la structure interne de l'objet diffractant. On verra quelles limitations on obtient dans ces deux visions.\\

L'objet périodique par excellence est le réseau.

  \section{Diffraction par un réseau}
  Voir D. Mauras p194.
  \subsection{Définition}
  Il en existe de différents types (amplitude par transmission, phase par transmission, amplitude par réflexion (spectrographe) et phase par réflexion). On va s'intéresser pour l'instant aux réseaux d'amplitude par transmission. Faire un dessin en définissant le pas du réseau.

  \subsection{Formule fondamentale du réseau}
  On va considérer ici un montage de type Fraunhofer (source à l'infini, observation à l'infini) : faire le dessin (voir D. Mauras p195) en forçant le trait sur le réseau pour faire apparaitre l'angle $\theta_0$. Les ondes diffractées par les fentes donnent lieu à des interférences non localisées. On les observe à l'infini dans la direction $\theta$, réalité dans le plan focal de la lentille (L$_2$) au point M(x). Le principe de retour inverse (source fictive en M) et le théorème de Malus nous permet de montrer que \textcolor{red}{pour avoir interférences constructives} au point d'observation M :
  \begin{equation}
      \sin\left(\theta_p\right) - \sin\left(\theta_0\right) = \frac{p\lambda_0}{na}
  \end{equation}
  qui est la formule fondamentale des réseaux. Remarque : pour un réseau blazé (cf \url{http://olivier.sigwarth.free.fr/CoursTS2/Ch5/Chap5.pdf}), on remplace $\sin(\theta_p)$ par $\sin(\theta_p+\alpha)$. $p$ est l'ordre d'interférence :
  \begin{itemize}
      \item $p=0$ : la lumière se propage en ligne droite selon les lois de l'optique géométrique,
      \item l'ordre d'interférence est borné car $-1\leq sin(\theta_p)\leq 1 \rightarrow |p|\leq \frac{a}{\lambda}$ donc plus a est grand, plus on peut avoir des ordres d'interférences élevés. Plus la longueur d'onde est faible, plus il y a d'ordres d'interférences. Ex : pour un réseau à 500 traits/mm, si $\lambda=550$~nm, p=-3, -2, -1, 0, 1, 2 ,3 (7 ordres d'interférences).
  \end{itemize}
  Utilisation en lumière polychromatique : comme $\theta_p$ dépend de $\lambda$, le réseau est donc un disperseur de la lumière. D'autre part :
  \begin{align*}
      d\theta_p\cos(\theta_p) &= p\frac{d\lambda}{a} \\
      \frac{d\theta_p}{d\lambda} &= \frac{p\lambda}{a\sqrt{1-(sin(\theta_0)+p\frac{\lambda}{a})^2}}
  \end{align*}
  On commente :
  \begin{enumerate}
      \item Pour un ordre donné, la dispersion augmente avec la longueur d'onde contrairement au prisme,
      \item la dispersion augmente avec l'ordre d'interférence,
  \end{enumerate}
  On va voir visuellement tout ça à l'aide de l'expérience ci-après.
  
  \subsection{Mesure des raies spectrales de la lampe à vapeur de mercure}
  On va ici faire une application de l'utilisation des réseaux.\\
  Matériel :
  \begin{itemize}
      \item un réseau (prendre celui où on peut changer le pas du réseau ENSP 3637), choisir le 3000 traits/mm,
      \item lampe à vapeur de mercure + condenseur de 8cm,
      \item une lentille convergente de 15-20cm de focale + une autre de focale 10cm,
      \item une fente réglable,
      \item un écran blanc avec une feuille blanche et du scotch
      \item un miroir, une règle de 1m (ou un mètre).
  \end{itemize}
  \textcolor{blue}{Expérience quantitative :} On dispose d'une source à vapeur de mercure. On fait passer la lumière par une fente source de largeur réglable. On met une lentille de focale 15-20cm pour faire une image sur un écran éloigné. On intercale un réseau entre les deux. On oriente le réseau pour obtenir le minimum de déviation. On mesure l'angle $\tan{D_{p,min}}=\frac{d_{ecran}}{L_{ecran-reseau}}$. On en déduit $\lambda$ (avec incertitudes) la formule des réseaux à l'angle de déviation minimum.\\
  
  Mettre sur slide l'angle de déviation minimum :
  \begin{align*}
      D_p &= \theta_p - \theta_0 \\
      \sin{\theta_p} &= \sin(\theta_0) + p\frac{\lambda}{a} \\
      \frac{dD_p}{d\theta_0} &= \frac{\cos(\theta_0)}{\cos(\theta_p)}-1 =0 \rightarrow \cos(\theta_p)=\cos(\theta_0) \\
  \end{align*}
  Dériver $D_p$ une seconde fois par rapport à $\theta_0$ pour montrer que c'est un angle de déviation minimum. En prenant $\theta_{p,min}=-\theta_{0,min}$, le rayon émergeant est symétrique du rayon incident par rapport au plan du réseau et l'angle de déviation vaut :
  \begin{equation}
      2\sin\left(\frac{D_{p,min}}{2}\right) = p\frac{\lambda}{a}
  \end{equation}

  \textcolor{red}{Transition :} Ce qu'on a vu pour l'instant c'est qu'on peut mesurer le spectre d'émission d'une source polychromatique. Quelles sont ses limitations ? Il faut étudier plus en détail l'intensité résultante à travers une struture périodique la structure des pics d'interférences.

  \section{Facteur de structure et facteur de forme}
  
  \subsection{Intensité de la diffraction}

  \section{Application : diffraction des solides cristallins}
  Voir Kittel Chapitre 2.
  \subsection{Conditions de diffraction de Bragg}
  La condition des interférences contructives entre deux plans réticulaires séparés par la distance d s'écrit :
  \begin{equation}
      2d\sin{\theta} = n\lambda
  \end{equation}
  \textbf{Remarque :} pour qu'il y ait diffraction, on doit avoir $2d\sim 5\angstrom\leq\lambda$ ce qui montre qu'on ne peut pas utiliser la lumière pour résoudre la structure cristalline.


\section*{Conclusion}
Ouverture sur la diffraction des rayons X. Tâches de diffraction des papillons.
\end{reportBlock}




\newpage 
%%%%%%%%%%%%%%%%%%%%%%%%%%%%%%%%%%%%%%%%%%%%%%%%%%%%
%%%% En-tête leçon
\begin{headerBlock}
  \chapter{Absorbtion et émission de la lumière}
    \label{LP_Absorption}
\end{headerBlock}

%%%%%%%%%%%%%%%%%%%%%%%%%%%%%%%%%%%%%%%%%%%%%%%%%%%%
%%%% Références
\begin{center}
\begin{tabularx}{\textwidth}{| X | X | c | c |}
  \hline
  \rowcolor{gray!20}\multicolumn{4}{c}{Bibliographie de la leçon : } \\
  \hline 
  Titre & Auteurs & Editeur (année) & ISBN \\
  \hline
  Optique Physique & R. Taillet & de Boeck &   \\
  \hline 
  Physique Statistique & Landau et Lifshitz & Ellipses &  \\
  \hline 
   Poly laser & A. Maitre &  &  \\
\hline
 Sextant & & Hermann & \\
 \hline 
 Tout-en-un PC/PC* & M.-N. Sanz & Dunod & \\
 \hline 
 Physique en PC/PC* & Pascal Olive & Ellipses & \\
 \hline
\end{tabularx}
\end{center}

%%%%%%%%%%%%%%%%%%%%%%%%%%%%%%%%%%%%%%%%%%%%%%%%%%%%
\begin{reportBlock}{Plan détaillé}
  \textbf{Niveau choisi pour la leçon :} Licence
  \newline
  \textbf{Prérequis : }Milieu LHI, distribution de Boltzmann
  \newline
  
  \textbf{Déroulé détaillé de la leçon: }   \newline
On verra deux modèles qui permettront d'expliquer le phénomène d'émission et absorption de la lumière dans les milieux.
  \section{Modèle du corps noir}
  
  \subsection{Electron élastiquement lié}


\section{Modèle d'Einstein}
\subsection{Equation d'Einstein}
Soit un système à deux niveaux d'énergie $E_a$ et $E_b$ associée aux états propres $\ket{a}$ et $\ket{b}$. On note $N_a$ et $N_b$ la population de ces niveaux telles que $N_a+N_b=N$. La proba d'occupation est $n_i=\frac{N_i}{N}$.\\
La lumière que reçoit ce système est caractérisé par une densité spectrale $u(\nu)=\frac{dU}{d\nu}$ avec $U=\frac{\epsilon_0E^2}{2}+\frac{B^2}{2\mu_0}$.\\


A l'équilibre : $\frac{dn_a}{dt}=-\frac{dn_b}{dt}$ et donc : 
\begin{equation}
    \frac{dn_a}{dt} = -An_b + B_{21}u(\nu_0)
\end{equation}
\subsection{Coefficients d'Einstein}

En posant $h\nu_0 = E_b - E_a$, on a $\frac{n_a}{n_b}=\exp{-\frac{E_a+E_b}{k_BT}} = \exp{\frac{h\nu_0}{b_BT}}$. On en déduit que $B_{12} = B_{21} = B$ et $\frac{A}{B}=\frac{8\pi h\nu_0^3}{c^3}$. L'équation d'Einstein devient : 
\begin{equation}
    \frac{dn_b}{dt} = -\frac{dn_a}{dt} = -An_a + Bu(\nu_0)(n_a-n_b)
\end{equation}

\subsection{Processus de transfert d'énergie}
La puissance transférée de l'atome vers le champ $P_{at\rightarrow ch}=-h\nu_0\frac{dn_b}{dt} = h\nu_0B(n_b-n_a)u(\nu_0)$. On distingue alors deux régimes : 
\begin{itemize}
    \item Si $n_a>n_b$, $P_{at\rightarrow ch}<0$ $\longrightarrow$ Absorbtion
    \item Si $n_a<n_b$, $P_{at\rightarrow ch}>0$ $\longrightarrow$ Inversion de population
\end{itemize}

Initialement $n_a=1$,$n_b=0$ : 
\begin{itemize}
    \item $n_a(\infty) = \frac{A + Bu(\nu_0)}{A + 2Bu(\nu_0)}$
    \item $n_b(\infty) = \frac{Bu(\nu_0)}{A + 2Bu(\nu_0)}$
\end{itemize}
\end{reportBlock}




\newpage
%%%%%%%%%%%%%%%%%%%%%%%%%%%%%%%%%%%%%%%%%%%%%%%%%%%%
%%%% En-tête leçon
\begin{headerBlock}
  \chapter{Propriétés macroscopiques des corps ferromagnétiques}
  \label{LP_Ferromagnetisme} 
\end{headerBlock}




%%%%%%%%%%%%%%%%%%%%%%%%%%%%%%%%%%%%%%%%%%%%%%%%%%%%
%%%% Références
\begin{center}
\begin{tabularx}{\textwidth}{| X | X | c | c |}
  \hline
  \rowcolor{gray!20}\multicolumn{4}{c}{Bibliographie de la leçon : } \\
  \hline 
  Titre & Auteurs & Editeur (année) & ISBN \\
  \hline
  Électromagnétisme. Tome 4 - Milieux diélectriques et milieux aimantés & M. Bertin, J.P. Faroux et J. Renault  & Dunod (1984) &    \\
  \hline 
  Electromagnétisme - Fondements et applictions & J-Ph. Pérez & Dunod 4ème édition (2019) &    \\
  \hline 
  Physique Spé. PSI*, PSI & S. Olivier, C. More, H. Gié & Tec \& Doc (2000) &    \\
  \hline 
  Physique de l'état solide & C. Kittel & Dunod 7ème édition (1998) &    \\
  \hline
\end{tabularx}
\end{center}

%%%%%%%%%%%%%%%%%%%%%%%%%%%%%%%%%%%%%%%%%%%%%%%%%%%%

%%%%%%%%%%%%%%%%%%%%%%%%%%%%%%%%%%%%%%%%%%%%%%%%%%%%
%%%% Plan
\begin{reportBlock}{Plan détaillé}

  \textbf{Niveau choisi pour la leçon :} Licence 3
  \newline
  \textbf{Prérequis} : \begin{itemize}
      \item Electrocinétique
      \item Induction
      \item Notions sur la paramagnétisme et le diamagnétisme
      \item Milieu LHI
  \end{itemize}

  \textbf{Déroulé détaillé de la leçon: }  
  
  \section*{Introduction}

Introduction sur les matériaux ferromagnétiques en prenant l'exemple de la magnétite (Fe$_2$O$_3$).
Définition : corps qui, sous l'action d'un champ EM extérieur, s'aimante très fortement.

\section{Aimantation d'un corps ferromagnétique (1min15)}

\subsection{Magnétostatique dans un milieu aimanté}

Dans la matière, il y a des électrons et des atomes qui portent des moments magnétiques. 

Définition de l'aimantation : $\mathbf M = \frac{\ud \mathbf m}{\ud t}$ (en A.m$^{-1}$).

Vecteur excitation magnétique : $\mathbf H = \frac{\mathbf B}{\mu_0} - \mathbf M$. 

Equation de Maxwell-Ampère : $\nabla \times \mathbf H j_{\text{libre}}$.
Equation de Maxwell-flux : $\nabla \cdot \mathbf B = 0$.

Pour un LHI : $\mathbf M = \chi_m \mathbf H$, avec $\chi_m$ la susceptibilité magnétiques.

Pour les milieux paramagnétiques et diamagnétiques, $\mid\chi_m \mid << 1$. Alors que pour les ferromagnétiques : 
\begin{itemize}
    \item $\chi_m(\mathbf H)$ : relation non linéaire entre $\mathbf M$ et $\mathbf H$.
    \item $\mid\chi_m \mid >> 1$
\end{itemize}

\subsection{Courbe de première aimantation (7min10)}

Courbe $M(H)$ pour un matériau ferrmoagnétique initialement non aimanté : (1) courbe linéaire pour les petits $H$ puis (2) fortement croissante puis (3) sature progressivement jusqu'à $M_{sat}$.

$\mu_0 M_{sat}$ est le champ magnétique maximal d'un ferromagnétique à $T$ donnée et dépend du matériau. $M_{sat}(Fe) = 2.1 T$


\subsection{Interprétation microscopique (9min12)}
Slide : Domaines de Weiss. Déplacement des domaines réversible à faible $B$, mais irreversible pour $B$ plus élevé du fait de la présence d'impuretés dans le matériau.\\

Une propriété des ferromagnétiques est la canalisation des lignes de champs magnétiques. Slide: illustrations pour différentes géométries. Pour un ferromagnétique torique, les lignes de champ sont complètement canalisées.

\section{Cycle d'hystérésis (14min)}

Présentation de l'expérience. Transformateur: un primaire et un secondaire. \textbf{C'est l'expérience "Étude du cycle d'hystérésis du fer d'un transformateur" du TP Conversion de puissance électrique, version 2023}.

\subsection{Etude du noyau de fer d'un transformateur(15min)}

Schéma électrique du montage (cf. TP Conversion de puissance électrique)

Théorème d'Ampère : $L H = n_1 i_1 + n_2 i_2$
$L$ : longueur totale du tore (fer doux), $n_i$ nombre de spires de la bobine $i$. $H = \frac{n_1 i_1 + n_2 i_2}{L}$ donne (avec $i_2$ négligeable devant $i_1$ car la résistance imposée dans le secondaire ($\sim 10$~k$\Omega$) associée est bien plus élevée que celle du primaire ($\sim 30$~$\Omega$)) $i_1 = \frac{L}{n_1} H$. Ainsi, si on mesure la tension aux bornes de la résistance du circuit primaire : 

$V_x = R i_1 = \frac{R L}{n_1} H$ avec ici $\frac{R L}{n_1} = 62.4 V/(Am^{-1})$ qu'on place sur la voie 1 de l'oscillo.

Dans le circuit secondaire, on a (Loi de Faraday) : $e = -\frac{\ud \phi}{\ud t} = -n_2 S \frac{\ud B}{\ud t} = R' i_2 + \int \frac{i_2}{C}$ (cf. TP). $R'$ et $C$ sont choisies de telle sorte que $\int \frac{i_2}{C}$ soit négligeable devant $R' i_2$. Il vient :

$i_2 = \frac{S}{R'} \frac{\ud B}{\ud t}$.

Conséquences, aux bornes du condensateur: $U_c = \int \frac{n_2 S}{c R'} \frac{\ud B}{\ud t} \ud t$. Finalement :

$V_y = U_c = \frac{n2 S}{R' C} B$ qu'on place sur la voie 2 de l'oscillo.

\subsection*{Début de l'expérience (24min45)}

\begin{itemize}
    \item Visualisation du cycle d'hystérésis avec le mode XY.
    \item Tracé du cycle $B(H)$ au tableau et définition du champ coercitif $H_c$ (pour $B=0$) et du champ rémanent $B_r$ (pour $H=0$).
    \item Mesure expérimentale de $B_r$. Valeurs : $V_y = 2.70 \pm 0.02$~V donne $B_r = 0.532 \pm 0.004 T$. 
    \item Mesure expérimentale de $H_c$. Valeurs : $V_x = 2.10 \pm 0.01$~V donne $H_c = 313 \pm 1$~A/m. Valeur caractéristique des ferro doux. Plus cette valeur est faible, plus l'excitation à devoir appliquer pour désaimanter le matériau sera faible: ce type de matériau se désaimante facilement.  
\end{itemize}

On distingue deux types de ferro (slide tableau comparatif). Ferro doux (Transformateurs, inductance à haute fréquence) et ferro durs (application générateur électrique, RMN, etc.).

\subsection{Bilan de puissance (35min38)}

Loi des mailles : $U i_1 + e i_1 - R i_1 = 0$. Premier terme: ; dernier terme : puissance dissipée par effet Joule. $e i_1 = - \frac{\ud \phi}{\ud t} i_1$. $\delta W = - \ud \phi i_1 = - \frac{SHL}{n_1} = \ud B$. $P = \frac{1}{T} SL \oint H _ud B$. ¨

\section*{Conclusion (39min50)}

Application : disques durs.
Fin : 40min35.

\end{reportBlock}


%%%%%%%%%%%%%%%%%%%%%%%%%%%%%%%%%%%%%%%%%%%%%%%%%%%%
%%%% Questions
\begin{reportBlock}{Questions posées par l’enseignant (avec réponses)}
  \textbf{Q: Si je lis votre relation, si j'augmente le nombre de spires, je vais diminuer la perte par hystérésis ?} \textcolor{purple}{Non, il y a une erreur dans ma formule, elle ne dépend pas du nombre de spires.} \newline
  
  \textbf{Q: Si je regarde le cycle d'hystérésis, le champ $B$ sature ?} \textcolor{purple}{Non, il continue à croître linéairement.} \newline
  
   \textbf{Q: Pourquoi on utilise des ferro doux pour l'inductance à haute fréquence.?} \textcolor{purple}{Ce qui compte dans une inductance c'est la variation du flux. Mais dans le ferro dur, quand c'est saturé, certes le champ est fort, mais il n'est plus sensible au champ extérieur. Un ferro doux, en première approximation, c'est linéaire et la pente est la susceptibilité. Mais pour le ferro dur, la pente est nulle.} \newline
  
  \textbf{Q: Pourquoi à haute fréquence ?} \textcolor{purple}{Pour minimiser les pertes par courants de Foucault.} \newline
  
   \textbf{Q: Quels matériaux qui minimisent ces pertes à hautes fréquence?} \textcolor{purple}{Utiliser des isolants (ferrites), on va limiter ainsi des pertes par courants de Foucault.} \newline
  
  \textbf{Q: Est-ce que la canalisation des champ est générique à tous les ferro ?} \textcolor{purple}{Ce n'est pas le cas pour les ferros durs, que pour les ferros doux. Toutes les applications qui utilisent $\chi$ ou $\mu_r$ très grand c'est les ferro doux, car il n'y a plus de pente pour les ferro durs.} \newline
  
   \textbf{Q: Dans quel état sont les ferromagnétiques ?} \textcolor{purple}{Solides cristallin. L'état ferro provient d'interaction au niveau des atomes qui n'existent pas à l'état fluide.} \newline
  
  \textbf{Q: L'aimantation à saturation et le champ coercitif dépendent des matériaux ?} \textcolor{purple}{Varie de quelques magnétons de Bohr mais reste du même ordre de grandeur alors que $H_c$ varie beaucoup : un ferro doux a un champ coercitif faible (10$^{-3}$~T), un ferro dur très grand (0.1~T) matériau à un autre.} \newline
  
   \textbf{Q: Sur l'histoire du transformateur, vous avez appliqué le théorème d'Ampère. C'est évident que $\oint H.\ud l = H L$ ? } \textcolor{purple}{Il faut utiliser les relations de passage des champs B et H pour démontrer la canalisation des lignes de champs dans le ferro en l'absence de courants surfaciques à l'interface fer$\rightarrow$air. Ensuite, les symétries et invariances du tore donnent $\mathbf{H}=H(r)\mathbf{e_{\theta}}$ et le théorème d'Ampère le long d'une ligne de champ permet d'avoir la formule donnée si on considère que la section du tore est faible devant la distance à l'axe.%Vous n'avez pas utilisé $\nabla \cdot B = 0$, il faut l'utiliser
   .} \newline
  
  \textbf{Q: Dans un éléctroaimant ?} \textcolor{purple}{Il n'y a pas conservation de la norme de $H$. Pour relier $H$ dans l'entre-fer et dans le milieu, il faut utiliser la conservation du flux.} \newline
  
   \textbf{Q: Pourquoi le système forme les domaines de Weiss ?} \textcolor{purple}{Au niveau microscopique, il y a une compétition entre l'énergie qu'il va falloir fournir pour créer ces interfaces et le coût en énergie pour créer un champ via l'alignement des moments magnétiques.} \newline
  
  \textbf{Q: La taille des domaines ?} \textcolor{purple}{De l'ordre du micromètre.} \newline
  
   \textbf{Q: Est-ce qu'il y a des directions priviligiées au départ dans champ extérieur? Est-ce que je peux avoir une courbe d'hystérésis qui dépend du champ $B$?} \textcolor{purple}{Oui, il y a un axe de facile aimantation. Les champs coercitifs vont être plus forts dans l'axe de facile aimantation. Cela va exister dans des monocristaux. Les axes de facile et difficile aimantation sont définis par rapport à l'orientation cristalline du matériau et des interactions entre les moments magnétiques dans le matériau. Il n'y a pas de raison pour que l'orientation du domaine soit dans la direction du champ appliqué.} \newline
  
  \textbf{Q:  Quels sont les conditions qui vous permettent de lire directement $B$?} \textcolor{purple}{$e = U_{R'} + U_c = R' i_2 + \frac{1}{jC\omega} i_2 = R'(1+\frac{1}{jRC\omega}) i_2$. On veut alors $R'C >> \frac{1}{\omega}$ avec $\omega=2\pi f$ et $f=50$~Hz (fréquence du secteur).} \newline
  
  \textbf{Q: Les applications des ferros durs?} \textcolor{purple}{Tous les aimants permanents. } 
  
 
  
  \end{reportBlock}
  
%%%%%%%%%%%%%%%%%%%%%%%%%%%%%%%%%%%%%%%%%%%%%%%%%%%%
%%%% Commentaires
\begin{reportBlock}{Commentaires lors de la correction de la leçon}

Agréable à suivre. Le rythme était un peu lent. Parfois vous vous répétez un peu, c'est très bien pour un vrai cours mais pas nécéssaire pour une leçon d'agrégation, ce qui permet de gagner un peu de temps pour la partie ferro dur/ferro doux qui est un point central, et aussi pour l'histoire du $\nabla \cdot B$. Ne pas faire la démonstration pour la canalisation est un choix, vous auriez pu le faire sur transparent pour gagner du temps. Votre exploitation de la manipulation est remarquable. Il faut donner les chiffres des ordres de grandeurs. 

\end{reportBlock}



%%%%%%%%%%%%%%%%%%%%%%%%%%%%%%%%%%%%%%%%%%%%%%%%%%%%
%%%% Correction
\begin{reportBlock}{Partie réservée au correcteur}
  \textbf{Avis général sur la leçon (plan, contenu, etc.) :}
  
  
  \textbf{Notions fondamentales à aborder, secondaires, délicates :}
  
  
  \textbf{Expériences possibles (en particulier pour l'agrégation docteur) :}
  
  
  \textbf{Bibliographie conseillée :}
\end{reportBlock}

\newpage
%%%%%%%%%%%%%%%%%%%%%%%%%%%%%%%%%%%%%%%%%%%%%%%%%%%%
%%%% En-tête leçon
\begin{headerBlock}
  \chapter{Mécanismes de la conduction électrique dans les solides}
    \label{LP_Conduction}
\end{headerBlock}

%%%%%%%%%%%%%%%%%%%%%%%%%%%%%%%%%%%%%%%%%%%%%%%%%%%%
%%%% Références
\begin{center}
\begin{tabularx}{\textwidth}{| X | X | c | c |}
  \hline
  \rowcolor{gray!20}\multicolumn{4}{c}{Bibliographie de la leçon : } \\
  \hline 
  Titre & Auteurs & Editeur (année) & ISBN \\
  \hline
Physique des Solides (Chap 1 à 3)  & N. Ashcroft et D.Mermin   &  EDP Sciences (2002) & 2-86883-577-5  \\
  \hline 
     Slides de cours & Gwendal Fève &  &  Site Montrouge \\
  \hline 
  Physique des Solides & C. Kittel &  Dunod &  \\
\hline
\end{tabularx}
\end{center}

%%%%%%%%%%%%%%%%%%%%%%%%%%%%%%%%%%%%%%%%%%%%%%%%%%%%

\section{Approche classique}
\subsection{Modèle de Drude (1902) 10min max}
%Définition résistivité : $\rho=\frac{RS}{L}$.\\
Contexte historique : découverte de l'électron par Thomson (1899), avant expérience de Rutherford.\\
Cristal métallique : [grosses sphères chargés + avec des électrons de c\oe ur] = ions métalliques  + électrons de valence pour assurer électroneutralité.\\
Remarque : $n_{gaz}=10^{25}$m$^{-3}$ tandis que $n_{e^{-}}=10^{29}$m$^{-3}$, cf p4 Ashcroft.\\
Hypothèses :\begin{itemize}
    \item pas d'interaction entre les électrons et les ions, et entre les électrons et les électrons : \textcolor{red}{"électrons libres"},
    \item \textcolor{red}{collisions et changement de vitesse instanées} entre les électrons de c\oe ur et les électrons de valence,
    \item électrons de valence (conduction) se déplacent en ligne droite jusqu'à collision avec proba $1/\tau$. $\tau$ est \textcolor{red}{le temps de collision} ou libre parcourt moyen, temps moyen de propagation entre deux chocs,
    \item \textcolor{red}{chaos moléculaire}, la distribution des vitesses suit une loi de Maxwell-Boltzmann comme le gaz classique, la direction des électrons est aléatoire. L'équilibre thermodynamique local des électrons avec leur entourage par le biais des collisions (seul mécanisme restant) : 
\end{itemize}

Ce modèle décrit relativement bien la conductivité dans un métal.
Déduire équation du mouvement + 
\subsection{Mise en défaut pratique du modèle de Drude}
Voici ce que Drude prévoit en supposant que la vitesse des électrons obéit à la statistique de Boltzmann :
\begin{itemize}
    \item dans un modèle de gaz parfait, la vitesse moyenne des électrons est : $v^*=\sqrt{\frac{3k_BT}{m_e}}=\frac{l}{tau}$ (utiliser le théorème d'équipartition.
    \item comme $\tau=\frac{\sigma m_e}{N_ee^2}$ avec le modèle de Drude, on obtient $\sigma\propto \frac{1}{\sqrt{T}}$
\end{itemize}

\textcolor{blue}{Manip quantitative : mesure 4 points d'un morceau de cuivre à différentes températures. Montrer (si possible) que la résistivité est linéaire en température.}\\
\begin{itemize}
    \item en utilisant $1/2mv^2=3/2k_BT$, on trouve $v=10^5$m/s alors que $v_{F}=10^6$m/s et indépendante de T,
    \item quid des matériaux isolants ? semi-conducteurs ?
    \item effet Hall classique dans l'aluminium (p18 Ashcroft) montre que n dépend fortement de B appliqué, ce qui est inenvisagable dans le modèle de Drude + constante de Hall parfois positive
\end{itemize}

Conclusion : il faut passer à un autre modèle plus poussé : un modèle quantique.

\section{Modèle quantique des électrons libres}
Electrons obéissent à la statistique de Fermi-Dirac et non à la statistique de Maxwell-Boltzmann.
\subsection{Modèle des électrons libres}
Cf Kittel chap 8 ou Ashcroft chap 2 et 13. Modèle de Sommerfeld. Faire les calculs Justifier qu'on peut se placer à T=0K.

\section{Théorie des bandes}

\subsection{Electrons dans un potentiel périodique}
Modèle : électrons en intéraction avec le potentiel périodique du réseau cristallins.\\
Cf Kittel (chap 7) + TD 1 Physique des Solides : mettre en prérequis pour les calculs.

\subsection{Bandes d'énergie}
Slide : schéma bande avec isolant+ tableau périodique 

\subsection{Remplissage des bandes}

Distinguer métaux, semi-conducteurs et isolant. Prévoir lesquels seront métalliques avec le demi-remplissage, lesquels seront isolants. Montrer que ça ne marche pas toujours.
\subsection{Conductivité dans un semi-conducteur}
Densité d'état varie en $\exp(-\Delta/k_BT)$ donc $\sigma$ aussi.\\

\textcolor{blue}{Mesure de la conductivité dans un semi-conducteur dopé pour déterminer le gap.}, cf TP semiconducteurs. Durée 30min environ, $\Delta_{exp}=0.6$eV alors que $\Delta_{th}=0.7$eV. Interprétation : matériel vieux, gap a pu changer. Matériau pas pur.\\
\section{Conclusion}
Ouverture sur l'ingénierie des semi-conducteurs : micro-électronique, jonctions pn, transistors. Prix Nobel 2014 pour l'invention de la diode bleue (en 1992).\\
Ouverture sur la supraconductivité.\\
Ouverture possible sur la conduction thermique, loi de Wiedemann-Franz.
\newpage
%%%%%%%%%%%%%%%%%%%%%%%%%%%%%%%%%%%%%%%%%%%%%%%%%%%%
%%%% En-tête leçon
\begin{headerBlock}
  \chapter{Phénomènes de résonance dans différents domaines de a physique}
  \label{LP_resonance} 
\end{headerBlock}




%%%%%%%%%%%%%%%%%%%%%%%%%%%%%%%%%%%%%%%%%%%%%%%%%%%%
%%%% Références
\begin{center}
\begin{tabularx}{\textwidth}{| X | X | c | c |}
  \hline
  \rowcolor{gray!20}\multicolumn{4}{c}{Bibliographie de la leçon : } \\
  \hline 
  Titre & Auteurs & Editeur (année) & ISBN \\
  \hline
  & & & \\
  \hline 
  & & &    \\
  \hline 
  & & &    \\
  \hline 
\end{tabularx}
\end{center}

%%%%%%%%%%%%%%%%%%%%%%%%%%%%%%%%%%%%%%%%%%%%%%%%%%%%

%%%%%%%%%%%%%%%%%%%%%%%%%%%%%%%%%%%%%%%%%%%%%%%%%%%%
%%%% Plan
\begin{reportBlock}{Plan détaillé}

  \textbf{Niveau choisi pour la leçon :} Licence 3
  \newline
  \textbf{Prérequis} : \begin{itemize}
      \item Mécanique newtonienne
      \item Optique ondulatoire
      \item 
  \end{itemize}

  \textbf{Déroulé détaillé de la leçon: }  
  
  \section*{Introduction}
Définition : pour un système auquel qu'on soumet à une excitation sinusoïdale de pulsation $\omega$, la réponse du système est maximale à la pulsation $\omega_0$, appelée fréquence de résonance.
  \section{Oscillateurs harmoniques amortis}
  \subsection{Résonance en vitesse en régime forcé} 
\textcolor{red}{Attention, faire la discussion sur la phase}. Interprétation énergétique possible.
  \section{Oscillateurs couplés}
\begin{center}
    \includegraphics[scale=0.5]{LP_Resonance/Oscillateurs_couple.jpg}
\end{center}    


\textcolor{green}{Expérience : montrer l'influence du couplage sur la résonance du premier circuit RLC à l'aide d'un deuxième circuit RLC.}
  \section{}
\end{reportBlock}
\newpage 
%%%%%%%%%%%%%%%%%%%%%%%%%%%%%%%%%%%%%%%%%%%%%%%%%%%%
%%%% En-tête leçon
\begin{headerBlock}
  \chapter{Oscillateurs ; portraits de phase et non-linéarités.}
  \label{LP_PortaitPhase} 
\end{headerBlock}




%%%%%%%%%%%%%%%%%%%%%%%%%%%%%%%%%%%%%%%%%%%%%%%%%%%%
%%%% Références
\begin{center}
\begin{tabularx}{\textwidth}{| X | X | c | c |}
  \hline
  \rowcolor{gray!20}\multicolumn{4}{c}{Bibliographie de la leçon : } \\
  \hline 
  Titre & Auteurs & Editeur (année) & ISBN \\
  \hline
   \url{https://uhincelin.pagesperso-orange.fr/LP49_BUP_portrait_phase_oscil.pdf} & H. Gié &  BUP n°744&    \\
  \hline 
   &  & &    \\
  \hline 
\end{tabularx}
\end{center}

%%%%%%%%%%%%%%%%%%%%%%%%%%%%%%%%%%%%%%%%%%%%%%%%%%%%

%%%%%%%%%%%%%%%%%%%%%%%%%%%%%%%%%%%%%%%%%%%%%%%%%%%%
%%%% Plan
\begin{reportBlock}{Plan détaillé}

  \textbf{Niveau choisi pour la leçon :} 
  \newline
  \textbf{Prérequis} : \begin{itemize}
      \item 
  \end{itemize}

  \textbf{Déroulé détaillé de la leçon: }  
  
  \section*{Introduction}
Messages à faire passer : définition d'un portait de phase, portait de phase amorti/non-amorti, stabilité des points fixes.
  \section{Oscillateurs non-amortis}
  \subsection{Equation du mouvement}
  Exemple de la physique : pendule simple à petites oscillations, circuit LC
  \subsection{Approche énergétique}
  Parler de stabilité, conservation de l'énergie mécanique.

  \subsection{Portrait de phase}

  \section{Effets non linéaires}
  Formule de borda pendule.\\

  \textcolor{blue}{Manip quanti :} mettre en évidence non linéarités, déterminer formule de Borda.
  


\end{reportBlock}
\newpage
%%%%%%%%%%%%%%%%%%%%%%%%%%%%%%%%%%%%%%%%%%%%%%%%%%%%
%%%% En-tête leçon
\begin{headerBlock}
  \chapter{Cinématique relativiste. Expérience de Michelson et Morlay}
    \label{LP_CinematiqueRelativiste}
\end{headerBlock}

%%%%%%%%%%%%%%%%%%%%%%%%%%%%%%%%%%%%%%%%%%%%%%%%%%%%
%%%% Références
\begin{center}
\begin{tabularx}{\textwidth}{| X | X | c | c |}
  \hline
  \rowcolor{gray!20}\multicolumn{4}{c}{Bibliographie de la leçon : } \\
  \hline 
  Titre & Auteurs & Editeur (année) & ISBN \\
  \hline
\end{tabularx}
\end{center}

%%%%%%%%%%%%%%%%%%%%%%%%%%%%%%%%%%%%%%%%%%%%%%%%%%%%
\begin{reportBlock}{Plan détaillé}
  \textbf{Niveau choisi pour la leçon :} L3
  \newline
  \textbf{Prérequis : }
  \newline


\section{Emergence de la relativité restreinte}

\subsection{Transformation de galilée en électromagnétisme}

\subsection{Expérience de Michelson et Morlay}

\subsection{Postulat d'Einstein}

\section{Changement de référentiel}

\subsection{Evènements}

\subsection{Transformation de Lorentz}

\subsection{Intervalle d'espace-temps}

\section{Conséquences physiques}
\subsection{Dilatation du temps}
\textcolor{blue}{Expérience :} utilisaion du programme Gum\_C pour la détermination de $\gamma$ des muons d'après l'expérience de Frish et Smith.
\subsection{Contraction des longueurs}

\end{reportBlock}
\newpage
%%%%%%%%%%%%%%%%%%%%%%%%%%%%%%%%%%%%%%%%%%%%%%%%%%%%
%%%% En-tête leçon
\begin{headerBlock}
  \chapter{Effet tunnel. Application à la radioactivité alpha.}
    \label{LP_EffetTunnel}
\end{headerBlock}

%%%%%%%%%%%%%%%%%%%%%%%%%%%%%%%%%%%%%%%%%%%%%%%%%%%%
%%%% Références
\begin{center}
\begin{tabularx}{\textwidth}{| X | X | c | c |}
  \hline
  \rowcolor{gray!20}\multicolumn{4}{c}{Bibliographie de la leçon : } \\
  \hline 
  Titre & Auteurs & Editeur (année) & ISBN \\
  \hline
  Polycopié & Jean Hare &  & \\
  \hline
  Tout-en-un PC/PC* & M.-N. Sanz & Dunod & \\
  \hline
\end{tabularx}
\end{center}

%%%%%%%%%%%%%%%%%%%%%%%%%%%%%%%%%%%%%%%%%%%%%%%%%%%%
\begin{reportBlock}{Plan détaillé}
  \textbf{Niveau choisi pour la leçon :} 
  \newline
  \textbf{Prérequis : }
  \newline

\section*{Introducion}

\section{Equation de Schrödinger dans un potentiel}

\subsection{Position du problème}

\subsection{Coefficient de transmission}

\section{Application à la radioactivité alpha}

\subsection{Modèle de Gamow}
\textcolor{blue}{Expérience :} Code python simulant Gamow (cf Alexandre).

\subsection{Loi de Geiger-Nutall}


\section*{Conclusion}


\end{reportBlock}
\newpage
\part{Leçons de chimie}

%\vspace*{\stretch{0.5}}
%\begin{center}
 %   \vspace{5cm}
  %  \Huge{\textbf{Leçons de chimie}}
%\end{center}
%\vspace*{\stretch{1}}

\newpage
\begin{changemargin}{-1.5cm}{-0cm}

Code couleur pour les manips :
\begin{center}
\textcolor{green}{Expérience déjà faite, sur laquelle je suis à l'aise}\\
\textcolor{yellow}{Retravailler un geste, l'interprétation de l'expérience}\\
\textcolor{red}{Manip à faire en priorité}\\
\end{center}

\begin{tabularx}{\paperwidth-2cm}{| X | X | c | X |}
  \hline
  \rowcolor{gray!20}\multicolumn{4}{c}{Avancement préparation oraux Leçons Chimie} \\
  \hline 
  Titre de la leçon & Elément imposé & Expérience & Niveau \\
  \hline
  \textbf{De la structure à la polarité d'une entité} & logiciel de représentation molécule & \textcolor{red}{\textbf{0\%}}  & 1ère générale - spécialité  \\
  \hline
  \textbf{Evolution d'un système chimique} & Python, déterminer la compo d'un système & \textcolor{red}{\textbf{0\%}}  & 1ère générale - spécialité, Python : cf agreg ou CAPES 2023  \\
  \hline
  \textbf{Dosage} & Dosage par étalonnage (ex: sérum phy) & \textcolor{red}{\textbf{0\%}}  & 1ère générale - spécialité \\
  \hline
  \textbf{Synthèse, traitement, caractérisations} p\pageref{LC_SyntheseTraitement} & Filtration sous vide & Synthèse acide benzoïque  & 1ère générale - spécialité \\
  \hline
  \hline
  \textbf{Oxydants et réducteurs} & Réaliser une pile & \textcolor{green}{Pile Daniell}  & Term générale - spécialité \\
  \hline
  \textbf{Chimie analytique quantitative et fiabilité} & Titrage par conductimétrie & \textcolor{green}{Titrage ions Cl dans sérum phy}  & Term générale - spécialité  \\
  \hline
  \textbf{Evolution spontanée d'un système} & Déterminer $Q_{final}$ & \textcolor{red}{\textbf{0\%}}  & Term générale - spécialité  \\
  \hline
  \textbf{Cinétique et catalyse} & Mise en évidence d'un effet catalyseur & \textcolor{red}{\textbf{0\%}}  & Term générale - spécialité  \\
  \hline
  \textbf{Stratégies en synthèse organique} & Protocole d'optimisation d'un rendement/vitesse & \textcolor{red}{\textbf{0\%}}  & Term générale - spécialité  \\
  \hline
  \hline
   \textbf{Structure spatiale des molécules} & Différencier 2 stéréoisomères & \textcolor{red}{\textbf{0\%}}  & 1ère STL  \\
  \hline
  \textbf{Réactions acide-base en solution acqueuse} & Préparer une solution tampon & \textcolor{red}{\textbf{0\%}}  & 1ère STL  \\
  \hline
  \textbf{Solvants et solubilité} & Etudier facteur influençant la solubilité & \textcolor{red}{\textbf{0\%}}  & 1ère STL  \\
  \hline
\end{tabularx}
\end{changemargin}

\newpage

\begin{changemargin}{-1.5cm}{0cm}
\begin{tabularx}{\paperwidth-2cm}{| X | X | X | X |}
  \hline
  \rowcolor{gray!20}\multicolumn{4}{c}{Avancement préparation oraux Leçons Chimie} \\
  \hline 
  Titre de la leçon & Elément imposé & Expérience & Niveau \\
  \hline
  \textbf{Synthèse, purification, contrôle pureté liquide organique} & CCM & \textcolor{red}{\textbf{0\%}}  & 1ère SPCL  \\
  \hline
  \textbf{Réactivité des alcools} & Recristallisation & \textcolor{green}{Aspirine}  & 1ère SPCL  \\
  \hline
  \textbf{Réactions de synthèse, sites electrophiles, nucléophiles, formalisme flèches courbes} & Montage à reflux & \textcolor{red}{\textbf{0\%}}  & 1ère SPCL \\
  \hline
  \textbf{Stéréochimie de configuration}, p\pageref{LC_Stéréochimie} & Différencier 2 stéréoisomères & \textcolor{red}{Hydro-halogénation régiosélective}  & Terminale SPCL \\
  \hline
  \textbf{Distillation et diagrammes binaires}, p\pageref{LC_Distillation} & Distillation & \textcolor{green}{Distillation eau-éthanol}  & Terminale SPCL \\
  \hline
   \textbf{Techniques spectroscopiques} & Déterminer C d'après une courbe d'étallonnage & \textcolor{green}{Dosage bleu brillant dans le curaçao}  & Terminale SPCL \\
  \hline
  \textbf{Conductivité} & Titrage par précipitation & \textcolor{green}{Déterminer concentration Cl dans le sérum phy}  & Terminale SPCL, revoir plan pour caler les expériences\\
  \hline
  \textbf{Solubilité} & Extraire sélectivement les ions d'un mélange & \textcolor{green}{Extraction sélective des ions fer dans une solution de nitrate de fer et de sulfate de cuivre (Mesplède p193)}  & Terminale SPCL \\
  \hline
  \textbf{Oxydoréduction} & Titrage dont la réaction support est une oxydo-reduction & \textcolor{green}{Titrage des ions hypochlorite contenus dans le Dakin/eau de Javel}  & Terminale SPCL \\
  \hline
  \textbf{Réactivité des dérivés d'acide} & Montage Dean-Stark & \textcolor{green}{Synthèse arôme de banane}  & Terminale SPCL \\
  \hline
  \textbf{Electrolyse, électrosynthèse} & Réaliser une électrolyse à anode soluble & \textcolor{red}{Elec à anode soluble}- Florence Porteu-de-Buchère - Dunod p190 & Terminale SPCL \\
  \hline
\end{tabularx}
\end{changemargin}

\newpage

\begin{changemargin}{-1.5cm}{-0cm}
\begin{tabularx}{\paperwidth-2cm}{| X | X | X | X |}
  \hline
  \rowcolor{gray!20}\multicolumn{4}{c}{Avancement préparation oraux Leçons Chimie} \\
  \hline 
  Titre de la leçon & Elément imposé & Expérience & Niveau \\
  \hline
  \textbf{Energie chimique - exemple des combustions} & Estimer pouvoir calo d'une combustion & \textcolor{green}{Voir Nathan ST2S}  & 1ère/terminale STI2D \\
  \hline
  \textbf{Oxydoréduction} p\pageref{LC_Oxydoreduction_STI2D} & Déterminer la capacité d'un accumulateur & \textcolor{red}{Accumulateur au plomb}  & 1ère/terminale STI2D \\
   \hline  
  \hline
  \textbf{Molécules d'intérêt biologique} & Différencier aldéhyde/cétone & \textcolor{green}{Voir Nathan ST2S} & 1ère ST2S \\
  \hline
  \textbf{Gestion des risques au laboratoire de chimie} p\pageref{LC_GestionRisquesLabo} & Mettre en \oe uvre un protocol de neutralisation & \textcolor{green}{Neutralisation acide} & 1ère ST2S \\
  \hline
   \textbf{Biomolécules et énergie} p\pageref{LC_BiomoleculesEnergie} & Réaliser l'hydrolyse d'un glucide complexe & \textcolor{green}{Voir Nathan ST2S}  & 1ère ST2S \\
  \hline 
  \textbf{L'eau, propriétés physiques et chimiques} & Réaction une extraction liquide/liquide & \textcolor{red}{0\%} & 1ère ST2S \\
  \hline 
  \hline
  \textbf{Chimie et alimentation} & Teneur en vitamine C d'un aliment/médicament & \textcolor{red}{0\%} & Terminale ST2S \\
  \hline
  \textbf{Contrôle qualité de l'eau et de l'air} & Réaliser un dosage conductimétrique & \textcolor{green}{Dosage des ions $SO_4^{2-}$ dans la Contrex} & Terminale ST2S \\
  \hline
  \end{tabularx}
\end{changemargin}

\newpage 
\begin{changemargin}{-1.5cm}{-0cm}
  \begin{tabularx}{\paperwidth-2cm}{| X | X | X | X |}
  \hline
  \textbf{Transformations chimiques en solution acqueuse} & Déterminer $K_{eq}$ d'une réaction (A/B, précipitation, redox) & \textcolor{red}{0\%} & MPSI \\
  \hline
  \textbf{Acides et bases} & Déterminer $K_a$ & \textcolor{red}{0\%} & MPSI \\
  \hline
  \textbf{Solvants} & Déterminer constante de partage & \textcolor{red}{0\%} & MPSI \\
  \hline
  \textbf{Structure et prop des solides} & Utiliser avogadro ou vesta et déterminer des paramètres géométriques & \textcolor{red}{0\%} & MPSI \\
  \hline
  \textbf{Cinétique homogène} & Déterminer $E_a$ & \textcolor{red}{hydrolyse du chlorure de tertiobutyle suivie par pH-métrie} - F. Porteu de Buchère p40& MPSI \\
  \hline
  \textbf{Diagrammes E-pH} p\pageref{LC_DiagrammeEpH} & Mettre en \oe uvre une médiamutation & \textcolor{red}{0\%} & MPSI \\
  \hline
  \textbf{Liaisons chimiques} & Utiliser Vesta ou avogadro & \textcolor{red}{0\%} & MPSI \\
  \hline
\end{tabularx}
\end{changemargin}

\newpage
\begin{changemargin}{-1.5cm}{-0cm}
\begin{tabularx}{\paperwidth-2cm}{| X | X | X | X |}
  \hline
  \rowcolor{gray!20}\multicolumn{4}{c}{Avancement préparation oraux Leçons Chimie} \\
  \hline 
  Titre de la leçon & Elément imposé & Expérience & Niveau \\
  \hline
  \textbf{Utilisation 1er principe pour la détermination de grandeurs physico-chimiques} & Déterminer $\Delta_rH_{reaction}$ & \textcolor{red}{pas fait} - F. Porteu p75 & PSI \\
  \hline
  \textbf{Optimisation d'un procédé chimique} & Mettre en évidence influence T ou P ou catalyseur sur une réaction & \textcolor{red}{0\%} & PSI \\
  \hline
  \textbf{Corrosion humide des métaux} & Protection contre la corrosion (clou+gel agar-agar) & \textcolor{red}{Passivation du fer}-Cachau Redox & PSI \\
  \hline
  \textbf{Générateurs électrochimiques} & Montrer influence de facteurs sur $e_{vide}$ d'une pile & \textcolor{red}{0\%} & PSI \\
  \hline
  \textbf{Conversion d'énergie électrique en énergie chimique} & Déterminer $\eta_{faradique}$ d'un électrolyseur & \textcolor{red}{0\%} & PSI \\
  \hline
  \textbf{Cinétique électrochimique} & Tracer courbe i-E & \textcolor{red}{0\%} & PSI \\
  \hline
  \textbf{Application 2nd principe à une transfo chimique} & Déterminer une constante d'équilibre thermo & \textcolor{red}{0\%} & PSI \\
  \hline
  \hline
  \textbf{Diagramme E-pH} & Réaliser un procédé industriel à l'échelle du labo & \textcolor{red}{0\%} & TSI 2 \\
  \hline
  \textbf{Déplacement de l'équilibre chimique} & Utiliser un bain thermostaté & \textcolor{red}{0\%} & TSI 2 \\
  \hline  
\end{tabularx}

\end{changemargin}
\newpage
\begin{headerBlock}{\huge{Idées en vracs}}
\begin{itemize}
    \item Pour l'utilisation de la trompe à eau ou de la pompe à vide : briser le vide d'abord, couper l'eau ou l'alimentation de la pompe après
    \item Site de l'INRS pour les fiches toxicologiques des composés chimiques : \url{https://www.inrs.fr/publications/bdd/fichetox.html}
    \item 
\end{itemize}

\end{headerBlock}


\newpage
\begin{headerBlock}
\chapter{L'eau, propriétés physiques et chimiques}
\label{LC_Eau}
 \end{headerBlock}

%%%%%%%%%%%%%%%%%%%%%%%%%%%%%%%%%%%%%%%%%%%%%%%%%%%%
%%%% Références


%%%%%%%%%%%%%%%%%%%%%%%%%%%%%%%%%%%%%%%%%%%%%%%%%%%%
%%%% Plan
\begin{reportBlock}{Bibliographie}

\begin{center}
\begin{tabular}{|c|c|c|c|}\hline
Titre & Auteur(s) & Editeur (année) & ISBN \\ \hline
BUP n$^0$790 ~ & ~ & ~ & ~ \\
\hline
\end{tabular}
\end{center}

\end{reportBlock}

\begin{reportBlock}{Plan détaillé}

\underline{Niveau} : 1èr ST2S \\

\section*{Introduction pédagogique}


\paragraph*{Prérequis}
\begin{itemize}
\item Barycentre
\item Formules développées et semi-développées
\end{itemize}

\paragraph*{Contexte :}
Place de la leçon : milieu d'année.

\paragraph*{Notions importantes}

\begin{itemize}
\item Constante de solubilité
\item sens de l'évolution
\item influence de la solubilité (T, pH, etc...)
\item protocole 
\end{itemize}

\paragraph*{Objectifs}

\begin{itemize}
\item prédire la précipitation, dissolution
\item déterminer la solubilité et prévoir son évolution
\end{itemize}

\paragraph*{Difficultés}

\begin{itemize}
\item Appréhender le caractère polaire/apolaire liée à la géométrie
\item distinctions entre liaisons covalente et hydrogène
\item miscibilité, solubilité
\end{itemize}
Pour y remédier, activité préliminiaire pour manipuler les grandeurs
\section*{Introduction }
Dissolution courante dans la vie de tous les jours.\\
\textbf{Question:}

\paragraph*{Manipulation qualitative:} \textcolor{green}{}

\section{Electronégativité}

\subsection{Liaison polarisée}

\textcolor{red}{Def électronégativité : }Grandeur sans dimension caractérisant la capacité qu'à un atome engagé dans une liaison à attirer les électrons vers lui.\\
Tableau périodique : $\chi(O)>\chi(H)$.\\

\textcolor{red}{Def charge partielle : }$\delta^{\pm}$

\textcolor{red}{Def polarisation d'une liaison chimique : }si $\Delta\chi=0$ : liaison apolaire. si $\Delta\chi\neq 0$ : liaison polaire.

\subsection{La liaison hydrogène}
\textcolor{red}{Def liaison hydrogène : }liaison de type attractive entre deux entités polaires. $E_{liaison-H}=$qqkJ/mol. C'est une liaison électrostatique attractive entre deux charges partielles de signes opposés. Permet d'expliquer, en partie, les différences de température d'ébullition entres des corps purs.


\section{Propriétés chimiques de l'eau}

\textcolor{green}{Manipulation : dépollution de l'eau :} mélange d'eau et de I$_2$ pour faire passer $I_2$ de l'eau au cyclohexane. Puis extraction liquide-liquide avec une ampoule à décanter.

\section*{Conclusion} 
\end{reportBlock}
\newpage
\input{LC_SPCL/LC_Solubilite.tex}
\newpage
\input{LC_SPCL/LC_Diagramme_binaire.tex}
\newpage
\begin{headerBlock}
\chapter{Stéréochimie de configuration}
\label{LC_Stéréochimie}
 \end{headerBlock}

%%%%%%%%%%%%%%%%%%%%%%%%%%%%%%%%%%%%%%%%%%%%%%%%%%%%
%%%% Références


%%%%%%%%%%%%%%%%%%%%%%%%%%%%%%%%%%%%%%%%%%%%%%%%%%%%
%%%% Plan
\begin{reportBlock}{Bibliographie}

\begin{center}
\begin{tabularx}{\textwidth}{| X | X | c | c |}\hline
Titre & Auteur(s) & Editeur (année) & ISBN \\ \hline
\url{http://ressources-stl.fr/wp-content/uploads/2020/08/Structure-spatiale-des-especes-chimiques.pdf} & Ressource STL & ~ & ~ \\
\hline
\url{https://truejulosdu13.github.io/assets/organic_chemistry_lessons/Cours2.pdf} & Jules Schleinitz &  ~ & ~ \\ \hline
 \url{https://spcl.ac-montpellier.fr/moodle/course/view.php?id=61} & Académie de Montpellier & Chapitre 9 & ~ \\ 
 \hline
\end{tabularx}
\end{center}

\end{reportBlock}

\begin{reportBlock}{Plan détaillé}

\underline{Niveau} : Tle STL - SPCL \\

\section*{Introduction pédagogique}


\paragraph*{Prérequis}
\begin{itemize}
\item représentation de Lewis, Cram
\item formalismes des flèches courbes
\end{itemize}

\paragraph*{Contexte :}
Place de la leçon : dernier cours de chimie.s

\paragraph*{Notions importantes}

\begin{itemize}
\item Stéréoisomèrie (enantiomère, diastéréoisomère)
\item stabilité des carbocations,
\item polarimétrie de Laurent
\end{itemize}

\paragraph*{Objectifs}

\begin{itemize}
\item géométrie carbocation
\item déterminer excès énantiomérique par la loi de Biot
\item règles CIP
\end{itemize}

\paragraph*{Difficultés}

\begin{itemize}
\item vocabulaire,
\item représentation dans l'espace,
\item manipuler la loi de Biot
\end{itemize}


\section*{Introduction}
On a vu les différents types de réactions à l'aide de mécanismes réactionnels, on va s'intéresser ici à la géométrie des produits obtenus.\\

\textcolor{green}{Expérience :} \url{https://spcl.ac-montpellier.fr/moodle/pluginfile.php/18438/mod_resource/content/3/Chapitre\%209\%20-\%20Aspets\%20microscopiques\%20-\%20Activite\%203.pdf}

\section{Intermédiaire réactionnel}

\subsection{Exemple de mécanisme}


\subsection{Stabilité}

\subsection{Stéréoisomère}


\section{Polarimétrie de Laurent}

\subsection{Principe}

\subsection{Loi de Biot}

\subsection{Excès énantiomérique}



\section{}

\subsection{}






\section{Conclusion} 


\end{reportBlock}






\newpage
\begin{headerBlock}
\chapter{Electrolyse - électrosynthèse}
\label{LC_Electrolyse_SPCL}
 \end{headerBlock}

%%%%%%%%%%%%%%%%%%%%%%%%%%%%%%%%%%%%%%%%%%%%%%%%%%%%
%%%% Références


%%%%%%%%%%%%%%%%%%%%%%%%%%%%%%%%%%%%%%%%%%%%%%%%%%%%
%%%% Plan
\begin{reportBlock}{Bibliographie}

\begin{center}
\begin{tabularx}{\textwidth}{| X | X | c | c |}\hline
Titre & Auteur(s) & Editeur (année) & ISBN \\ \hline
 Chimie tout-en-un PC/PC* Chap 13 & A. Demolliens & Nathan (2009) &  \\ 
 \hline
 \url{https://spcl.ac-montpellier.fr/moodle/course/view.php?id=62&section=10} & Académie Montpellier & & \\
 \hline
 Epreuve orale de Chimie (p194 et p398) & F. Porteu-de-Buchère & Dunod (2017) &  \\
 \hline
\end{tabularx}
\end{center}

\end{reportBlock}

\begin{reportBlock}{Plan détaillé}

\underline{Niveau} : Terminale STL-SPCL \\

\section*{Introduction pédagogique}


\paragraph*{Prérequis}
\begin{itemize}
\item 
\end{itemize}
\paragraph*{Contexte :}


\paragraph*{Notions importantes}

\begin{itemize}
\item 
\end{itemize}

\paragraph*{Objectifs}

\begin{itemize}
\item 
\end{itemize}

\paragraph*{Difficultés}

\begin{itemize}
\item 
\end{itemize}

\section*{Introduction}

\section{}

\section*{Conclusion} 

\end{reportBlock}
\newpage
\input{LC_SPCL/LC_Oxydoreduction_SPCL}
\newpage
\input{LC_SPCL/LC_Conductimetrie/LC22_Conductimétrie.tex}
\newpage
\begin{headerBlock}
\chapter{Cinétique électrochimique}
\label{LC_CinetiqueElectrochimique}
 \end{headerBlock}

%%%%%%%%%%%%%%%%%%%%%%%%%%%%%%%%%%%%%%%%%%%%%%%%%%%%
%%%% Références


%%%%%%%%%%%%%%%%%%%%%%%%%%%%%%%%%%%%%%%%%%%%%%%%%%%%
%%%% Plan
\begin{reportBlock}{Bibliographie}

\begin{center}
\begin{tabularx}{\textwidth}{| X | X | c | c |}\hline
Titre & Auteur(s) & Editeur (année) & ISBN \\ \hline
 &  &  &  \\ 
 \hline
\end{tabularx}
\end{center}

\end{reportBlock}

\begin{reportBlock}{Plan détaillé}

\underline{Niveau} : PSI \\

\section*{Introduction pédagogique}


\paragraph*{Prérequis}
\begin{itemize}
\item Oxydoréduction
\item Thermochimie
\end{itemize}
\paragraph*{Contexte :}


\paragraph*{Notions importantes}

\begin{itemize}
\item Vitesse de réaction
\end{itemize}

\paragraph*{Objectifs}

\begin{itemize}
\item Tracer une courbe intensité-potentiel
\item déterminer la solubilité et prévoir son évolution
\end{itemize}

\paragraph*{Difficultés}

\begin{itemize}
\item Réaction en milieu hétérogène
\item Pk montage à 3 électrodes
\end{itemize}
Pour y remédier,différencier milieu homogène, milieu hétérogène. 
\section*{Introduction }
Réaction à la surface d'une électrode. On va relier vitesse et thermodynamique.

\section{Réaction électrochimique}

\subsection{Description}
On s'intéresse à l'équilibre :
\begin{equation}
    Red = Ox + ne^-
\end{equation}

Ex : Fe$^{2+}$ = Fe$^{3+}$ + 1 e$^-$. Cette réaction n'a pas de réalité physique car l'électron n'existe pas en solution. En revanche, à la surface d'une électrode métallique, il y a formation d'un électron libre : c'est ce qu'on appelle une réaction électrochimique.

\subsection{Facteur cinétique}
En milieu homogène : 
\begin{itemize}
    \item La proba de renconre entre deux réactifs
    \item La proba que la réaction se produise quand les réactifs sont en contact
\end{itemize}
En milieu hétérogène :
\begin{itemize}
    \item Les réactifs doivent atteindre a surface de l'électrode
    \item le transfert d'électron doit se produire
    \item le produit doit s'éloigner de la surface de l'électrode
\end{itemize}
Schéma (slide) étapes d'une réaction électrochimique.

\subsection{Intensité et vitesse de réaction}
\underline{Rappel :} la vitesse de réaction $v$ est égale à la dérivée temporelle de l'avancement $\xi$.\\
En milieu homogène, on s'intéresse à des vitesses volumiques. Ici, il est plus adéquat de considérer des vitesses surfaciques.\\
\textcolor{green}{définition :} En notant $A$ la surface immergée de l'électrode : 
\begin{equation}
    v = \frac{1}{A}\frac{d\xi}{dt} = \frac{1}{nA}\frac{dn_{e^-}}{dt}
\end{equation}
Ex : pour Fe$^{2+}$ = Fe$^{3+}$ + 1 e$^-$, $n=1$. Comme la variation de la quantité de matière d'électron échangée est proportionnelle à la variation de charge :
\begin{equation}
    dq = Fdn_{e^-}
\end{equation}
On a finalement :
\begin{equation}
    \frac{1}{nAF}\frac{dq}{dt} = \frac{i}{nAF} = \frac{j}{nF}
    \end{equation}

\textcolor{green}{convention :} On définit positivement le courant anodique $i_a$ \textit{i.e.} lié à l'oxydation et négativement le courant cathodique $i_c$ provoqué par la réduction.

\section{Courbe intensité-potentiel}

\subsection{Montage à 3 électrodes}
Slide montage à deux électrodes : ne fonctionne pas.\\
Objectif : étudier le courant en imposant un potentiel à l'électrode de travail. Pour garder un potentiel fixe à l'électrode de travail on a besoin d'un montage à 3 électrodes. (24')\\
Schéma du montage à 3 électrodes.\\ 
La contre-électrode permet de fermer le circuit électrique sans que du courant ne passe par l'électrode de référence. Utilité d'un générateur : sert à imposer une différence de potentiel. Le potentiostat permet d'imposer la différence de potentiel.

\subsection{Limitation de la cinétique}
\begin{itemize}
    \item \textbf{transfert de charge :} transfert électronique à l'interface électrode/solution
    \item \textbf{transfert de matière :} Comprend les phénomènes de diffusion et de convection 
\end{itemize}

\subsection{Système lent (34')}
Transfert de charge cinétiquement déterminant. Il existe une surtension à appliquer pour observer un courant dans l'électrode.\\
Définition surtension anodique/cathodique à l'aide d'un schéma.
\subsection{Système rapide (40')}
\textcolor{blue}{expérience :} Tracer d'une courbe intensité-potentiel 
\section{Conclusion} 


\end{reportBlock}
\newpage
\begin{headerBlock}
\chapter{Diagrammes E-pH}
\label{LC_DiagrammeEpH}
 \end{headerBlock}

%%%%%%%%%%%%%%%%%%%%%%%%%%%%%%%%%%%%%%%%%%%%%%%%%%%%
%%%% Références


%%%%%%%%%%%%%%%%%%%%%%%%%%%%%%%%%%%%%%%%%%%%%%%%%%%%
%%%% Plan
\begin{reportBlock}{Bibliographie}

\begin{center}
\begin{tabular}{|c|c|c|c|}\hline
Titre & Auteur(s) & Editeur (année) & ISBN \\ \hline
BUP n$^0$790 ~ & ~ & ~ & ~ \\
\hline
\end{tabular}
\end{center}

\end{reportBlock}

\begin{reportBlock}{Plan détaillé}

\underline{Niveau} : TSI2 \\

\section*{Introduction pédagogique}


\paragraph*{Prérequis}
\begin{itemize}
\item Réactions redox
\item Réactions A/B
\item Potentiel de Nernst
\end{itemize}

\paragraph*{Contexte :}
Place de la leçon : milieu d'année.

\paragraph*{Notions importantes}

\begin{itemize}
\item 
\end{itemize}

\paragraph*{Objectifs}

\begin{itemize}
\item
\end{itemize}

\paragraph*{Difficultés}

\begin{itemize}
\item
\end{itemize}

\section*{Introduction }

\textbf{Question:}

\paragraph*{Manipulation qualitative:} \textcolor{green}{}

\section{Lecture d'un diagramme E-pH}

\subsection{}
Déf : diagramme E-pH. Exemple du diagramme du Fer.

\subsection{Calculs des équations aux frontières}


\section{Utilisation d'un diagramme potentiel pH}

\subsection{Diagramme potentiel-pH de l'eau}

\section{Utilisation d'un diagramme E-pH à des fins expérimentales}

\subsection{Procédé industriel à l'aide d'un diagramme E-pH}
Présentation de la méthode de Winkler.

\section{Conclusion} 


\end{reportBlock}

\begin{reportBlock}{Questions posées}

\begin{itemize}

\item 
\textcolor{purple}{}

\end{itemize}


\end{reportBlock}

\begin{reportBlock}{Commentaires}

\end{reportBlock}


\begin{reportBlock}{Expérience 1}
% bloc à dupliquer autant de fois que d'expériences

\underline{Titre} :  \\

\underline{Référence complète} :  \\ 

\underline{But de la manip} : \\

\underline{Commentaire éventuel} : 

\underline{Phase présentée au jury} :\\

\underline{Durée de la manip} : \\

\end{reportBlock}



\begin{reportBlock}{Expérience 2}
% bloc à dupliquer autant de fois que d'expériences

\underline{Titre} : 

\underline{Référence complète} :  \\ 

\underline{Équation chimique et but de la manip} : \\

\underline{Phase présentée au jury} :  \\

\underline{Durée de la manip} : \\

\end{reportBlock}


\begin{reportBlock}{Expérience 3}
% bloc à dupliquer autant de fois que d'expériences

\underline{Titre} : \\

\underline{Référence complète} : \\ 

\underline{Équation chimique et but de la manip} :  \\


\underline{Commentaire éventuel :} 

\underline{Phase présentée au jury} \\

\underline{Durée de la manip} :  \\

\end{reportBlock}



\begin{reportBlock}{Compétence \og Autour des valeurs de la République et des thématiques relevant de la laïcité et de la citoyenneté \fg{}}

\underline{Question posée} : \\

\underline{Réponse proposée} : \\ 

\underline{Commentaire du correcteur} : \\

\end{reportBlock}


\begin{reportBlock}{Champ libre pour le correcteur}
% compléments, propositions de manipulation, bibliographie etc.

\paragraph*{Remarques sur le plan}


\paragraph*{Vocabulaire}

\paragraph*{Équipements de protection individuelle}

\paragraph*{Autre expérience possible} 

\end{reportBlock}
\newpage
\begin{headerBlock}
\chapter{Synthèse, traitement et caractérisations}
\label{LC_SyntheseTraitement}
 \end{headerBlock}

%%%%%%%%%%%%%%%%%%%%%%%%%%%%%%%%%%%%%%%%%%%%%%%%%%%%
%%%% Références


%%%%%%%%%%%%%%%%%%%%%%%%%%%%%%%%%%%%%%%%%%%%%%%%%%%%
%%%% Plan
\begin{reportBlock}{Bibliographie}

\begin{center}
\begin{tabularx}{\textwidth}{| X | X | c | c |}\hline
Titre & Auteur(s) & Editeur (année) & ISBN \\ \hline
 &  &  &  \\ 
 \hline
\end{tabularx}
\end{center}

\end{reportBlock}

\begin{reportBlock}{Plan détaillé}

\underline{Niveau} : 1ère Générale \\

\section*{Introduction pédagogique}


\paragraph*{Prérequis}
\begin{itemize}
\item Modélisation d'une transfo chimique par une réaction chimique
\item Caractéristiques physico-chimiques d'espèces chimiques
\item Techniques expérimentales
\end{itemize}
\paragraph*{Contexte :}
Leçon de milieu d'année après Structure des entités organiques. 2ème chimique de chimie orga.

\paragraph*{Notions importantes}

\begin{itemize}
\item 
\end{itemize}

\paragraph*{Objectifs}

\begin{itemize}
\item Compétences expérimentales : montage à reflux, filtration, lavage, séparation. Caractérisations par CCM
\item 
\end{itemize}

\paragraph*{Difficultés}

\begin{itemize}
\item Vocabulaire spécifique
\item Aspect inventaire
\end{itemize}

\section*{Introduction }
Synthèse organique dans la vie de tous les jours  synthèse du paracétamol, synthèse de l'indigo (pour les jeans). Intérêt : produire de façon massive des molécules qui existent déjà dans la vie de tous les jours.

\section{Synthèse en chimie organique}

\subsection{Qu'est-ce-qu'une synthèse ?}
\textcolor{green}{Définition :} Une synthèse  = ensemble des étapes de fabrication d'une ou plusieurs espèces chimiques purs, implicant a transformation chimique de réactifs.\\

Ex : acide benzoïque (formule chimique semi-développée pas à connaitre). Présentation du montage. A la fin de synthèse, on a retirer le produit obtenu et on l'a filtré à l'aide d'une papier filtre. On va étudier le filtrat. Présentation de la filtration sur Büchner.\\

Etapes d'une synthèse :
\begin{enumerate}
    \item Transformation chimique
    \item Extraction du milieu réactionnel
    \item Identification du produit et analyse de la pureté
    \item Purification éventuelle
\end{enumerate}

\subsection{La transformation chimique}
Les réactifs sont transformés en produits selon l'équation de réaction.\\

Ex : Synthèse de l'acide benzoïque 
\begin{chemmath}
4MnO_4^-(aq) + 3C_6H_5CH_2OH(l) \longrightarrow 3C_6H_5COO^-(aq) + 4MnO_2(s) +4HO^-(aq) + 4H_2O(l)
\end{chemmath}

\underline{Paramètres :}
\begin{itemize}
    \item Chauffage : accélère la réaction (ex : chauffage à reflux)
    \item Solvant : les réactifs doivent y être tous solubles si possible
    \item Autres : agitation, concentrations réactifs, catalyseur, ...
\end{itemize}

\section{Extraction du produit du mélange réactionnel}
\subsection{Cas où l'espèce est un liquide}
\begin{itemize}
    \item Si l'espèce est non-miscible avec le solvant : décantation
    \item Si l'espèce est miscible avec le solvant et que sa température d'ébullition diffère de plus de 20°C avec celle du solvant : distillation fractionnée
\end{itemize}
\subsection{Cas où l'espèce est un solide}
On réalise une filtration simple ou sur filtre sur Büchner.
\subsection{Cas où l'espèce est un soluté}
On change les conditions expérimentales. On peut :
\begin{itemize}
    \item baisser le pH
    \item diminuer la température
    \item introduire une espèce plus soluble : cristalliser le soluté
    \item ou sinon on peut réaliser une extraction liquide-liquide
\end{itemize}

\section{Efficacité de la synthèse}
\subsection{Identification du produit}
\begin{itemize}
    \item CCM,
    \item mesure de la température de fusion,
    \item spectre infrarouge,
    \item mesure de la masse molaire,
    \item mesure $T_{eb}$ pour un liquide.
\end{itemize}

\subsection{Purification}
Cette étape permet d'éliminer les impuretés contenues dans le brut.\\

Ex : recristallisation. On dissout le produit d'intérêt dans un solvant dans lequel il est soluble à chaud mais pas à froid, les impuretés sont solubles à chaud ou à froid. Etalonnage du banc Kofler.\\

\section*{Conclusion}
Dans une prochaine leçon, on verra les notions de rendements.
\end{reportBlock}
\newpage
\begin{headerBlock}
\chapter{Biomolécules et énergie}
\label{LC_BiomoleculesEnergie}
 \end{headerBlock}

%%%%%%%%%%%%%%%%%%%%%%%%%%%%%%%%%%%%%%%%%%%%%%%%%%%%
%%%% Références


%%%%%%%%%%%%%%%%%%%%%%%%%%%%%%%%%%%%%%%%%%%%%%%%%%%%
%%%% Plan
\begin{reportBlock}{Bibliographie}

\begin{center}
\begin{tabularx}{\textwidth}{| X | X | c | c |}\hline
Titre & Auteur(s) & Editeur (année) & ISBN \\ \hline
 &  &  &  \\ 
 \hline
\end{tabularx}
\end{center}

\end{reportBlock}

\begin{reportBlock}{Plan détaillé}

\underline{Niveau} : 1ère ST2S \\

\section*{Introduction pédagogique}


\paragraph*{Prérequis}
\begin{itemize}
\item Description des molécules
\item Molécules d'intérêt biologique (glucide, lipide, protides (acides aminés, petites protéines))
\item Energie en alimentation (calorie)
\item réaction de combustion
\end{itemize}
\paragraph*{Contexte :}
Dernière leçon sur les biomolécules

\paragraph*{Notions importantes}

\begin{itemize}
\item 
\end{itemize}

\paragraph*{Objectifs}

\begin{itemize}
\item Comprendre l'intérêt des biomolécules
\item déterminer les transfo possibles
\end{itemize}

\paragraph*{Difficultés}

\begin{itemize}
\item vocabulaire spécifique
\item nombreuses transformations chimiques avec des molécules de grandes tailles
\end{itemize}

\section*{Introduction }
Quels intérêts des molécules pour le corps humain ? Energie apportée par ces molécules. 

\section{Les biomolécules pour l'organisme}

\subsection{Les biomolécules comme sources d'énergie}
3 nutriments majoritaires :
\begin{itemize}
    \item protides : 4kcal/g de protides, servent aux structures pour les autres composés biologiques du corps et au fonctionnement. Elles représentent 17\% de l'énergie apporté au corps humain
    \item lipides : 9kcal/g. Effort physique assez important. 35-40\% des besoins énergétiques de l'organisme. Problème : le transformer en énergie par le corps humain
    \item glucides : 4kcal/g. Source d'énergie majoritaire pour le corps humain. Moitié des besoins en énergie pour l'organisme.
\end{itemize}
Apport journalier en énergie : 2000kcal pour un humain normalement constitué.\\

\textcolor{blue}{Manip quali : tests de caractérisation lipides/glucides/protides} : test de bi-urée (en milieu basique).

\subsection{Exemple des glucides}
Glucides : hydrates de carbone : C$_x$(H$_2$O)
\begin{itemize}
    \item simple : non hydrolysable, monosaccharide
    \item complexe : combinaison de plusieurs glucides simples : hydrolysable.
\end{itemize}

Ex simple : glucose, fructose, galactose qui sont des isomères de constitution C$_6$H$_{12}$O.\\

Ex complexe : bisaccharide : saccharose : glucose + fructose, maltose : 2 glucoses, lactose : glucose + galactose. Polysaccharides : amidons, glycogène.

\textcolor{blue}{lancement manip imposée : hydrolyse d'un glucide complexe} cf Nathan ST2S.

\section{Transformation biochimique dans l'organisme}

\subsection{La réaction d'hydrolyse}

\textcolor{green}{Définition :} Une réaction de rupture de liaison covalente au sein d'une molécule par action d'une molécule d'eau. Elle permet de former deux plus petites molécules.\\

Ex : hydrolyse de saccharose dont l'équation bilan est :
\begin{equation}
    C_{12}H_{22}O_{11} + H_2O = C_6H_{12}O_6 + C_6H_{12}O_6
\end{equation}
Cependant c'est une réaction lente. C'est pourquoi on procède à un montage à reflux pour augmenter la température et on a ajouté un catalyseur : acide chlorhydrique. Dans le corps humain, les catalyseurs sont les enzymes.

\subsection{Transformation du glucose et production d'énergie}

2 voies possibles :
\begin{itemize}
    \item aérobie en présence d'O$_2$
    \item aérobie sans d'O$_2$
\end{itemize}

1ère étape de glycolyse (acide pyruvique):
\begin{equation}
Blabla    
\end{equation}

\textcolor{blue}{On arrête la manip. Pas forcément utilisable dans le temps imparti de la leçon. Mais hydrolyse faite en préparation et utilisable. Test à la liqueur de Fehling.}

\section{Conclusion} 
On a pu voir les molécules qui apportent l'énergie.
\end{reportBlock}
\newpage
\begin{headerBlock}
\chapter{Gestion des risques en laboratoire de chimie}
\label{LC_GestionRisquesLabo}
 \end{headerBlock}

%%%%%%%%%%%%%%%%%%%%%%%%%%%%%%%%%%%%%%%%%%%%%%%%%%%%
%%%% Références


%%%%%%%%%%%%%%%%%%%%%%%%%%%%%%%%%%%%%%%%%%%%%%%%%%%%
%%%% Plan
\begin{reportBlock}{Bibliographie}

\begin{center}
\begin{tabularx}{\textwidth}{| X | X | c | c |}\hline
Titre & Auteur(s) & Editeur (année) & ISBN \\ \hline
 &  &  &  \\ 
 \hline
\end{tabularx}
\end{center}

\end{reportBlock}

\begin{reportBlock}{Plan détaillé}

\underline{Niveau} : 1ère ST2S \\

\section*{Introduction pédagogique}


\paragraph*{Prérequis}
\begin{itemize}
\item 
\end{itemize}

\paragraph*{Contexte :}
Leçon de début d'année.

\paragraph*{Notions importantes}

\begin{itemize}
\item 
\end{itemize}

\paragraph*{Objectifs}

\begin{itemize}
\item 
\end{itemize}

\paragraph*{Difficultés}

\begin{itemize}
\item 
\end{itemize}

\section*{Introduction }
Produit ménagers = 12000 morts/an en France. Phrase d'accroche : on a ici de l'eau de javel, ici un détartant comme le vinaigre. Il ne faut surtout pas les mélangers ! On va essayer de comprendre pourquoi.
\section{Manipuler en sécurité}

\subsection{Les bons gestes}

\begin{itemize}
    \item Porter une blouse
    \item Porter des lunettes
    \item Attacher ses cheveux
    \item Porter des gants
\end{itemize}

En cas d'accident :
\begin{itemize}
    \item Contact avec l pau : rincer abondament
    \item contact avec les yeux : rincer avec le rince \oe il
    \item Inhalation : respirer de l'air frais
    \item boire de l'eau
\end{itemize}

\subsection{Savoir lire les signes}

\begin{itemize}
    \item Pictogrammes de sécurité (description sur slide, source \url{profpinedon.ekablog.com}
    \item étiquettes : ex du méthanol
\end{itemize}


\section{Sécurité des produits acides et basiques}

\subsection{Couple acide/base}

\textcolor{green}{Définition (au sens de Brönsted) :} Espèce chimique est une espèce susceptible de donner un proton. Une base est une espèce susceptible de capter un proton.\\

Cas particulier de l'eau (espèce ampholyte).\\

\begin{itemize}
    \item Plus la solution est acide, plus $[H_3O^+]$ est importante.
    \item Plus la solution est acide, plus $[HO^-]$ est importante.
\end{itemize}

\subsection{pH et concentration}

\textcolor{green}{définition :} le pH quantifie l'acidité/la basicité d'une espèce chimique. $[H_3O^+]=-10^{-pH}$.\\

En pratique, on vérifie au papier pH l'acidité d'une solution. On peut aussi utiliser le pH mètre si on veut être précis lors d'une expérience.

\subsection{Neutralisation}

\textcolor{blue}{Manipulation imposée : neutralisation d'un acide}

\section{Sécurité des produits oxydants et réducteurs}

Antiseptique : agit sur les tissus vivants. Ex : I$_2$ dans la bétadine.\\

Désinfectant : tue tout objet inerte. Ex : ions ClO$^-$ dans l'eau de javel.

\subsection{Couple oxydo-réducteur}
\textcolor{green}{Définition :} Un réducteur = espèce chimique capable de capter des électrons. Un oxydant = espèce chimique capable de céder des électrons.

\subsection{Précautions d'emploi}
\begin{itemize}
    \item lire l'étiquette
    \item diluer
    \item ne pas mélanger
\end{itemize}

Ex : eau de javel + détartrant = formation de dichlore gazeux = mortel.

\subsection{Neutralisation}

\begin{itemize}
    \item dilution
    \item neutralisation par mélange. Ex : $I_2+2S_2O_3^{2-} = 2I^-+S_4O_6^{2-}$
\end{itemize}
\section{Conclusion} 


\end{reportBlock}
\newpage
\input{LC_STI2D/LC_Oxydroreduction}
\end{document}