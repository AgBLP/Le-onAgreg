\begin{headerBlock}
\chapter{L'eau, propriétés physiques et chimiques}
\label{LC_Eau}
 \end{headerBlock}

%%%%%%%%%%%%%%%%%%%%%%%%%%%%%%%%%%%%%%%%%%%%%%%%%%%%
%%%% Références


%%%%%%%%%%%%%%%%%%%%%%%%%%%%%%%%%%%%%%%%%%%%%%%%%%%%
%%%% Plan
\begin{reportBlock}{Bibliographie}

\begin{center}
\begin{tabular}{|c|c|c|c|}\hline
Titre & Auteur(s) & Editeur (année) & ISBN \\ \hline
BUP n$^0$790 ~ & ~ & ~ & ~ \\
\hline
\end{tabular}
\end{center}

\end{reportBlock}

\begin{reportBlock}{Plan détaillé}

\underline{Niveau} : 1èr ST2S \\

\section*{Introduction pédagogique}


\paragraph*{Prérequis}
\begin{itemize}
\item Barycentre
\item Formules développées et semi-développées
\end{itemize}

\paragraph*{Contexte :}
Place de la leçon : milieu d'année.

\paragraph*{Notions importantes}

\begin{itemize}
\item Constante de solubilité
\item sens de l'évolution
\item influence de la solubilité (T, pH, etc...)
\item protocole 
\end{itemize}

\paragraph*{Objectifs}

\begin{itemize}
\item prédire la précipitation, dissolution
\item déterminer la solubilité et prévoir son évolution
\end{itemize}

\paragraph*{Difficultés}

\begin{itemize}
\item Appréhender le caractère polaire/apolaire liée à la géométrie
\item distinctions entre liaisons covalente et hydrogène
\item miscibilité, solubilité
\end{itemize}
Pour y remédier, activité préliminiaire pour manipuler les grandeurs
\section*{Introduction }
Dissolution courante dans la vie de tous les jours.\\
\textbf{Question:}

\paragraph*{Manipulation qualitative:} \textcolor{green}{}

\section{Electronégativité}

\subsection{Liaison polarisée}

\textcolor{red}{Def électronégativité : }Grandeur sans dimension caractérisant la capacité qu'à un atome engagé dans une liaison à attirer les électrons vers lui.\\
Tableau périodique : $\chi(O)>\chi(H)$.\\

\textcolor{red}{Def charge partielle : }$\delta^{\pm}$

\textcolor{red}{Def polarisation d'une liaison chimique : }si $\Delta\chi=0$ : liaison apolaire. si $\Delta\chi\neq 0$ : liaison polaire.

\subsection{La liaison hydrogène}
\textcolor{red}{Def liaison hydrogène : }liaison de type attractive entre deux entités polaires. $E_{liaison-H}=$qqkJ/mol. C'est une liaison électrostatique attractive entre deux charges partielles de signes opposés. Permet d'expliquer, en partie, les différences de température d'ébullition entres des corps purs.


\section{Propriétés chimiques de l'eau}

\textcolor{green}{Manipulation : dépollution de l'eau :} mélange d'eau et de I$_2$ pour faire passer $I_2$ de l'eau au cyclohexane. Puis extraction liquide-liquide avec une ampoule à décanter.

\section{Conclusion} 


\end{reportBlock}

\begin{reportBlock}{Questions posées}

\begin{itemize}

\item 
\textcolor{purple}{}

\end{itemize}


\end{reportBlock}

\begin{reportBlock}{Commentaires}

\end{reportBlock}


\begin{reportBlock}{Expérience 1}
% bloc à dupliquer autant de fois que d'expériences

\underline{Titre} :  \\

\underline{Référence complète} :  \\ 

\underline{But de la manip} : \\

\underline{Commentaire éventuel} : 

\underline{Phase présentée au jury} :\\

\underline{Durée de la manip} : \\

\end{reportBlock}



\begin{reportBlock}{Expérience 2}
% bloc à dupliquer autant de fois que d'expériences

\underline{Titre} : 

\underline{Référence complète} :  \\ 

\underline{Équation chimique et but de la manip} : \\

\underline{Phase présentée au jury} :  \\

\underline{Durée de la manip} : \\

\end{reportBlock}


\begin{reportBlock}{Expérience 3}
% bloc à dupliquer autant de fois que d'expériences

\underline{Titre} : \\

\underline{Référence complète} : \\ 

\underline{Équation chimique et but de la manip} :  \\


\underline{Commentaire éventuel :} 

\underline{Phase présentée au jury} \\

\underline{Durée de la manip} :  \\

\end{reportBlock}



\begin{reportBlock}{Compétence \og Autour des valeurs de la République et des thématiques relevant de la laïcité et de la citoyenneté \fg{}}

\underline{Question posée} : \\

\underline{Réponse proposée} : \\ 

\underline{Commentaire du correcteur} : \\

\end{reportBlock}


\begin{reportBlock}{Champ libre pour le correcteur}
% compléments, propositions de manipulation, bibliographie etc.

\paragraph*{Remarques sur le plan}


\paragraph*{Vocabulaire}

\paragraph*{Équipements de protection individuelle}

\paragraph*{Autre expérience possible} 

\end{reportBlock}