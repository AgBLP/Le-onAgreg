%%%%%%%%%%%%%%%%%%%%%%%%%%%%%%%%%%%%%%%%%%%%%%%%%%%%
%%%% En-tête leçon
\begin{headerBlock}
  \chapter{Ondes acoustiques.}
  \label{LP_OndeAcoustique} 
\end{headerBlock}




%%%%%%%%%%%%%%%%%%%%%%%%%%%%%%%%%%%%%%%%%%%%%%%%%%%%
%%%% Références
\begin{center}
\begin{tabularx}{\textwidth}{| X | X | c | c |}
  \hline
  \rowcolor{gray!20}\multicolumn{4}{c}{Bibliographie de la leçon : } \\
  \hline 
  Titre & Auteurs & Editeur (année) & ISBN \\
  \hline
  Tout-en-un PC/PC* & M.-N. Sanz & Dunod (2022) & \\
  \hline 
   \url{http://www.etienne-thibierge.fr/agreg/ondes_poly_2015.pdf} & Ethienne Thibierge & 2015 &  \\
  \hline 
  Ondes H-prépa & J-M Brébec & Hachette (2004) & \\
  \hline
  \url{https://dropsu.sorbonne-universite.fr/s/nyD9Ppz3kH6BHZE} & Clément Sayrin & & \\
  \hline
  Ondes acoustiques & A. Chaigne & Ecole Polytechnique (2011) & \\
  \hline
\end{tabularx}
\end{center}

%%%%%%%%%%%%%%%%%%%%%%%%%%%%%%%%%%%%%%%%%%%%%%%%%%%%

%%%%%%%%%%%%%%%%%%%%%%%%%%%%%%%%%%%%%%%%%%%%%%%%%%%%
%%%% Plan
\begin{reportBlock}{Plan détaillé}

  \textbf{Niveau choisi pour la leçon :} 
  \newline
  \textbf{Prérequis} : \begin{itemize}
      \item 
      \item 
      \item 
  \end{itemize}
  
  \section*{Introduction}

  \section{}

  \subsection{}   

  \section{}

  \subsection{}
  
  \section{}
  


\end{reportBlock}