%%%%%%%%%%%%%%%%%%%%%%%%%%%%%%%%%%%%%%%%%%%%%%%%%%%%
%%%% En-tête leçon
\begin{headerBlock}
  \chapter{Ondes acoustiques.}
  \label{LP_OndeAcoustique} 
\end{headerBlock}




%%%%%%%%%%%%%%%%%%%%%%%%%%%%%%%%%%%%%%%%%%%%%%%%%%%%
%%%% Références
\begin{center}
\begin{tabularx}{\textwidth}{| X | X | c | c |}
  \hline
  \rowcolor{gray!20}\multicolumn{4}{c}{Bibliographie de la leçon : } \\
  \hline 
  Titre & Auteurs & Editeur (année) & ISBN \\
  \hline
  Tout-en-un PC/PC* & M.-N. Sanz & Dunod (2022) & \\
  \hline 
   \url{http://www.etienne-thibierge.fr/agreg/ondes_poly_2015.pdf} & Ethienne Thibierge & 2015 &  \\
  \hline 
  Ondes H-prépa & J-M Brébec & Hachette (2004) & \\
  \hline
  \url{https://dropsu.sorbonne-universite.fr/s/nyD9Ppz3kH6BHZE} & Clément Sayrin & & \\
  \hline
  Ondes acoustiques & A. Chaigne & Ecole Polytechnique (2011) & \\
  \hline
  Optique Physique et électronique & D. Mauras & PUF (2001) & \\
  \hline
\end{tabularx}
\end{center}

\begin{reportBlock}{Commentaires des années précédentes :}
    \begin{itemize}
        \item \textbf{2017 :} La contextualisation et des applications de la vie courante ne doivent pas être oubliées dans cette leçon qui se résume souvent à une suite de calculs. De plus, les fluides ne sont pas les seuls milieux dans lesquels les ondes acoustiques peuvent être étudiées,
        \item \textbf{2014 :} Cette leçon peut être l’occasion de traiter les ondes acoustiques dans les fluides ou dans les milieux périodiques, certes, mais elle peut aussi être l’occasion de traiter les deux cas qui donnent lieu à des phénoménologies très différentes,
        \item \textbf{2013 :} Le candidat est libre d’étudier les ondes acoustiques dans un fluide ou dans un solide élastique.
    \end{itemize}
\end{reportBlock}

%%%%%%%%%%%%%%%%%%%%%%%%%%%%%%%%%%%%%%%%%%%%%%%%%%%%

%%%%%%%%%%%%%%%%%%%%%%%%%%%%%%%%%%%%%%%%%%%%%%%%%%%%
%%%% Plan
\begin{reportBlock}{Plan détaillé}

  \textbf{Niveau choisi pour la leçon :} 
  \newline
  \textbf{Prérequis} : \begin{itemize}
      \item équations de la mécanique des fluides
      \item coefficient de compressibilité isentropique
      \item 
  \end{itemize}
  
  \section*{Introduction}
  Nous allons décrire dans cette leçon la propagation des ondes acoustiques. Deux points sont importants dans cette description :
  \begin{itemize}
      \item La propagation nécessite un milieu matériel pour avoir lieu : expérience de la cloche sous vide (expérience de Otto von Guericke seconde moitié du XVIIème ou Robert Boyle plus performant) \url{https://www.youtube.com/watch?v=BC9Pod4cnpk},
      \item La propagation est issue d'un couplage entre les variations de pression du fluide et les variations de vitesse des particules.
  \end{itemize}
  
  \section{Description des ondes acoustiques}
  Les équations de la mécanique des fluides sont des équations non linéaires et donc difficile voir impossible à résoudre. Pour décrire la propagation des ondes dans les fluides, on va devoir faire quelques hypothèses qu'on pourra justifier a posteriori.
  
  \subsection{Approximation acoustique}   
  On va considéré un fluide parfait (i.e. sans prendre en compte les effets de viscosité, déformations réversibles du milieu) qui est caractérisé par :
  \begin{itemize}
      \item Une pression au repos $P_0$,
      \item une masse volumique uniforme $\rho_0$,
      \item une vitesse moyenne particulaire nulle
  \end{itemize}
  Une onde sonore est une perturbation par rapport à cet état d'équilibre. L'état du fluide est donc décrit par :
  \begin{itemize}
      \item la pression $P(M,t)=P_0+P_1(M,t)$ où $P_1$ est appelé \textcolor{green}{pression acoustique ou surpression},
      \item la masse volumique (compression/dilatation du fluide) $\rho(M,t)=\rho_0 + \rho_1(M,t)$,
      \item la vitesse particulaire $\mathbf{v}(M,t)=\mathbf{0}+\mathbf{v_{1}}(M,t)$.
  \end{itemize}
  \textcolor{red}{approximation acoustique :} on se place dans le cadre de faibles perturbations $\lvert P_1(M,t) \rvert << P_0$, $\lvert rho_1(M,t) \rvert << \rho_0$ et $\lvert \mathbf{v_1}(M,t) \rvert << V_0$ avec $V_0$ une vitesse qu'on déterminera, et on négligera les termes d'ordre 2.\\
  
  \textcolor{red}{Transition :} Maintenant qu'on a placé le cadre de notre étude, on va pouvoir étudier la propagation. Et pour cela on va utiliser les équations de la mécanique des fluides.
  
  \subsection{Equation de propagation du son}
  Voir Mauras p5.\\
  On a à notre disposition deux équations en mécanique des fluides, l'équation locale de la conservation de la masse et l'équation d'Euler (fluide parfait) en négligeant la gravité :
  \begin{align}
      \partialD{\rho}{t} + div(\rho\mathbf{v})(M,t) &= 0 \\
      \rho(M,t)\left(\partialD{\mathbf{v}}{t}(M,t) + (\mathbf{v}\cdot\mathbf{grad})\mathbf{v}(M,t) \right) &= -\mathbf{grad}P(M,t)
  \end{align}
  On va linéariser ces équations en ne gardant que les termes à l'ordre 1. Pour l'équations de conservation de la masse :
  \begin{equation}
      \partialD{\rho_1}{t} + \rho_0div(\mathbf{v_1})(M,t) = 0
  \end{equation}
  où on fait apparaître une relation entre $\rho_1$ et $v_1$ alors qu'on préfèrerait entre $v_1$ et $P_1$. Problème, on a 4 équations pour 5 inconnues. Il en faut donc une cinquième. On va supposer que le comportement du fluide est décrit par une équation du type $\rho =f(P)$. Si on suppose que la passage de l'onde se fait suffisament rapidement pour la transformation du milieu soit adiabatique (p872 Dunod + réversible comme écoulement parfait), on peut utiliser la définition du coefficient de compressibilité isentropique $\chi_S$ :
  \begin{equation}
      \chi_S = -\frac{1}{V}\left(\partialD{V}{P}\right)_S = \frac{1}{\rho}\left(\partialD{\rho}{P}\right)_S
  \end{equation}
  et un développement de Taylor permet d'obtenir $\rho_1 = \chi_S\rho_0P_1$ ce qui donne pour l'équation locale de la conservation de la masse :
  \begin{equation}
      \chi_S\partialD{P_1}{t} + div(\mathbf{v_1})(M,t) = 0
  \end{equation}
  L'hypothèse $\mathbf{v_1}$ petite consiste à négliger le terme $\mathbf{v}\cdot\mathbf{grad})\mathbf{v}(M,t)$ devant $\partialD{\mathbf{v}}{t}$ ce qui est possible si $v<<c=V_0$ qui est la vitesse introduite au début de cette partie. On trouve en linéarisant l'équation d'Euler :
  \begin{equation}
      \rho_0\partialD{\mathbf{v_1}}{t}(M,t) = -\mathbf{grad}P_1(M,t)
  \end{equation}
  En utilisant le théorème de Schwarz, on trouve les équations de propagation de la supression et de la vitesse :
  \begin{align}
    \partialD{^2P_1}{t^2} &= \frac{1}{c^2}\Delta P_1 \\
    \partialD{^2\mathbf{v_1}}{t^2} &= \frac{1}{c^2}\mathbf{\Delta v_1}     
  \end{align}
  avec $c = \frac{1}{\sqrt{\chi_S\rho_0}}$ la vitesse du son dans l'air.\\

  Comme on a une transformation isentropique, on peut utiliser la loi de Laplace :
  \begin{align*}
      PV^{\gamma} &= cste \\
      dPV^{\gamma} + \gamma PV^{\gamma-1}dV &= 0 \\
      \partialD{V}{P} &= -\frac{V}{\gamma P} \\
      \chi_S = -\frac{1}{V}\left( \partialD{V}{P}\right) &= \frac{1}{\gamma P} = \frac{M}{\gamma\rho RT} \\
      c &= \frac{1}{\sqrt{\chi_S\rho}}=\sqrt{\frac{\gamma RT}{M_{air}}}
  \end{align*}
  
  \subsection{Mesure de la vitesse des ondes ultrasonores}
  \textcolor{blue}{Expérience quantitative :} On prend deux piézo-électriques sur un banc optique. Un est relié à un GBF, l'autre à l'oscillo. On mesure un décalage de plusieurs longueurs d'ondes (10$\lambda$) et on prend les positions relatives des capteurs. On détermine $c=\lambda f$. On mesure ici la vitesse de phase. Pour mesurer la vitesse de groupe, utiliser la fonction Burst du GBF, mesurer le temps de vol pour plusieurs positions.\\

  D. Mauras p6, application : séparation isotopiques de l'uranium par diffusion gazeuse à partir d'hexafluorure d'uranium gazeux.
  
  \section{Aspect énergétique}
  


\end{reportBlock}