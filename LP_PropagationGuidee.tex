%%%%%%%%%%%%%%%%%%%%%%%%%%%%%%%%%%%%%%%%%%%%%%%%%%%%
%%%% En-tête leçon
\begin{headerBlock}
  \chapter{Propagation guidée des ondes.}
  \label{LP_PropagationGuidee} 
\end{headerBlock}




%%%%%%%%%%%%%%%%%%%%%%%%%%%%%%%%%%%%%%%%%%%%%%%%%%%%
%%%% Références
\begin{center}
\begin{tabularx}{\textwidth}{| X | X | c | c |}
  \hline
  \rowcolor{gray!20}\multicolumn{4}{c}{Bibliographie de la leçon : } \\
  \hline 
  Titre & Auteurs & Editeur (année) & ISBN \\
  \hline
   Electromagnétisme & Pérez & Dunod & \\
  \hline 
    & & &    \\
  \hline 
   &  & &    \\
  \hline 
\end{tabularx}
\end{center}

%%%%%%%%%%%%%%%%%%%%%%%%%%%%%%%%%%%%%%%%%%%%%%%%%%%%

%%%%%%%%%%%%%%%%%%%%%%%%%%%%%%%%%%%%%%%%%%%%%%%%%%%%
%%%% Plan
\begin{reportBlock}{Plan détaillé}

  \textbf{Niveau choisi pour la leçon :} Licence 3
  \newline
  \textbf{Prérequis} : \begin{itemize}
      \item 
  \end{itemize}

  \textbf{Déroulé détaillé de la leçon: }  
  
  \section*{Introduction}

  \section{Propagation}
  \subsection{Champs couplés}
  \subsection{Impédance caractéristique}

  
  \section{Ondes TEM dans un câble coaxial}

  \subsection{Equation des télgraphistes}

  \subsection{Rapport des ondes stationnaires (ROS)}

  \section{Ondes centimétrique en propagation guidée}

  \textcolor{blue}{Expérience quantitative :} Mesure de la relation de dispersion dans un guide et mesure de la fréquence de propagation dans un guide d'onde (TP Onde 2).
  

\end{reportBlock}