%%%%%%%%%%%%%%%%%%%%%%%%%%%%%%%%%%%%%%%%%%%%%%%%%%%%
%%%% En-tête leçon
\begin{headerBlock}
  \chapter{Notions de viscosité d'un fluide. Ecoulement visqueux.}
  \label{LP_Viscosite} 
\end{headerBlock}




%%%%%%%%%%%%%%%%%%%%%%%%%%%%%%%%%%%%%%%%%%%%%%%%%%%%
%%%% Références
\begin{center}
\begin{tabularx}{\textwidth}{| X | X | c | c |}
  \hline
  \rowcolor{gray!20}\multicolumn{4}{c}{Bibliographie de la leçon : } \\
  \hline 
  Titre & Auteurs & Editeur (année) & ISBN \\
  \hline
  Hydrodynamique physique & Guyon, Hulin, Petit & EDP Sciences & \\
  \hline 
   Poly de cours & Marc Rabaud & &    \\
  \hline 
   &  & &    \\
  \hline 
\end{tabularx}
\end{center}

%%%%%%%%%%%%%%%%%%%%%%%%%%%%%%%%%%%%%%%%%%%%%%%%%%%%

%%%%%%%%%%%%%%%%%%%%%%%%%%%%%%%%%%%%%%%%%%%%%%%%%%%%
%%%% Plan
\begin{reportBlock}{Plan détaillé}

  \textbf{Niveau choisi pour la leçon :} Licence 3
  \newline
  \textbf{Prérequis} : \begin{itemize}
      \item 
  \end{itemize}

  \textbf{Déroulé détaillé de la leçon: }  
  
  \section*{Introduction}
\textcolor{blue}{Expérience qualitative :} Diffusion de la quantité de mouvement dans un cylindre \url{https://www.youtube.com/watch?v=i4wGWJdB7mg}.

  \section{Modèle macroscopique de la viscosité}
cf Chap 2 Guyon Hulin Petit.
  \subsection{Ecoulement de Couette plan}

  \subsection{Tenseur des déformations}
  Voir Guyon Hulin Petit Chap 3 p125.


  \section{Dynamique des fluide visqueux}
  
\subsection{Tenseur des contraintes}

\subsection{Equation de Navier-Stokes}

\subsection{Aspect énergétique}
Fluide visqueux = dissipation d'énergie. Démo p265 Guyon Hulin Petit.
\subsection{Conditions aux limites cinématiques et dynamiques}
Tableau de comparaison fluides parfaits fluide réels.

\section{Mesures de la viscosité}
\subsection{Ecoulement autour d'une sphère : force de Stokes}
\subsection{Mesure de la viscosité de la glycérine}

\end{reportBlock}