%%%%%%%%%%%%%%%%%%%%%%%%%%%%%%%%%%%%%%%%%%%%%%%%%%%%
%%%% En-tête leçon
\begin{headerBlock}
  \chapter{Notions de viscosité d'un fluide. Ecoulement visqueux.}
  \label{LP_Viscosite} 
\end{headerBlock}




%%%%%%%%%%%%%%%%%%%%%%%%%%%%%%%%%%%%%%%%%%%%%%%%%%%%
%%%% Références
\begin{center}
\begin{tabularx}{\textwidth}{| X | X | c | c |}
  \hline
  \rowcolor{gray!20}\multicolumn{4}{c}{Bibliographie de la leçon : } \\
  \hline 
  Titre & Auteurs & Editeur (année) & ISBN \\
  \hline
  Hydrodynamique physique & Guyon, Hulin, Petit & EDP Sciences & \\
  \hline 
  Poly de cours & Marc Rabaud & &    \\
  \hline 
  Physique Spé PC/PC* & S. Olivier & Tec\&Doc (2000) &    \\
  \hline 
\end{tabularx}
\end{center}

%%%%%%%%%%%%%%%%%%%%%%%%%%%%%%%%%%%%%%%%%%%%%%%%%%%%

%%%%%%%%%%%%%%%%%%%%%%%%%%%%%%%%%%%%%%%%%%%%%%%%%%%%
%%%% Plan
\begin{reportBlock}{Plan détaillé}

  \textbf{Niveau choisi pour la leçon :} CPGE 2ème année
  \newline
  \textbf{Prérequis} : \begin{itemize}
  \item cinématiques des fluides
      \item loi de l'hydrostatique, poussée d'Archimède
      \item diffusion
  \end{itemize}

  \begin{reportBlock}{Commentaires des années précédentes :}
    \begin{itemize}
        \item \textbf{2017 :} Il peut être judicieux de présenter le fonctionnement d’un viscosimètre dans cette leçon,
        \item \textbf{2016 :} Le jury invite les candidats à réfléchir d’avantage à l’origine des actions de contact mises en jeu entre un fluide et un solide,
        \item \textbf{2014, 2013, 2012, 2011 :} L’exemple de l’écoulement de Poiseuille cylindrique n’est pas celui dont les conclusions sont les plus riches. Les candidats doivent avoir réfléchi aux différents mécanismes de dissipation qui peuvent avoir lieu dans un fluide. L’essentiel de l’exposé doit porter sur les fluides newtoniens : le cas des fluides non newtoniens, s’il peut être brièvement mentionné ou résenté, ne doit pas prendre trop de temps et faire perdre de vue le message principal.
    \end{itemize}
\end{reportBlock}

  \textbf{Déroulé détaillé de la leçon: }  
  
  \section*{Introduction}
  On a décrit dans un cours précédent l'équation du mouvement pour un fluide parfait. Hors, il existe beaucoup de situation ou le modèle du fluide parfait ne permet pas d'expliquer le comportement de l'écoulement du fluide. \textcolor{blue}{Manip qualitative :} on voit bien que l'écoulement de l'eau et du miel ne se fait pas à la même vitesse. D'autre part, on peut observer avec le miel que les couches verticales du fluide se mettent successivement en mouvement (vitesse nulle au fond du bécher qui augmente progressivement). Il y a ce qu'on appelle une force de viscosité qui va diffuser la quantité de mouvement du fluide.

  \section{Notion de viscosité}
cf Chap 2 Guyon Hulin Petit.
\subsection{Viscosité de cisaillement}
Voir Tec\&Doc p418. On considère un écoulement de la forme $\mathbf{v}=v_x(z)$ dans un fluide. On considère un élément de surface $\mathbf{dS}=dxdy\mathbf{\hat{u_z}}$ d'ordonnée z séparant le fluide situé au-dessus de z et en dessous de z. L'action de contact, appelée \textcolor{green}{force de viscosité}, exercée par le fluide situé au-dessus de z sur le fluide situé en-dessous de z est tangentielle à la vitesse de l'écoulement :
\begin{equation}
    d\mathbf{F_t} = \eta\partialD{v_x}{z}dS\mathbf{\hat{u_x}}
\end{equation}
où on introduit le coefficient de viscosité cinématique $\eta$ de dimension Pa.s$^{-1}$ comme on peut le voir avec la formule. \textcolor{green}{Donner quelques exemples de $\eta$ sur slide (miel et eau)}. 
\textbf{Remarques :}
\begin{itemize}
    \item Plus $\eta$ est grand plus la force de viscosité de cisaillement est élevée (c'est pour ça qu'on dit que le miel est plus visqueux que l'eau),
    \item force non nulle si champ de vitesse inhomogène
    \item le sens de la force est tel qu'il tend à homogénéiser le champ des vitesses : la couche de fluide du dessus qui va plus vite met en mouvement la couche de dessous : \textbf{il y a une diffusion de la quantité de mouvement} des couches du fluide les plus rapides vers les couches les moins rapides.
\end{itemize} 

  \subsection{Equivalent volumique de la force de viscosité}
Considérons un pavé élementaire de volume $d\tau=dxdydz$ en prenant le même champ de vitesse que précédemment. Un bilan des forces conduit à :
\begin{itemize}
    \item $d\mathbf{F}_{z+dz}=\eta\partialD{v_x}{z}(z+dz)\mathbf{u_x}$ car le fluide sur la couche de dessus va plus vite,
    \item  $d\mathbf{F}_{z}=-\eta\partialD{v_x}{z}(z)\mathbf{u_x}$ car le fluide sur la couche de dessous va moins vite.
\end{itemize}
Un développement de Taylor à l'ordre 1 permet d'écrire la résultante comme :
\begin{align}
    d\mathbf{F} &= \eta\partialD{^2v_x}{d_z^2}d\tau\mathbf{u_x} \\
    \mathbf{f}_{visc}&=\frac{d\mathbf{F}}{d\tau} = \eta\mathbf{\Delta v}
\end{align}
\textbf{La dernière formulation est valable pour les écoulements incompressibles !}. C'était une hypothèse dont on se passait pour les fluides parfaits.


  %\subsection{Tenseur des déformations}
  %Voir Guyon Hulin Petit Chap 3 p125.
  
  \textcolor{red}{Transition :} Maintenant qu'on a vu la force volumique de viscosité pour un fluide visqueux, on peut l'injecter dans les équations du mouvement du champ de vitesse eulérien du fluide.

  \section{Dynamique d'un fluide visqueux}

  \subsection{Equation de Navier-Stokes}
  Le principe fondamental de la dynamique appliquée à une particule fluide suivant le champ de vitesse eulérien $\mathbf{v}$ du fluide s'écrit désormais :
  \begin{equation}
      \rho\left(\partialD{\mathbf{v}}{t} + (\mathbf{v}\cdot\grad)\mathbf{v}\right) = -\grad P + \rho \mathbf{g} + \eta\Delta\mathbf{v} + \mathbf{f}_{vol}
  \end{equation}
  C'est l'équation de Navier-Stokes.\\
  \textbf{Remarques :}
  \begin{itemize}
      \item comme l'équation d'Euler, cette équation est non-linéaire donc très difficile à résoudre numériquement ou analytiquement (problème du millénaire en maths),
      \item 
  \end{itemize}
  Analysons ce qui se passe lorsqu'on suppose le terme convectif nul (ce qui est possible par exemple pour un écoulement parallèle tel que celui décrit précédemment


  \subsection{Transport diffusif de la quantité de mouvement}
\subsection{Tenseur des contraintes}

\subsection{Equation de Navier-Stokes}

\subsection{Aspect énergétique}
Fluide visqueux = dissipation d'énergie. Démo p265 Guyon Hulin Petit.
\subsection{Conditions aux limites cinématiques et dynamiques}
Tableau de comparaison fluides parfaits fluide réels.

\section{Mesures de la viscosité}
\subsection{Ecoulement autour d'une sphère : force de Stokes}
\subsection{Mesure de la viscosité de la glycérine}

\end{reportBlock}