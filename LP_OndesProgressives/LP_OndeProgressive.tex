%%%%%%%%%%%%%%%%%%%%%%%%%%%%%%%%%%%%%%%%%%%%%%%%%%%%
%%%% En-tête leçon
\begin{headerBlock}
  \chapter{Ondes progressives, ondes stationnaires.}
  \label{LP_OndesProgressives} 
\end{headerBlock}




%%%%%%%%%%%%%%%%%%%%%%%%%%%%%%%%%%%%%%%%%%%%%%%%%%%%
%%%% Références
\begin{center}
\begin{tabularx}{\textwidth}{| X | X | c | c |}
  \hline
  \rowcolor{gray!20}\multicolumn{4}{c}{Bibliographie de la leçon : } \\
  \hline 
  Titre & Auteurs & Editeur (année) & ISBN \\
  \hline
  Tout-en-un PC/PC* & M.-N. Sanz & Dunod (2022) & \\
  \hline 
   \url{http://www.etienne-thibierge.fr/agreg/ondes_poly_2015.pdf} & Ethienne Thibierge & 2015 &    \\
   \hline
   \url{http://images.math.cnrs.fr/Spectre.html\#nh7} &  San Vu Ngoc & CNRS & \\
  \hline 
  Ondes H-prépa & J-M Brébec & Hachette (2004) & \\
  \hline
  \url{https://dropsu.sorbonne-universite.fr/s/nyD9Ppz3kH6BHZE} & Clément Sayrin & & \\
\end{tabularx}
\end{center}

\begin{reportBlock}{Commentaires des années précédentes :}
    \begin{itemize}
        \item \textbf{2015 :} Les candidats doivent être attentifs à bien équilibrer leur exposé entre ces deux familles d’ondes qui, d’ailleurs, ne s’excluent pas entre elles,
        \item \textbf{2014 :} À l’occasion de cette leçon, le jury tient à rappeler une évidence : avec un tel titre, la leçon doit être équilibrée et ne peut en aucun cas se limiter pour l’essentiel aux ondes progressives.
    \end{itemize}
\end{reportBlock}

%%%%%%%%%%%%%%%%%%%%%%%%%%%%%%%%%%%%%%%%%%%%%%%%%%%%

%%%%%%%%%%%%%%%%%%%%%%%%%%%%%%%%%%%%%%%%%%%%%%%%%%%%
%%%% Plan
\begin{reportBlock}{Plan détaillé}

  \textbf{Niveau choisi pour la leçon :} CPGE 2ème année
  \newline
  \textbf{Prérequis} : \begin{itemize}
      \item Forces, énergie
      \item 
      \item 
  \end{itemize}
  
  \section*{Introduction}
\textcolor{green}{Slide} Quel est le point commun entre tous ces phénomènes : Olà dans un stade, effet d'une goutte qui tombe dans une flaque, un séisme, une corde soumise à des vibrations ? L'objectif de cette leçon est de donner les similitudes entre ces phénomènes et de pouvoir en retirer des caractéristiques générales.

  \section{Approches du phénomène de propagation}

  \subsection{Elongation d'une corde sans raideur}
  Voir Hprépa p30.
  \textbf{Hypothèses :} On regarde des petites perturbations de la hauteur de la corde notée $y(x,t)$. 
  Obtenir l'équation de d'Alembert unidimentionnelle.\\
  
  Donner les équations couplant $y(x,t)$ et $T(x,t)$ (p32) en disant que le phénomène de propagation est contenue dans ces équations liant vitesse transverse $\fracD{y}{t}$ et la projection de la tension suivant $\mathbf{\hat{u_y}}$.\\

  \textcolor{red}{Transition : Ce couplage entre deux grandeurs rappelle celui de la tension et du courant. On peut effectivement faire l'analogie de la propagation de U et i dans un câble coaxial avec la tension et la déformation de la corde.}

  \subsection{Analogie électrocinétique : le câble coaxial} 
  Voir HPrépa p58. \textcolor{blue}{Prendre câble coaxial, montrer l'âme et la gain et le diélectrique}. Faire le schéma d'un câble coaxial avec âme et gaine et le schéma électriue équivalent.\\
  \textbf{Hypothèses :} On ne se place plus dans l'ARQS pour le câble tout entier car on veut voir des phénomènes de propagation mais on peut le découper en tronçons dans lesquels la propagation est négligeable. \\
  
  Obtenir les équations couplées (loi des mailles et loi des n\oe uds) et obtenir l'équation de d'Alembert unidimentionnelle.\\
  \textcolor{blue}{Manip quantitative : mesure de la vitesse de propagation de i ou u dans le câble coaxial avec un câble de 100m de long. Envoyer un pulse d'impulsion 100ns, montrer le décalage sur la voie Y et mesurer ce décalage avec les curseurs. On obtient $v=\frac{2c}{3}$. On peut en déduire $n=\frac{v}{c}=\sqrt{\epsilon_r}$}.

  Définir la notion d'onde (poly Thibierge p10) : \og Une onde correspond à la propagation d’une perturbation à travers un milieu. Cette perturbation est générée par une source, qui apporte de l’énergie au milieu. L'existence de deux grandeurs couplées qui se créent l'une l'autre est à la base du phénomène de propagation.\\

  Dans le cas de la corde, une personne à mis en mouvement la corde. Dans le cas de la ligne, un expérimentateur a décidé de mettre un GBF. Dans le cas de la olà, un groupe de personne déclanche ce phénomène. Dans le cas de la goutte d'eau, un robinet qui fuit a amené de l'eau au milieu.\\

  \textcolor{green}{Slide : tableau analogie}
  
  \textcolor{red}{On a décrit le phénomène de propagation comme l'existence de deux grandeurs couplées qui se propagent dans un milieu. On va maintenant voir la forme des solutions de l'équation de d'Alembert.}
  
  

  \section{Ondes progressives}
  \subsection{Solution de l'équation de d'Alembert unidimentionnelle}
  A voir si on a le temps de la démontrer.
  
  \subsection{Interprétation}
  Voir Tec\&Doc p128 ou Dunod p806-807 et Thibierge Exos C1. Faire les dessins avec correspondance temporelle et spatiale.

  \subsection{OPPH}
  Idée : tout signal périodique peut se décomposer en série de Fourier dont chaque terme est solution de l'équation de d'Alembert. Cela est possible par la linéarité de l'équation.\\
  Obtenir la relation de dispersion. Définir vitesse de phase et vitesse de groupe. 
  
  \section{Onde stationnaire}
  On peut imaginer une onde progressive et une onde regressive qui se propagent et qui se rencontrent. 
  

  \subsection{Corde de Melde}

  \section{Conclusion}
  Conclure sur absorption, dispersion, impédance. Prendre exemple Dunod p818 en introduisant une force de frottement.


\end{reportBlock}