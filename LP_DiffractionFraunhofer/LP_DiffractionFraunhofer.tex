%%%%%%%%%%%%%%%%%%%%%%%%%%%%%%%%%%%%%%%%%%%%%%%%%%%%
%%%% En-tête leçon
\begin{headerBlock}
  \chapter{Diffraction de Fraunhofer.}
  \label{LP_DiffractionFraunhofer} 
\end{headerBlock}




%%%%%%%%%%%%%%%%%%%%%%%%%%%%%%%%%%%%%%%%%%%%%%%%%%%%
%%%% Références
\begin{center}
\begin{tabularx}{\textwidth}{| X | X | c | c |}
  \hline
  \rowcolor{gray!20}\multicolumn{4}{c}{Bibliographie de la leçon : } \\
  \hline 
  Titre & Auteurs & Editeur (année) & ISBN \\
  \hline
  Tout-en-un PC/PC* & M.-N. Sanz & Dunod (2022) & \\
  \hline
\end{tabularx}
\end{center}

%%%%%%%%%%%%%%%%%%%%%%%%%%%%%%%%%%%%%%%%%%%%%%%%%%%%
\begin{reportBlock}{Commentaires des années précédentes :}
    \begin{itemize}
        \item \textbf{2017 :} Les conditions de Fraunhofer et leurs conséquences doivent être présentées, ainsi que le lien entre les dimensions caractéristiques d’un objet diffractant et celles de sa figure de diffraction,
        \item \textbf{2014-2011 :} Les conditions de l’approximation de Fraunhofer doivent être clairement énoncées. Pour autant, elles ne constituent pas le coeur de la leçon.
    \end{itemize}
\end{reportBlock}
%%%%%%%%%%%%%%%%%%%%%%%%%%%%%%%%%%%%%%%%%%%%%%%%%%%%
%%%% Plan
\begin{reportBlock}{Plan détaillé}

  \textbf{Niveau choisi pour la leçon :} 
  \newline
  \textbf{Prérequis} : \begin{itemize}
      \item 
      \item 
      \item 
  \end{itemize}
  
  \section*{Introduction}

  \section{Conditions de diffraction}

  \subsection{Théorème de Huygens-Fresnel} 
  Enoncé, faire les dessins.
  \subsection{Diffraction par une structure}
  Voir TD Optique Pauline - Diffraction. Arriver à Fraunhofer vs Fresnel. 

  \subsection{Conditions de Fraunhofer}
  Bien distinguer les conditions + exemples
  Parler de théorème de Van Citter-Zernike ? Peut être utile par la suite et ça fait transition.
  \section{? Exemples de systèmes diffractants ?}
  
  \subsection{Diffraction par une fente d'Young}
  \textcolor{blue}{Manip qualitative : caméra CCD, éclairement par une source à l'infini, image à l'infini }

  \subsection{Diffraction par un trou d'Young}
  Arriver à la formule
  
  \section{Application au filtrage spatial}

  \subsection{Expérience d'Abbe}


\end{reportBlock}