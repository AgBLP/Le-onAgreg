%%%%%%%%%%%%%%%%%%%%%%%%%%%%%%%%%%%%%%%%%%%%%%%%%%%%
%%%% En-tête leçon
\begin{headerBlock}
  \chapter{Mécanismes de la conduction électrique dans les solides}
    \label{LP_Conduction}
\end{headerBlock}

%%%%%%%%%%%%%%%%%%%%%%%%%%%%%%%%%%%%%%%%%%%%%%%%%%%%
%%%% Références
\begin{center}
\begin{tabularx}{\textwidth}{| X | X | c | c |}
  \hline
  \rowcolor{gray!20}\multicolumn{4}{c}{Bibliographie de la leçon : } \\
  \hline 
  Titre & Auteurs & Editeur (année) & ISBN \\
  \hline
Physique des Solides (Chap 1 à 3)  & N. Ashcroft et D.Mermin   &  EDP Sciences (2002) & 2-86883-577-5  \\
  \hline 
     Slides de cours & Gwendal Fève &  &  Site Montrouge \\
  \hline 
  Physique des Solides & C. Kittel &  Dunod &  \\
\hline
\end{tabularx}
\end{center}

%%%%%%%%%%%%%%%%%%%%%%%%%%%%%%%%%%%%%%%%%%%%%%%%%%%%

\section{Approche classique}
\subsection{Modèle de Drude (1902) 10min max}
%Définition résistivité : $\rho=\frac{RS}{L}$.\\
Contexte historique : découverte de l'électron par Thomson (1899), avant expérience de Rutherford.\\
Cristal métallique : [grosses sphères chargés + avec des électrons de c\oe ur] = ions métalliques  + électrons de valence pour assurer électroneutralité.\\
Remarque : $n_{gaz}=10^{25}$m$^{-3}$ tandis que $n_{e^{-}}=10^{29}$m$^{-3}$, cf p4 Ashcroft.\\
Hypothèses :\begin{itemize}
    \item pas d'interaction entre les électrons et les ions, et entre les électrons et les électrons : \textcolor{red}{"électrons libres"},
    \item \textcolor{red}{collisions et changement de vitesse instanées} entre les électrons de c\oe ur et les électrons de valence,
    \item électrons de valence (conduction) se déplacent en ligne droite jusqu'à collision avec proba $1/\tau$. $\tau$ est \textcolor{red}{le temps de collision} ou libre parcourt moyen, temps moyen de propagation entre deux chocs,
    \item \textcolor{red}{chaos moléculaire}, la distribution des vitesses suit une loi de Maxwell-Boltzmann comme le gaz classique, la direction des électrons est aléatoire. L'équilibre thermodynamique local des électrons avec leur entourage par le biais des collisions (seul mécanisme restant) : 
\end{itemize}

Ce modèle décrit relativement bien la conductivité dans un métal.
Déduire équation du mouvement + 
\subsection{Mise en défaut pratique du modèle de Drude}
Voici ce que Drude prévoit en supposant que la vitesse des électrons obéit à la statistique de Boltzmann :
\begin{itemize}
    \item dans un modèle de gaz parfait, la vitesse moyenne des électrons est : $v^*=\sqrt{\frac{3k_BT}{m_e}}=\frac{l}{tau}$ (utiliser le théorème d'équipartition.
    \item comme $\tau=\frac{\sigma m_e}{N_ee^2}$ avec le modèle de Drude, on obtient $\sigma\propto \frac{1}{\sqrt{T}}$
\end{itemize}

\textcolor{blue}{Manip quantitative : mesure 4 points d'un morceau de cuivre à différentes températures. Montrer (si possible) que la résistivité est linéaire en température.}\\
\begin{itemize}
    \item en utilisant $1/2mv^2=3/2k_BT$, on trouve $v=10^5$m/s alors que $v_{F}=10^6$m/s et indépendante de T,
    \item quid des matériaux isolants ? semi-conducteurs ?
    \item effet Hall classique dans l'aluminium (p18 Ashcroft) montre que n dépend fortement de B appliqué, ce qui est inenvisagable dans le modèle de Drude + constante de Hall parfois positive
\end{itemize}

Conclusion : il faut passer à un autre modèle plus poussé : un modèle quantique.

\section{Modèle quantique des électrons libres}
Electrons obéissent à la statistique de Fermi-Dirac et non à la statistique de Maxwell-Boltzmann.
\subsection{Modèle des électrons libres}
Cf Kittel chap 8 ou Ashcroft chap 2 et 13. Modèle de Sommerfeld. Faire les calculs Justifier qu'on peut se placer à T=0K.

\section{Théorie des bandes}

\subsection{Electrons dans un potentiel périodique}
Modèle : électrons en intéraction avec le potentiel périodique du réseau cristallins.\\
Cf Kittel (chap 7) + TD 1 Physique des Solides : mettre en prérequis pour les calculs.

\subsection{Bandes d'énergie}
Slide : schéma bande avec isolant+ tableau périodique 

\subsection{Remplissage des bandes}

Distinguer métaux, semi-conducteurs et isolant. Prévoir lesquels seront métalliques avec le demi-remplissage, lesquels seront isolants. Montrer que ça ne marche pas toujours.
\subsection{Conductivité dans un semi-conducteur}
Densité d'état varie en $\exp(-\Delta/k_BT)$ donc $\sigma$ aussi.\\

\textcolor{blue}{Mesure de la conductivité dans un semi-conducteur dopé pour déterminer le gap.}, cf TP semiconducteurs. Durée 30min environ, $\Delta_{exp}=0.6$eV alors que $\Delta_{th}=0.7$eV. Interprétation : matériel vieux, gap a pu changer. Matériau pas pur.\\
\section{Conclusion}
Ouverture sur l'ingénierie des semi-conducteurs : micro-électronique, jonctions pn, transistors. Prix Nobel 2014 pour l'invention de la diode bleue (en 1992).\\
Ouverture sur la supraconductivité.\\
Ouverture possible sur la conduction thermique, loi de Wiedemann-Franz.