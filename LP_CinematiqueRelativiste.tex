%%%%%%%%%%%%%%%%%%%%%%%%%%%%%%%%%%%%%%%%%%%%%%%%%%%%
%%%% En-tête leçon
\begin{headerBlock}
  \chapter{Cinématique relativiste. Expérience de Michelson et Morlay}
    \label{LP_CinematiqueRelativiste}
\end{headerBlock}

%%%%%%%%%%%%%%%%%%%%%%%%%%%%%%%%%%%%%%%%%%%%%%%%%%%%
%%%% Références
\begin{center}
\begin{tabularx}{\textwidth}{| X | X | c | c |}
  \hline
  \rowcolor{gray!20}\multicolumn{4}{c}{Bibliographie de la leçon : } \\
  \hline 
  Titre & Auteurs & Editeur (année) & ISBN \\
  \hline
  Relativité & M. Boratev et R. Kerner & Ellipses & \\
  \hline
  La relativité & A. Einstein & Payot & \\
  \hline
  Relativité et Invariance - Fondements et applications & J.-P. Pérez & Dunod & \\
  \hline
  Relativité Restreinte (exos) & Y. Simon & Armand Colin & \\
  \hline
  Mécanique 1 (Chap 1) & BFR & Dunod (1984) & \\
  \hline
\end{tabularx}
\end{center}

%%%%%%%%%%%%%%%%%%%%%%%%%%%%%%%%%%%%%%%%%%%%%%%%%%%%
\begin{reportBlock}{Plan détaillé}
  \textbf{Niveau choisi pour la leçon :} L3
  \newline
  \textbf{Prérequis : }
  \newline


\section{Emergence de la relativité restreinte}

\subsection{Transformation de galilée en électromagnétisme}
Les équations de l'électromagnétisme ne sont valables que dans le référentiel de l'éther pour concilier mécanique classique et électromagnétisme.
\subsection{Expérience de Michelson et Morlay}
Montrer les résultats de l'expérience de Michelson et Morlay (Michelson et Morlay, 1887). 

\subsection{Postulat d'Einstein}

\section{Changement de référentiel}

\subsection{Evènements}

\subsection{Transformation de Lorentz}

\subsection{Intervalle d'espace-temps}

\section{Conséquences physiques}
\subsection{Dilatation du temps}
\textcolor{blue}{Expérience :} utilisaion du programme Gum\_C pour la détermination de $\gamma$ des muons d'après l'expérience de Frish et Smith.
\subsection{Contraction des longueurs}

\end{reportBlock}