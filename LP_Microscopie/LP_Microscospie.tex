%%%%%%%%%%%%%%%%%%%%%%%%%%%%%%%%%%%%%%%%%%%%%%%%%%%%
%%%% En-tête leçon
\begin{headerBlock}
  \chapter{Microscopies optiques}
  \label{LP_Microscopie} 
\end{headerBlock}




%%%%%%%%%%%%%%%%%%%%%%%%%%%%%%%%%%%%%%%%%%%%%%%%%%%%
%%%% Références
\begin{center}
\begin{tabularx}{\textwidth}{| X | X | c | c |}
  \hline
  \rowcolor{gray!20}\multicolumn{4}{c}{Bibliographie de la leçon : } \\
  \hline 
  Titre & Auteurs & Editeur (année) & ISBN \\
  \hline
  Les instruments d'optique & Luc Detwiller & Ellipses (1997) & \\
  \hline 
  \url{http://ressources.agreg.phys.ens.fr/media/ressources/RessourceFichiers/11-Maxime_Dahan_-_Microscopie_pour_la_biologie.pdf} & M. Dahan & Vraiment bien & \\
  \hline
  \url{https://www.nikonsmallworld.com/} & Nikon & &    \\
  \hline 
  Optique & Eugène Hecht & Pearson &   \\
  \hline 
  Optique & Sylvain Houard & de Boeck & \\
  \hline
\end{tabularx}
\end{center}

%%%%%%%%%%%%%%%%%%%%%%%%%%%%%%%%%%%%%%%%%%%%%%%%%%%%

%%%%%%%%%%%%%%%%%%%%%%%%%%%%%%%%%%%%%%%%%%%%%%%%%%%%
%%%% Plan
\begin{reportBlock}{Plan détaillé}

  \textbf{Niveau choisi pour la leçon :} 
  \newline
  \textbf{Prérequis} : \begin{itemize}
      \item Optique géométrique : lentilles, construction images par une lentille
      \item Diffraction par un cercle
      \item 
  \end{itemize}

  \textbf{Déroulé détaillé de la leçon: }  
  
  \section*{Introduction}
  L'homme a voulu voler, voir loin dans l'Univers mais aussi voir l'infiniment petit.\\
  On appelle micrsocopie l'ensemble des techniques permettant de rendre discernables et visuels des objets indiscernables à l'\oe il nu (environ 1 minute d'arc = $0.017\degree$ pour une vision 10/10 soit environ 100km de la surface de la Lune). 

  \section{Le microscope \og d'autrefois \fg }

  \subsection{Présentation du dispositif}
  Faire le schéma du microscope.\\
  \textcolor{green}{L'objectif} : lentille convergente de courte focale qui fait une image intermédiaire $A_1B_1$ de l'objet $AB$ à aggrandir. \textcolor{green}{L'oculaire} : lentille convergente faisant une image à l'infini de $A_1B_1$ pour que l'\oe il qui s'y accole n'accomode pas. On doit donc avoir $\bar{O_2A_1}=-f_2'$.

  \subsection{Grossissement commercial}
  \textcolor{green}{Grossissement oculaire }: $G_{c,oc}$, rapport entre l'angle sous lequel est vu l'objet ($A_1B_1$) à travers l'oculaire et l'angle sous lequel est vu le même objet à travers $G_{c,oc}=\frac{\alpha'}{\alpha_1}=\frac{A_1B_1/f_2'}{A_1B_1/d_m}=\frac{d_m}{f_2'}$.\\
  \textcolor{green}{Grandissement objectif} : $`\gamma_{obj}=\frac{\bar{A_1B_1}}{AB}=-\frac{\Delta}{f_1'}$ par le théorème de Thalès+formule de conjugaison de Descartes.\\
  \textcolor{green}{Grandissement commercial} : $G_{c}=\frac{\alpha'}{\alpha}=\frac{A_1B_1/f_2'}{AB/d_m}=\lvert \gamma_{obj} \rvert G_{c,oc}$\\  
  Le grossissement commercial d'un microscope est donné par :
  \begin{equation}
     G_{com} = \lvert \gamma_{ob}\rvert G_{oc} = \frac{\alpha'}{\alpha}
  \end{equation}
 \textcolor{blue}{Expérience quantitative :} Faire l'image d'une mire micrométrique par un microscope optique. Pour cela : \begin{itemize}
      \item utiliser une lampe quartz-iode/LED,
      \item 1 condenseur de 8 ou 12 cm pour focaliser l lumière sur la mire
      \item un microscope avec une mire micrométrique (pas 0.1mm),
      \item une lentille de focale 1m ou 150cm,
      \item fixer l'écran à la distance focale de la lentille,
      \item ajuster le microscope pour avoir une image nette sur l'écran
  \end{itemize} 
 On mesure $\alpha'= \frac{Taille-objet-sur-l'écran}{distance-objet-écran}$ ainsi que les incertitudes associées. 
 \importantbox{On doit faire des traits bien droits pour la mesure de la distance à l'écran. Pour cela, prendre une feuille et la coller à l'écran, reproduire les traits sur la feuille. Faire la mesure à la règle proprement sur la feuille en traçant des angles droits.} En préparation, j'ai trouvé $G_{com}=40.5(3)$ à comparer à 40. Bon ordre de grandeur, la valeur peut-être différente de 40, le microscope ne coûte pas cher et pas d'incertitudes sur la valeur du constructeur.\\
  
  Parler de : 
  \begin{itemize}
      \item conditions d'éclairement (éclairage Köhler
      \item résolution optique (ouverture numérique, diffraction, aberrations) 
      \item contraste
      \item microscopie plein champ/ point par point
  \end{itemize}

  \subsection{Eclairage de Köhler}
  
  \textbf{Transition :} on va voir comment améliorer le microscope classique


  \section{Contraste et résolution}

  \subsection{Résolution verticale : profondeur de champ}
  
  \subsection{Résolution latéral : critère de Rayleigh}

\section{Microscopie à contraste de phase}
Cette technique s'intéresse en particulier à des échantillons transparents dont les épaisseurs sont faibles \textcolor{green}{Slide photos avec ou sans contraste de phase + photos microscopies Nikon}. Elle a valu le prix Nobel à Frederik Zernike en 1953.
\subsection{Principe}


\section*{Ouverture}
Microscopie électronique par effet tunnel

\end{reportBlock}