\begin{headerBlock}
\chapter{Biomolécules et énergie}
\label{LC_BiomoleculesEnergie}
 \end{headerBlock}

%%%%%%%%%%%%%%%%%%%%%%%%%%%%%%%%%%%%%%%%%%%%%%%%%%%%
%%%% Références


%%%%%%%%%%%%%%%%%%%%%%%%%%%%%%%%%%%%%%%%%%%%%%%%%%%%
%%%% Plan
\begin{reportBlock}{Bibliographie}

\begin{center}
\begin{tabularx}{\textwidth}{| X | X | c | c |}\hline
Titre & Auteur(s) & Editeur (année) & ISBN \\ \hline
 &  &  &  \\ 
 \hline
\end{tabularx}
\end{center}

\end{reportBlock}

\begin{reportBlock}{Plan détaillé}

\underline{Niveau} : 1ère ST2S \\

\section*{Introduction pédagogique}


\paragraph*{Prérequis}
\begin{itemize}
\item Description des molécules
\item Molécules d'intérêt biologique (glucide, lipide, protides (acides aminés, petites protéines))
\item Energie en alimentation (calorie)
\item réaction de combustion
\end{itemize}
\paragraph*{Contexte :}
Dernière leçon sur les biomolécules

\paragraph*{Notions importantes}

\begin{itemize}
\item 
\end{itemize}

\paragraph*{Objectifs}

\begin{itemize}
\item Comprendre l'intérêt des biomolécules
\item déterminer les transfo possibles
\end{itemize}

\paragraph*{Difficultés}

\begin{itemize}
\item vocabulaire spécifique
\item nombreuses transformations chimiques avec des molécules de grandes tailles
\end{itemize}

\section*{Introduction }
Quels intérêts des molécules pour le corps humain ? Energie apportée par ces molécules. 

\section{Les biomolécules pour l'organisme}

\subsection{Les biomolécules comme sources d'énergie}
3 nutriments majoritaires :
\begin{itemize}
    \item protides : 4kcal/g de protides, servent aux structures pour les autres composés biologiques du corps et au fonctionnement. Elles représentent 17\% de l'énergie apporté au corps humain
    \item lipides : 9kcal/g. Effort physique assez important. 35-40\% des besoins énergétiques de l'organisme. Problème : le transformer en énergie par le corps humain
    \item glucides : 4kcal/g. Source d'énergie majoritaire pour le corps humain. Moitié des besoins en énergie pour l'organisme.
\end{itemize}
Apport journalier en énergie : 2000kcal pour un humain normalement constitué.\\

\textcolor{blue}{Manip quali : tests de caractérisation lipides/glucides/protides} : test de bi-urée (en milieu basique).

\subsection{Exemple des glucides}
Glucides : hydrates de carbone : C$_x$(H$_2$O)
\begin{itemize}
    \item simple : non hydrolysable, monosaccharide
    \item complexe : combinaison de plusieurs glucides simples : hydrolysable.
\end{itemize}

Ex simple : glucose, fructose, galactose qui sont des isomères de constitution C$_6$H$_{12}$O.\\

Ex complexe : bisaccharide : saccharose : glucose + fructose, maltose : 2 glucoses, lactose : glucose + galactose. Polysaccharides : amidons, glycogène.

\textcolor{blue}{lancement manip imposée : hydrolyse d'un glucide complexe} cf Nathan ST2S.

\section{Transformation biochimique dans l'organisme}

\subsection{La réaction d'hydrolyse}

\textcolor{green}{Définition :} Une réaction de rupture de liaison covalente au sein d'une molécule par action d'une molécule d'eau. Elle permet de former deux plus petites molécules.\\

Ex : hydrolyse de saccharose dont l'équation bilan est :
\begin{equation}
    C_{12}H_{22}O_{11} + H_2O = C_6H_{12}O_6 + C_6H_{12}O_6
\end{equation}
Cependant c'est une réaction lente. C'est pourquoi on procède à un montage à reflux pour augmenter la température et on a ajouté un catalyseur : acide chlorhydrique. Dans le corps humain, les catalyseurs sont les enzymes.

\subsection{Transformation du glucose et production d'énergie}

2 voies possibles :
\begin{itemize}
    \item aérobie en présence d'O$_2$
    \item aérobie sans d'O$_2$
\end{itemize}

1ère étape de glycolyse (acide pyruvique):
\begin{equation}
Blabla    
\end{equation}

\textcolor{blue}{On arrête la manip. Pas forcément utilisable dans le temps imparti de la leçon. Mais hydrolyse faite en préparation et utilisable. Test à la liqueur de Fehling.}

\section{Conclusion} 
On a pu voir les molécules qui apportent l'énergie.
\end{reportBlock}