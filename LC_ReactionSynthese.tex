\begin{headerBlock}
\chapter{Reaction de synthèse}
\label{LC_ReactionSynthese}
 \end{headerBlock}

%%%%%%%%%%%%%%%%%%%%%%%%%%%%%%%%%%%%%%%%%%%%%%%%%%%%
%%%% Références


%%%%%%%%%%%%%%%%%%%%%%%%%%%%%%%%%%%%%%%%%%%%%%%%%%%%
%%%% Plan
\begin{reportBlock}{Bibliographie}

\begin{center}
\begin{tabularx}{\textwidth}{| X | X | c | c |}\hline
Titre & Auteur(s) & Editeur (année) & ISBN \\ \hline
 &  &  &  \\ 
 \hline
\end{tabularx}
\end{center}

\end{reportBlock}

\begin{reportBlock}{Plan détaillé}

\underline{Niveau} : 1ère STL-SPCL \\

\section*{Introduction pédagogique}


\paragraph*{Prérequis}
\begin{itemize}
\item synthèse
\item CCM
\item groupes chimiques
\item acide/base, oxydation
\end{itemize}
\paragraph*{Contexte :}


\paragraph*{Notions importantes}

\begin{itemize}
\item 
\end{itemize}

\paragraph*{Objectifs}

\begin{itemize}
\item Tracer une courbe intensité-potentiel
\item déterminer la solubilité et prévoir son évolution
\end{itemize}

\paragraph*{Difficultés}

\begin{itemize}
\item Réaction en milieu hétérogène
\item Pk montage à 3 électrodes
\end{itemize}
Pour y remédier,différencier milieu homogène, milieu hétérogène. 
\section*{Introduction }
Synthèse : réaction et production d'une espèce chimique en laboratoire.

\section{Réactivité}
\subsection{Différentes transformations}
\begin{itemize}
    \item substitution
    \item addition
    \item élimination
\end{itemize}

\subsection{Notion électrophile et nucléophile}
Définitions et exemple.

\subsection{Formalisme des flèches courbes}

\section{Mise en place d'une synthèse : esterification}

\subsection{Principe du montage à reflux}
Schéma du montage.
\textcolor{blue}{Expérience :} Synthèse d'un arome de banane.

\subsection{Mécanisme des flèches courbes appliqué à l'estérifcation}

\end{reportBlock}