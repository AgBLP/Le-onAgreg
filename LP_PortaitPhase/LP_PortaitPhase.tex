%%%%%%%%%%%%%%%%%%%%%%%%%%%%%%%%%%%%%%%%%%%%%%%%%%%%
%%%% En-tête leçon
\begin{headerBlock}
  \chapter{Oscillateurs ; portraits de phase et non-linéarités.}
  \label{LP_PortaitPhase} 
\end{headerBlock}




%%%%%%%%%%%%%%%%%%%%%%%%%%%%%%%%%%%%%%%%%%%%%%%%%%%%
%%%% Références
\begin{center}
\begin{tabularx}{\textwidth}{| X | X | c | c |}
  \hline
  \rowcolor{gray!20}\multicolumn{4}{c}{Bibliographie de la leçon : } \\
  \hline 
  Titre & Auteurs & Editeur (année) & ISBN \\
  \hline
   \url{https://uhincelin.pagesperso-orange.fr/LP49_BUP_portrait_phase_oscil.pdf} & H. Gié &  BUP n°744&    \\
  \hline 
   \url{https://www.lpens.ens.psl.eu/wp-content/uploads/2022/09/MecaniquePourChimistes-3.pdf} & Jules Fillette & &  \\
  \hline 
\end{tabularx}
\end{center}

%%%%%%%%%%%%%%%%%%%%%%%%%%%%%%%%%%%%%%%%%%%%%%%%%%%%
\begin{reportBlock}{Commentaires des années précédentes :}
    \begin{itemize}
        \item \textbf{2017 :} Les définitions d’un oscillateur et d’un portrait de phase sont attendues. La leçon doit présenter des systèmes comportant des non-linéarités,
        \item \textbf{2015 :} L’intérêt de l’utilisation des portraits de phase doit ressortir de la leçon,
        \item \textbf{2013 :} Les aspects non-linéaires doivent être abordés dans cette leçon sans développement calculatoire excessif, en utilisant judicieusement la notion de portrait de phase. Une simulation numérique bien présentée peut enrichir cette leçon.
    \end{itemize}
\end{reportBlock}
%%%%%%%%%%%%%%%%%%%%%%%%%%%%%%%%%%%%%%%%%%%%%%%%%%%%
%%%% Plan
\begin{reportBlock}{Plan détaillé}

  \textbf{Niveau choisi pour la leçon :} CPGE 2ème année
  \newline
  \textbf{Prérequis} : \begin{itemize}
      \item TMC,
      \item Electrocinétique : schéma électrique, loi des mailles, résistance, condensateur, inductance
  \end{itemize}

  \textbf{Déroulé détaillé de la leçon: }  
  
  \section*{Introduction}
Messages à faire passer : définition d'un portait de phase, portait de phase amorti/non-amorti, stabilité des points fixes, centres attracteurs.\\

Le notion de portait de phase est un outil très riche pour l'analyse de nombreux systèmes mécaniques, en particulier pour les oscillateurs. C'est que nous allons essayer de montrer dans cette leçon à travers quelques exemples.

  \section{Oscillateurs non-amortis}
  On va s'intéresser à des systèmes mécaniques ou électrocinétiques dont on va dans un premier temps négliger les phénomènes de dissipation. On va commencer par prendre l'exemple très simple du pendule pesant.
  \subsection{Equation du mouvement d'un pendule pesant}
  Exemple de la physique : pendule simple à petites oscillations. On applique le théorème du moment cinétique appliqué à la masse $m$ dans le référentiel terrestre qu'on suppose galiléen :
  \begin{equation}
     \frac{d\mathbf{L_O}(M)}{dt} = mL^2\ddot{\theta}\mathbf{\hat{u_z}} = \mathbf{M_O}(\mathbf{P}) = -mgLsin(\theta)\mathbf{\hat{u_z}}
  \end{equation}
  On en déduit l'équation d'un oscillateur harmonique de pulsation propre $\omega_0=\sqrt{\frac{g}{L}}$ :
  \begin{equation}
      \ddot{\theta} + \omega_0^2\sin(\theta) = 0
  \end{equation}
  \textbf{Analogies :} courant dans un circuit LC suit la même équation.
  \subsection{Portrait de phase}
  Définition d'un portait de phase pour le pendule pesant.
  \subsection{Approche énergétique}
  Parler de stabilité, conservation de l'énergie mécanique.  

  \section{Effets non linéaires}
  Formule de borda pendule.\\
  \subsection{Etude des effets non linéaires du pendule pesant}
  Faire une acquisition à grand angle. Parler pendant l'acquisition du programme bordaThomas.ipynb

  \section{Bifurcation}
  \subsection{}
  \section*{Conclusion}
  Ouvrir sur les oscillateurs entretenus ? Van der Pol, résistance négative,

  \textcolor{blue}{Manip quanti :} mettre en évidence non linéarités, déterminer formule de Borda.
  


\end{reportBlock}