%%%%%%%%%%%%%%%%%%%%%%%%%%%%%%%%%%%%%%%%%%%%%%%%%%%%
%%%% En-tête leçon
\begin{headerBlock}
  \chapter{Absorbtion et émission de la lumière}
    \label{LP_Absorption}
\end{headerBlock}

%%%%%%%%%%%%%%%%%%%%%%%%%%%%%%%%%%%%%%%%%%%%%%%%%%%%
%%%% Références
\begin{center}
\begin{tabularx}{\textwidth}{| X | X | c | c |}
  \hline
  \rowcolor{gray!20}\multicolumn{4}{c}{Bibliographie de la leçon : } \\
  \hline 
  Titre & Auteurs & Editeur (année) & ISBN \\
  \hline
  Optique Physique & R. Taillet & de Boeck &   \\
  \hline 
  Physique Statistique & Landau et Lifshitz & Ellipses &  \\
  \hline 
   Poly laser & A. Maitre &  &  \\
\hline
 Sextant & & Hermann & \\
 \hline 
 Tout-en-un PC/PC* & M.-N. Sanz & Dunod & \\
 \hline 
 Physique en PC/PC* & Pascal Olive & Ellipses & \\
 \hline
\end{tabularx}
\end{center}

%%%%%%%%%%%%%%%%%%%%%%%%%%%%%%%%%%%%%%%%%%%%%%%%%%%%
\begin{reportBlock}{Plan détaillé}
  \textbf{Niveau choisi pour la leçon :} Licence
  \newline
  \textbf{Prérequis : }Milieu LHI, distribution de Boltzmann
  \newline
  
  \textbf{Déroulé détaillé de la leçon: }   \newline
On verra deux modèles qui permettront d'expliquer le phénomène d'émission et absorption de la lumière dans les milieux.
  \section{Modèle du corps noir}
  
  \subsection{Electron élastiquement lié}


\section{Modèle d'Einstein}
\subsection{Equation d'Einstein}
Soit un système à deux niveaux d'énergie $E_a$ et $E_b$ associée aux états propres $\ket{a}$ et $\ket{b}$. On note $N_a$ et $N_b$ la population de ces niveaux telles que $N_a+N_b=N$. La proba d'occupation est $n_i=\frac{N_i}{N}$.\\
La lumière que reçoit ce système est caractérisé par une densité spectrale $u(\nu)=\frac{dU}{d\nu}$ avec $U=\frac{\epsilon_0E^2}{2}+\frac{B^2}{2\mu_0}$.\\


A l'équilibre : $\frac{dn_a}{dt}=-\frac{dn_b}{dt}$ et donc : 
\begin{equation}
    \frac{dn_a}{dt} = -An_b + B_{21}u(\nu_0)
\end{equation}
\subsection{Coefficients d'Einstein}

En posant $h\nu_0 = E_b - E_a$, on a $\frac{n_a}{n_b}=\exp{-\frac{E_a+E_b}{k_BT}} = \exp{\frac{h\nu_0}{b_BT}}$. On en déduit que $B_{12} = B_{21} = B$ et $\frac{A}{B}=\frac{8\pi h\nu_0^3}{c^3}$. L'équation d'Einstein devient : 
\begin{equation}
    \frac{dn_b}{dt} = -\frac{dn_a}{dt} = -An_a + Bu(\nu_0)(n_a-n_b)
\end{equation}

\subsection{Processus de transfert d'énergie}
La puissance transférée de l'atome vers le champ $P_{at\rightarrow ch}=-h\nu_0\frac{dn_b}{dt} = h\nu_0B(n_b-n_a)u(\nu_0)$. On distingue alors deux régimes : 
\begin{itemize}
    \item Si $n_a>n_b$, $P_{at\rightarrow ch}<0$ $\longrightarrow$ Absorbtion
    \item Si $n_a<n_b$, $P_{at\rightarrow ch}>0$ $\longrightarrow$ Inversion de population
\end{itemize}

Initialement $n_a=1$,$n_b=0$ : 
\begin{itemize}
    \item $n_a(\infty) = \frac{A + Bu(\nu_0)}{A + 2Bu(\nu_0)}$
    \item $n_b(\infty) = \frac{Bu(\nu_0)}{A + 2Bu(\nu_0)}$
\end{itemize}
\end{reportBlock}



