%%%%%%%%%%%%%%%%%%%%%%%%%%%%%%%%%%%%%%%%%%%%%%%%%%%%
%%%% En-tête leçon
\begin{headerBlock}
  \chapter{Absorbtion et émission de la lumière}
    \label{LP_Absorption}
\end{headerBlock}

%%%%%%%%%%%%%%%%%%%%%%%%%%%%%%%%%%%%%%%%%%%%%%%%%%%%
%%%% Références
\begin{center}
\begin{tabularx}{\textwidth}{| X | X | c | c |}
  \hline
  \rowcolor{gray!20}\multicolumn{4}{c}{Bibliographie de la leçon : } \\
  \hline 
  Titre & Auteurs & Editeur (année) & ISBN \\
  \hline
  Optique Physique & R. Taillet & de Boeck &   \\
  \hline 
  Physique Statistique & Landau et Lifshitz & Ellipses &  \\
 \hline 
   Poly laser & A. Maitre &  &  \\
 \hline
  Sextant & & Hermann & \\
 \hline 
 Tout-en-un PC/PC* & M.-N. Sanz & Dunod (2019-2022) & \\
 \hline 
 Physique en PC/PC* & Pascal Olive & Ellipses & \\
 \hline
 Laser & CAGNAC & & \\
 \hline
 Optique & S. Houard & de boeck (2011) & \\
 \hline
\end{tabularx}
\end{center}

\begin{reportBlock}{Commentaires des années précédentes :}
    \begin{itemize}
        \item \textbf{2017 :} Cette leçon ne peut se résumer à une présentation des relations d’Einstein,
        \item \textbf{2015 :} Cette leçon peut être traitée de façons très variées, mais il est bon que les candidats aient réfléchi aux propriétés des diverses formes de rayonnements émis, aux dispositifs exploitant ces propriétés et au cadre théorique permettant de les comprendre,
        \item \textbf{2014-2011 : }Trop souvent, il y a confusion entre les processus élémentaires pour un atome et un ensemble d’atomes. De même le candidat doit préciser au cours de sa leçon le caractère monochromatique ou non du champ de rayonnement qu’il considère et plus généralement les caractéristiques du rayonnement stimulé.
    \end{itemize}
\end{reportBlock}

%%%%%%%%%%%%%%%%%%%%%%%%%%%%%%%%%%%%%%%%%%%%%%%%%%%%
\begin{reportBlock}{Plan détaillé}
  \textbf{Niveau choisi pour la leçon :} Licence 3
  \newline
  \textbf{Prérequis : }
  \begin{itemize}
      \item diffraction par un réseau,
      \item modèle du corps noir, loi de Planck,
      \item physique statistique : loi de Boltzmann
  \end{itemize}

  
  \textbf{Déroulé détaillé de la leçon: }   \newline
Spectre émission discret (lumière spectrale) vs spectre d'émission continu (lumière corps noir).\\
L'approche classique : modèle de Drude-Lorentz qui prédisait relativement bien des fréquences d'absorption lumineuse dans les isolants. Mais il restait deux problèmes :
\begin{itemize}
    \item structure discontinue des spectres atomiques,
    \item divergence de la répartition spectrale du rayonnement thermique dans le domaine ultraviolet ("catastrophe ultravolette")
\end{itemize}
Il faut attendre que Planck et Einstein propose la quantification des niveaux d'énergie des atomes/ions/molécules.\\
On va dans un premier temps faire une étude phénoménologique de l'absorption et de l'émission spontanée.
  \section{Absorption et émission spontanée}
  Etudions d'abord ce qu'il se passe pour une collection d'atomes.
  \subsection{Spectre d'émission d'un atome}
  Faire un dessin. Einstein : quantification des niveaux d'énergie des atomes (raies de l'hydrogène mais aussi d'autres espèces) en utilisant les relations de Planck-Einstein. Présenter spectre d'émission et d'absorption d'éléments chimiques. \\
  Utilité : composition de l'atmosphère d'une planète par exemple \url{https://planet-terre.ens-lyon.fr/ressource/atmospheres-systeme-solaire.xml#analyse}
  \subsection{Mesure spectroscopique du spectre d'émission du mercure Hg}
  \textcolor{blue}{Expérience quantitative :} On se propose ici de mesurer quelques raies d'émission du spectre de la lampe à vapeur de mercure. Pour cela on a besoin de :
  \begin{itemize}
      \item un réseau (prendre celui où on peut changer le pas du réseau ENSP 3637), choisir le 3000 traits/mm mais on peut montrer ce que ça change en changeant le pas,
      \item lampe à vapeur de mercure + condenseur de 8cm,
      \item une lentille convergente de 15-20cm de focale,
      \item une fente réglable,
      \item un écran blanc avec une feuille blanche et du scotch.
  \end{itemize}
On fait l'image de la fente sur l'écran à une distance assez éloignée. On intercale le réseau juste après la lentille. On mesure la distance $L\pm u(L)$ entre le réseau et l'écran. On cherche le minimum de déviation en tournant le réseau. Au minimum de déviation, $\tan{D_{pm}}=\frac{d_{ecran}}{L}$. On mesure la distance à l'écran entre l'ordre 0 et l'ordre 1 (ou 2 si on peut). On en déduit $\lambda_p$ :
\begin{equation}
    \lambda_p = \frac{2a}{p}\sin\left(\frac{\arctan{\frac{d_{ecran}}{L}}}{2}\right)
\end{equation}
Comparer aux longueurs d'onde attendues pour le spectre du mercure (voir slide et \url{https://fr.wikipedia.org/wiki/Raie_spectrale}).\\
On a bien une quantification des niveaux d'énergie !

\subsection{Mise en évidence expérimentale de l'absorption}
Faire le spectre d'absorption/transmission de la rhodamine. On prend une lampe quartz-iode, et on éclaire la rhodamine avec un condenseur. On regarde avec une fibre optique connectée à spectra-suite sur l'ordi. On fait le blanc et le noir. On regarde le spectre d'absortion. Youhou.\\

\textcolor{red}{Transition :} On va maintenant donner l'approche d'Einstein pour modéliser l'émission et l'aborption. 

\section{Modèle d'Einstein}
En plus de l'absorption et l'émission qu'on a mis en évidence dans la partie précédente, Einstein a l'idée de rajouter un terme d'émission stimulée pour rendre compte des propriétés microscopiques du rayonnement thermique. 
\subsection{Probabilités de transition}
Voir Pascal Olive p786 ou Dunod p1118.\\
Soit un système à deux niveaux d'énergie $E_1$ et $E_2$ associée aux états propres $\ket{1}$ et $\ket{2}$. On note $N_1$ et $N_2$ la population de ces niveaux telles que $N_1+N_2=N=cste$.\\

La lumière que reçoit ce système est caractérisée par une densité spectrale $u(\nu)=\frac{du_{em}}{d\nu}$ (en J.s.m$^{-3}$) avec la densité volumique d'énergie EM contenue dans l'intervalle $[\nu-\frac{\Delta\nu}{2},nu+\frac{\Delta\nu}{2}]$ $du_{em}=d\left(\frac{\epsilon_0E^2}{2}+\frac{B^2}{2\mu_0}\right)=u_{\nu}d\nu$.\\

Comme N se conserve : $\frac{dn_1}{dt}=-\frac{dn_2}{dt}$ et donc : 
\begin{align}
    \frac{dn_2}{dt} &= \left(\frac{dN_2}{dt}\right)_{em.st}+\left(\frac{dN_2}{dt}\right)_{em.sp} + \left(\frac{dN_2}{dt}\right)_{abs} \\
    \frac{dn_2}{dt} &= -B_{21}N_2u(\nu_{12})-A_{21}N_2 + B_{12}N_1u(\nu_{12})
\end{align}
\subsection{Relations entre les coefficients d'Einstein}

Voir P. Olive p786. Dire deux choses :
\begin{itemize}
    \item on se place en régime stationnaire,
    \item le système \{N atomes\} est placé dans un milieu thermostaté à T quelconque baignant dans le rayonnement d'équilibre à cette température.
\end{itemize}
La proba d'occupation des deux niveaux à l'équilibre est donnée par la loi de Boltzmann $n_i=\frac{N_i}{N}=e^{\frac{E_i}{k_BT}}$. Le rapport $\frac{N_1}{N_2}=e^{\frac{E_2-E_1}{k_BT}}=e^{\frac{h\nu_{12}}{k_BT}}$.\\
En prenant un rayonnement électromagnétique de type corps noir : $u(\nu_{12})=\frac{8\pi h\nu_{12}^3}{c^3}$

\subsection{Processus de transfert d'énergie}
La puissance transférée de l'atome vers le champ $P_{at\rightarrow ch}=-h\nu_0\frac{dn_2}{dt} = h\nu_0B(n_2-n_1)u(\nu_0)$. On distingue alors deux régimes : 
\begin{itemize}
    \item Si $n_1>n_2$, $P_{at\rightarrow ch}<0$ $\longrightarrow$ Absorbtion
    \item Si $n_1<n_2$, $P_{at\rightarrow ch}>0$ $\longrightarrow$ Inversion de population
\end{itemize}

Initialement $n_1=1$,$n_2=0$ : 
\begin{itemize}
    \item $n_1(\infty) = \frac{A + Bu(\nu_0)}{A + 2Bu(\nu_0)}$
    \item $n_2(\infty) = \frac{Bu(\nu_0)}{A + 2Bu(\nu_0)}$
\end{itemize}

\section{Application au laser}
\subsection{Principe}
Parler de 

\subsection{Modèle à 4 niveaux d'énergie : le laser de TP He-Ne}


\section*{Conclusion}
Modèle de l'électron élastiquement lié qui prédit l'absorption électromagnétique de la matière et le rayonnement dipolaire ?
\end{reportBlock}



