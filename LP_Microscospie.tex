%%%%%%%%%%%%%%%%%%%%%%%%%%%%%%%%%%%%%%%%%%%%%%%%%%%%
%%%% En-tête leçon
\begin{headerBlock}
  \chapter{Microscopies optiques}
  \label{LP_Microscopie} 
\end{headerBlock}




%%%%%%%%%%%%%%%%%%%%%%%%%%%%%%%%%%%%%%%%%%%%%%%%%%%%
%%%% Références
\begin{center}
\begin{tabularx}{\textwidth}{| X | X | c | c |}
  \hline
  \rowcolor{gray!20}\multicolumn{4}{c}{Bibliographie de la leçon : } \\
  \hline 
  Titre & Auteurs & Editeur (année) & ISBN \\
  \hline
  Slide de cours & Agnès Maître & Site Montrouge & \\
  \hline 
  \url{http://ressources.agreg.phys.ens.fr/media/ressources/RessourceFichiers/11-Maxime_Dahan_-_Microscopie_pour_la_biologie.pdf} & M. Dahan & & \\
  \hline
  \url{https://www.microscopyu.com/} & Nikon & &    \\
  \hline 
  Optique & Eugène Hecht & Pearson &   \\
  \hline 
  Optique & Sylvain Houard & de Boeck & \\
  \hline
\end{tabularx}
\end{center}

%%%%%%%%%%%%%%%%%%%%%%%%%%%%%%%%%%%%%%%%%%%%%%%%%%%%

%%%%%%%%%%%%%%%%%%%%%%%%%%%%%%%%%%%%%%%%%%%%%%%%%%%%
%%%% Plan
\begin{reportBlock}{Plan détaillé}

  \textbf{Niveau choisi pour la leçon :} 
  \newline
  \textbf{Prérequis} : \begin{itemize}
      \item 
  \end{itemize}

  \textbf{Déroulé détaillé de la leçon: }  
  
  \section*{Introduction}


  \section{Le microscope \og d'autrefois \fg }

  \subsection{Présentation du dispositif}
  Faire le schéma du microscope : objectif, oculaire, image secondaire, etc ...

  \subsection{Propriétés}
  Faire le calcul de grossissement commercial $G_{com}=\frac{\alpha'}{\alpha}$\\
  Parler d'ouverture numérique, profondeur de champ, système d'éclairement, aberrations (chromatiques, géométriques)
  
  \subsection{Mesure du grossissement}
 
  \textcolor{blue}{Expérience quantitative :} Faire l'image d'une grille par un microscope optique, remonter au grossissement et comparer par rapport à la valeur de référence.

  \subsection{Limitations}
  Parler de : 
  \begin{itemize}
      \item conditions d'éclairement (éclairage Köhler
      \item résolution optique (ouverture numérique, diffraction, aberrations) 
      \item contraste
      \item microscopie plein champ/ point par point
  \end{itemize}


\textbf{Transition :} on va voir comment améliorer le microscope classique

\section{Microscopie confocal}

\section*{Ouverture}
Microscopie par effet tunnel

\end{reportBlock}