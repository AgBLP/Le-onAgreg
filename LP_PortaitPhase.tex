%%%%%%%%%%%%%%%%%%%%%%%%%%%%%%%%%%%%%%%%%%%%%%%%%%%%
%%%% En-tête leçon
\begin{headerBlock}
  \chapter{Oscillateurs ; portraits de phase et non-linéarités.}
  \label{LP_PortaitPhase} 
\end{headerBlock}




%%%%%%%%%%%%%%%%%%%%%%%%%%%%%%%%%%%%%%%%%%%%%%%%%%%%
%%%% Références
\begin{center}
\begin{tabularx}{\textwidth}{| X | X | c | c |}
  \hline
  \rowcolor{gray!20}\multicolumn{4}{c}{Bibliographie de la leçon : } \\
  \hline 
  Titre & Auteurs & Editeur (année) & ISBN \\
  \hline
   \url{https://uhincelin.pagesperso-orange.fr/LP49_BUP_portrait_phase_oscil.pdf} & H. Gié &  BUP n°744&    \\
  \hline 
   &  & &    \\
  \hline 
\end{tabularx}
\end{center}

%%%%%%%%%%%%%%%%%%%%%%%%%%%%%%%%%%%%%%%%%%%%%%%%%%%%

%%%%%%%%%%%%%%%%%%%%%%%%%%%%%%%%%%%%%%%%%%%%%%%%%%%%
%%%% Plan
\begin{reportBlock}{Plan détaillé}

  \textbf{Niveau choisi pour la leçon :} 
  \newline
  \textbf{Prérequis} : \begin{itemize}
      \item 
  \end{itemize}

  \textbf{Déroulé détaillé de la leçon: }  
  
  \section*{Introduction}
Messages à faire passer : définition d'un portait de phase, portait de phase amorti/non-amorti, stabilité des points fixes.
  \section{Oscillateurs non-amortis}
  \subsection{Equation du mouvement}
  Exemple de la physique : pendule simple à petites oscillations, circuit LC
  \subsection{Approche énergétique}
  Parler de stabilité, conservation de l'énergie mécanique.

  \subsection{Portrait de phase}

  \section{Effets non linéaires}
  Formule de borda pendule.\\

  \textcolor{blue}{Manip quanti :} mettre en évidence non linéarités, déterminer formule de Borda.
  


\end{reportBlock}