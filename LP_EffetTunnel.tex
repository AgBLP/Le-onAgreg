%%%%%%%%%%%%%%%%%%%%%%%%%%%%%%%%%%%%%%%%%%%%%%%%%%%%
%%%% En-tête leçon
\begin{headerBlock}
  \chapter{Effet tunnel. Application à la radioactivité alpha.}
    \label{LP_EffetTunnel}
\end{headerBlock}

%%%%%%%%%%%%%%%%%%%%%%%%%%%%%%%%%%%%%%%%%%%%%%%%%%%%
%%%% Références
\begin{center}
\begin{tabularx}{\textwidth}{| X | X | c | c |}
  \hline
  \rowcolor{gray!20}\multicolumn{4}{c}{Bibliographie de la leçon : } \\
  \hline 
  Titre & Auteurs & Editeur (année) & ISBN \\
  \hline
  Reflexion totale frustrée & Polycopié de Jean Hare &  & \\
  \hline
  Tout-en-un PC/PC* & M.-N. Sanz & Dunod & \\
  \hline
  Leçons de Physique & Patrick Charmont & Dunod (2000) & \\
  \hline
  \url{https://cahier-de-prepa.fr/pc*-poincare/download?id=1007} & & & \\
  \hline
\end{tabularx}
\end{center}

%%%%%%%%%%%%%%%%%%%%%%%%%%%%%%%%%%%%%%%%%%%%%%%%%%%%
\begin{reportBlock}{Plan détaillé}
  \textbf{Niveau choisi pour la leçon :} 
  \newline
  \textbf{Prérequis : }
  \begin{itemize}
      \item Postulats de la mécanique quantique
      \item Particule dans un puit de potentiel harmonique
  \end{itemize}

\section*{Introducion}
Les lois de la mécanique quantique sont très différentes de la physique classique. Par exemple, du point de vue classique : si une particule voit un puit de potentiel, elle rebondit si elle n'a pas assez d'énergie. En revanche, la dualité onde-particule de la matière (introduite par de Broglie) va montrer quelques effets intéressants : l'aspect ondulatoire de la matière va permettre la transmission de la particule même si celle-ci possède une énergie inférieure au potentiel : c'est l'effet tunnel.

\textcolor{red}{Décrire le modèle en disant ce qu'on}
\section{Etats liés vs états de diffusion}
\subsection{Fonction d'onde et équation de Schrodinger}

Probabilité : $|\psi(x,t)|^2dx$ = probabilité de trouver la particule entre x et x+dx. Interprétation possible que si $\int_{-\infty}^{+\infty}|\psi(x,t)|^2dx=1$.\\

Equation de Schrodinger : $i\hbar\partialD{\psi(x,t)}{t}=-\frac{\hbar^2}{2m}\partialD{^2\psi(x,t)}{x^2}+V(x)\psi(x,t)$\\

Equation linéaire, on peut choisir $\psi = \chi(x)\exp\left(-\frac{iE}{h}t\right)$ avec le signe - dans l'exponentiel pour faire correspondre l'énergie de la particule.\\

Energie = valeur propre de H, c'est ce qu'on cherche à résoudre tout le temps en mécanique quantique. 
\begin{itemize}
    \item Si V confinement $V\sim x^2$, alors E est quantifié. on a des \textbf{états liés}
    \item Si V non confinement, E est continue et on a des \textbf{états de diffusion (ou libres)}
\end{itemize}

Etats de diffusion : ce sont les états qui vont nous intéresser pour l'effet tunnel. Comment les interpréter ?

\subsection{Courant de probabilité}
Introduire le courant de probabilité et la densité de proba : 
\begin{itemize}
    \item $\rho(x,t)=|\psi(x,t)|^2$ densité de proba
    \item $\mathbf{j}(x,t)=\frac{\hbar}{2m}\left(\psi^*\partialD{\psi}{x}-(\partialD{\psi^*}{x})\psi\right)$, courant de probabilité
\end{itemize}

Dans le cas d'une onde plane (non intégrable): 
\begin{itemize}
    \item onde progressive : $\chi(x)=A\exp\left(ikx\right)$ et $\mathbf{j}(x,t)=\frac{\hbar k}{m}|A|^2\mathbf{e_x}$ courant de proba dans le sens $+\mathbf{e_x}$
    \item onde régressive : $\chi(x)=A\exp\left(-ikx\right)$ et $\mathbf{j}(x,t)=\frac{\hbar k}{m}|A|^2\mathbf{e_x}$ courant de proba dans le sens $-\mathbf{e_x}$
\end{itemize}
Schrödinger : $\partialD{\rho}{t}+\nabla\cdot\mathbf{j}=0$ loi de conservation (démo p34 poly Jean Hare).

\section{Effet tunnel : transmission}

\subsection{Le puit carré modèle intégrable de la barrière de potentiel}
Schéma du problème.\\
\begin{itemize}
    \item $E>V_0$ : état de diffusion, pas forcément super intéressant
    \item $E<V_0$ : état de diffusion possible avec une certaine probabilité : c'est l'effet tunnel !
\end{itemize}
Calcul des fonctions d'ondes dans les zones 1, 2 et 3. Dire que $B_3$ s'annule car pas d'onde provenant de $+\infty$.\\

Donner les conditions aux limites $x=0$ et $x=a$. On obtient 4 équations avec 5 inconnues.



\subsection{Coefficients de réflexion et transmission}
Cf Dunod p1252.

\importantbox{Ne pas faire la résolution complète, ça prend beaucoup trop de temps et ce n'est pas intéressant}

On trouve R et T, on verifie qu'on a bien R+T=1 (conservation de l'énergie). \\

\textcolor{blue}{Manip qualitative : montrer logiciel colorado qu'il y a transmission.}

\subsection{Approximation de la barrière épaisse}
Trouver $\delta$, l'interpréter. Garder l'analogie avec réflexion totale frustrée pour les questions.

\section{Application à la radioactivité alpha}

\subsection{Désintégration alpha}
Découverte de la radioactivité. Montrer carte de la stabilité des noyaux. Radioactivité $\alpha$ = noyaux lourds.
\subsection{Modèle de Gamow}
\textcolor{blue}{Expérience :} Code python simulant Gamow (cf Alexandre).

\subsection{Loi de Geiger-Nutall}
\textcolor{blue}{Manip quanti/quali : faire regression $log(T_{1/2}) = \frac{A}{\sqrt{E}}+B$ en utilisant un tableau de données préconstruit.}

\section*{Conclusion}


\end{reportBlock}