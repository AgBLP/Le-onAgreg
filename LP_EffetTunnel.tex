%%%%%%%%%%%%%%%%%%%%%%%%%%%%%%%%%%%%%%%%%%%%%%%%%%%%
%%%% En-tête leçon
\begin{headerBlock}
  \chapter{Effet tunnel. Application à la radioactivité alpha.}
    \label{LP_EffetTunnel}
\end{headerBlock}

%%%%%%%%%%%%%%%%%%%%%%%%%%%%%%%%%%%%%%%%%%%%%%%%%%%%
%%%% Références
\begin{center}
\begin{tabularx}{\textwidth}{| X | X | c | c |}
  \hline
  \rowcolor{gray!20}\multicolumn{4}{c}{Bibliographie de la leçon : } \\
  \hline 
  Titre & Auteurs & Editeur (année) & ISBN \\
  \hline
  Polycopié & Jean Hare &  & \\
  \hline
  Tout-en-un PC/PC* & M.-N. Sanz & Dunod & \\
  \hline
\end{tabularx}
\end{center}

%%%%%%%%%%%%%%%%%%%%%%%%%%%%%%%%%%%%%%%%%%%%%%%%%%%%
\begin{reportBlock}{Plan détaillé}
  \textbf{Niveau choisi pour la leçon :} 
  \newline
  \textbf{Prérequis : }
  \newline

\section*{Introducion}

\section{Equation de Schrödinger dans un potentiel}

\subsection{Position du problème}

\subsection{Coefficient de transmission}

\section{Application à la radioactivité alpha}

\subsection{Modèle de Gamow}
\textcolor{blue}{Expérience :} Code python simulant Gamow (cf Alexandre).

\subsection{Loi de Geiger-Nutall}


\section*{Conclusion}


\end{reportBlock}