%%%%%%%%%%%%%%%%%%%%%%%%%%%%%%%%%%%%%%%%%%%%%%%%%%%%
%%%% En-tête leçon
\begin{headerBlock}
  \chapter{Modèle de l'écoulement parfait d'un fluide}
    \label{LP_EcoulementParfait}
\end{headerBlock}

%%%%%%%%%%%%%%%%%%%%%%%%%%%%%%%%%%%%%%%%%%%%%%%%%%%%
%%%% Références
\begin{center}
\begin{tabularx}{\textwidth}{| X | X | c | c |}
  \hline
  \rowcolor{gray!20}\multicolumn{4}{c}{Bibliographie de la leçon : } \\
  \hline 
  Titre & Auteurs & Editeur (année) & ISBN \\
  \hline
  Hydrodynamique & Guyon, Hulin, Petit &  & \\
  \hline
  Tout-en-un PC/PC* & M.-N. Sanz & Dunod & \\
  \hline
\end{tabularx}
\end{center}

%%%%%%%%%%%%%%%%%%%%%%%%%%%%%%%%%%%%%%%%%%%%%%%%%%%%
\begin{reportBlock}{Plan détaillé}
  \textbf{Niveau choisi pour la leçon :} 
  \newline
  \textbf{Prérequis : }
  \newline

\section*{Introducion}
Différents fluides, différentes viscosité. On va s'intéresser à un modèle pour décrire les fluides faiblement visqueux.

\section{Modèle de l'écoulement parfait}


\subsection{Cadre du modèle}
\subsection{Equation d'Euler}
\subsection{Ecoulement de Bernouilli}

\section{Applications}

\subsection{Les sondes Pitot}
Du nom d'Henri Pitot (1732) qui voulait mesurer le débit de la Seine.
\subsection{Effet Venturi}


\section*{Conclusion}


\end{reportBlock}