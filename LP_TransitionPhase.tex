%%%%%%%%%%%%%%%%%%%%%%%%%%%%%%%%%%%%%%%%%%%%%%%%%%%%
%%%% En-tête leçon
\begin{headerBlock}
  \chapter{Transition de phase}
  \label{LP_TransitionPhase} 
\end{headerBlock}




%%%%%%%%%%%%%%%%%%%%%%%%%%%%%%%%%%%%%%%%%%%%%%%%%%%%
%%%% Références
\begin{center}
\begin{tabularx}{\textwidth}{| X | X | c | c |}
  \hline
  \rowcolor{gray!20}\multicolumn{4}{c}{Bibliographie de la leçon : } \\
  \hline 
  Titre & Auteurs & Editeur (année) & ISBN \\
  \hline
  Thermodynamique & BFR & Dunod & \\
  \hline 
  Thermodynamique & Diu &  &    \\
  \hline 
   &  & &    \\
  \hline 
\end{tabularx}
\end{center}

%%%%%%%%%%%%%%%%%%%%%%%%%%%%%%%%%%%%%%%%%%%%%%%%%%%%

%%%%%%%%%%%%%%%%%%%%%%%%%%%%%%%%%%%%%%%%%%%%%%%%%%%%
%%%% Plan
\begin{reportBlock}{Plan détaillé}

  \textbf{Niveau choisi pour la leçon :} Licence 3
  \newline
  \textbf{Prérequis} : \begin{itemize}
      \item 
  \end{itemize}

  \textbf{Déroulé détaillé de la leçon: }  
  
  \section*{Introduction}
Faire une accroche phénoménologique. Comment décrire l'eau qui bout ? Pourquoi la température reste constante ? Pourquoi c'est la même chose pour l'eau qui gèle ? Est-ce que c'est la même chose pour toutes les transitions de phase ?

  \section{Transition de phase du 1er ordre}

  
  \subsection{Description du phénomène}

  
  \section{Etude du diagramme (P,V) de l'eau/azote}
  
  
  \subsection{Enthalpie de changement d'état} 


  \section{La transition ferro-para}
  Coexistence entre deux phases. 
\subsection{Transition paramagnétisme-ferromagnétisme}


\end{reportBlock}