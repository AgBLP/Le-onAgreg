%%%%%%%%%%%%%%%%%%%%%%%%%%%%%%%%%%%%%%%%%%%%%%%%%%%%
%%%% En-tête leçon
\begin{headerBlock}
  \chapter{Phénomènes de transport}
    \label{LP_Transport}
\end{headerBlock}

%%%%%%%%%%%%%%%%%%%%%%%%%%%%%%%%%%%%%%%%%%%%%%%%%%%%
%%%% Références
\begin{center}
\begin{tabularx}{\textwidth}{| X | X | c | c |}
  \hline
  \rowcolor{gray!20}\multicolumn{4}{c}{Bibliographie de la leçon : } \\
  \hline 
  Titre & Auteurs & Editeur (année) & ISBN \\
  \hline
  Thermodynamique & BFR & Dunod (1989) & \\
\end{tabularx}
\end{center}

%%%%%%%%%%%%%%%%%%%%%%%%%%%%%%%%%%%%%%%%%%%%%%%%%%%%
\begin{reportBlock}{Plan détaillé}
  \textbf{Niveau choisi pour la leçon :} 
  \newline
  \textbf{Prérequis : }
  \newline


\section{Généralités sur les phénomènes de transport}

\textcolor{blue}{Expérience qualitative :} Barreau  chauffé + caméra thermique
\subsection{Type de transport}
\subsection{Système hors équilibre}
\subsection{Equilibre thermodynamique local}

\section{Diffusion de particules}

\subsection{Loi de Fick (1855)}
\subsection{Conservation du nombre de particules et équation de diffusion}
\subsection{Exemple de solution dans un cas non-stationnaire}

\section{Correspondance avec d'autres phénomènes de diffusion}

\subsection{Analogie}


\section*{Conclusion}
Ouverture sur le théorème de N\oe ther.

\end{reportBlock}