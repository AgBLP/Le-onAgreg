%%%%%%%%%%%%%%%%%%%%%%%%%%%%%%%%%%%%%%%%%%%%%%%%%%%%
%%%% En-tête leçon
\begin{headerBlock}
  \chapter{Ondes progressives, ondes stationnaires.}
  \label{LP_OndesProgressives} 
\end{headerBlock}




%%%%%%%%%%%%%%%%%%%%%%%%%%%%%%%%%%%%%%%%%%%%%%%%%%%%
%%%% Références
\begin{center}
\begin{tabularx}{\textwidth}{| X | X | c | c |}
  \hline
  \rowcolor{gray!20}\multicolumn{4}{c}{Bibliographie de la leçon : } \\
  \hline 
  Titre & Auteurs & Editeur (année) & ISBN \\
  \hline
  Tout-en-un PC/PC* & M.-N. Sanz & Dunod & \\
  \hline 
   Poly Ondes (prépa ENS Lyon) & Ethienne Thibierge &  &    \\
   \hline
   \url{http://images.math.cnrs.fr/Spectre.html\#nh7} &  San Vu Ngoc & CNRS & \\
  \hline 
  Dictionnaire de physique & Richard Taillet & de Boeck & \\
  \hline
  Ondes &  & H-prépa & \\
  \hline
\end{tabularx}
\end{center}

%%%%%%%%%%%%%%%%%%%%%%%%%%%%%%%%%%%%%%%%%%%%%%%%%%%%

%%%%%%%%%%%%%%%%%%%%%%%%%%%%%%%%%%%%%%%%%%%%%%%%%%%%
%%%% Plan
\begin{reportBlock}{Plan détaillé}

  \textbf{Niveau choisi pour la leçon :}
  \newline
  \textbf{Prérequis} : \begin{itemize}
      \item 
  \end{itemize}

  \textbf{Déroulé détaillé de la leçon: }  
  
  \section*{Introduction}

  \section{Propagation d'une onde}
  
  \subsection{Définition et propriétés} 
  Cf partie 1 agrégation 2004

  \subsection{Equation de d'Alembert}
  
  \section{Onde stationnaire}

  \subsection{Corde de Melde}

  \section{}


\end{reportBlock}