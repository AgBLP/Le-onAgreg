%%%%%%%%%%%%%%%%%%%%%%%%%%%%%%%%%%%%%%%%%%%%%%%%%%%%
%%%% En-tête leçon
\begin{headerBlock}
  \chapter{Lois de conservation en mécanique}
    \label{LP_LoisConservation}
\end{headerBlock}

%%%%%%%%%%%%%%%%%%%%%%%%%%%%%%%%%%%%%%%%%%%%%%%%%%%%
%%%% Références
\begin{center}
\begin{tabularx}{\textwidth}{| X | X | c | c |}
  \hline
  \rowcolor{gray!20}\multicolumn{4}{c}{Bibliographie de la leçon : } \\
  \hline 
  Titre & Auteurs & Editeur (année) & ISBN \\
  \hline
  Mécanique 1 & H. Gié et J.-P. Sarmant & Tec\&Doc (1984) & \\
  \hline
  Mécanique 2ème année & H. Gié et J.-P. Sarmant & Tec\&Doc (1996) & \\
  \hline
  Mécanique PCSI-MPSI & P. Brasselet & PUF (2000) & \\
  \hline
\end{tabularx}
\end{center}

\begin{reportBlock}{Commentaires des années précédentes :}
    \begin{itemize}
        \item \textbf{2017 :} Des exemples concrets d’utilisation des lois de conservation sont attendus,
        \item \textbf{2016 :} Lors de l’entretien avec le jury, la discussion peut aborder d’autres domaines que celui de la mécanique classique,
        \item \textbf{2015 :} Cette leçon peut être traitée à des niveaux très divers. L’intérêt fondamental des lois de conservation et leur origine doivent être connus et la leçon ne doit pas se limiter à une succession d’applications au cours desquelles les lois de conservation se résument à une propriété anecdotique du problème considéré.
    \end{itemize}
\end{reportBlock}

%%%%%%%%%%%%%%%%%%%%%%%%%%%%%%%%%%%%%%%%%%%%%%%%%%%%
\begin{reportBlock}{Plan détaillé}
  \textbf{Niveau choisi pour la leçon :} CPGE 2ème année
  \newline
  \textbf{Prérequis : }
  \begin{itemize}
      \item Lois de la mécanique classique,
      \item Définition énergie mécanique, moment cinétique, moment d'une force, quantité de mouvement
      \item Système fermé
  \end{itemize}


\section*{Introduction}
Les principes de conservation sont nombreux en physique : conservation de la charge en électromagnétisme, conservation de la masse ou de l'énergie pour un système thermodynamique fermé (1er principe). Nous allons ici nous restreindre aux principes de conservations en mécanique newtonienne.\\

En mécanique, nous connaissons trois lois d'évolution :
\begin{enumerate}
    \item le \underline{\textbf{théorème de la résultante cinétique :}} $\frac{d\mathbf{p}}{dt}=\sum_{i}\mathbf{F_i}=\mathbf{R}$,
    \item le \underline{\textbf{théorème du moment cinétique :}} $\frac{d\mathbf{L_O}}{dt}=\sum_{i}\mathbf{M_{O,i}}$,
    \item le \underline{\textbf{théorème de la puissance mécanique :}} $\frac{dE_m}{dt}=\sum_{i}P_i$.
\end{enumerate}

\textcolor{red}{Transition :}Les termes $\sum_{i}\mathbf{F_i}$, $\sum_{i}\mathbf{M_{O,i}}$ et $\sum_{i}P_i$ jouent le rôle de \textit{sources} pour les grandeurs $\mathbf{p}$, $L_O$ et $E_m$ . On va s'intéresser aux relations de conservation qui découlent de l'annulation de ces termes de sources 

\section{Conservation de la quantité de mouvement}

\subsection{Système isolé}
\textcolor{green}{Définition :} Système fermé qui ne subit aucune action extérieure donc $\forall i, \mathbf{F_i}=0$. \\

En pratique cela n'arrive jamais car il y a toujours des forces gravitationnelles. Prenons un système isolé constitué de deux sous-système A et B. Si A intéragit avec B et reçoit de la quantité de mouvement $\Delta \mathbf{P_A}$ alors ce dernier a dû lui céder une quantité de mouvement $\Delta \mathbf{P_B}$ telle que :
\begin{equation}
    \Delta(\mathbf{P_A}+\mathbf{P_B})= 0 \Leftrightarrow \Delta\mathbf{P_A} = -\Delta\mathbf{P_B}
\end{equation}
Voir Tec\&Doc 2ème année p110. Exemple de la diffusion de Rutherford où il y a un transfert de la quantité de mouvement. Déduire la norme de la vitesse de recul d'un atome.\\
Ca se voit aussi avec le recul d'une arme à feu.
\subsection{Système pseudo-isolé}
\textcolor{green}{Définition :} un système est dit pseudo-isolé s'il est soumis à des actions extérieures qui se compensent c'est-à-dire dont la résultante $\mathbf{R}=\sum_i\mathbf{F_i}$ est nulle.\\

Exemple : mobile placé sur coussin d'air.
\subsection{Collision élastique}
\textcolor{blue}{Expérience : }Conservation de $\mathbf{p}$ de mobiles autoporteurs.

\section{Conservation de l'énergie}

\subsection{Forces conservatives}
Brasselet p71. \textcolor{green}{Définition :} une force est conservative si son travail entre deux positions quelconques ne dépend que de celles-ci et pas du chemin que le système a suivi entre ces deux points.\\
L'exemple le plus parlant est la force de pensanteur dont le travail entre deux points d'altitude $z_1$ et $z_2$ s'écrit :
\begin{equation}
    W_{1\rightarrow2}=-mg(z_2-z_1) = E_{p2}-E_{p1}
\end{equation}
Brasselet p73. Par définition du gradient : $dE_p=\nabla E_p dr = -\delta W = -\mathbf{F}\cdot\mathbf{dr}$ donc $\mathbf{F}=-\nabla E_p$.
\section{Conservation du moment cinétique}

\subsection{TMC}
\subsection{Système pseudo-isolé}
\subsection{Cas d'une force centrale}


\section*{Conclusion}
Ouverture sur les théorèmes de N\oe ther (Emmi N\oe ther, mathématicienne allemande (1882-1935):
\begin{enumerate}
    \item La conservation de la quantité de mouvement est une conséquence de \textbf{l'invariance des lois de la physique par translation spatiale}, c'est-à-dire de l'homogénéité de l'espace,
    \item  La conservation du moment cinétique est une conséquence de \textbf{l'invariance des lois physiques par rotation}, c'est-à-dire de l'isotropie de l'espace,
    \item La conservation de l'énergie est une conséquence de \textbf{l'invariance des lois physiques par translation temporelle}, c'est-à-dire de l'homogénéité du temps.
\end{enumerate}

\end{reportBlock}