\begin{changemargin}{-1.5cm}{-0cm}

\begin{center}
\begin{tabularx}{\paperwidth-2cm}{| X | X | c | X |}
  \hline
  \rowcolor{gray!20}\multicolumn{4}{c}{Avancement préparation oraux Leçons Physique} \\
  \hline 
  Titre de la leçon & Expériences & Avancement & Commentaires \\
  \hline
\textbf{Gravitation} p\pageref{LP_Gravitation}  & \textcolor{green}{Pendule simple : mesure de g} & \textcolor{yellow}{\textbf{60\%}}  & Reprendre intro, partie marées à modifier, rajouter exemples effets de marée dans le système solaire, limite de Roche à bosser  \\
  \hline 
  \hline
     \textbf{Lois de conservation en mécanique} p\pageref{LP_LoisConservation} & \textcolor{green}{Conservation de E chute bille} & \textcolor{orange}{\textbf{20\%}} & plan détaillé \\
  \hline 
    \textbf{Notions de viscosité d'un fluide. \'{E}coulements visqueux} p\pageref{LP_Viscosite} & \textcolor{green}{Détermination de $\eta_{silicone}$ par la chute des billes}  & \textcolor{orange}{\textbf{30\%}} & Plan trop ambitieux, faire démo loi de Stokes dans partie dynamique + faire manip \\
  \hline 
  \textbf{Modèle de l'écoulement parfait d'un fluide} p\pageref{LP_EcoulementParfait} & \textcolor{green}{Mesure $\rho_{air}$ sonde Pitot + anémomètre} & \textcolor{green}{\textbf{100\%}} & Bien revoir couches limites \\
  \hline
  \textbf{Phénomènes interfaciaux impliquant des fluides} p\pageref{LP_PhenomenesInterfaciaux} & \textcolor{green}{Loi de Laplace - expériences quali bulles savon }& \textcolor{green}{\textbf{80\%}} & Transition gravité-mouillage \\
  \hline 
  \hline
  \textbf{Premier principe de la thermodynamique} p\pageref{LP_PremierPrincipe} & \textcolor{orange}{Mesure $c_{fer}$ dans calorimètre} - \textcolor{red}{Expérience Joule} & \textcolor{orange}{\textbf{50\%}} & Leçon Lydia a bien marché \\
  \hline 
  \textbf{Transitions de phase} p\pageref{LP_TransitionPhase} & \textcolor{green}{Mesure chaleur latente de l'eau} & \textcolor{red}{\textbf{20\%}} & \\
  \hline 
  \textbf{Phénomènes de transport} p\pageref{LP_Transport} & \textcolor{green}{Diffusion du glycérol dans l'eau} & \textcolor{green}{\textbf{80\%}} & à répéter\\
  \hline 
  \hline
  \textbf{Conversion de puissance électromécanique} p\pageref{LP_ConversionPuissance} & \textcolor{green}{Etude MCC (résistance, rendement, ...)} & \textcolor{yellow}{\textbf{60\%}} & Finir moteur synchrone\\
  \hline 
  \textbf{Induction électromagnétique} p\pageref{LP_Induction} & Mesure inductance circuit LC + aimant & \textcolor{red}{\textbf{0\%}} & Leçon Lydia assez complet\\
  \hline
  \textbf{Rétroaction et oscillations} p\pageref{LP_RetroactionOscillation} & \textcolor{red}{Quali : MCC asservi.}, \textcolor{yellow}{Quanti : Pont de Wien} & \textcolor{red}{\textbf{0\%}} & \\
  \hline
  \textbf{Traitement d'un signal. Étude spectrale} p\pageref{LP_TraitementSignal} & \textcolor{green}{Filtre RC et démodulation par détection synchrone} & \textcolor{yellow}{\textbf{60\%}} & Finir partie 3\\
  \hline
\end{tabularx}
\end{center}

\begin{center}
\begin{tabularx}{\paperwidth-2cm}{| X | X | c | X |}
\hline
  \textbf{Ondes progressives, ondes stationnaires} p\pageref{LP_OndesProgressives}~& \textcolor{red}{Corde Melde - étude câble coax} & \textcolor{orange}{\textbf{40\%}} & Faire partie onde progressive et onde stationnaire \\
  \hline
  \textbf{Ondes acoustiques} p\pageref{LP_OndeAcoustique}~& ?? Mesure $c_{air}$ piézo ?? & \textcolor{red}{\textbf{0\%}} &  \\
  \hline
  \textbf{Propagation guidée des ondes} p\pageref{LP_PropagationGuidee} & \textcolor{green}{Quali : piezo + tube}, \textcolor{green}{Quanti : guide d'onde centi}& \textcolor{green}{\textbf{80\%}} & Ouverture à finir + Pertes et ROS à regarder. \\
  \hline
  \textbf{Microscopies optiques} p\pageref{LP_Microscopie}~& \textcolor{green}{Microscope maison, mesure grandissement}~& \textcolor{red}{\textbf{0\%}} &  \\
  \hline
  \textbf{Interférences à deux ondes en optique} p\pageref{LP_InterferencesDeuxOndes}~& Mesure largeur fentes d'Young & \textcolor{orange}{\textbf{40\%}} & Plan à revoir en détail\\
  \hline
  \textbf{Interférométrie à division d'amplitude} p\pageref{LP_DivisionAmplitude} & \textcolor{green}{Michelson : doublet sodium} & \textcolor{green}{\textbf{100\%}} & A revoir vite fait coin d'air \\
  \hline
  \textbf{Diffraction de Fraunhofer} p\pageref{LP_DiffractionFraunhofer} & \textcolor{red}{Diffraction par une fente+CCD-MHTEX/Filtrage spatial }& \textcolor{orange}{\textbf{30\%}} &  \\
  \hline
  \textbf{Diffraction par des structures périodiques} p\pageref{LP_DiffractionPeriodique} & \textcolor{red}{Diffraction par un réseau} & \textcolor{red}{\textbf{0\%}} &  \\
  \hline
  \textbf{Absorption et émission de la lumière} p\pageref{LP_Absorption} & \textcolor{yellow}{Corps noir ?} & \textcolor{red}{\textbf{0\%}} & Leçon Elric pas mal \\
  \hline
  \hline
  \textbf{Propriétés macroscopiques des corps ferromagnétiques } p\pageref{LP_Ferromagnetisme} & \textcolor{green}{Cycle hystérésis transfo} & \textcolor{green}{\textbf{90\%}} & Mettre en relief appli ferro durs/doux + démo confinement lignes de champ slides \\
  \hline
  \textbf{Mécanismes de la conduction électrique dans les solides} p\pageref{LP_Conduction} & \textcolor{green}{Mesure conductivité cuivre vs T} & \textcolor{yellow}{\textbf{50\%}} & Transitions délicates, revoir partie 1 \\
  \hline
  \textbf{Phénomènes de résonance dans différents domaines de la physique} p\pageref{LP_resonance} & \textcolor{green}{Résonance en tension sur la capacité du circuit RLC + couplage mutuel circuits RLC couplés} & \textcolor{yellow}{\textbf{40\%}} & Plan à détailler  \\
  \hline
  \textbf{Oscillateurs ; portraits de phase et non-linéarités} p\pageref{LP_PortaitPhase}~& \textcolor{yellow}{Borda sur le pendule + portait de phase} & \textcolor{red}{\textbf{0\%}} & Leçon Basile pour ref  \\
  \hline
  \hline
  \textbf{Cinématique relativiste. Expérience de Michelson et Morley} p\pageref{LP_CinematiqueRelativiste} & \textcolor{red}{logiciel Gum\_MC, exp Frish et Smith} & \textcolor{red}{\textbf{20\%}} & Plan ok, manip à bosser  \\
  \hline
  \textbf{Effet tunnel : application à la radioactivité alpha} & \textcolor{green}{Simulation site colorado}& \textcolor{green}{\textbf{80\%}} & Faire tableau avec $T_{1/2}$ et E. Interprétation $T_{1/2}$ avec particule à travailler. \\
  \hline
\end{tabularx}
\end{center}

\end{changemargin}