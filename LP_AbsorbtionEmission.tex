%%%%%%%%%%%%%%%%%%%%%%%%%%%%%%%%%%%%%%%%%%%%%%%%%%%%
%%%% En-tête leçon
\begin{headerBlock}
  \chapter{Absorbtion et émission de la lumière}
    \label{LP_Absorption}
\end{headerBlock}

%%%%%%%%%%%%%%%%%%%%%%%%%%%%%%%%%%%%%%%%%%%%%%%%%%%%
%%%% Références
\begin{center}
\begin{tabularx}{\textwidth}{| X | X | c | c |}
  \hline
  \rowcolor{gray!20}\multicolumn{4}{c}{Bibliographie de la leçon : } \\
  \hline 
  Titre & Auteurs & Editeur (année) & ISBN \\
  \hline
  Optique Physique & R. Taillet & de Boeck &   \\
  \hline 
  Physique Statistique & Landau et Lifshitz & Ellipses &  \\
  \hline 
   Poly laser & A. Maitre &  &  \\
\hline
 Sextant & & Hermann & \\
 \hline 
 Tout-en-un PC/PC* & M.-N. Sanz & Dunod & \\
 \hline 
 Physique en PC/PC* & Pascal Olive & Ellipses & \\
 \hline
\end{tabularx}
\end{center}

%%%%%%%%%%%%%%%%%%%%%%%%%%%%%%%%%%%%%%%%%%%%%%%%%%%%
\begin{reportBlock}{Plan détaillé}
  \textbf{Niveau choisi pour la leçon :} Licence
  \newline
  \textbf{Prérequis : }Milieu LHI, distribution de Boltzmann
  \newline
  
  \textbf{Déroulé détaillé de la leçon: }   \newline
On verra deux modèles qui permettront d'expliquer le phénomène d'émission et absorption de la lumière dans les milieux.
  \section{Modèle de Lorentz}
  
  \subsection{Electron élastiquement lié}
  Modèle classique : électron gravite autour d'un proton et voit le champ électrique de la lumière $\mathbf{E}$. Un bilan des forces conduit à :
  \begin{itemize}
      \item Force de Lorentz : $-e\mathbf{E}$
      \item Force de rappel : $-K\mathbf{r}$
      \item phénomène dissipatif : $-\frac{m_e}{\tau}\mathbf{E}$
  \end{itemize}
Le PFD donne :
\begin{equation}
    \frac{d^2\mathbf{r}}{dt^2} + \frac{\omega_0}{Q}\frac{d\mathbf{r}}{dt} + \omega^2\mathbf{r} = \frac{-e\mathbf{E}}{m_e}
\end{equation}
avec $Q=\omega_0\tau$ et $\omega_0=\sqrt{\frac{K}{m_e}}$.\\

Soit le moment dipolaire $\mathbf{p}=-e\mathbf{r}$ et le vecteur polarisation $\mathbf{P}=N\mathbf{p}=\epsilon_0\chi_e\mathbf{E}$ dans un milieu LHI. On choisit un champ $\mathbf{E}=\mathbf{E_0}\exp{j(\omega t - \mathbf{k}\cdot\mathbf{r})}$ et on en déduit : 
\begin{equation}
    \chi_e = \frac{\chi_0}{1 + jQ\frac{\omega}{\omega_0}-(\frac{\omega}{\omega_0})^2}
\end{equation}
avec $\chi_0=\frac{Ne^2}{n_e\epsilon_0\omega_0^2}$\\

Problème : ce modèle ne prend pas du tout en compte l'émission de la lumière. Il faut alors introduire le modèle d'Einstein sur l'émission.

\section{Modèle d'Einstein}
\subsection{Equation d'Einstein}
Soit un système à deux niveaux d'énergie $E_a$ et $E_b$ associée aux états propres $\ket{a}$ et $\ket{b}$. On note $N_a$ et $N_b$ la population de ces niveaux telles que $N_a+N_b=N$. La proba d'occupation est $n_i=\frac{N_i}{N}$.\\
La lumière que reçoit ce système est caractérisé par une densité spectrale $u(\nu)=\frac{dU}{d\nu}$ avec $U=\frac{\epsilon_0E^2}{2}+\frac{B^2}{2\mu_0}$.\\


A l'équilibre : $\frac{dn_a}{dt}=-\frac{dn_b}{dt}$ et donc : 
\begin{equation}
    \frac{dn_a}{dt} = -An_b + B_{21}u(\nu_0)
\end{equation}
\subsection{Coefficients d'Einstein}

En posant $h\nu_0 = E_b - E_a$, on a $\frac{n_a}{n_b}=\exp{-\frac{E_a+E_b}{k_BT}} = \exp{\frac{h\nu_0}{b_BT}}$. On en déduit que $B_{12} = B_{21} = B$ et $\frac{A}{B}=\frac{8\pi h\nu_0^3}{c^3}$. L'équation d'Einstein devient : 
\begin{equation}
    \frac{dn_b}{dt} = -\frac{dn_a}{dt} = -An_a + Bu(\nu_0)(n_a-n_b)
\end{equation}

\subsection{Processus de transfert d'énergie}
La puissance transférée de l'atome vers le champ $P_{at\rightarrow ch}=-h\nu_0\frac{dn_b}{dt} = h\nu_0B(n_b-n_a)u(\nu_0)$. On distingue alors deux régimes : 
\begin{itemize}
    \item Si $n_a>n_b$, $P_{at\rightarrow ch}<0$ $\longrightarrow$ Absorbtion
    \item Si $n_a<n_b$, $P_{at\rightarrow ch}>0$ $\longrightarrow$ Inversion de population
\end{itemize}

Initialement $n_a=1$,$n_b=0$ : 
\begin{itemize}
    \item $n_a(\infty) = \frac{A + Bu(\nu_0)}{A + 2Bu(\nu_0)}$
    \item $n_b(\infty) = \frac{Bu(\nu_0)}{A + 2Bu(\nu_0)}$
\end{itemize}
\end{reportBlock}


\begin{reportBlock}{Questions posées par l’enseignant (avec réponses)}
  \textbf{C : Que peut-on dire sur le mouvement dans le modèle élastiquement lié ?}  \textcolor{purple}{Il y a une force centrale (force électrostatique) donc on a des mouvements circulaires.} Uniquement circulaires ? \textcolor{purple}{Non, il peut y avoir des trajectoires elliptiques qui sont des solutions pour un état lié (comme le système \{planète-Soleil\} pour la gravité).} \newline
  \textbf{C : Pas d'interaction répulsive ?}  \textcolor{purple}{La force de rappel élastique est répulsive si l'allongement est plus faible que la longueur à vide.}\\
  \textbf{C : Pourquoi il y a un phénomène dissipatif ?}  \textcolor{purple}{Je ne sais plus. Un électron accéléré perd forcément de l'énergie par rayonnement (dipolaire par exemple), sinon il peut avoir une vitesse infinie.}\\
  \textbf{C : Le champ $\mathbf{E}$ n'est pas uniforme si ?}  \textcolor{purple}{La longueur d'onde du champ est très grande devant la taille de l'atome donc on peut le considérer uniforme.} Y-a-t'il quand même un effet ? Le proton est-il fixe ? \textcolor{purple}{Oui le proton bouge, il faudrait se placer dans le référentiel du centre de masse. Mais comme $m_{proton}>>m_{e^-}$, ça ne change pas vraiment la position du problème.}\\
  \textbf{C : C'est quoi le lien entre le modèle de Lorentz et le titre de la leçon ?} \textcolor{purple}{Je voulais arriver jusqu'à la définition des indices optiques.} Qu'est-ce que ce modèle prédit ? Où va l'énergie ? \textcolor{purple}{Ce modèle prédit l'évolution de la polarisabilité électronique en fonction de la fréquence de l'onde incidente. Pour certaines fréquences, le dipôle absorber un photon et s'orienter et créer une polarisation d'origine  électronique. Voir le BFR Tome 4, Chapitre 4. L'énergie se transfert vers les phonons, dans le rayonnement dipolaire. C'est un phénomène de diffusion. Le transfert d'énergie se fait par l'intermédiaire de la partie imagnaire de la susceptibilité. En revanche ce modèle ne prédit pas d'amplification (énergie amenée par les atomes au champ électrique, en gros on ne peut pas avoir un photon à l'entrée = deux photons à la sortie).}\\
   \textbf{C : Revenons à l'interprétation des coefficients d'Einstein, les explications paraissaient un peu triviales} \textcolor{purple}{A : désexcitation spontanée car niveau b instable, B : désexcitation stimulée par le champ électrique. } Il y a un temps de vie des électrons sur les niveaux ? \textcolor{purple}{D'après le principe d'incertitude de Heisenberg $\Delta E\Delta t>\hbar/2$. Autrement dit, les électrons ont une durée de vie $\Delta t$ finie sur les niveaux d'énergie de largeur $\Delta E$.} C'est quoi le $\Delta E$ ? \textcolor{purple}{La largeur d'une bande d'énergie d'un niveau électronique.} Du coup c'est quoi la différence entre A et B ? \textcolor{purple}{A traduit le taux de probabilité liée à la désexcitation spontanée de l'atome (en $s^{-1}$), B est le taux de probabilité de transition $\ket{A}\rightarrow \ket{B}$ lié à l'absorption du rayonnement égal au taux d'émission stimulée $\ket{B}\rightarrow \ket{A}$.}\\
   \textbf{C : Pourquoi $u$ n'apparait pas dans l'émission spontanée ?} \textcolor{purple}{Ce terme n'est pas lié au rayonnement thermique mais au fait que les électrons ont une durée de vie finie sur le niveau d'énergie $\ket{b}$ du fait du principe d'incertitude d'Heisenberg.}
  
\end{reportBlock}


