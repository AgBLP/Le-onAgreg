\begin{headerBlock}
\chapter{Diagrammes E-pH}
\label{LC_DiagrammeEpH}
 \end{headerBlock}

%%%%%%%%%%%%%%%%%%%%%%%%%%%%%%%%%%%%%%%%%%%%%%%%%%%%
%%%% Références


%%%%%%%%%%%%%%%%%%%%%%%%%%%%%%%%%%%%%%%%%%%%%%%%%%%%
%%%% Plan
\begin{reportBlock}{Bibliographie}

\begin{center}
\begin{tabular}{|c|c|c|c|}\hline
Titre & Auteur(s) & Editeur (année) & ISBN \\ \hline
BUP n$^0$790 ~ & ~ & ~ & ~ \\
\hline
\end{tabular}
\end{center}

\end{reportBlock}

\begin{reportBlock}{Plan détaillé}

\underline{Niveau} : TSI2 \\

\section*{Introduction pédagogique}


\paragraph*{Prérequis}
\begin{itemize}
\item Réactions redox
\item Réactions A/B
\item Potentiel de Nernst
\end{itemize}

\paragraph*{Contexte :}
Place de la leçon : milieu d'année.

\paragraph*{Notions importantes}

\begin{itemize}
\item 
\end{itemize}

\paragraph*{Objectifs}

\begin{itemize}
\item
\end{itemize}

\paragraph*{Difficultés}

\begin{itemize}
\item
\end{itemize}

\section*{Introduction }

\textbf{Question:}

\paragraph*{Manipulation qualitative:} \textcolor{green}{}

\section{Lecture d'un diagramme E-pH}

\subsection{}
Déf : diagramme E-pH. Exemple du diagramme du Fer.

\subsection{Calculs des équations aux frontières}


\section{Utilisation d'un diagramme potentiel pH}

\subsection{Diagramme potentiel-pH de l'eau}

\section{Utilisation d'un diagramme E-pH à des fins expérimentales}

\subsection{Procédé industriel à l'aide d'un diagramme E-pH}
Présentation de la méthode de Winkler.

\section{Conclusion} 


\end{reportBlock}

\begin{reportBlock}{Questions posées}

\begin{itemize}

\item 
\textcolor{purple}{}

\end{itemize}


\end{reportBlock}

\begin{reportBlock}{Commentaires}

\end{reportBlock}


\begin{reportBlock}{Expérience 1}
% bloc à dupliquer autant de fois que d'expériences

\underline{Titre} :  \\

\underline{Référence complète} :  \\ 

\underline{But de la manip} : \\

\underline{Commentaire éventuel} : 

\underline{Phase présentée au jury} :\\

\underline{Durée de la manip} : \\

\end{reportBlock}



\begin{reportBlock}{Expérience 2}
% bloc à dupliquer autant de fois que d'expériences

\underline{Titre} : 

\underline{Référence complète} :  \\ 

\underline{Équation chimique et but de la manip} : \\

\underline{Phase présentée au jury} :  \\

\underline{Durée de la manip} : \\

\end{reportBlock}


\begin{reportBlock}{Expérience 3}
% bloc à dupliquer autant de fois que d'expériences

\underline{Titre} : \\

\underline{Référence complète} : \\ 

\underline{Équation chimique et but de la manip} :  \\


\underline{Commentaire éventuel :} 

\underline{Phase présentée au jury} \\

\underline{Durée de la manip} :  \\

\end{reportBlock}



\begin{reportBlock}{Compétence \og Autour des valeurs de la République et des thématiques relevant de la laïcité et de la citoyenneté \fg{}}

\underline{Question posée} : \\

\underline{Réponse proposée} : \\ 

\underline{Commentaire du correcteur} : \\

\end{reportBlock}


\begin{reportBlock}{Champ libre pour le correcteur}
% compléments, propositions de manipulation, bibliographie etc.

\paragraph*{Remarques sur le plan}


\paragraph*{Vocabulaire}

\paragraph*{Équipements de protection individuelle}

\paragraph*{Autre expérience possible} 

\end{reportBlock}