\begin{headerBlock}
\chapter{Phénomène de corrosion humide}
\label{LC_Corrosion}
 \end{headerBlock}

%%%%%%%%%%%%%%%%%%%%%%%%%%%%%%%%%%%%%%%%%%%%%%%%%%%%
%%%% Références


%%%%%%%%%%%%%%%%%%%%%%%%%%%%%%%%%%%%%%%%%%%%%%%%%%%%
%%%% Plan
\begin{reportBlock}{Bibliographie}

\begin{center}
\begin{tabular}{|c|c|c|c|}\hline
Titre & Auteur(s) & Editeur (année) & ISBN \\ \hline
PC Tout-en-un PSI/PSI* ~ & B. Fosset, J.-B. Baudin ~ & Dunod (2022) ~ & ~ \\
Tout-en-un PC/PC* & P. Frajman ~ &  Hatier (2009) ~ & ~ \\
\hline
\end{tabular}
\end{center}

\end{reportBlock}

\begin{reportBlock}{Plan détaillé}

\underline{Niveau} : PSI \\

\underline{Pré-requis} :
\begin{itemize}
\item Courbe intensité-potentielle
\end{itemize}


\section*{Introduction pédagogique}



\paragraph*{Prérequis}
\begin{itemize}
\item 
\end{itemize}

\paragraph*{Notions importantes}

\begin{itemize}
\item
\end{itemize}

\paragraph*{Objectifs}

\begin{itemize}
\item
\end{itemize}

\paragraph*{Difficultés}

\begin{itemize}
\item 
\end{itemize}

\section*{Introduction}


\section{Conductance et conductivité}

\section{}

\subsection{}


\section{Conclusion}

\end{reportBlock}

