\begin{headerBlock}
\chapter{Cinétique électrochimique}
\label{LC_CinetiqueElectrochimique}
 \end{headerBlock}

%%%%%%%%%%%%%%%%%%%%%%%%%%%%%%%%%%%%%%%%%%%%%%%%%%%%
%%%% Références


%%%%%%%%%%%%%%%%%%%%%%%%%%%%%%%%%%%%%%%%%%%%%%%%%%%%
%%%% Plan
\begin{reportBlock}{Bibliographie}

\begin{center}
\begin{tabularx}{\textwidth}{| X | X | c | c |}\hline
Titre & Auteur(s) & Editeur (année) & ISBN \\ \hline
 &  &  &  \\ 
 \hline
\end{tabularx}
\end{center}

\end{reportBlock}

\begin{reportBlock}{Plan détaillé}

\underline{Niveau} : PSI \\

\section*{Introduction pédagogique}


\paragraph*{Prérequis}
\begin{itemize}
\item Oxydoréduction
\item Thermochimie
\end{itemize}
\paragraph*{Contexte :}


\paragraph*{Notions importantes}

\begin{itemize}
\item Vitesse de réaction
\end{itemize}

\paragraph*{Objectifs}

\begin{itemize}
\item Tracer une courbe intensité-potentiel
\item déterminer la solubilité et prévoir son évolution
\end{itemize}

\paragraph*{Difficultés}

\begin{itemize}
\item Réaction en milieu hétérogène
\item Pk montage à 3 électrodes
\end{itemize}
Pour y remédier,différencier milieu homogène, milieu hétérogène. 
\section*{Introduction }
Réaction à la surface d'une électrode. On va relier vitesse et thermodynamique.

\section{Réaction électrochimique}

\subsection{Description}
On s'intéresse à l'équilibre :
\begin{equation}
    Red = Ox + ne^-
\end{equation}

Ex : Fe$^{2+}$ = Fe$^{3+}$ + 1 e$^-$. Cette réaction n'a pas de réalité physique car l'électron n'existe pas en solution. En revanche, à la surface d'une électrode métallique, il y a formation d'un électron libre : c'est ce qu'on appelle une réaction électrochimique.

\subsection{Facteur cinétique}
En milieu homogène : 
\begin{itemize}
    \item La proba de renconre entre deux réactifs
    \item La proba que la réaction se produise quand les réactifs sont en contact
\end{itemize}
En milieu hétérogène :
\begin{itemize}
    \item Les réactifs doivent atteindre a surface de l'électrode
    \item le transfert d'électron doit se produire
    \item le produit doit s'éloigner de la surface de l'électrode
\end{itemize}
Schéma (slide) étapes d'une réaction électrochimique.

\subsection{Intensité et vitesse de réaction}
\underline{Rappel :} la vitesse de réaction $v$ est égale à la dérivée temporelle de l'avancement $\xi$.\\
En milieu homogène, on s'intéresse à des vitesses volumiques. Ici, il est plus adéquat de considérer des vitesses surfaciques.\\
\textcolor{green}{définition :} En notant $A$ la surface immergée de l'électrode : 
\begin{equation}
    v = \frac{1}{A}\frac{d\xi}{dt} = \frac{1}{nA}\frac{dn_{e^-}}{dt}
\end{equation}
Ex : pour Fe$^{2+}$ = Fe$^{3+}$ + 1 e$^-$, $n=1$. Comme la variation de la quantité de matière d'électron échangée est proportionnelle à la variation de charge :
\begin{equation}
    dq = Fdn_{e^-}
\end{equation}
On a finalement :
\begin{equation}
    \frac{1}{nAF}\frac{dq}{dt} = \frac{i}{nAF} = \frac{j}{nF}
    \end{equation}

\textcolor{green}{convention :} On définit positivement le courant anodique $i_a$ \textit{i.e.} lié à l'oxydation et négativement le courant cathodique $i_c$ provoqué par la réduction.

\section{Courbe intensité-potentiel}

\subsection{Montage à 3 électrodes}
Slide montage à deux électrodes : ne fonctionne pas.\\
Objectif : étudier le courant en imposant un potentiel à l'électrode de travail. Pour garder un potentiel fixe à l'électrode de travail on a besoin d'un montage à 3 électrodes. (24')\\
Schéma du montage à 3 électrodes.\\ 
La contre-électrode permet de fermer le circuit électrique sans que du courant ne passe par l'électrode de référence. Utilité d'un générateur : sert à imposer une différence de potentiel. Le potentiostat permet d'imposer la différence de potentiel.

\subsection{Limitation de la cinétique}
\begin{itemize}
    \item \textbf{transfert de charge :} transfert électronique à l'interface électrode/solution
    \item \textbf{transfert de matière :} Comprend les phénomènes de diffusion et de convection 
\end{itemize}

\subsection{Système lent (34')}
Transfert de charge cinétiquement déterminant. Il existe une surtension à appliquer pour observer un courant dans l'électrode.\\
Définition surtension anodique/cathodique à l'aide d'un schéma.
\subsection{Système rapide (40')}
\textcolor{blue}{expérience :} Tracer d'une courbe intensité-potentiel 
\section{Conclusion} 


\end{reportBlock}