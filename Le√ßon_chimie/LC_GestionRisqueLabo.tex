\begin{headerBlock}
\chapter{Gestion des risques en laboratoire de chimie}
\label{LC_GestionRisquesLabo}
 \end{headerBlock}

%%%%%%%%%%%%%%%%%%%%%%%%%%%%%%%%%%%%%%%%%%%%%%%%%%%%
%%%% Références


%%%%%%%%%%%%%%%%%%%%%%%%%%%%%%%%%%%%%%%%%%%%%%%%%%%%
%%%% Plan
\begin{reportBlock}{Bibliographie}

\begin{center}
\begin{tabularx}{\textwidth}{| X | X | c | c |}\hline
Titre & Auteur(s) & Editeur (année) & ISBN \\ \hline
 &  &  &  \\ 
 \hline
\end{tabularx}
\end{center}

\end{reportBlock}

\begin{reportBlock}{Plan détaillé}

\underline{Niveau} : 1ère ST2S \\

\section*{Introduction pédagogique}


\paragraph*{Prérequis}
\begin{itemize}
\item 
\end{itemize}

\paragraph*{Contexte :}
Leçon de début d'année.

\paragraph*{Notions importantes}

\begin{itemize}
\item 
\end{itemize}

\paragraph*{Objectifs}

\begin{itemize}
\item 
\end{itemize}

\paragraph*{Difficultés}

\begin{itemize}
\item 
\end{itemize}

\section*{Introduction }
Produit ménagers = 12000 morts/an en France. Phrase d'accroche : on a ici de l'eau de javel, ici un détartant comme le vinaigre. Il ne faut surtout pas les mélangers ! On va essayer de comprendre pourquoi.
\section{Manipuler en sécurité}

\subsection{Les bons gestes}

\begin{itemize}
    \item Porter une blouse
    \item Porter des lunettes
    \item Attacher ses cheveux
    \item Porter des gants
\end{itemize}

En cas d'accident :
\begin{itemize}
    \item Contact avec l pau : rincer abondament
    \item contact avec les yeux : rincer avec le rince \oe il
    \item Inhalation : respirer de l'air frais
    \item boire de l'eau
\end{itemize}

\subsection{Savoir lire les signes}

\begin{itemize}
    \item Pictogrammes de sécurité (description sur slide, source \url{profpinedon.ekablog.com}
    \item étiquettes : ex du méthanol
\end{itemize}


\section{Sécurité des produits acides et basiques}

\subsection{Couple acide/base}

\textcolor{green}{Définition (au sens de Brönsted) :} Espèce chimique est une espèce susceptible de donner un proton. Une base est une espèce susceptible de capter un proton.\\

Cas particulier de l'eau (espèce ampholyte).\\

\begin{itemize}
    \item Plus la solution est acide, plus $[H_3O^+]$ est importante.
    \item Plus la solution est acide, plus $[HO^-]$ est importante.
\end{itemize}

\subsection{pH et concentration}

\textcolor{green}{définition :} le pH quantifie l'acidité/la basicité d'une espèce chimique. $[H_3O^+]=-10^{-pH}$.\\

En pratique, on vérifie au papier pH l'acidité d'une solution. On peut aussi utiliser le pH mètre si on veut être précis lors d'une expérience.

\subsection{Neutralisation}

\textcolor{blue}{Manipulation imposée : neutralisation d'un acide}

\section{Sécurité des produits oxydants et réducteurs}

Antiseptique : agit sur les tissus vivants. Ex : I$_2$ dans la bétadine.\\

Désinfectant : tue tout objet inerte. Ex : ions ClO$^-$ dans l'eau de javel.

\subsection{Couple oxydo-réducteur}
\textcolor{green}{Définition :} Un réducteur = espèce chimique capable de capter des électrons. Un oxydant = espèce chimique capable de céder des électrons.

\subsection{Précautions d'emploi}
\begin{itemize}
    \item lire l'étiquette
    \item diluer
    \item ne pas mélanger
\end{itemize}

Ex : eau de javel + détartrant = formation de dichlore gazeux = mortel.

\subsection{Neutralisation}

\begin{itemize}
    \item dilution
    \item neutralisation par mélange. Ex : $I_2+2S_2O_3^{2-} = 2I^-+S_4O_6^{2-}$
\end{itemize}
\section{Conclusion} 


\end{reportBlock}