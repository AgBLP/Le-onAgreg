\begin{headerBlock}
\chapter{Réactivité des derivés d'acides}
\label{LC_Conductimétrie}
 \end{headerBlock}

%%%%%%%%%%%%%%%%%%%%%%%%%%%%%%%%%%%%%%%%%%%%%%%%%%%%
%%%% Références

\begin{reportBlock}{Bibliographie}

\begin{center}
\begin{tabular}{|c|c|c|c|}\hline
Titre & Auteur(s) & Editeur (année) & ISBN \\ \hline
Chimie organique: tout en fiche &  Maddaluno et al. &  Dunod &  \\ \hline
Sources internet pour STL SPCL & ~ &  ~ & ~ \\ \hline
Techniques expérimentales en chimie & Bernard et al. &  Dunod &  \\ \hline
\url{https://culturesciences.chimie.ens.fr/thematiques/chimie-organique/methodes-et-outils/les-acides-amines-et-la-synthese-peptidique} & Sur les acides aminés et liaison peptidique & & \\
\hline
\url{https://spcl.ac-montpellier.fr/moodle/pluginfile.php/3381/mod_resource/content/6/CH6\%20Syntheses\%20organiques_activite4.pdf} & Pour synthèse Dean-Stark & & \\
\hline
\end{tabular}
\end{center}

\end{reportBlock}

\begin{reportBlock}{Plan détaillé}

\underline{Niveau} : Tle STL - SPCL \\

\underline{Élément imposé} : Montage Dean-Stark \\

\underline{Pré-requis} :
\begin{itemize}
\item Représentation topologique des molécules
\item Facteurs cinétiques d'une réaction chimique
\item Site électrophile/nucléophile
\item Électronégativité
\end{itemize}


\section*{Introduction pédagogique}

\subsection*{Objectifs}

\begin{itemize}
\item Introduire de nouveaux groupements fonctionnels
\item Présenter l'estérification
\item Montrer le déplacement d'équilibre chimique
\item Présenter le montage de Dean-Stark
\end{itemize}


\subsection*{Place de la leçon dans l'année}

Avant la distillation et après la notion de rendement de synthèse.

\subsection*{Difficultés et remédiations}

\begin{itemize}
\item Nomenclature
\item Identification et différenciation des groupements fonctionnels
\end{itemize}

\textcolor{red}{Fin de l'intro pédagogique : 3min} \\

Intro: arômes dans les bonbons synthétisés. 

\section{Dérivés d'acides}

\subsection{Réactivité de l'acide carboxylique}

\paragraph{Ex} Acide éthanoïque

Schéma de la molécule.

%\begin{center}
%\includegraphics[scale=0.6]{acideEt2}
%\end{center}

\subsection{Exemples de dérivées d'acide}

%\begin{center}
%\includegraphics[scale=0.2]{tableau}
%\end{center}

%\begin{center}
%\includegraphics[scale=0.25]{tableau2}
%\end{center}


\subsection{Estérification [17'51]} 

Equation de la réaction d'estérification : \\
Acide carboxylique + alcool $\rightarrow$ ester + H$_2$O.

\paragraph{Ex} ester de banane

Acide acétique (acide éthanoïque) + alcool isoamylique (3-méthylbutan-1-ol) $\rightarrow$ ester de banane

Constante de réaction $K = 4$.

%\begin{center}
%\includegraphics[scale=0.2]{banane}
%\end{center}

\paragraph{Particularité} Réaction lente, équilibrée.

\section{Optimisation d'une estérification [22'50]}


\subsection{Tableau d'avancement}

%\begin{center}
%\includegraphics[scale=0.2]{avancement}
%\end{center}

Quantité d'eau à la fin de la réaction : $\xi_f = 7.63 10^{-2}$ mol. \\
$\eta = \frac{\xi_f}{n_1} = 84\%$.

Déplacer l'équilibre en supprimant un produit (l'eau).

\subsection{Montage de Dean Stark [28'57]}

Présentation du montage (schéma sur slide) et du principe.

On joue sur la miscibilité de l'eau avec le solvant dans le milieu réactionnel.

\paragraph{Expérience 1 [32'19]} Synthèse de l'ester de banane (Dean Stark).

Rappel des règles de sécurité. 

\section*{Conclusion}

Optimisation du rendement $\rightarrow$ chimie verte. Dans une prochaine leçon: expliquer la distillation.

[40'12]

\end{reportBlock}

\begin{reportBlock}{Questions posées}

\begin{itemize}

\item Vous avez parlé de mécanisme à l'oral. Est-ce que vous comptez le présenter à un autre moment ? Que doivent savoir faire les élèves ? \textcolor{purple}{Je ne voulais pas en parler. Je pense que c'est faisable de présenter le mécanisme. Ils doivent savoir identifier les réactions pour une estérification.}

\item Vous avez présenté différents dérivés d'acides. Pour les amides vous avez parlé de la liaison peptidique.Est-ce que vous comptez en parler ? \textcolor{purple}{Peut-être dans une activité documentaire. }


\item Quels sont les débouchés de la filière STL SPCL ? \textcolor{purple}{Études en chimie ou en biologie plutôt orientés vers les applications.}


\item Applications concrètes de cette leçon. \textcolor{purple}{Synthèses organiques : synthèse d'un arôme (estérification), savons (saponification).}


\item C'est quoi l'intérêt de partir d'un anhydride ? \textcolor{purple}{Totale et plus rapide.}


\item Pourquoi plus rapide ? \textcolor{purple}{La charge du carbocation est moins stabilisée sur l'anhydride car l'effet mésomère est partagé entre deux carbocation: si l'un est stabilisé ce n'est pas le cas de l'autre.}


\item C'est quoi un halogénure d'alkyle ? \textcolor{purple}{Alcanes dont un ou plusieurs atomes d'hydrogène sont remplacés par des atomes d'halogène.}

\item C'est quoi un halogénure d'acyle ? \textcolor{purple}{Dérivés halogénés des acides carboxyliques.}


\item Vous avez dit qu'on utilise les dérivées d'acides dans la vie de tous les jours. \textcolor{purple}{Je voulais parler des molécules qui existent dans la nature et qu'on peut synthétiser.}



\item Différence entre naturel, artificiel et synthétique ? \textcolor{purple}{Les espèces chimiques naturelles sont les espèces chimiques non inventées par l'être humain (ex: eau). Les espèces chimiques artificielles sont les espèces chimiques inventées par l'être humain (ex: polymères dérivés du pétrole). Les espèces chimiques synthétiques sont des espèces chimiques, naturelles ou artificielles, qui ont été fabriquées par des êtres humains.}



\item Techniques pour extraire des composés naturels? \textcolor{purple}{Filtration, distillation, hydrodistillation.}



\item Sur quoi ça repose l'hydrodistillation ? \textcolor{purple}{Permet de séparer un mélange constitué d'eau et d'un autre composé non miscible à l'eau. Basé sur les diagrammes binaires liquide/vapeur de l'eau et du composé.}



\item Comment on peut déterminer la quantité maximal d'eau (si le rendement était de 100\%) ? \textcolor{purple}{C'est la quantité du réactif limitant. Ici: $1.8$ mL.}


\item Et ce que tu récupères du Dean Stark ? C'est plus. Comment ça se fait ? \textcolor{purple}{Il y a de l'eau dans l'APTS.}


\item A quoi sert l'APTS ? \textcolor{purple}{À catalyser la réaction.}


\item C'est quoi la catalyse ? \textcolor{purple}{Accélération de la cinétique de réaction au moyen d'un catalyseur.}


\item Quel est le rôle de l'APTS dans cette réaction ? \textcolor{purple}{Il va jouer le rôle d'acidificateur dans la première étape.}


\end{itemize}


\end{reportBlock}

\begin{reportBlock}{Commentaires}



\end{reportBlock}


\begin{reportBlock}{Expérience 1}
% bloc à dupliquer autant de fois que d'expériences

\underline{Titre} : Synthèse de l'ester de banane  \\

\underline{Référence complète} : Hatier, Physique-Chimie Tale spé, 2012, p.261.  \\ 
Techniques expérimentales de Chimie pour le montage. \\

\underline{Équation chimique et but de la manip} : cf. leçon. Objectif: augmenter le rendement de la synthèse de l'ester de banane. Calculer ce rendement. \\

\underline{Modification par rapport au mode opératoire décrit} : 

\underline{Commentaire éventuel} : Attention: l'APTS contient de l'eau, ça fausse la quantité d'eau obtenue avec la réaction: il faut le prendre en compte. \\

\underline{Phase présentée au jury} : Prélèvement du cyclohexane. Extraire l'eau obtenue pendant la synthèse (préparation) et la peser.

\underline{Durée de la manip} : 32'19- fin.  \\

\end{reportBlock}



\begin{reportBlock}{Compétence \og Autour des valeurs de la République et des thématiques relevant de la laïcité et de la citoyenneté \fg{}}

\underline{Question posée} : Est-ce une bonne idée de créer un compte sur un réseau social pour communiquer avec les élèves. \\

\underline{Réponse proposée} :  Pas forcément. Tout le monde n'a pas un compte sur un réseau social en particulier. Il y a d'autres moyens de communiquer, via l'ENT des établissement. Je ne vois pas l'intérêt de l'utiliser pour communiquer avec les élèves. Ça peut poser problème pour des questions d'accessibilité. Il y a aussi des questions de confidentialité, si les informations sont rendues publiques.  \\

\underline{Commentaire du correcteur} : 

\end{reportBlock}




\begin{reportBlock}{Champ libre pour le correcteur}
% compléments, propositions de manipulation, bibliographie etc.
\end{reportBlock}

\end{document}