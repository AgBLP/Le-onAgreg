\begin{headerBlock}
\chapter{Réactions de synthèse, sites électrophiles, nucléophiles, formalisme des flèches courbes}
\label{LC_ReactionSynthese_SPCL}
 \end{headerBlock}

%%%%%%%%%%%%%%%%%%%%%%%%%%%%%%%%%%%%%%%%%%%%%%%%%%%%
%%%% Références


%%%%%%%%%%%%%%%%%%%%%%%%%%%%%%%%%%%%%%%%%%%%%%%%%%%%
%%%% Plan
\begin{reportBlock}{Bibliographie}

\begin{center}
\begin{tabularx}{\textwidth}{| X | X | c | c |}\hline
Titre & Auteur(s) & Editeur (année) & ISBN \\ \hline
 Chimie PCSI Chap 3 & J-B Baudin & Dunod (2019) &  \\ 
 \hline
 \url{https://spcl.ac-montpellier.fr/moodle/course/view.php?id=61&section=4} & Académie Montpellier & & \\
 \hline
 \url{https://ww2.ac-poitiers.fr/sc_phys/sites/sc_phys/IMG/pdf/stl_bio_physique_chimie.pdf} & Sujet bac pour l'hydrolyse de l'eau dans l'ISS & & \\
\end{tabularx}
\end{center}

\end{reportBlock}

\begin{reportBlock}{Plan détaillé}

\underline{Niveau} : Terminale STL-SPCL \\

\section*{Introduction pédagogique}


\paragraph*{Prérequis}
\begin{itemize}
\item 
\end{itemize}
\paragraph*{Contexte :}


\paragraph*{Notions importantes}

\begin{itemize}
\item 
\end{itemize}

\paragraph*{Objectifs}

\begin{itemize}
\item 
\end{itemize}

\paragraph*{Difficultés}

\begin{itemize}
\item 
\end{itemize}

\section*{Introduction }

\section{Les biomolécules pour l'organisme}

\section{Conclusion} 

\end{reportBlock}