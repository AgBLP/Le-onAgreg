\begin{headerBlock}
\chapter{Techniques spectroscopiques}
\label{LC_ControleQualite}
 \end{headerBlock}

%%%%%%%%%%%%%%%%%%%%%%%%%%%%%%%%%%%%%%%%%%%%%%%%%%%%
%%%% Références


%%%%%%%%%%%%%%%%%%%%%%%%%%%%%%%%%%%%%%%%%%%%%%%%%%%%
%%%% Plan
\begin{reportBlock}{Bibliographie}

\begin{center}
\begin{tabularx}{\textwidth}{| X | X | c | c |}\hline
Titre & Auteur(s) & Editeur (année) & ISBN \\ \hline
Chimie PCSI Tout-en-un Chap 8 & B. Fosset, J-B Baudin & Dunod (2019) &  \\ 
 \hline
 
\end{tabularx}
\end{center}

\end{reportBlock}

\begin{reportBlock}{Plan détaillé}

\underline{Niveau} : Terminale ST2S \\

\section*{Introduction pédagogique}


\paragraph*{Prérequis}
\begin{itemize}
\item
\end{itemize}
\paragraph*{Contexte :}


\paragraph*{Notions importantes}

\begin{itemize}
\item 
\end{itemize}

\paragraph*{Objectifs}

\begin{itemize}
\item 
\end{itemize}

\paragraph*{Difficultés}

\begin{itemize}
\item 
\end{itemize}


\section*{Introduction }
\textcolor{blue}{Manip qualitative : prendre une masse reliée à un ressort. Si j'agite trop lentement ou trop rapidement, le ressort de se tend pas. Si j'agite à la bonne fréquence, il y a une élongation maximale : résonance.}

\section{Spectroscopie IR}
Mettre loi de Planck, états vibratoires de la molécule. Faire le lien avec la manip qualitative. Spectre typique IR, ordre de grandeur des énergies des états vibratoires des liaisons.

\section{Spectroscopie visible}
Loi de Beer-Lambert, faire la manip 

\section{Spectroscopie RMN}

\end{reportBlock}