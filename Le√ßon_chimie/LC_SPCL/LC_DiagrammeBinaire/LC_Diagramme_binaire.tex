%%%% En-tête leçon
\begin{headerBlock}
\chapter{Distillation}
    \label{LC_Distillation}
\end{headerBlock}




%%%%%%%%%%%%%%%%%%%%%%%%%%%%%%%%%%%%%%%%%%%%%%%%%%%%
%%%% Références
\begin{reportBlock}{Bibliographie}

\begin{center}
\begin{tabularx}{\textwidth}{| X | X | c | c |}\hline
Titre & Auteur(s) & Editeur (année) & ISBN \\ \hline
Tout-en-un Chimie PC/PC* ~ & B. Fosset, J-B. Baudin  ~ & Dunod (2022)~ & ~ \\
\hline 
\url{http://vle-calc.com/} & Site sympa pour les diagrammes ~ &  ~ & ~ \\
\hline
\url{https://spcl.ac-montpellier.fr/moodle/course/view.php?id=61&section=7} & Académie de Montpellier & Eduscol & ~ \\ \hline
\end{tabularx}
\end{center}

\end{reportBlock}

%%%%%%%%%%%%%%%%%%%%%%%%%%%%%%%%%%%%%%%%%%%%%%%%%%%%
%%%% Plan
\begin{reportBlock}{Rappels de définitions, concepts à aborder lors de la leçon : }
\textbf{Rappels : } cf le BO de la spécialité SPCL de STL.

\begin{enumerate}
    \item Pouvez-vous rappelez les difficultés associées à cette leçon ? \textcolor{red}{Lecture et compréhension de diagramme binaire, lien avec la distillation, courbe d'analyse thermique (mélange binaire), nombreuses définitions avec lesquelles on joue (fraction molaire/massique/titre/masse molaire, etc...), verrerie associée au montage}. Y-a-t’il une difficulté en terme de vocabulaire ? \textcolor{red}{Clairement, on doit insister sur les définitions et les différents noms de la verrerie spécifique à la distillation} Comment s’appelle le ballon ? \textcolor{red}{Le ballon est un bon exemple pour insister sur les définitions. On le nomme "bouilleur".} C’est un terme qu’on utilise dans d’autres situations ? \textcolor{red}{Dans les chaufferies, sert à faire bouillir le fluide chauffant. Dans l'industrie du sel, présent dans le processus de déssalement.}
    \item L'hydrodistillation est-elle au programme de terminale STL-SPCL ? \textcolor{red}{Oui, dans le thème « Chimie et développement durable ».}
    \item Diagramme binaire, notion théorique ou expérimentale ? \textcolor{red}{C'est une notion théorique à connotation fortement expérimentale car très utile pour déterminer les fractions massiques d'un mélange. Notion théorique car c'est applicable en théorique à tous les mélanges binaires et relié aux concepts thermodynamique (capacité thermique, température de changement d'état donc enthalpie de changement d'état, équation d'état, principes de la thermodynamique, etc...).}
      
\end{enumerate}

\end{reportBlock}

\begin{reportBlock}{Avis sur le plan proposé, choix des exemples et des expériences : }
\textbf{Plan :}\\
\section{Mélanges binaires homogènes}
Pour étudier les mélanges binaires homogènes, on a avoir besoin d'un certains nombre de définitions qui seront importantes dans tout le déroulé de la leçon.
\subsection{Mélange binaire}
Définition d'un mélange homogène : mélange dont on ne peut distinguer à l'\oe il nu ses différents constituants.
Définition d'un corps pur : matière qui ne comporte qu'une espèce chimique.
\subsection{Fraction molaire}
Définition fraction molaire d'un mélange A+B : la fraction molaire d'un corps A est : $x_{A}=\frac{n_A}{n_A+n_B}$, avec $n_i$ la quantité de matière du corps $i$
\subsection{Fraction massique}
Définition fraction massique d'un mélange A+B : la fraction massique d'un corps A est : $w_{A}=\frac{m_A}{m_A+m_B}$, avec m$_i$ la masse du corps $i$.
\section{Diagrammes binaires}
\subsection{Diagramme liquide-vapeur isobare}
Présentation et explication des graphes T=f(x ou w) : on montre la courbe de rosée et la courbe d'ébullition à une pression donnée. On décrit l'exemple particulier du diagramme eau/éthanol.
\subsection{Courbes d'analyse thermique}
Tracé des graphes T(t) pour un corps pur : on part de la phase liquide, on arrive à la température d'ébullition qui marque un pallier jusqu'à ce que la dernière goutte de liquide disparaisse puis elle réaugmente dans la phase gazeuse.\\
Pour un mélange binaire homogène, le plateau de température n'est pas constant car les corps A et B du mélange ont des températures de fusion différentes.\\ 
Définition de corps volatil : on appelle de volatil la capacité d'un solide ou d'un liquide à se vaporiser facilement. On va jouer sur les différences de volatilité entre les corps d'un mélange homogène pour réaliser une distillation par exemple.
\subsection{Azéotrope}
Definition azéotrope : mélange présentant une composition particulière pour lequel il se comporte comme un corps pur (phases L et V à l'équilibre ont la même composition). On le montre explicitement sur le diagramme binaire.
\subsection{Exploitation}
On prend un exemple en se plaçant à une certaine fraction molaire (ou massique) d'un mélange à une température ou le mélange est liquide. On fait la lecture des températures d'ébullition et de condensation. On parle de la composition des premières vapeurs (ou des premières gouttes). On rappelle les propriétés de volatilité des corps, et on fait un exemple sur une seule recondensation en présentant la nature du distillat obtenu.
\section{Application à la distillation}
\subsection{Distillation simple}
Sur slide : rappel du montage, lecture sur le diagramme. Ce montage a déjà été vu dans les cours de première et on va voir un nouveau montage qui est celui de la distillation fractionnée.
\subsection{Distillation fractionnée}
Sur slide et sur la paillasse : rappel montage, lecture sur le diagramme. On insiste sur les noms de la verrerie : ballon à fond rond, colonne de Vigreux, condenseur, erlenmeyer.\\

\textcolor{green}{\textbf{\underline{Expérience :}}}\\


On procède à l'expérience de distillation fractionnée sur un mélange eau/éthanol en pesant au préalable les masses d'eau et d'éthanol pour voir où on se place dans le diagramme binaire. On fait chauffer le mélange et on attend de voir les premières gouttes dans le distillat.\\
La réaction est assez rapide suivant la hauteur de la colonne de Vigreux (10min en préparation, 5min au cours de la leçon, ça dépend aussi de le composition du mélange au départ). Il faut faire attention à ce que la température du thermomètre placé au niveau du coude \{colonne de Vigreux - condenseur\} ne dépasse pas la température de l'azéotrope, sinon cela veut dire qu'on commence à faire recondenser de l'eau dans le distillat ce qui dégrade la distillation.\\
On détermine de la fraction massique à l'aide d'une courbe d'étallonage masse volumique vs titre alcoolémique. Puis on détermine le rendement donné par : $\eta=\frac{m_{\text{eth,distillat}}}{m_{\text{eth,initial}}}$.
\section*{Conclusion}
Ouverture sur l'application de la distillation dans la pétrochimie : on montre le schéma du distillat pour le pétrole (gaz, fioul, pétrole brut etc...) en expliquant que ces différentes phases correspondent au raffinement (pourcentage de distillation) du pétrole brut.\\

\textbf{Avis:} Bonne leçon globalement, attention à la trace écrite et les sous-parties vides.\\
Bien d’avoir choisi un sytème réel, mais garder eau+éthanol pas « A+B ». Bien pour les prérequis mais pas toujours clair sur ce qui est connu. Le dire dans l’intro péda.\\
Utilisation diagrammes binaires solide-liquide : eau + sel pour abaisser le point de fusion sur les routes.\\
Attention ne pas être trop vague sur les réponses, essayer de bien détailler.\\
L’olive : fond rond = ballon, fond plat = bécher\\
Montage à bien expliquer avec un schéma en direct ou un montage en direct. Attention au vocabulaire (bouilleur, distillat, etc…) .\\
Difficulté, la lecture des graphiques peut ne pas être intuitif. \\
Calorifuger la colonne de vigreux pour augmenter la vitesse de distillation. \\
Incertitudes pas adaptées car faits sur la calculatrice et pas sur tableur.\\
A l’aise à l’oral, bien pour les couleurs, attention dans le langage « j’ai 30s donc je me dépêche », être naturel pour être bien en maîtrise (même si dans la tête c’est pas du tout ça).

\end{reportBlock}

\begin{reportBlock}{Remarques sur des points spécifiques de la leçon: }
\begin{enumerate}
    \item Comment faire le lien entre diagramme binaire isobare et isotherme ? Comment à partir d’un diagramme à un fuseau vous pouvez tracer le diagramme binaire correspondant ? \textcolor{red}{Similaire mais avec courbe d'ébullition en haut et courbe de rosée en bas.}
    \item Titre en alcool, fraction molaire/massique, comment vous les reliez ? Comment l'expliquer aux étudiants ? \textcolor{red}{On peut partir des définitions, par exemple le titre massique $w_A=\frac{m_A}{m_A+m_B}$ et utiliser la relation $m_A=n_A\times M_A$ avec $n_A$ la quantité de matière et $M_A$ la masse molaire du corps A. Ensuite en multipliant en haut en en bas de la fraction par $n_A+n_B$, on voit apparaître les fractions molaires $x_{A/B}=\frac{n_A}{n_A+n_B}$ ce qui donne $w_{A}=\frac{x_AM_A}{x_AM_A+x_BM_B}$. Pour le titre massique, on fait la même chose en remplaçant $m_A = \rho_A V_A$.}
    \item En industrie, quel est l'ordre de grandeur des rendements ? Pour l'éthanol ? \textcolor{red}{En général, ils sont optimisés et donc très bons. L'ordre de grandeur pour l'éthanol est de l'ordre de celui du point azéotrope soit 96.5\% à $P=1$~bar pour la distillation fractionnée.} On peut aller plus loin en pureté ? \textcolor{red}{Oui, mais il faut introduire un troisième corps (le benzène pour l'éthanol) et utiliser les diagrammes ternaires. Le rendement est bien plus élevé (l'éthanol anhydre, utilisé pour des réactions très sensibles à l'eau, contient de l'eau à $50$~ppm selon Wikipédia).}
    \item Quelle différence entre l'hydro-distillation et l’entrainement à la vapeur ? En terme de montage expérimental ? \textcolor{red}{L'entraînement à la vapeur d'eau et l'hydrodistillation sont des procédés d'extraction ou de séparation de certaines substances organiques de l'eau. Ces deux termes n'ont pas la même signification. "Hydrodistillation" désigne la distillation d’un mélange hétérogène d’eau et d’un liquide organique. L'hydrostillation est beaucoup utilisé en parfumerie et utilise la décomposition des réactifs (molécules odorantes d'un composé organique libérées après décomposition de ce composé sous l'effet de la chaleur). "L’entraînement à la vapeur" est applicable aux composés peu ou pas solubles dans l'eau, dotés d'une tension de vapeur assez importante vers les $100$~$^{\circ}$C. L’avantage de cette technique réside en l'abaissement de la température de distillation les composés sont donc entraînés à des températures beaucoup plus basses que leur température d’ébullition, ce qui évite leur décomposition.}
    \item Pourquoi cette leçon est dans le cadre du programme "Systèmes et Procédés" ? \textcolor{red}{Le thème « Systèmes et procédés » a pour objectif d’étudier des systèmes réels en analysant les flux d’information, de matière et d’énergie. Il comporte un sous-thème "transport et transformation des flux de matière" dont l'étude est un élément important pour l’analyse et la compréhension des procédés physico-chimiques comme ceux liés à la distillation}
    \item C’est quoi la physique derrière les diagrammes binaires ? Comment obtenir l’équation des courbes rosée/ébullition ? \textcolor{red}{cf. cours de thermochimie.}
    \item Liquide/vapeur seuls diagrammes ? \textcolor{red}{Diagrammes solide-liquide.} Avec solides/liquides que pouvez-vous expliquer ? \textcolor{red}{Salage des routes, mélange eau+sel a une température de fusion plus basse que celle de l'eau pure.} 
    
    \end{enumerate}

\end{reportBlock}

\begin{reportBlock}{Discussion sur les manipulations présentées au cours du montage(objectifs de l’expérience, phases de manipulations intéressantes, difficultés théoriques et techniques) :}
 \begin{enumerate}
      \item	Vous n'avez pas utilisé de gants ? Quelles sont les précautions à prendre pour votre montage ? \textcolor{red}{Non, l'éthanol n'est pas dangereux, on en boit (avec modération) et on l'utilise couramment avec le gel hydroalcoolique par exemple. En revanche il est très inflammable donc il faut l'éloigner des sources de chaleurs.}
      \item Préciser la position du thermomètre sur le distillat ? \textcolor{red}{Il faut qu'il s situe au niveau du coude entre la partie colonne de vigreux et le réfrigérent. On veut connaître la température de la phase vapeur qui sort de la colonne de vigreux et vérifier qu'elle ne dépasse pas la température de l'azéotrope, sinon on commence à évaporer une phase pure en la phase qu'on ne souhaite pas voir présent dans le distillat.}
      \item Pourquoi avez-vous été surprise sur la rapidité de la distillation ? Si on attend encore plus, qu’est-ce qu’il se passe ? \textcolor{red}{C'était un peu plus rapide que pendant la préparation. Ca peut être un peu critique si on attend trop car on va recondenser l'eau dans le distillat et modifier la pureté de l'éthanol distillé.}
      \item Au niveau sécurité ? Quels conseils aux étudiants pour remplir le chauffe ballon ? \textcolor{red}{On fera attention à la fragilité de la verrerie. On fera attention à ce que l'évaporation ne soit pas complète car on peut endommager les équipements et la verrerie. C'est aussi pour cela qu'on utilise des chariots élévateurs pour empêcher le contact bouilleur-chauffage rapidement si besoin.}
      \item Que faut-il utiliser pour homogénéiser la solution dans votre cas ? \textcolor{red}{Que olive avec la forme adaptée (bombée) pour le ballon.} Que feraient les étudiants ? \textcolor{red}{Probablement ils ne se poseraient pas la question sur la forme de l'olive à choisir.} L’olive c’est le seul outil ? \textcolor{red}{Oui ?} C’est quoi une bonne agitation ? \textcolor{red}{Une agitation qui homogénéise bien la solution et que ne créé pas d'éclaboussures.}
      \item Le thermomètre servait à quoi dans le montage ? \textcolor{red}{A vérifier que le température ne dépassait pas celle du point azéotrope du diagramme binaire eau/éthanol}
      \item Si la distillation est trop lente, comment faire pour l'accélérer ? \textcolor{red}{On peut calorifuger les parois de la colonne.}
      \item Si on revient aux incertitudes sur les masses, est-elle représentative de l’incertitude de l’expérience ? \textcolor{red}{Non, l'incertitude va surtout porter sur les pertes de matières lors des différentes manipulations (transfert d'un récipient à un autre, etc.). Ces incertitudes ne peuvent être qu'estimer à vue d'oeil.}
      \item Quel est l’intérêt de présenter les incertitudes ? \textcolor{red}{Elle est intéressante pour faire comprendre le sens physique qu'il y a derrière, voir la propagation des erreurs (incertitudes sur une fraction, une somme) et pour comparer aux valeurs tabulées dans la littérature (fraction massique de l'azéotrope).}
      
  \end{enumerate}

\end{reportBlock}

\begin{reportBlock}{Propositions de manipulations –Bibliographie :}
\begin{enumerate}
    \item Quel TP vous feriez faire aux étudiants dans cette leçon ? Quels composés vous distilleriez dans ce cours-là ? \textcolor{red}{La même expérience que réalisée ici (distillation d'un mélange eau/éthanol) est intéressante. On peut aussi regarder l'effet de la taille de la colonne de vigreux et aussi faire la comparaison distillation fractionnée/distillation simple.}
    \item Si vous avez le choix, vous aborderiez ces notions comme vous l’avez fait ou autrement ? \textcolor{red}{Une analyse documentaire peut être pertinente : proposer des exemples de distillation utilisant les diagrammes binaires en industrie, faire réfléchir sur des montages possibles. Cela permet de se familiariser avec le vocabulaire.}. Où prendriez-vous les docs pour une analyse documentaire ? \textcolor{red}{Internet, Dunod PC, liens vidéos ludiques, simulation ou animation}. Quelles sont les sources que vous avez utilisées sur les diagrammes binaires ? Que pouvez-vous faire avec VLE-calc ? \textcolor{red}{Tracer des diagrammes binaires liquide-vapeur, obtenir les infos sur les azéotropes, etc.} Pas de références papier ? \textcolor{red}{Dunod PC, techniques expérimentales en chimie, principalement internet.}
\end{enumerate} 
\end{reportBlock}



\begin{reportBlock}{Autour des valeurs de la République et des thématiques relevant de la laïcité et de la citoyenneté :}
 Quelle est la place de l’humour sur la copie et à l’oral sur la relation élève professeur ?\\
 \textcolor{red}{L'humour peut être utilisé mais il faut faire attention à ses limites. L'humour à outrance peut décridibiliser l'autorité et se retourner contre l'enseignant, surtout en début d'année. L'humour pour pointer une difficulté d'un élève en classe ou sur la copie peut être très mal pris par l'élève, à voir évidemment avec son caractère en cours d'année. On ne peut pas rire de tout avec les élèves en classe, mais on peut faire de l'humour avec un élève en particulier à condition de connaître son caractère et de respecter des limites de décence. Un exemple humouristique (par exemple montrer une image des Dupontd dans Tintin qui font des bonds sur la lune main dans la main avec des cheveux colorés pour illustrer la gravitation) peut être attrayant et source de motivation/d'intérêt pour l'élève. L'humour est aussi une question de sujets. }
\end{reportBlock}
