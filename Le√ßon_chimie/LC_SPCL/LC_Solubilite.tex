\begin{headerBlock}
\chapter{Solubilité}
\label{LC_Solubilité}
 \end{headerBlock}

%%%%%%%%%%%%%%%%%%%%%%%%%%%%%%%%%%%%%%%%%%%%%%%%%%%%
%%%% Références


%%%%%%%%%%%%%%%%%%%%%%%%%%%%%%%%%%%%%%%%%%%%%%%%%%%%
%%%% Plan
\begin{reportBlock}{Bibliographie}

\begin{center}
\begin{tabularx}{\textwidth}{| X | X | c | c |}\hline
Titre & Auteur(s) & Editeur (année) & ISBN \\ \hline
100 manipulations de chimie (p193)~ & Jacques Mesplède, Jérôme Randon ~ & Bréal (2004) ~ & ~ \\
\hline
\url{http://je-plante-mon-agreg.com/Documents/Chimie_exp\%C3\%A9rimentale-Listes_TP_chimie.pdf} &  &  ~ & ~ \\ \hline
 \url{https://spcl.ac-montpellier.fr/moodle/pluginfile.php/11573/mod_resource/content/2/Chapitre\%201\%20-\%20Solubilite\%20-\%20\%20Fiche\%20de\%20synthese.pdf} & ~ &  & ~ \\ 
 \hline
 \url{http://leonardvinci.e-monsite.com/medias/files/02.solubilite.pdf} & ~ &  & ~ \\ 
 \hline
\end{tabularx}
\end{center}

\end{reportBlock}

\begin{reportBlock}{Plan détaillé}

\underline{Niveau} : Tle STL - SPCL \\

\section*{Introduction pédagogique}


\paragraph*{Prérequis}
\begin{itemize}
\item pH, pH-métrie
\item quantité de matière, masse volumique, concentrations
\end{itemize}

\paragraph*{Contexte :}
Place de la leçon : début d'année.

\paragraph*{Notions importantes}

\begin{itemize}
\item Constante de solubilité
\item prédire l'évolution
\item influence de la solubilité (T, pH, etc...)
\end{itemize}

\paragraph*{Objectifs}

\begin{itemize}
\item prédire la précipitation, dissolution
\item déterminer la solubilité et prévoir son évolution
\end{itemize}

\paragraph*{Difficultés}

\begin{itemize}
\item Difficultés mathématiques (unités, etc...)
\item Lien entre constante d'équilibre et pH
\end{itemize}
Pour y remédier, activité préliminiaire pour manipuler les grandeurs
\section*{Introduction }
Dissolution courante dans la vie de tous les jours. Ex : sucre dans le café.\\
\paragraph*{Manipulation qualitative:} \textcolor{green}{Exp qualitative : mettre du sel dans l'eau}. Quelle est la quantité max de sel qu'on peut mettre dans l'eau ?

\section{Solubilité et produit de solubilité}

\subsection{Solubilité}

\textcolor{red}{Def solubilité : }Notée s, quantité de matière/masse maximale d'une espèce que l'on peut dissoudre dans un litre de solvant (en $mol.L^{-1}$ ou $g.L^{-1}$)\\
ex: s(NaCl)=358.5g.L$^{-1}$ à 20°C.

\subsection{Produit de solubilité}
Pour la réaction : NaCl(s) = Na$^+$(aq) + Cl$^-$(aq)

\begin{equation}
    K_s = \frac{[Na^+]_{eq}[Cl^-]_{eq}}{(C^{0})^2}
    \end{equation}

Tableau d'avancement : $s=\sqrt{K_s}$

\section{Quotient de réaction}

\begin{equation}
    Q_r = \frac{[Na^+]_{eq}[Cl^-]_{eq}}{(C^{0})^2}
\end{equation}
\begin{itemize}
    \item $Q_r<K_s$ : sens direct : dissolution
    \item $Q_r=K_s$ : équilibre
    \item  $Q_r>K_s$ : sens indirect : precipitation
\end{itemize}
\section{Facteurs influençant sur la solubilité}
\subsection{Le pH}
\textcolor{red}{Rappel :} $pH = -\log([H_3O^+])$ \\
Le pH influe sur la solubilité $s$ : Fe(OH)$_3$(s) = Fe$^{3+}$(aq) + 3HO$^-$(aq).
\begin{equation}
    K_s = \frac{[Fe^{3+}]_{eq}[HO^-]_{eq}^3}{(C^{0})^4}
\end{equation}

\textcolor{blue}{Contexte :} application industrielle pour séparer les ions métalliques d'un minerai par exemple.\\
\textcolor{green}{Manipulation principale :} On va mélanger deux solutions à 1M, une solution de sulfate de cuivre (de couleur bleue) et une solution de nitrate de fer (de couleur orange).\\ %schéma expérimental à mettre
On va faire précipiter d'abord les ions fer Fe$^{2+}$ avec une solution de soude à 1M puis filtrer le précipité sur fritté.\\
Les réaction possibles sont :
\begin{itemize}
    \item Fe(OH)$_2$(s) = Fe$^{2+}$(aq) + 2HO$^-$(aq) possible à $pH\sim2$
    \item Cu(OH)$_2$(s) = Cu$^{2+}$(aq) + 2HO$^-$(aq) possible à pH$\sim 4.8$
\end{itemize}
C'est pour ça qu'on utilise un pH-mètre pour contrôler qu'on ne dépasse pas un pH d'environ 4.\\
La manip fonctionne bien, on voit que la couleur de la solution redevient bleue après filtrage du précipité de Fe(OH)$_2$(s).
\subsection{Température}
En général, la solubilité augmente avec la température.\\
\textcolor{red}{Attention : } Il y a des exceptions. $s$ peut diminuer avec la température (ex: calcaire dans la bouilloire).


\section{Conclusion} 
Applications : dépollution de l'eau, séparation des métaux, décalcairisation en milieu acide.

\end{reportBlock}

\begin{reportBlock}{Questions posées}

\begin{itemize}

\item Slide intro, je suis d'accord sur la position de leçon, obectifs : première capacité expérimentale du BO pas mentionnée ? \textcolor{purple}{Non, mais je la travaillerai en TP en demi-groupe.} Quelle technique peut-on utiliser pour extraire le solide ? \textcolor{purple}{Extraction liquide-liquide.}

\item Est-ce que l'appareil de l'extraction a déjà été  vu ? \textcolor{purple}{Je ne me rappelle plus s'ils l'ont vu l'année dernière, mais je ferai une séance d'intro de la verrerie en début d'année. Donc à mettre en préréquis.}

\item Le point à $1mL$ dévie de la droite, pourquoi ? \textcolor{purple}{Cela peut être dû au fait qu'à ce volume, on n'est pas forcément dans le régime linéaire de la loi de Kohlrausch.}

\item Vous donner comme exemple le titrage des ions nitrate. Comment proposez-vous d'en déterminer la concentration ? \textcolor{purple}{En réalisant le même titrage que pour les ions chlorure.} Peut-on dans ce cas déterminer séparément la concentration des 2 analytes (\ce{NO_3^-} et \ce{Cl^-}) ? \textcolor{purple}{Non.}

\item Tu n'as pas du tout parlé de quantité de matière, concentration, etc... ? \textcolor{purple}{J'ai oublié de faire un calcul de la solubilité dans la leçon...} 

\item Pas de problème de vocabulaire sur les difficultés ? C'est quoi un précipité si tu devais le présenter à tes élèves ? \textcolor{purple}{Un solide qui refuse de se dissoudre dans un solvant} 

\item Dans ton expérience introductive, est-ce qu'il reste du sel ? \textcolor{purple}{Quasiment pas, mais j'aurais du utiliser un autre solide coloré pour que ça soit visible (PbI$_2$) par exemple.}

\item Dans la définition de la solubilité, tu as oublié la température. \textcolor{purple}{Oui...}

\item Qu'est-ce qu'il se passe à K$_s$ ? \textcolor{purple}{Il y a coexistence du précipité et des ions en solutions.}

\item Différence entre un état final et un état d'équilibre ? \textcolor{purple}{Oui, il y a une différence. }

\item Peut-on définir K$_s$ pour d'autres composés que ioniques ? \textcolor{purple}{Non, on peut avoir dissolution d'un gaz dans l'eau par exemple.} Que se passe-t'il si P augmmente ? \textcolor{purple}{s augmente}

\item D'autres moyen pour faire varier la solubilité ? \textcolor{purple}{Effet d'ions communs.} 

\item Comment as-tu défini la solution saturée ? \textcolor{purple}{Quand on a ajouté $s$ ou plus de solide dans le solvant.}

\item Lien entre pH et concentration en HO$^-$ ?
\textcolor{purple}{$[HO^-]=10^{-14-pH}$}

\item Comment on fait en pratique pour les minerais ? \textcolor{purple}{Lixiviation, on se place à un pH super acide et on broit les minerais.}

\item Pour la manip, tu voulais faire qualitatif ou quantitatif ?  \textcolor{purple}{Qualitatif.}Comment tu aurais fait pour la rendre quantitatif ? Peux-tu faire le calcul du pH d'apparition de Fe(OH)$_2$(s) ? \textcolor{purple}{$[HO^-]=(\frac{K_s}{[Fe^{2+}]})^{1/3}$. Avec [Fe$^{2+}$]=0.25mol.L$^{-1}$ environ (rajout d'environ le double d'eau du mélange de solution de 10mL de chaque réactif. On trouve $pH=1,87$.} Il aurait fallu faire ce calcul qui est abordable en terminale. Cela aurait permi de mieux comprendre pourquoi on ne devait pas dépasser ce pH.

\item Si tu devais refaire faire la manip aux étudiants, que dirais-tu niveau sécurité pour la fiole à vide ? \textcolor{purple}{Il faut l'attacher avec une pince 3 doigts et une potence.}

\item Pourquoi dans le cas du calcaire c'est une exception ? \textcolor{purple}{Si on augmente la température, la solubilité du CO$_2$ dans l'eau diminue ce qui augmente le pH. }

\item Que proposez-vous en séance de travaux pratiques ? \textcolor{purple}{Le titrage du sérum physiologique, c'est pédagogique car c'est un produit du quotidien.}

\item 
\textcolor{purple}{}

\end{itemize}


\end{reportBlock}

\begin{reportBlock}{Commentaires}
Plan très bien, explications sont claires et pédagogiques. Choix de la manip impecc. Il manque le quantitatif dans la manip. Gestes pour la manip très bien.

\end{reportBlock}


\begin{reportBlock}{Expérience 1}
% bloc à dupliquer autant de fois que d'expériences

\underline{Titre} : Illustration solubilité du sel dans l'eau. \\

\underline{Référence complète} : \\ 

\underline{But de la manip} :En faible quantité ajouté dans l'eau, le sel se dissous complètement. Au-delà d'une certaine concentration de sel, il reste du précipité.\\

\underline{Commentaire éventuel} :On peut faire la manip avec une espèce coloré comme PbI$_2$ pour que ça soit plus visuel.

\underline{Durée de la manip} : 2' \\

\end{reportBlock}



\begin{reportBlock}{Expérience 2}
% bloc à dupliquer autant de fois que d'expériences

\underline{Titre} : Extraction sélective des ions fer dans une solution de nitrate de fer et de sulfate de cuivre. \\

\underline{Référence complète} : 100 manipulations de chimie, J. Mesplède et J. Randon, p193 \\ 

\underline{Équation chimique et but de la manip} : Fe(OH)$_2$(s) = Fe$^{2+}$(aq) + 2HO$^-$(aq) \\

\underline{Phase présentée au jury} : Précipitation des ions fer en maintenant le pH en-dessous de 4. Puis filtration sur fritté. \\

\underline{Commentaire éventuel} : Prévoir un fritté très fin.\\

\underline{Durée de la manip} : 10'\\

\end{reportBlock}

\begin{reportBlock}{Compétence \og Autour des valeurs de la République et des thématiques relevant de la laïcité et de la citoyenneté \fg{}}

\underline{Question posée} : En quoi la démarche scientifique fais de nous des bons citoyens ? \\

\underline{Réponse proposée} : La démarche scientfique c'est : observer de l'environnement, vérifier, raisonner, garder confiance en soi et avoir foi en les autres, être opimiste. Avoir un esprit critique sur des sujets de société. Beaucoup de complotisme de nos jours, la démarche scientifique permet de démonter cela.\\

\underline{Commentaire du correcteur} : 

\end{reportBlock}


\begin{reportBlock}{Champ libre pour le correcteur}
% compléments, propositions de manipulation, bibliographie etc.

\end{reportBlock}