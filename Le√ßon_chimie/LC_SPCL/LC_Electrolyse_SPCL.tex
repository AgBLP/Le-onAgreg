\begin{headerBlock}
\chapter{Electrolyse - électrosynthèse}
\label{LC_Electrolyse_SPCL}
 \end{headerBlock}

%%%%%%%%%%%%%%%%%%%%%%%%%%%%%%%%%%%%%%%%%%%%%%%%%%%%
%%%% Références


%%%%%%%%%%%%%%%%%%%%%%%%%%%%%%%%%%%%%%%%%%%%%%%%%%%%
%%%% Plan
\begin{reportBlock}{Bibliographie}

\begin{center}
\begin{tabularx}{\textwidth}{| X | X | c | c |}\hline
Titre & Auteur(s) & Editeur (année) & ISBN \\ \hline
 Chimie tout-en-un PC/PC* Chap 13 & A. Demolliens & Nathan (2009) &  \\ 
 \hline
 \url{https://spcl.ac-montpellier.fr/moodle/course/view.php?id=62&section=10} & Académie Montpellier & & \\
 \hline
 Epreuve orale de Chimie (p194 et p398) & F. Porteu-de-Buchère & Dunod (2017) &  \\
 \hline
\end{tabularx}
\end{center}

\end{reportBlock}

\begin{reportBlock}{Plan détaillé}

\underline{Niveau} : Terminale STL-SPCL \\

\section*{Introduction pédagogique}


\paragraph*{Prérequis}
\begin{itemize}
\item 
\end{itemize}
\paragraph*{Contexte :}


\paragraph*{Notions importantes}

\begin{itemize}
\item 
\end{itemize}

\paragraph*{Objectifs}

\begin{itemize}
\item 
\end{itemize}

\paragraph*{Difficultés}

\begin{itemize}
\item 
\end{itemize}

\section*{Introduction}

\section{}

\section*{Conclusion} 

\end{reportBlock}