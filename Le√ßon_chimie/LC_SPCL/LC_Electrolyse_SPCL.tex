\begin{headerBlock}
\chapter{Electrolyse - électrosynthèse}
\label{LC_Electrolyse_SPCL}
 \end{headerBlock}

%%%%%%%%%%%%%%%%%%%%%%%%%%%%%%%%%%%%%%%%%%%%%%%%%%%%
%%%% Références


%%%%%%%%%%%%%%%%%%%%%%%%%%%%%%%%%%%%%%%%%%%%%%%%%%%%
%%%% Plan
\begin{reportBlock}{Bibliographie}

\begin{center}
\begin{tabularx}{\textwidth}{| X | X | c | c |}\hline
Titre & Auteur(s) & Editeur (année) & ISBN \\ \hline
 Chimie tout-en-un PC/PC* Chap 13 & A. Demolliens & Nathan (2009) &  \\ 
 \hline
 \url{https://spcl.ac-montpellier.fr/moodle/course/view.php?id=62&section=10} & Académie Montpellier & & \\
 \hline
 Epreuve orale de Chimie (p194 et p398) & F. Porteu-de-Buchère & Dunod (2017) &  \\
 \hline
\end{tabularx}
\end{center}

\end{reportBlock}

\begin{reportBlock}{Plan détaillé}

\underline{Niveau} : Terminale STL-SPCL \\

\section*{Introduction pédagogique}


\paragraph*{Prérequis}
\begin{itemize}
\item 
\end{itemize}
\paragraph*{Contexte :}


\paragraph*{Notions importantes}

\begin{itemize}
\item 
\end{itemize}

\paragraph*{Objectifs}

\begin{itemize}
\item 4 premiers points du BO
\end{itemize}

\paragraph*{Difficultés}

\begin{itemize}
\item Identification des électrodes
\item Sens de déplacement des ions et des électrons
\item Difficultés calculatoires : décomposer les étapes de calculs
\item st\oe chiométrie des équations
\end{itemize}

\section*{Introduction}
On a vu dans un cours précédent que des réactions sont spontanées : rouille du fer, dépot vert dans les canalisations. Peut-on faire la réaction inverse ?\\
\textcolor{blue}{Expérience introductive :} tube en U avec des ions cuivres 2+, du solide Cu(s) se forme sur une électrode. C'est ce qu'on appelle un électrolyseur. On va essayer de comprendre comment ça marche.

\section{L'électrolyseur}
\subsection{Principe}
Schéma de l'expérience précédente. L'électrolyte est du sulfate de cuivre CuSO$_4^{2-}$, les électrodes sont en graphite. Les électrons se déplacent dans le sens opposé au courant.
\importantbox{Le générateur impose le sens du courant et donc le sens de déplacement des électrons.}

\subsection{Equations aux électrodes}
\underline{\textbf{Electrode connectée à la borne -}} : arrivée d'électrons, on a réductions des ions cuivres : c'est \textcolor{green}{la cathode}.\\

\underline{\textbf{Electrode connectée à la borne +}} : on a oxydation de l'eau : 2H$_2$O $\rightarrow$ 4H$^+$ + O$_2$ + 4e$^-$ : c'est \textcolor{green}{l'anode}.\\

On appelle transformation chimique \textbf{forcée} une transformation qui a lieu dans le sens opposé au sens thermodynamiquement favorisé.\\

L'équation de la réaction de l'électrolyse s'écrit : 
\begin{equation}
blabla
\end{equation}

\section{Bilan de matière sur un électrolyseur}
\subsection{Conservation de l'électricité}
La quantité d'électricité traversant un circuit électrique est :
\begin{equation}
    Q = I\Delta t
\end{equation}
Q peut aussi être liée à la quantité d'électrons échangés :
\begin{equation}
    Q = n_{e^-}F
\end{equation}
avec F=96500C.mol$^{-1}$ la constante de Faraday.\\

Application : $Cu^{2+}+2e^- \rightarrow Cu(s)$ donne $n_{Cu}=\frac{I\Delta t}{2F}$ et pour I=0.1A pendant 10min donne $m_{Cu}=30mg$.

\subsection{Rendement de l'électrolyse}
Le rendement de l'électrolyse est donné par :
\begin{equation}
    \eta = \frac{m_{cu,expérimental}}{m_{Cu,theorique}}
\end{equation}
\begin{itemize}
    \item $\eta=0$, pas effective,
    \item $\eta=1$, rendement maximal.
\end{itemize}

\section{Application : purification}
Dans l'industrie, on utilise une électrolyse à anode soluble pour purifier les minerais. Le minerai pas pur en cuivre va être soumis à un potentiel et va libérer le cuivre qui va se déposer sur une électrode plus pure. Schéma de l'expérience.\\

\textcolor{blue}{Manipulation quantitative :} on impose $I$ avec un générateur sur deux électrodes de cuivre. Le courant à t=0 vaut 470mA mais se stabilise à 470mA au bout de 2 minutes. On observe une couleur saumon sur la cathode qui vient du dépot de cuivre. On rince la cathode à l'eau distillée. On sèche délicatement (le dépot est fragile). On sèche au décapeur thermique.\\

Calcul du rendement, propagation d'incertitude.

\section*{Conclusion} 
Utilité : synthèse de l'eau de javel, synthèse de dihydrogène pour la pile à combustible, électrozingage.
\end{reportBlock}