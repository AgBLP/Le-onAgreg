%%%%%%%%%%%%%%%%%%%%%%%%%%%%%%%%%%%%%%%%%%%%%%%%%%%%
%%%% En-tête leçon
\begin{headerBlock}
  \chapter{Interférences à deux ondes en optique}
    \label{LP_InterferencesDeuxOndes}
\end{headerBlock}

%%%%%%%%%%%%%%%%%%%%%%%%%%%%%%%%%%%%%%%%%%%%%%%%%%%%
%%%% Références
\begin{center}
\begin{tabularx}{\textwidth}{| X | X | c | c |}
  \hline
  \rowcolor{gray!20}\multicolumn{4}{c}{Bibliographie de la leçon : } \\
  \hline 
  Titre & Auteurs & Editeur (année) & ISBN \\
  \hline
Physique Spé MP-MP* & Olivier, Gié, Sarmant & Tec \& Doc & \\
  \hline 
  Sextant &  & Hermann &  \\
  \hline 
   Tout-en-un, MP & M.-N. Sanz. & Dunod &  \\
   \hline
   Optique & Eugene HECHT & Pearson (2005) & \\
   \hline
   Optique & S. Houard & de Boeck & \\
   \hline
\end{tabularx}
\end{center}

%%%%%%%%%%%%%%%%%%%%%%%%%%%%%%%%%%%%%%%%%%%%%%%%%%%%
\begin{reportBlock}{Plan détaillé}
  \textbf{Niveau choisi pour la leçon :} CPGE
  \newline
  \textbf{Prérequis : }Modèle scalaire d'une onde, chemin optique, différence de marche, intensité lumineuse, formules trigonométriques
  \newline
  
  \textbf{Déroulé détaillé de la leçon: } \newline
\textcolor{green}{Manip introductive :} si on superpose deux lasers, il ne se passe rien. Si on les fait passer à travers un dispositif qui élargit le faisceau + une fente source + une bifente : on voit une figure d'interférence.
  \section{Interférences à deux ondes}
  \textcolor{red}{Définition :} phénomène ondulatoire qui résulte d'une interaction entre deux ondes (lumineuses) qui produit une intensité totale qui diffère de la somme des intensités individuelles.
  \subsection{Superposition de deux ondes}
  On considère deux sources ponctuelles $S_1$ et $S_2$ et des amplitudes vibratoires $a_i(M,t)=A_i\cos\left(\omega_it-\phi_{S_i} - \frac{2\pi[S_iM]}{\lambda_0i}\right)$. L'amplitude totale est : $a(M,t)=a_1(M,t)+a_2(M,t)$. L'intensité est : $I(M,t) = <a^2(M,t)>$.\\

  En développant, on obtient :
  \begin{equation}
      I = I_1 + I_2 + I_{1,2}
  \end{equation}
  avec $I_{1,2} = 2A_1A_2\cos\left(\omega_1t-\phi_1(M)\right)\cos\left(\omega_2t-\phi_2(M)\right)$

  \subsection{Conditions d'interférence, notion de cohérence}
  \begin{itemize}
      \item $I_{1,2}\neq$, dans ce cas on dit que les ondes sont cohérentes,
      \item si $\omega_1\neq\omega_2$, $I_{1,2}=0$
  \end{itemize}

  \underline{\textcolor{red}{Condition 1 :}} deux ondes de pulsations différentes sont incohérentes.\\

  Présentation du modèle du train d'onde : paquet d'onde séparés par un temps $\tau$. Comme $\phi_{S1}$ et $\phi_{S2}$ varient aléatoirement, on obtient $I_{1,2}$ non nulles sur le détecteur si :\\
  \underline{\textcolor{red}{Condition 2 :}} Il faut que les deux ondes soient issus du même train d'onde. \\

  On obtient alors la formule de Fresnel :
  \begin{equation}
      I = I_1 + I_2 + 2\sqrt{I_1I_2}\cos{\Delta\phi(M)}  \end{equation}
      où $\Delta\phi(M) = \frac{2\pi([S_2M]-[S_1M])}{\lambda_0}$

\section{Exemple d'interféromètre : les trous d'Young}
Cf photo.

\begin{equation}
    I = 2I_0\left[1+\cos\left(\frac{2\pi ax}{D}\right)\right]
\end{equation}
Succesion de franges brillantes et de franges sombres. La distance entre deux franges brillantes est appelée \textcolor{red}{interfrange} notée $i$ qui vaut ici : $i=\frac{\lambda_0D}{a}$.\\
\textcolor{green}{Manipulation 2 (quantitative) :} Mesure de l'interfrange de la figure d'interférences pour en déduire $a$. On mesure $a=0.16\pm0.04$mm à comparer avec la valeur $a_{fabricant}=0.2$mm.

\section{Notion de cohérence spatiale}
Effet de la largeur de la source en reprenant le problème avec deux sources séparées par une distance $b$. On obtient à l'aide des formules obtenues dans la partie précédente : 
\begin{equation}
    I_{tot} = 4I_0\left[1+\cos\left(\frac{\pi ab}{\lambda D}\right)\cos\left(\frac{2\pi a x}{\lambda D}+\frac{2\pi a b}{\lambda D}\right)\right]
\end{equation}
\section{Conclusion}
Ouverture sur les dispositifs à division du front d'onde et division d'amplitude.
\end{reportBlock}


\begin{reportBlock}{Questions posées par l’enseignant (avec réponses)}
   \textbf{C : D'autres phénomènes d'interférences autres que lumineuses ?}  \textcolor{purple}{Oui, exemple de la cuve à onde.} Qu'est-ce qui fait la spécificité des interférences des ondes lumineuses ? \textcolor{purple}{On peut faire des mesures super précises.}\\
   \textbf{C : Conditions de cohérence pour l'eau ?}  \textcolor{purple}{On somme directement les amplitudes, il n'y a pas de notion de cohérence pour une onde mécanique.}\\
   \textbf{C : Dépendence de la durée d'intégration ? Odg temps de réponse d'un détecteur ?}  \textcolor{purple}{Période de la lumière  : $10^{-15}$s, \oe uil : $10^{-2}$s,  photorésistance $10^{-2}$s, photodiode (standard): $10^{-6}$s, thermopile : $1$s}\\
   \textbf{C : lien entre intensité $I$ et éclairement $\epsilon$ ?}  \textcolor{purple}{On a $\epsilon = KI = K<s^2(M,t)>$, où $<...>$ représente la valeur moyenne temporelle, K est une constante qui dépend du détecteur et $s(M,t)$ représente une composante du champ électrique de la lumière par rapport à un axe perpendiculaire à sa direction de propagation. L'éclairement est la puissance surfacique moyenne de l'onde lumineuse (autrement dit la valeur moyenne temporelle du vecteur de Poynting).}\\
   \textbf{C : Pourquoi il faut un vide entre deux trains d'ondes ?}  \textcolor{purple}{Lié à la désexcitation de l'atome, la durée de vie d'un niveau d'énergie.} Un train d'onde c'est un photon du coup ? \textcolor{purple}{C'est l'aspect ondulatoire du photon.}\\
   \textbf{C : C'est quoi la cause de l'incohérence spatiale ?} \textcolor{purple}{Emission de trains d'onde de phase à l'origine aléatoire suivant l'atome émetteur.}\\

   \textbf{C : Différences/avantages interférométrie à division d'amplitude/ division du front d'onde ?} \textcolor{purple}{Division du front d'onde : on fait interférer de la lumière provenant de deux sources différentes. Les interférences ne sont pas localisées mais il y a un problème de brouillage du fait de la cohérence spatiale des sources. Division d'amplitude : on fait interférer de la lumière provenant d'un même faisceau incident dont on a séparé en deux (au moins) l'amplitude. Il n'y a pas de problème lié à la cohérence spatiale de la source mais le prix à payer est la localisation des interférences (à l'infini pour une lame d'air, à distance finie pour un coin d'air). L'avantage est de pouvoir utiliser des sources de lumière très étendues, on gagne en luminosité.}\\

   \textbf{C : Stratégies à mettre en \oe uvre pour éviter $20\%$ d'erreur sur les mesures ?} \textcolor{purple}{Caméra CCD, mettre une lentille pour agrandir l'image} Ca change quoi avec une lentille ? \textcolor{purple}{On remplace $D$ par $f'$ dans la formule de $I_{tot}$.} C'est mieux du coup ? \textcolor{purple}{On peut mesurer $f'$ de façon assez précise} Quoi d'autre ? \textcolor{purple}{Pied à coulisse, banc optique, ...}\\
   
\end{reportBlock}
