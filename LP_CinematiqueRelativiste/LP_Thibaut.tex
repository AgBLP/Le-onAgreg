%%%%%%%%%%%%%%%%%%%%%%%%%%%%%%%%%%%%%%%%%%%%%%%%%%%%
%%%% En-tête leçon
\begin{headerBlock}
  \chapter{Cinématique relativiste. Expérience de Michelson et Morlay}
    \label{LP_CinematiqueRelativiste}
\end{headerBlock}

%%%%%%%%%%%%%%%%%%%%%%%%%%%%%%%%%%%%%%%%%%%%%%%%%%%%
%%%% Références
\begin{center}
\begin{tabularx}{\textwidth}{| X | X | c | c |}
  \hline
  \rowcolor{gray!20}\multicolumn{4}{c}{Bibliographie de la leçon : } \\
  \hline 
  Titre & Auteurs & Editeur (année) & ISBN \\
  \hline
  Relativité & C. Semay & Dunod & \\
  \hline
  La relativité & A. Einstein & Payot & \\
  \hline
  Relativité et Invariance - Fondements et applications & J.-P. Pérez & Dunod & \\
  \hline
  Relativité Restreinte (exos) & Y. Simon & Armand Colin & \\
  \hline
  Mécanique 1 (Chap 1) & BFR & Dunod (1984) & \\
  \hline
  \url{http://blog.ac-versailles.fr/profbcasanova/public/TS4_phychim/P8_muons-2014.pdf} & & & \\
  \hline
  Mécanique 1 (Chap 15) & H. Gié et J.-P. Sarmant & Tec\&Doc (1984) & \\
\end{tabularx}
\end{center}

\begin{reportBlock}{Commentaires des années précédentes :}
    \begin{itemize}
        \item \textbf{2016 :} Les notions d’événement et d’invariant sont incontournables dans cette leçon,
        \item \textbf{2015 :} Le jury rappelle qu’il n’est pas forcément nécessaire de mettre en \oe uvre des vitesses relativistes pour être capable de détecter et de mesurer des effets relativistes,
        \item \textbf{2014 :} Cette leçon exige une grande rigueur dans l’exposé tant sur les notions fondamentales de relativité restreinte que sur les référentiels en jeu. Elle invite les candidats à faire preuve d’une grande pédagogie pour présenter des notions a priori non intuitives et faire ressortir les limites de l’approche classique. Un exposé clair des notions d’invariant relativiste est attendu.
    \end{itemize}
\end{reportBlock}

%%%%%%%%%%%%%%%%%%%%%%%%%%%%%%%%%%%%%%%%%%%%%%%%%%%%
\begin{reportBlock}{Plan détaillé}
  \textbf{Niveau choisi pour la leçon :} L3
  \newline
  \textbf{Prérequis : }
  \newline

\section*{Introduction}

\section{Emergence de la relativité restreinte}

\subsection{Transformation de galilée en électromagnétisme}
Les équations de l'électromagnétisme ne sont valables que dans le référentiel de l'éther pour concilier mécanique classique et électromagnétisme.
\subsection{Expérience de Michelson et Morlay}
Montrer les résultats de l'expérience de Michelson et Morlay (Michelson et Morlay, 1887). 

\subsection{Postulat d'Einstein}

\section{Changement de référentiel relativiste}

\subsection{Evènements}

\subsection{Transformation de Lorentz}

\subsection{Intervalle d'espace-temps}

\section{Conséquences physiques}
\subsection{Dilatation du temps}
\textcolor{blue}{Expérience :} utilisaion du programme Gum\_C pour la détermination de $\gamma$ des muons d'après l'expérience de Frish et Smith.
\subsection{Contraction des longueurs}

\end{reportBlock}