%%%%%%%%%%%%%%%%%%%%%%%%%%%%%%%%%%%%%%%%%%%%%%%%%%%%
%%%% En-tête leçon
\begin{headerBlock}
  \chapter{Conversion de puissance électromécanique}
    \label{LP_ConversionPuissance}
\end{headerBlock}

%%%%%%%%%%%%%%%%%%%%%%%%%%%%%%%%%%%%%%%%%%%%%%%%%%%%
%%%% Références
\begin{center}
\begin{tabularx}{\textwidth}{| X | X | c | c |}
  \hline
  \rowcolor{gray!20}\multicolumn{4}{c}{Bibliographie de la leçon : } \\
  \hline 
  Titre & Auteurs & Editeur (année) & ISBN \\
  \hline
  TP Moteur &  & Site agreg Montrouge &   \\
  \hline 
  Cours Jérémy Neveu & J. Neveu & Site agreg Montrouge & \\
  \hline
  Tout-en-un PSI/PSI* (Chap 27 et 28) & S. Cardini & Dunod (2022) & \\
  \hline
\end{tabularx}
\end{center}

\begin{reportBlock}{Commentaires des années précédentes :}
    \begin{itemize}
        \item \textbf{2017 :} Une approche à l’aide des seules forces de Laplace est insuffisante. Les candidats doivent aussi s’interroger sur l’intérêt d’utiliser des matériaux ferromagnétiques dans les machines électriques,
        \item \textbf{2016 :} Afin de pouvoir aborder des machines électriques de forte puissance, le rôle essentiel du fer doit être considéré car les forces électromagnétiques ne se réduisent pas aux seules actions de Laplace s’exerçant sur les conducteurs traversés par des courants,
        \item \textbf{2015 :} Il est souhaitable de préciser le rôle de l’énergie magnétique lors de l’étude des convertisseurs électromécaniques constitués de matériaux ferromagnétiques linéaires non saturés,
        \item Dans le cas des machines électriques, les candidats sont invités à réfléchir au rôle du fer dans les actions électromagnétiques qui peuvent également être déterminées par dérivation d’une grandeur énergétique par rapport à un paramètre de position.
    \end{itemize}
\end{reportBlock}

%%%%%%%%%%%%%%%%%%%%%%%%%%%%%%%%%%%%%%%%%%%%%%%%%%%%
\begin{reportBlock}{Plan détaillé}
  \textbf{Niveau choisi pour la leçon :} 
  \newline
  \textbf{Prérequis : }
  \newline

\section*{Introduction}
Parler de l'intérêt dans la vie de tous les jours :
\begin{itemize}
    \item Conversion mécanique - électrique : ex : éolienne, production d'électricité, courants de Foucault
    \item Conversion électrique - mécanique : moteur, haut-parleur, freinage par courants de Foucault
\end{itemize}

\section{Principe de la conversion}
\subsection{Exemple du rail de Laplace}
Faire le schéma en faisant attention à la convention d'orientation : choisir U et I, le reste c'est avec le bonhomme de Maxwell la règle de la main droite.\\

Neutralité du conducteur : $\rho_{e^-}=\rho_{ions}$ implique que la force de Lorentz s'écrit comme la force de Laplace $d\mathbf{F_{L}}=\mathbf{j}d\tau\wedge\mathbf{B}=I\mathbf{dl}\wedge\mathbf{B}$.\\

\subsection{Bilan de puissance}
Puissance mécanique : $dP_L = d\mathbf{F_{L}}\cdot\mathbf{v}=-I(\mathbf{v}\wedge\mathbf{B})\cdot\mathbf{dl}$. Comme le champ électromoteur s'écrit par définition : $\mathbf{E_m}=\mathbf{v}\wedge\mathbf{B}-\partialD{\mathbf{A}}{t}$, et la force électromotrice s'écrit $e=\mathbf{E_m}\cdot\mathbf{dl}$, on a \textcolor{red}{\underline{\textbf{en régime permanent }}} la relation fondamentale de conversion de puissance électromécanique :
\begin{equation}
    dP_L=-Ide\Rightarrow \boxed{P_{L} + P_{e} = 0}
\end{equation}
\importantbox{I est dans le sens de $\mathbf{dl}$}.

\subsection{Interprétation}
Deux cas de figure :
\begin{itemize}
    \item Pour un circuit électrique mobile dans un champ B permanent, si de la puissance électrique est fourni au circuit via des phénomènes d'induction ($P_e>0$) alors la même puissance est retirée du système mécanique ($P_L<0$). Par exemple, les moteurs.
    \item Si de la puissance est fourni au système mécanique via des forces de Laplace ($P_L>0$) alors la même puissance est retirée du système électrique via des phénomènes inductifs ($P_e<0$). Par exemple : les alternateurs dans les éoliennes/centrales électrique pour produire de l'électricité.
\end{itemize}

\textbf{Transition :} On va étudier dans une deuxième partie le principe du moteur à courant continu.

\section{Moteur à courant continu}
Cf cours de Jérémy.\\
Intérêt : commander la vitesse de rotation de l'arbre du moteur par la tension qu'on lui impose. Utilser pour les moteurs de faibles puissance (jusqu'à une centaine de kW).
\subsection{Structure}
\textcolor{blue}{Expérience qualitative :} Montrer le MCC de la collection. Décrire le stator, rotor, collecteur-balais. Shéma à une spire (cf Cours Jérémy Neveu).\\
Arriver sur les relations fondamentales du MCC :
\begin{align}
    E &= K\Phi_S\Omega \\
    <\Gamma(t)> &= K\Phi_SI_R
\end{align}
    
\subsection{Etude du MCC}
Avantage du moteur synchrone : réversible (si mode générateur : alternateur)
Attention : utiliser des multimètres métrix, poids de 100g à 5kg (x2 à chaque fois). Pied à coulisse pour $r$, chrono pour mesure $\Omega$ à la main, typiquement $v$ stable pour $\Delta x=10$~cm (cf tracker).\\
\textcolor{blue}{Manip 1 : }Déterminer $K_{em}$ la résistance du moteur et la comparer à la valeur donnée dans la notice. Je trouve $K_{em}=1.30(2)$~V.rad.s$^{-1}$.\\
\textcolor{blue}{Manip 2 :} Tracer $\eta$ en fonction de la puissance utile et comparer à charge nominale = 2kg et $\eta_{nom}=50\%$. Le courant lu est assez critique, ne pas hésiter à refaire plusieurs fois.\\

\textcolor{yellow}{Bonus :} Montrer la réversibilité du moteur avec une charge lourde pour voir apparaître ue tension lors de la descente.
\textbf{Transition : }Nous voyons bien que nous n'avons pas un rendement de 100\%, il y a des pertes.
\subsection{Bilan de puissance}
Faire schéma fonctionnnement du moteur (cf Tout-en-un PSI/PSI* p818).
\begin{itemize}
    \item Pertes cuivre : effet Joule dans les conducteurs constituant les bobinages de la machine,
    \item Pertes fer : puissance dissipée par le parcourt du cycle d'hystérésis des aimants en fer doux + puissance dissipée par effet Joule par les courants de Foucault 
    
\end{itemize}

\textbf{Transition :} Ces moteurs sont aujourd'hui mis en concurrence par les moteurs à courant alternatif dont nous allons détailler le fonctionnement pour les moteurs synchrones 

\section{Machines synchrones}
\subsection{Génération d'un champ magnétique tournant (bonus ?)}
\subsection{Champ rotorique et champ statorique}
\subsection{Point de fonctionnement et stabilité}

\section*{Ouverture}
Moteur asynchrone.
\end{reportBlock}