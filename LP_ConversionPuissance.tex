%%%%%%%%%%%%%%%%%%%%%%%%%%%%%%%%%%%%%%%%%%%%%%%%%%%%
%%%% En-tête leçon
\begin{headerBlock}
  \chapter{Conversion de puissance électromécanique}
    \label{LP_ConversionPuissance}
\end{headerBlock}

%%%%%%%%%%%%%%%%%%%%%%%%%%%%%%%%%%%%%%%%%%%%%%%%%%%%
%%%% Références
\begin{center}
\begin{tabularx}{\textwidth}{| X | X | c | c |}
  \hline
  \rowcolor{gray!20}\multicolumn{4}{c}{Bibliographie de la leçon : } \\
  \hline 
  Titre & Auteurs & Editeur (année) & ISBN \\
  \hline
  TP Moteur &  & Site agreg Montrouge &   \\
  \hline 
  Cours Jérémy Neveu & J. Neveu & Site agreg Montrouge & \\
  \hline
\end{tabularx}
\end{center}

%%%%%%%%%%%%%%%%%%%%%%%%%%%%%%%%%%%%%%%%%%%%%%%%%%%%
\begin{reportBlock}{Plan détaillé}
  \textbf{Niveau choisi pour la leçon :} 
  \newline
  \textbf{Prérequis : }
  \newline

\section*{Introduction}
Parler de l'intérêt dans la vie de tous les jours :
\begin{itemize}
    \item Conversion mécanique - électrique : ex : éolienne, production d'électricité, courants de Foucault
    \item Conversion électrique - mécanique : moteur, haut-parleur, freinage par courants de Foucault
\end{itemize}

\section{Principe de la conversion}
\subsection{Rails de Laplace}
\subsection{Bilan de puissance}

\section{Moteurs synchrones}
Cf cours de Jérémy
\subsection{Etude du MCC}
Avantage du moteur synchrone : réversible (si mode générateur : alternateur)
Attention : utiliser des multimètres métrix, poids de 100g à 5kg (x2 à chaque fois). Pied à coulisse pour $r$, chrono pour mesure $\Omega$ à la main, typiquement $v$ stable pour $\Delta x=10$~cm (cf tracker).\\
\textcolor{blue}{Manip 1 : }Déterminer $K_{em}$ la résistance du moteur et la comparer à la valeur donnée dans la notice. Je trouve $K_{em}=1.30(2)$~V.rad.s$^{-1}$.\\
\textcolor{blue}{Manip 2 :} Tracer $\eta$ en fonction de la puissance utile et comparer à charge nominale = 2kg et $\eta_{nom}=50\%$. Le courant lu est assez critique, ne pas hésiter à refaire plusieurs fois.

\subsection{Bilan de puissance}

\end{reportBlock}