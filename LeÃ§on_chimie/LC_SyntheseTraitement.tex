\begin{headerBlock}
\chapter{Synthèse, traitement et caractérisations}
\label{LC_SyntheseTraitement}
 \end{headerBlock}

%%%%%%%%%%%%%%%%%%%%%%%%%%%%%%%%%%%%%%%%%%%%%%%%%%%%
%%%% Références


%%%%%%%%%%%%%%%%%%%%%%%%%%%%%%%%%%%%%%%%%%%%%%%%%%%%
%%%% Plan
\begin{reportBlock}{Bibliographie}

\begin{center}
\begin{tabularx}{\textwidth}{| X | X | c | c |}\hline
Titre & Auteur(s) & Editeur (année) & ISBN \\ \hline
 &  &  &  \\ 
 \hline
\end{tabularx}
\end{center}

\end{reportBlock}

\begin{reportBlock}{Plan détaillé}

\underline{Niveau} : 1ère Générale \\

\section*{Introduction pédagogique}


\paragraph*{Prérequis}
\begin{itemize}
\item Modélisation d'une transfo chimique par une réaction chimique
\item Caractéristiques physico-chimiques d'espèces chimiques
\item Techniques expérimentales
\end{itemize}
\paragraph*{Contexte :}
Leçon de milieu d'année après Structure des entités organiques. 2ème chimique de chimie orga.

\paragraph*{Notions importantes}

\begin{itemize}
\item 
\end{itemize}

\paragraph*{Objectifs}

\begin{itemize}
\item Compétences expérimentales : montage à reflux, filtration, lavage, séparation. Caractérisations par CCM
\item 
\end{itemize}

\paragraph*{Difficultés}

\begin{itemize}
\item Vocabulaire spécifique
\item Aspect inventaire
\end{itemize}

\section*{Introduction }
Synthèse organique dans la vie de tous les jours  synthèse du paracétamol, synthèse de l'indigo (pour les jeans). Intérêt : produire de façon massive des molécules qui existent déjà dans la vie de tous les jours.

\section{Synthèse en chimie organique}

\subsection{Qu'est-ce-qu'une synthèse ?}
\textcolor{green}{Définition :} Une synthèse  = ensemble des étapes de fabrication d'une ou plusieurs espèces chimiques purs, implicant a transformation chimique de réactifs.\\

Ex : acide benzoïque (formule chimique semi-développée pas à connaitre). Présentation du montage. A la fin de synthèse, on a retirer le produit obtenu et on l'a filtré à l'aide d'une papier filtre. On va étudier le filtrat. Présentation de la filtration sur Büchner.\\

Etapes d'une synthèse :
\begin{enumerate}
    \item Transformation chimique
    \item Extraction du milieu réactionnel
    \item Identification du produit et analyse de la pureté
    \item Purification éventuelle
\end{enumerate}

\subsection{La transformation chimique}
Les réactifs sont transformés en produits selon l'équation de réaction.\\

Ex : Synthèse de l'acide benzoïque 
\begin{chemmath}
4MnO_4^-(aq) + 3C_6H_5CH_2OH(l) \longrightarrow 3C_6H_5COO^-(aq) + 4MnO_2(s) +4HO^-(aq) + 4H_2O(l)
\end{chemmath}

\underline{Paramètres :}
\begin{itemize}
    \item Chauffage : accélère la réaction (ex : chauffage à reflux)
    \item Solvant : les réactifs doivent y être tous solubles si possible
    \item Autres : agitation, concentrations réactifs, catalyseur, ...
\end{itemize}

\section{Extraction du produit du mélange réactionnel}
\subsection{Cas où l'espèce est un liquide}
\begin{itemize}
    \item Si l'espèce est non-miscible avec le solvant : décantation
    \item Si l'espèce est miscible avec le solvant et que sa température d'ébullition diffère de plus de 20°C avec celle du solvant : distillation fractionnée
\end{itemize}
\subsection{Cas où l'espèce est un solide}
On réalise une filtration simple ou sur filtre sur Büchner.
\subsection{Cas où l'espèce est un soluté}
On change les conditions expérimentales. On peut :
\begin{itemize}
    \item baisser le pH
    \item diminuer la température
    \item introduire une espèce plus soluble : cristalliser le soluté
    \item ou sinon on peut réaliser une extraction liquide-liquide
\end{itemize}

\section{Efficacité de la synthèse}
\subsection{Identification du produit}
\begin{itemize}
    \item CCM,
    \item mesure de la température de fusion,
    \item spectre infrarouge,
    \item mesure de la masse molaire,
    \item mesure $T_{eb}$ pour un liquide.
\end{itemize}

\subsection{Purification}
Cette étape permet d'éliminer les impuretés contenues dans le brut.\\

Ex : recristallisation. On dissout le produit d'intérêt dans un solvant dans lequel il est soluble à chaud mais pas à froid, les impuretés sont solubles à chaud ou à froid. Etalonnage du banc Kofler.\\

\section*{Conclusion}
Dans une prochaine leçon, on verra les notions de rendements.
\end{reportBlock}