\begin{changemargin}{-1.5cm}{-0cm}

Code couleur pour les manips :
\begin{center}
\textcolor{green}{Expérience déjà faite, sur laquelle je suis à l'aise}\\
\textcolor{yellow}{Retravailler un geste, l'interprétation de l'expérience}\\
\textcolor{red}{Manip à faire en priorité}\\
\end{center}

\begin{tabularx}{\paperwidth-2cm}{| X | X | c | X |}
  \hline
  \rowcolor{gray!20}\multicolumn{4}{c}{Avancement préparation oraux Leçons Chimie} \\
  \hline 
  Titre de la leçon & Elément imposé & Expérience & Niveau \\
  \hline
  \textbf{De la structure à la polarité d'une entité} & logiciel de représentation molécule & \textcolor{red}{\textbf{0\%}}  & 1ère générale - spécialité  \\
  \hline
  \textbf{Evolution d'un système chimique} & Python, déterminer la compo d'un système & \textcolor{red}{\textbf{0\%}}  & 1ère générale - spécialité, Python : cf agreg ou CAPES 2023  \\
  \hline
  \textbf{Dosage} & Dosage par étalonnage (ex: sérum phy) & \textcolor{red}{\textbf{0\%}}  & 1ère générale - spécialité \\
  \hline
  \textbf{Synthèse, traitement, caractérisations} p\pageref{LC_SyntheseTraitement} & Filtration sous vide & Synthèse acide benzoïque  & 1ère générale - spécialité \\
  \hline
  \hline
  \textbf{Oxydants et réducteurs} & Réaliser une pile & \textcolor{green}{Pile Daniell}  & Term générale - spécialité \\
  \hline
  \textbf{Chimie analytique quantitative et fiabilité} & Titrage par conductimétrie & \textcolor{green}{Titrage ions Cl dans sérum phy}  & Term générale - spécialité  \\
  \hline
  \textbf{Evolution spontanée d'un système} & Déterminer $Q_{final}$ & \textcolor{red}{\textbf{0\%}}  & Term générale - spécialité  \\
  \hline
  \textbf{Cinétique et catalyse} & Mise en évidence d'un effet catalyseur & \textcolor{red}{\textbf{0\%}}  & Term générale - spécialité  \\
  \hline
  \textbf{Stratégies en synthèse organique} & Protocole d'optimisation d'un rendement/vitesse & \textcolor{red}{\textbf{0\%}}  & Term générale - spécialité  \\
  \hline
  \hline
   \textbf{Structure spatiale des molécules} & Différencier 2 stéréoisomères & \textcolor{red}{\textbf{0\%}}  & 1ère STL  \\
  \hline
  \textbf{Réactions acide-base en solution acqueuse} & Préparer une solution tampon & \textcolor{red}{\textbf{0\%}}  & 1ère STL  \\
  \hline
  \textbf{Solvants et solubilité} & Etudier facteur influençant la solubilité & \textcolor{red}{\textbf{0\%}}  & 1ère STL  \\
  \hline
  
  
\end{tabularx}
\end{changemargin}
\newpage
\begin{changemargin}{-1.5cm}{0cm}

Code couleur pour les manips :\\
\begin{center}
\textcolor{green}{Expérience déjà faite, sur laquelle je suis à l'aise}\\
\textcolor{yellow}{Retravailler un geste, l'interprétation de l'expérience}\\
\textcolor{red}{Manip à faire en priorité}\\
\end{center}

\begin{tabularx}{\paperwidth-2cm}{| X | X | X | X |}
  \hline
  \rowcolor{gray!20}\multicolumn{4}{c}{Avancement préparation oraux Leçons Chimie} \\
  \hline 
  Titre de la leçon & Elément imposé & Expérience & Niveau \\
  \hline
  \textbf{Synthèse, purification, contrôle pureté liquide organique} & CCM & \textcolor{red}{\textbf{0\%}}  & 1ère SPCL  \\
  \hline
  \textbf{Réactivité des alcools} & Recristallisation & \textcolor{green}{Aspirine}  & 1ère SPCL  \\
  \hline
  \textbf{Réactions de synthèse, sites electrophiles, nucléophiles, formalisme flèches courbes} & Montage à reflux & \textcolor{red}{\textbf{0\%}}  & 1ère SPCL \\
  \hline
  \hline
  \textbf{Distillation et diagrammes binaires}, p\pageref{LC_Distillation} & Distillation & \textcolor{red}{\textbf{0\%}}  & Terminale SPCL \\
  \hline
   \textbf{Techniques spectroscopiques} & Déterminer C d'après une courbe d'étallonnage & \textcolor{red}{\textbf{0\%}}  & Terminale SPCL \\
  \hline
  \textbf{Conductivité} & Titrage par précipitation & \textcolor{green}{\textbf{90\%}}  & Terminale SPCL, revoir plan pour caler les expériences\\
  \hline
  \textbf{Solubilité} & Extraire sélectivement les ions d'un mélange & \textcolor{red}{\textbf{0\%}}  & Term générale - spécialité \\
  \hline
  \textbf{Oxydoréduction} & Titrage dont la réaction support est une oxydo-reduction & \textcolor{red}{\textbf{0\%}}  & Term générale - spécialité \\
  \hline
  \textbf{Réactivité des dérivés d'acide} & Montage Dean-Stark & \textcolor{green}{Synthèse arôme de banane}  & Term générale - spécialité \\
  \hline
  \textbf{Electrolyse, électrosynthèse} & Réaliser une électrolyse à anode soluble & \textcolor{red}{Elec à anode soluble}- Florence Porteu-de-Buchère - Dunod p190 & Term générale - spécialité \\
  \hline
\end{tabularx}
\end{changemargin}

\newpage
\begin{changemargin}{-1.5cm}{-0cm}
    
\begin{tabularx}{\paperwidth-2cm}{| X | X | X | X |}
  \hline
  \rowcolor{gray!20}\multicolumn{4}{c}{Avancement préparation oraux Leçons Chimie} \\
  \hline 
  Titre de la leçon & Elément imposé & Expérience & Niveau \\
  \hline
   \textbf{Energie chimique - exemple des combustions} & Estimer pouvoir calo d'une combustion & \textcolor{green}{Voir Nathan ST2S}  & 1ère/terminale STI2D \\
  \hline
   \textbf{Distillation et diagrammes binaires} & Distillation & \textcolor{yellow}{}  & 1ère/terminale STI2D \\
   \hline  
   \hline
  \textbf{Molécules d'intérêt biologique} & Différencier aldéhyde/cétone & \textcolor{green}{Voir Nathan ST2S} & 1ère ST2S \\
  \hline
  \textbf{Gestion des risques au laboratoire de chimie} p\pageref{LC_GestionRisquesLabo} & Mettre en \oe uvre un protocol de neutralisation & \textcolor{green}{Neutralisation acide} & 1ère ST2S \\
  \hline
   \textbf{Biomolécules et énergie} p\pageref{LC_BiomoleculesEnergie} & Réaliser l'hydrolyse d'un glucide complexe & \textcolor{green}{Voir Nathan ST2S}  & 1ère ST2S \\
  \hline 
  \textbf{L'eau, propriétés physiques et chimiques} & Réaction une extraction liquide/liquide & \textcolor{red}{0\%} & 1ère ST2S \\
  \hline 
  \hline
  \textbf{Chimie et alimentation} & Teneur en vitamine C d'un aliment/médicament & \textcolor{red}{0\%} & Terminale ST2S \\
  \hline
  \hline
  \textbf{Transformations chimiques en solution acqueuse} & Déterminer $K_{eq}$ d'une réaction (A/B, précipitation, redox) & \textcolor{red}{0\%} & MPSI \\
  \hline
  \textbf{Acides et bases} & Déterminer $K_a$ & \textcolor{red}{0\%} & MPSI \\
  \hline
  \textbf{Solvants} & Déterminer constante de partage & \textcolor{red}{0\%} & MPSI \\
  \hline
  \textbf{Structure et prop des solides} & Utiliser avogadro ou vesta et déterminer des paramètres géométriques & \textcolor{red}{0\%} & MPSI \\
  \hline
  \textbf{Cinétique homogène} & Déterminer $E_a$ & \textcolor{red}{hydrolyse du chlorure de tertiobutyle suivie par pH-métrie} - F. Porteu de Buchère p40& MPSI \\
  \hline
  \textbf{Diagrammes E-pH} p\pageref{LC_DiagrammeEpH} & Mettre en \oe uvre une médiamutation & \textcolor{red}{0\%} & MPSI \\
  \hline
  \textbf{Liaisons chimiques} & Utiliser Vesta ou avogadro & \textcolor{red}{0\%} & MPSI \\
  \hline
\end{tabularx}
\end{changemargin}

\newpage
\begin{changemargin}{-1.5cm}{-0cm}
Code couleur pour les manips :\\
\begin{center}
\textcolor{green}{Expérience déjà faite, sur laquelle je suis à l'aise}\\
\textcolor{yellow}{Retravailler un geste, l'interprétation de l'expérience}\\
\textcolor{red}{Manip à faire en priorité}\\
\end{center}

\begin{tabularx}{\paperwidth-2cm}{| X | X | X | X |}
  \hline
  \rowcolor{gray!20}\multicolumn{4}{c}{Avancement préparation oraux Leçons Chimie} \\
  \hline 
  Titre de la leçon & Elément imposé & Expérience & Niveau \\
  \hline
  \textbf{Utilisation 1er principe pour la détermination de grandeurs physico-chimiques} & Déterminer $\Delta_rH_{reaction}$ & \textcolor{red}{pas fait} - F. Porteu p75 & PSI \\
  \hline
  \textbf{Optimisation d'un procédé chimique} & Mettre en évidence influence T ou P ou catalyseur sur une réaction & \textcolor{red}{0\%} & PSI \\
  \hline
  \textbf{Corrosion humide des métaux} & Protection contre la corrosion (clou+gel agar-agar) & \textcolor{red}{Passivation du fer}-Cachau Redox & PSI \\
  \hline
  \textbf{Générateurs électrochimiques} & Montrer influence de facteurs sur $e_{vide}$ d'une pile & \textcolor{red}{0\%} & PSI \\
  \hline
  \textbf{Conversion d'énergie électrique en énergie chimique} & Déterminer $\eta_{faradique}$ d'un électrolyseur & \textcolor{red}{0\%} & PSI \\
  \hline
  \textbf{Cinétique électrochimique} & Tracer courbe i-E & \textcolor{red}{0\%} & PSI \\
  \hline
  \textbf{Application 2nd principe à une transfo chimique} & Déterminer une constante d'équilibre thermo & \textcolor{red}{0\%} & PSI \\
  \hline
  \hline
  \textbf{Diagramme E-pH} & Réaliser un procédé industriel à l'échelle du labo & \textcolor{red}{0\%} & TSI 2 \\
  \hline
  \textbf{Déplacement de l'équilibre chimique} & Utiliser un bain thermostaté & \textcolor{red}{0\%} & TSI 2 \\
  \hline  
\end{tabularx}

\begin{headerBlock}
\begin{itemize}
    \item Réaliser l'hydrolyse d'un complexe (leçon biomolécules et énergie - 1ère ST2S)
    \item 
\end{itemize}

\end{headerBlock}
\end{changemargin}

