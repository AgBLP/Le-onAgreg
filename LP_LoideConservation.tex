%%%%%%%%%%%%%%%%%%%%%%%%%%%%%%%%%%%%%%%%%%%%%%%%%%%%
%%%% En-tête leçon
\begin{headerBlock}
  \chapter{Lois de conservation en mécanique}
    \label{LP_LoisConservation}
\end{headerBlock}

%%%%%%%%%%%%%%%%%%%%%%%%%%%%%%%%%%%%%%%%%%%%%%%%%%%%
%%%% Références
\begin{center}
\begin{tabularx}{\textwidth}{| X | X | c | c |}
  \hline
  \rowcolor{gray!20}\multicolumn{4}{c}{Bibliographie de la leçon : } \\
  \hline 
  Titre & Auteurs & Editeur (année) & ISBN \\
  \hline
  Mécanique 1 & H. Gié et J.-P. Sarmant & Tec\&Doc (1984) & \\
  \hline
  Mécanique 2ème année & H. Gié et J.-P. Sarmant & Tec\&Doc (1996) & \\
  \hline
\end{tabularx}
\end{center}

%%%%%%%%%%%%%%%%%%%%%%%%%%%%%%%%%%%%%%%%%%%%%%%%%%%%
\begin{reportBlock}{Plan détaillé}
  \textbf{Niveau choisi pour la leçon :} CPGE 2ème année
  \newline
  \textbf{Prérequis : }
  \newline


\section{Quantité de mouvement}

\subsection{PFD}
\subsection{Système pseudo-isolé}
\subsection{Collision élastique}
\textcolor{blue}{Expérience : }Conservation de $\mathbf{p}$ de mobiles autoporteurs.

\section{Moment cinétique}

\subsection{TMC}
\subsection{Système pseudo-isolé}
\subsection{Cas d'une force centrale}

\section{Energie}

\subsection{TEC}
\subsection{Forces conservatives}
\subsection{Cas du pendule simple}

\section*{Conclusion}
Ouverture sur le théorème de N\oe ther.

\end{reportBlock}