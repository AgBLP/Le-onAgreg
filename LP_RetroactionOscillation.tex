%%%%%%%%%%%%%%%%%%%%%%%%%%%%%%%%%%%%%%%%%%%%%%%%%%%%
%%%% En-tête leçon
\begin{headerBlock}
  \chapter{Rétroactions et oscillations.}
  \label{LP_RetroactionOscillation} 
\end{headerBlock}




%%%%%%%%%%%%%%%%%%%%%%%%%%%%%%%%%%%%%%%%%%%%%%%%%%%%
%%%% Références
\begin{center}
\begin{tabularx}{\textwidth}{| X | X | c | c |}
  \hline
  \rowcolor{gray!20}\multicolumn{4}{c}{Bibliographie de la leçon : } \\
  \hline 
  Titre & Auteurs & Editeur (année) & ISBN \\
  \hline
  Electronique & Pérez & Dunod & \\
  \hline 
  Tout-en-un PSI/PSI* & Cardini & Dunod &    \\
  \hline 
  Physique Spé. PSI/PSI* & Olivier, More, Gié & Tec\&Doc & \\
  \hline 
  Cours & Jérémy Neveu & & \\
  \hline
\end{tabularx}
\end{center}

%%%%%%%%%%%%%%%%%%%%%%%%%%%%%%%%%%%%%%%%%%%%%%%%%%%%

%%%%%%%%%%%%%%%%%%%%%%%%%%%%%%%%%%%%%%%%%%%%%%%%%%%%
%%%% Plan
\begin{reportBlock}{Plan détaillé}

  \textbf{Niveau choisi pour la leçon :} Licence 3
  \newline
  \textbf{Prérequis} : \begin{itemize}
      \item 
  \end{itemize}

  \textbf{Déroulé détaillé de la leçon: }  
  
  \section*{Introduction}

  \section{Rétroactions}
  \subsection{Enthalpie de changement d'état} 


  \section{Oscillations}
\subsection{Oscillateur à Pont de Wien}
Prendre $R_1\sim 1k\Omega$ pour l'amplification et $R=1k\Omega$ dans le filtre. Mettre $t-t_0$ pour le fit et enregistrer les oscillations par clé USB directement sur l'oscillo. Fitter par 
\begin{equation}
    A\exp\left(-a\omega_0(t+t_0)\right)\times\sin\left(\sqrt{1-\alpha^2}\omega_0(t+t_0)\right)
\end{equation}


\end{reportBlock}